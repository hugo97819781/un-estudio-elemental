%GENERALES
\newcommand{\tq}{\text{ $|$ }}
\newcommand{\midcup}{\mbox{$\bigcup$}\,}
\newcommand{\midcap}{\mbox{$\bigcap$}\,}
\newcommand{\ms}[1]{\mathscr{#1}}
% \coloneq (funciona en stix2)
\newcommand{\mycoloneq}{\mathrel{\text{\ooalign{\raisebox{0.5ex}{\scalebox{0.52}{$\bullet$}}\cr \raisebox{0.05ex}{\scalebox{0.52}{$\bullet$}}}}\mkern-5mu =}}
%RENOVACIÓN DE COMANDOS
\renewcommand{\emptyset}{\varnothing}
\renewcommand{\tau}{\ms{T}}

%\renewcommand{\mapsto}{\DOTSB\mapstochar\rightarrow}

%Nuevo "Setminus"
\newcommand{\mysetminusD}{\hbox{\tikz{\draw[line width=0.6pt,line cap=round] (3pt,0) -- (0,6pt);}}}
\newcommand{\mysetminusT}{\mysetminusD}
\newcommand{\mysetminusS}{\hbox{\tikz{\draw[line width=0.45pt,line cap=round] (2pt,0) -- (0,4pt);}}}
\newcommand{\mysetminusSS}{\hbox{\tikz{\draw[line width=0.4pt,line cap=round] (1.5pt,0) -- (0,3pt);}}}
\newcommand{\mysetminus}{\mathbin{\mathchoice{\mysetminusD}{\mysetminusT}{\mysetminusS}{\mysetminusSS}}}
\renewcommand{\setminus}{\mysetminus}
%OPERADORES
%\makeatletter
%	\renewcommand{\operator@font}{\opfont}
%\makeatother

\DeclareMathOperator{\inte}{int}
\DeclareMathOperator{\ext}{ext}
\DeclareMathOperator{\cla}{cl}
\DeclareMathOperator{\der}{der}
\DeclareMathOperator{\fron}{fr}
\DeclareMathOperator{\scl}{sqcl}
\DeclareMathOperator{\cf}{cf}
\DeclareMathOperator{\Id}{Id}
\DeclareMathOperator{\ima}{ima}
\DeclareMathOperator{\dom}{dom}
\DeclareMathOperator{\St}{St}
\DeclareMathOperator{\Osq}{so}
%TEXTOS
\DeclareMathOperator{\T}{T}
\DeclareMathOperator{\US}{US}
\DeclareMathOperator{\AN}{AN}
\DeclareMathOperator{\OR}{OR}
\DeclareMathOperator{\CAR}{CAR}
\DeclareMathOperator{\zfc}{ZFC}
\DeclareMathOperator{\zf}{ZF}
\DeclareMathOperator{\HC}{CH}
\DeclareMathOperator{\Ma}{MA}
\DeclareMathOperator{\Ac}{AC}
\DeclareMathOperator{\Pm}{MC}
\DeclareMathOperator{\Pdm}{WMC}
\DeclareMathOperator{\Ad}{AD}
\DeclareMathOperator{\Mad}{MAD}
\DeclareMathOperator{\Gen}{Gen}
\DeclareMathOperator{\MP}{MP}
%COCIENTE
%\newcommand{\quot}[2]{{\raisebox{.2em}{$#1$}\left/\raisebox{-.2em}{$#2$}\right.}}

%AUXILIARES
\newcommand{\CTT}{\textcolor{magenta}{(CITE) }}

\newcommand{\MroImagen}{
    \begin{tikzpicture}[scale=1.3]
        \fill[gray!20]
        (0,-.5)
        to[out=80,in=260] (.5,1)
        to[out=260+180,in=30] (-1.4,1)
        to[out=30+180,in=160] (-1.5,-2.5)
        to[out=160+180,in=290] (.5,-1.5)
        to[out=290+180,in=80+180] (0,-.5);

        \fill[fill=morado!50, fill opacity=0.5]
        (0,-.51)
        to[out=80+70,in=50] (-1.8,-.5)
        to[out=50+180,in=130] (-1.7,-1.2)
        to[out=130+180,in=-80-70] (0,-.51);

        \fill[fill=morado!50, fill opacity=0.5]
        (.505,1.02)
        to[out=80+70,in=50] (-1.3,0.5)
        to[out=50+180,in=130] (-1.1,-.4)
        to[out=130+180,in=-80-70] (.505,1.02);

        %\draw[morado, line width=.4]
        %(0,-.51)
        %to[out=80+70,in=50] (-1.8,-.5)
        %to[out=50+180,in=130] (-1.7,-1.2)
        %to[out=130+180,in=-80-70] (0,-.51);
        \node[above right] at (-1.8,-1) {$x \subseteq  \Psi(\mathscr{A})$};

        %\draw[morado, line width=.4]
        %(.505,1.02)
        %to[out=80+70,in=50] (-1.3,0.5)
        %to[out=50+180,in=130] (-1.1,-.4)
        %to[out=130+180,in=-80-70] (.505,1.02);
        \node[above right] at (-1.4,.1) {$y \subseteq  \Psi(\mathscr{A})$};

        \draw[line width=1.5, dash pattern=on 1pt off 1pt on 1pt off 1pt, azul]
        (.5,-1.5)
        to[out=290+180,in=80+180] (0,-.5)
        to[out=80,in=260] (.5,1)
        to[out=260+180,in=30] (-1.4,1);
        \node[above right] at (-1.4,1.15) {$\mathscr{A}$};

        \draw[fill=azul, azul] (0,-.51) circle (1pt);
        \node[above right] at (0,-.5) {$x \in \Psi(\mathscr{A})$};

        \draw[fill=azul, azul] (.505,1.02) circle (1pt);
        \node[above right] at (.5,1) {$y \in \Psi(\mathscr{A})$};

        \node[above right] at (-1,-2) {$\omega$};
    \end{tikzpicture}
}