\chapter*{Notas y consideraciones (BORRAR)}

\begin{flushright}
	Se tiene que aclarar
\end{flushright}

\begin{enumerate}
	\item Qué significa numerable.
	\item Notaciones $[X]^{* \kappa}$.
	\item Casi ajenidad
	\item notación de sucesiones
\end{enumerate}

\begin{flushright}
	Preliminares de \textbf{conjuntos}:
\end{flushright}
\begin{enumerate}
	\item Axiomas $\zfc$ y equivalencias de $\Ac$
	\item Teoría de ordenes parciales, elementos distinguidos, etc.
	      \begin{enumerate}
		      \item Notaciones de rayos $<_x$, $\leq_x$ (segmentos iniciales CHECAR LEMA DEBAJO DE
		            %\ref{familia-casi-ajena-de-cardinalidad-c}), etcétera.				
		      \item Casi contención y $\subseteq^*$, $=^*$, ideal de los finitos (?)
	      \end{enumerate}
	\item Ordinales.
	      \begin{enumerate}
		      \item Propiedades
		      \item Aritmética ordinal
		      \item Quién es $\omega$
	      \end{enumerate}
	\item Cardinales.
	      \begin{enumerate}
		      \item Propiedades
		      \item Artimética cardinal
		      \item Los $\aleph_\alpha$ y $\mathfrak{c}$
		      \item \HC, sus equivalencias y resultados
	      \end{enumerate}
\end{enumerate}

\begin{flushright}
	Preliminares de \textbf{topología}
\end{flushright}
\begin{enumerate}
	\item Todo lo básico de topología

	\item Axiomas de separación, espacios desarrollables y espacios metrizables.

	\item Teoremas de metrización, bases regulares (?)

	\item Espacios secuenciales y Fréchet
	      \begin{enumerate}
		      \item Definición, clausura secuencial, independencia respecto a subespacios
		      \item $1 \AN \to $ Fréchet $\to$ Secuencial.
		      \item Orden secuencial, lema de $\omega_1$ y cosas por el estilo
	      \end{enumerate}
\end{enumerate}

\begin{flushright}
	Cosas que falta revisar
\end{flushright}
\begin{itemize}
	\item Teorema que no sale (y sucesiones ahi).
	\item Sucesiones del capítulo 1 y resultados de los métodos para construir MADS
	\item doble derivado en corolarios de KR
\end{itemize}

\begin{flushright}
	Resultados que ya se usaron
\end{flushright}
\begin{itemize}
	\item CONJUNTOS
	      \begin{enumerate}
		      \item copo \textit{c.c.c} \hfill 660 chap 1
	      \end{enumerate}
	\item LOGICA
	      \begin{enumerate}
		      \item Demostrabilidad \hfill 668 chap 1
	      \end{enumerate}
	\item TOPOLOGÍA
	      \begin{enumerate}
		      \item Cero-dimensional, condición suficiente \hfill 120 chap 2
		      \item Cero-dimensional + $\T_1$ implica Tychonoff \hfill 122 chap 2
		      \item Cantor es universal sobre los cero-dimensionales \hfill 134 chap 2
		      \item Teorema de Categoría de Baire \hfill 236 chap 2, 352 chap 2
		      \item En espacios metrizables, sep imllica $2\AN$ \hfill 255 chap 2
		      \item $2\AN$ + loc compacto implica $\sigma$-comp \hfill 257 chap 2
		      \item $\sigma$-comp implica Lindelof \hfill 257 chap 2
		      \item Caracterizaciones básicas de pesudocompacidad	301 chap 2
		      \item Peudocompacto + Normal implica numerablemente compacto \hfill 306 chap 2
		      \item Lema de Jones \hfill 30 chap 4
		      \item Metrizable implica normal \hfill 31 chap 4
		      \item Teorema de bing \hfil 151 chap 4
	      \end{enumerate}
\end{itemize}