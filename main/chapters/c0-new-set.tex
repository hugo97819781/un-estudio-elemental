\setcounter{chapter}{-1}
\chapter{Preliminares}
\section{Conjuntos}

El presente trabajo se desarrolla dentro de la teoría de conjuntos usual, $\zfc$, entendido como $\zf$ junto con el axioma de elección $\Ac$. Ocasionalmente haremos referencia a axiomas adicionales, tales como la hipótesis del continuo $\HC$ (véase \pagp\pageref{CHDef}) o el axioma de Martin $\Ma$ (véase \pagp\pageref{MADef}). Los sistemas mencionados pueden consultarse en la \cref{apnd-ltc}.

\index[alph]{conjunto!potencia}\index[alph]{función!restricción de una}\index[alph]{clase}\index[sym]{$f \upharpoonright B$}\index[sym]{$\ms{P}(X)$}\index[alph]{conjunto!potencia}\index[alph]{enumeración}\index[alph]{conjunto!enumerado}
Dado un conjunto $X$, se denotará por $\ms{P}(X)$ a su conjunto \textit{potencia}. Ahora, si $f \colon A \to X$ es una función y $B \subseteq A$, se escribirá la \textit{restricción} de $f$ a $B$ como $f \upharpoonright B$ y denotaremos la composición de funciones por yuxtaposición. Las jerarquía de operaciones binarias que utilizaremos será: $\setminus, \times, \cup, \cap$.Identificaremos cualquier fórmula $\varphi$ de la teoría de conjuntos con una \textit{clase}, esto es, una colección $\{ x \tq \varphi \}$. Una \textit{enumeración} para un conjunto $A$ es una función biyectiva $I \to A$ ($i \mapsto a_i$), en esta situación, se dice que $A$ \textit{está enumerado} como $A=\{ A_i \tq i \in I \}$.


\subsection{Órdenes Parciales}
\index[alph]{conjunto!parcialmente ordenado}\index[alph]{orden!parcial}\index[alph]{orden!parcial antireflexivo}\index[alph]{orden!parcial antireflexivo}\index[sym]{$\leq$ (orden)}\index[sym]{$<$ (orden)}\index[alph]{orden}
Un \textit{orden parcial} sobre un conjunto $P$ es una relación reflexiva, transitiva y antisimétrica en $P$. Cuando $\leq$ sea un orden parcial en $P$, se denotará el \textit{orden antireflexivo} asociado a $\leq$ por $< \mycoloneq  \leq \setminus \Id_A$. Un \textit{conjunto parcialmente ordenado} es un par ordenado $(P,R)$, donde $R$ es un orden parcial en $P$, o un orden antireflexivo asociado a un orden parcial en $P$.

\index[alph]{isomorfismo de orden}\index[alph]{órdenes!isomorfos}\index[alph]{función!creciente}\index[alph]{función!estrictamente creciente}\index[alph]{función!isomorfismo de orden}\index[sym]{$(P,\leq) \cong (Q,\leq)$}\index[sym]{$(P,<) \cong (Q,<)$}
Una función $f \colon P \to Q$ entre órdenes parciales es \textit{creciente} (\textit{estrictamente creciente}, respectivamente) siempre que para todos $p,q \in P$, con $p < q$, se cumple $f(p) \leq f(q)$ ($f(p) < f(q)$, respectivamente). Si $f$ es biyectiva, creciente y $f^{-1}$ es creciente, se dice que $f$ es \textit{isomorfismo de órdenes}, en tal caso usaremos la notación $(P,\leq) \cong (Q,\leq)$ (o bien, $(P,<) \cong (Q,<)$).

\index[alph]{elemento!máximo}\index[alph]{elementos comparable con otro}\index[alph]{elemento!mínimo}\index[alph]{elemento!supremo}\index[alph]{elemento!ínfimo}\index[alph]{elemento!maximal}\index[alph]{elemento!minimal}\index[sym]{$\max(A)$}\index[sym]{$\min(A)$}\index[sym]{$\sup(A)$}\index[sym]{$\inf(A)$}\index[alph]{conjunto!totalmente ordenado}\index[alph]{orden!total}\index[alph]{orden!buen orden}\index[alph]{conjunto!bien ordenado}
Sean $\mathbb{P}=(P,\leq)$ un conjunto parcialmente ordenado. Seguiremos la notación estándar para los elementos \textit{máximo}, \textit{mínimo}, \textit{supremo} e \textit{ínfimo} de $A$ (cuando existan); esto es, $\max(A)$, $\min(A)$, $\sup(A)$, $\inf(A)$, respectivamente. Un elemento $r \in P$ es $\leq$-\textit{maximal} de $A$ si no existe $a \in A$ tal que $r < a$; de forma análoga se define el concepto de $\leq$-\textit{minimal} de $A$. Se dice que $p,q \in P$ son \textit{comparables} cuando $p \leq q$ o $q \leq p$.

Una \textit{cadena} de $\mathbb{P}$ es un subconjunto $C \subseteq P$ de elementos comparables. Cuando $P$ mismo sea una cadena, se dirá que $\mathbb{P}$ es un \textit{conjunto totalmente ordenado} y que $\leq$ (o bien, $<$) es un \textit{orden total} en $P$. Si cada $A \in \ms{P}(P) \setminus \{ \emptyset \}$ tiene mínimo, diremos que $\mathbb{P}$ es un \textit{conjunto bien ordenado} y que $\leq$ (o bien, $<$) es un \textit{buen orden} en $P$.

La prueba de la siguiente equivalencia para $\Ac$ se puede ver en \cite[\S ~ I.11]{munkresTopo}:
\begin{teorema}[Principio de Maximalidad de Hausdorff]\phantomsection\label{teo-PMO}\index[alph]{lema!principio de maximalidad de Hausdorff}\index[alph]{principio de maximalidad de Hausdorff}\index[alph]{Hausdorff!principio de maximalidad de}
    Todo conjunto parcialmente ordenado y no vacío $\mathbb{P}$ contiene una cadena, $\subseteq$-maximal del conjunto de cadenas de $\mathbb{P}$.
\end{teorema}

\index[alph]{casi!contenido}\index[alph]{casi!contención}\index[alph]{casi!iguales}\index[alph]{casi!igualdad}\index[alph]{casi!ajeno}\index[sym]{$\subseteq^*$}\index[sym]{$=^*$}

A continuación se presenta un objeto que, si bien no es un orden parcial, está inducido por orden parcial contención. Para dos conjuntos cualesquiera $A$ y $B$, diremos que $A$ está \textit{casi contenido} en $B$ si $A \setminus B$ es finito, esta situación se denotará por $A \subseteq^* B$. Se conviene que $A$ y $B$ son \textit{casi iguales}, denotado $A=^*B$, si y sólo si $A \subseteq^* B$ y $B \subseteq ^* B$.
\begin{proposicion}
    Sean $A,B$ y $C$ conjuntos arbitrarios, entonces:
    \begin{enumerate}
        \item Si $A \subseteq^* B$ si y sólo si existe $F \subseteq A$ finito tal que $A \setminus F \subseteq B$.
        \item $A \subseteq^* A$ u $A=^* B$.
        \item Si $A \subseteq^* B$ y $B \subseteq^* C$, entonces $A \subseteq^* C$.
        \item $=^*$ se comporta como relación de equivalencia.
        \item Si $A,B \subseteq C$, entonces $A \subseteq^* B$ si y sólo si $C \setminus B \subseteq^* C \setminus A$.
        \item Si $A,B \subseteq C$, entonces $A \subseteq^* B$ si y sólo si $C \setminus B \subseteq^* C \setminus A$.
    \end{enumerate}
\end{proposicion}

\begin{proposicion}
    Sean $X,Y$ conjuntos, $f \colon X \to Y$, $A,B \subseteq X$ y $C \subseteq Y$. Entonces se cumple que:
    \begin{enumerate}
        \item Si $A \subseteq^* B$, entonces $f[A] \subseteq^* f[B]$.
        \item Si $A \subseteq^* f^{-1}[C]$ entonces $f[A] \subseteq^* C$.
    \end{enumerate}
    Y si $f$ es inyectiva, ocurren los recíprocos.
\end{proposicion}


\subsection{Ordinales y Cardinales}

\index[alph]{número!ordinal}\index[sym]{$\alpha < \beta$ (ordinales)}\index[sym]{$\omega$}\index[alph]{número!ordinal!sucesor}\index[alph]{número!ordinal!límite}\index[alph]{número!ordinal!cero}
Como es estándar, se trabajará sobre el universo de Von Neumann. Así, $\alpha$ es un \textit{número ordinal} si $\alpha \subseteq \ms{P}(\alpha)$ y $(\alpha,\in)$ es un conjunto bien ordenado. Dados ordinales $\alpha$ y $\beta$, escribiremos $\alpha < \beta$ para indicar que $\alpha \in \beta$, equivalentemente, $\alpha \subsetneq \beta$. Toda clase no vacía de ordinales tiene mínimo \cite[Obs.~I.2.4]{jechSet}. Para cada ordinal $\alpha$, el sucesor de $\alpha$ es $\alpha+1=\alpha \cup \{\alpha\}$. Existen tres tipos de ordinales: \textit{cero} ($0=\emptyset$), los \textit{sucesor} (aquellos de la forma $\alpha+1$, para algún ordinal $\alpha$), y los \textit{límite} (ninguno de los anteriores). Denotaremos por $\omega$ al primer ordinal límite; $\omega$ es el conjunto de números naturales.

\index[alph]{sucesión}\index[alph]{sucesión!finita}\index[sym]{$(x_n)_{n \in \omega} \subseteq A$}\index[sym]{$(x_k)_{k \in n} \subseteq A$}
Una \textit{sucesión} (\textit{sucesión finita}, respectivamente) en un conjunto $A$ es una función $x \colon \omega \to A$ ($x \colon n \to A$, para algún $n \in \omega$, respectivamente). Esto se denota por $(x_n)_{n \in \omega} \subseteq A$, o bien, si $x$ es sucesión finita: $(x_k)_{k \in n} \subseteq A$.

Con relativa frecuencia se realizarán construcciones \textit{recursivas} en $\omega$ y $\omega_1$. El mecanismo detrás de esto es la siguiente restricción del teorema de Recursión Transfinita \cite[\S~ I.2.Induction~ and~ Recursion]{jechSet}.
\begin{teorema}[Recursión en ordinales]\index[alph]{teorema!de recursión en ordinales}\index[alph]{recursión!en ordinales}\index[alph]{recursión!teorema de}
    Sea $\alpha \neq 0$ un ordinal, entonces para todo conjunto $X$, $x_0 \in X$, $g \colon X \to X$ y $f \colon \ms{P}(X) \to X$ arbitrarias:
    \begin{enumerate}
        \item Existe una única $h \colon  \alpha \to X$ tal que para cada $\beta \in \alpha$, $h(\beta+1)=gh(\beta)$.
        \item Existe una única $j \colon  \alpha \to X$ de modo que: $j(0)=x_0$; para cada $\beta \in \alpha$, $j(\beta+1)=gj(\beta)$; y, para cada ordinal límite $\gamma \in \alpha$, $j(\gamma)=f(j[\gamma])$.
    \end{enumerate}
    
    
\end{teorema}

\index[alph]{número!cardinal}\index[alph]{ordinal}\index[alph]{cardinal}\index[sym]{$\aleph_\alpha$}\index[sym]{$|X|$}\index[sym]{$\omega_\alpha$}\index[sym]{$\aleph_\alpha$}\index[sym]{$\mathfrak{c}$}\index[alph]{conjunto!numerable}\index[alph]{conjunto!finito}\index[alph]{conjunto!más que numerable}\index[alph]{conjunto!no numerable}\index[alph]{conjunto!contable}\index[alph]{conjunto!a lo más numerable}\index[alph]{cardinalidad}\index[alph]{tamaño}\index[alph]{conjunto!cardinalidad de}\index[alph]{conjunto!tamaño de}
Un \textit{número cardinal} es un ordinal no biyectable con ninguno de sus elementos. Se usará la enumeración habitual para cardinales, $\aleph_\alpha=\omega_\alpha$. Si $A$ es conjunto, $|A|$ es al único cardinal biyectable con $A$, esto es, la \textit{cardinalidad} o el \textit{tamaño} de $A$. El conjunto $A$ es \textit{finito} si $|A|<\omega$; \textit{contable} o \textit{a lo más numerable} si $|A| \leq \omega$; \textit{numerable} si $|A|=\omega$ y \textit{más que numerable} o \textit{no numerable} si $|A|>\omega$.

De manera puntual, se hará referencia a la aritmética ordinal y cardinal, utilizando la notación convencional para la suma, el producto y la exponenciación \cite[\S.~ I.3, I.5]{jechSet}. En todo lo que sigue se usará implícitamente que:
\begin{proposicion}
    Si $\kappa,\lambda$ y $\mu$ son cardinales, entonces:
    \begin{enumerate}
        \item $\kappa + \lambda = \lambda + \kappa = \kappa  \lambda = \lambda  \kappa$.
        \item $(\lambda \kappa)^\mu = \lambda^\mu \kappa^\mu$ y $\kappa^{\lambda + \mu} = \kappa^\lambda \kappa ^\mu$.
        \item Si $\kappa \leq \lambda$, entonces:
        \begin{enumerate}
            \item $\kappa + \mu \leq \lambda + \mu$ y $\kappa \mu \leq \lambda \mu$.
            \item $\mu^ \kappa \leq \mu ^ \lambda$.
            \item Si $\mu \neq 0$, $\kappa^ \mu \leq \lambda ^ \mu$.
        \end{enumerate}
        \item Si $\kappa\geq \omega$, entonces $\kappa + \lambda = \max\{\kappa, \lambda\}$.
        \item Para cada conjunto $X$, $|\ms{P}(X)|=2^{|X|}>|X|$.
    \end{enumerate}

    Y si $\{\kappa_\alpha \tq \alpha \in I\}$ es una familia no vacía de cardinales y cada $\kappa_\alpha$ es infinito, o bien $I$ es infinito, entonces:
    \[ \sum_{\alpha \in  I} \kappa_\alpha = |I| \sup_{\alpha \in I} \kappa_\alpha \, . \]
\end{proposicion}

\index[sym]{$\mathfrak{c}$}\phantomsection\label{CHDef}\index[sym]{$\HC$}\index[alph]{hipótesis!del continuo}\index[alph]{continuo!hipótesis del}\index[alph]{continuo!cardinal}
La letra $\mathfrak{c}$ denota el cardinal del \textit{continuo}, es decir $\mathfrak{c}=|\mathbb{R}|=2^{\aleph_0}$. La formulación de la \textit{hipótesis del continuo}, $\HC$, que utilizaremos es: $\aleph_1=\mathfrak{c}$.

\index[sym]{$[X]^\kappa$}\index[sym]{$[X]^{<\kappa}$}\index[sym]{$[X]^{\leq\kappa}$}\index[sym]{$[X]^{>\kappa}$}\index[sym]{$[X]^{\geq\kappa}$}\index[sym]{$X^\kappa$}\index[sym]{$X^{\kappa}$}
Dado un conjunto $X$ y un cardinal $\kappa \leq |X|$, se denotarán por $[X]^\kappa$ y $[X]^{<\kappa}$ a las colecciones de todos los subconjuntos de $X$ de cardinalidad exactamente $\kappa$ y menor que $\kappa$, respectivamente. De forma análoga se definen los conjuntos $[X]^{\leq \kappa}$, $[X]^{>\kappa}$ y $[X]^{\geq \kappa}$. Por otro lado, $X^\kappa$ denota el conjunto de todas las funciones de $\kappa$ en $X$, mientras que $X^{<\kappa}$ es el conjunto $\{ f \tq \exists \alpha < \kappa \, (f \colon \alpha \to X) \}$.
\begin{proposicion}\phantomsection\label{cardi-subcolec}
    Sean $X$ un conjunto infinito y $\kappa \leq |X|$, entonces:
    \begin{enumerate}
        \item $|[X]^\kappa| = |X|^\kappa$, en particular; si $X$ es numerable, $|[X]^{\omega}| = \mathfrak{c}$.
        \item $|[X]^{<\omega}| = |X^{<\omega}| = |X|$.
    \end{enumerate}
\end{proposicion}
%\begin{proof}
%    (i) Para cada $A \in [X]^\kappa \subseteq \ms{P}(X)$ fíjese ($\Ac$) una biyección $g_A  \colon  \kappa \to A$. Entonces la función $A \mapsto g_A$ es inyección de $[X]^\kappa$ en $X^\kappa$. Para la desigualdad recíproca, como $\kappa \leq |X|$, existe una biyección $g \colon X \times \kappa \to X$. Cada $f \in X^\kappa$ es un subconjunto de tamaño $\kappa$ de $X \times \kappa$; por tanto, la asignación $f \mapsto g[f]$ es inyección de $X^\kappa$ en $[X]^\kappa$.

%    Ahora, si $|X|=\aleph_0$, entonces $|[X]^\omega| = \aleph_0^{\aleph_0}$. Como $2 \leq \aleph_0$, entonces $\aleph_0 ^{\aleph_0}$; y como $\aleph_0 < 2^\aleph_0$, entonces $\aleph_0^{\aleph_0} \leq (2^{\aleph_0})^{\aleph_0} = 2^{\aleph_0 \aleph_0} = 2^\aleph_0$.

 %   (iii) Obsérvese que $|X|\leq [X]^{<\omega}$, pues $x \mapsto \{x\}$ es inyección de $X$ en $[X]^{<\omega}$. Para la desigualdad recíproca, note que:
 %   \[ |[X]^{<\omega}| = \Bigg| \bigcup_{n \in \omega} [X]^n \Bigg| \leq \sum_{n \in \omega} |[X]^n| = \sum_{n \in \omega} |X| = |X| \cdot \aleph_0  = |X| \, . \]
 %   probando la igualdad deseada.
%\end{proof}

\subsection{Estructuras combinatorias}
\subsubsection{Árboles}
\label{sec-arbols}
\index[alph]{orden parcial!árbol}\index[alph]{árbol}\index[alph]{árbol!altura de un}\index[alph]{orden!en un árbol}\index[sym]{$o(x)$}
Un árbol es un conjunto parcialmente ordenado $\mathbb{T}=(T,\leq)$ tal que si $x \in T$, $\leq$ es buen orden en $\{ p \in T \tq p<x \}$. El \textit{orden} de $x \in T$, $o(x)$, es el único ordinal isomorfo a $(\{ p \in T \tq p<x \},<)$ \cite[\S.~I.2.12]{jechSet}. $\sup\{ o(x) +1 \tq x \in T \}$ es la \textit{altura de} $\mathbb{T}$. Una \textit{rama} de $\mathbb{T}$ es una cadena, $\subseteq$-maximal del conjunto de cadenas de $\mathbb{T}$.

\begin{proposicion}\phantomsection\label{2fin-es-arbol}
	$\mathbb{T} \mycoloneq (2^{<\omega}, \subseteq)$ es un árbol numerable de altura $\omega$.
\end{proposicion}

\begin{proof}
	Sean $f\in 2^{<\omega}$, $n=\dom(f)$ y $T_f:=\{ g \in 2^{<\omega} \tq g \subsetneq f \}$. Nótese que $T_f =\{ f \upharpoonright m \tq m < n \}$. Entonces $H \colon n \to T_f$, definida por $H(m)=f \upharpoonright m$, es isomorfismo de orden, pues es biyectiva y para cualesquiera $m,k \in n$, $k \in m$ si y sólo si $f \upharpoonright k \subsetneq f \upharpoonright m$.

	Por lo tanto $\mathbb{T}$ es un árbol y para cada $f \in 2^{<\omega}$, $o(f)=n$. Consecuentemente, la altura de $\mathbb{T}$ es $\sup\{ n+1 \tq n \in \omega \}=\omega$. Por \ref{cardi-subcolec}, $2^{<\omega}$ es numerable.
\end{proof}

\subsubsection{Cofinalidad y lema de Fodor}
\index[alph]{cofinalidad}\index[alph]{número!ordinal!regular}\index[alph]{número!cardinal!regular}\index[alph]{función!cofinal}\index[sym]{$\cf(\alpha)$}\index[alph]{regular!ordinal}\index[alph]{regular!cardinal}
Una función entre ordinales $f \colon \beta \to \alpha$ es \textit{cofinal} en $\alpha$ si para cada $\delta \in \alpha$ existe cierto $\varepsilon \in \beta$ de manera tal que $\delta \leq f(\varepsilon)$. Se define:
\[ \cf(\alpha) = \min \{  \beta \tq \exists f:\beta \to \alpha \, (f \text{ es cofinal en } \alpha ) \} \, . \]
Para cualquier ordinal $\alpha$: $\cf(\alpha)\leq \alpha$; $\cf(\alpha)$ es un cardinal; y siempre existe una función cofinal y estrictamente creciente $f \colon \cf(\alpha) \to \alpha$ \cite[\S.~I.3.Cofinality]{jechSet}. Un ordinal (cardinal, respectivamente) $\kappa$ es \textit{regular} si $\cf(\kappa)=\kappa$. $\omega_1$ es regular. \newpage

\index[alph]{conjunto!cerrado (en ordinales)}\index[alph]{conjunto!club}\index[alph]{conjunto!estacionario}
Sea $\gamma$ un ordinal límite. $C \subseteq \gamma$ se denomina \textit{cerrado} (en $\gamma$) si para todo $\alpha<\gamma$, $\midcup (C \cap \alpha)=\alpha$ implica que $\alpha \in C$. Si $C$ es cerrado y no cofinal en $\gamma$, diremos que $C$ es un \textit{club} (de $\gamma$). Un conjunto $S \subseteq \gamma$ se dice \textit{estacionario} (en $\gamma$) si y sólo si tiene intersección no vacía con todo club de $\gamma$. Con esta terminología se enuncia la siguiente últil herramienta \cite[Teo.~ I.8.7]{jechSet}:

\begin{teorema}[Lema de Fodor]\phantomsection\label{fodor-lem}\index[alph]{lema!de Fodor}\index[alph]{Fodor!lema de}\index[alph]{función!estacionaria}
    Sean $\kappa$ un cardinal regular, $S \subseteq \kappa$ estacionario y $f \colon S \to \kappa$ \textit{regresiva}, es decir, para cada $\alpha \in S \setminus \{0\}$, $f(\alpha)<\alpha$. Entonces existe un conjunto estacionario $T \subseteq S$ tal que $f \upharpoonright T$ es constante; equivalentemente, existe $\delta \in \kappa$ tal que $f^{-1}[\{\delta\}] \subseteq S$ es estacionario.
\end{teorema}

\subsubsection{Axioma de Martin}

\phantomsection
\label{copo-ccc}
\index[alph]{elemento!compatible con otro}\index[alph]{elemento!incompatible con otro}\index[sym]{$p \parallel q$}\index[sym]{$p \perp q$}\index[alph]{conjunto!anticadena}\index[alph]{anticadena}\index[alph]{propiedad!de anticadena contable}\index[alph]{propiedad!c.c.c.}\index[alph]{c.c.c.}\index[alph]{conjunto!denso (en un orden parcial)}
Por lo que resta de la sección, $\mathbb{P}=(P,\leq)$ será un conjunto parcialmente ordenado. Dos elementos $p,q \in P$ son \textit{compatibles} cuando existe $r \in P$ con $r\leq p$ y $r\leq p$, esto se denotará por $p \parallel q$. Si $p \not \parallel q$, diremos que $p$ y $q$ son \textit{incompatibles} y escribiremos $p \perp q$. Una \textit{anticadena} de $\mathbb{P}$ es un subconjunto $A \subseteq P$ de elementos incompatibles dos a dos. Cuando toda anticadena de $\mathbb{P}$ sea contable, diremos que $\mathbb{P}$ tiene la \textit{propiedad de anticadena contable} (\textit{c.c.c.}). Un conjunto $D \subseteq P$ es \textit{denso} (en $\mathbb{P}$) cuando para cualquier $p \in P$ existe $d \in D$ tal que $d \leq a$.

\index[alph]{filtro}\index[alph]{ideal}
Un \textit{filtro} es un conjunto no vacío $F \subseteq P$ tal que para cualesquiera $p,q \in P$:
\begin{enumerate}
    \item Si $p \in F$ y $p \leq q$, entonces $q \in F$, y
    \item Si $p,q \in F$, entonces existe $r \in F$ de manera que $r \leq p$ y $r \leq q$.
\end{enumerate}

\index[alph]{filtro!propio}\index[alph]{ideal!propio}
De manera dual, un \textit{ideal} es un filtro en $(P,\geq)$. Un filtro (ideal, respectivamente) es \textit{propio} cuando es diferente de $P$.
\begin{proposicion}
    Sean $X$ un conjunto y $F \subseteq \ms{P}(X)$. $F$ es filtro si y sólo si:
    \begin{enumerate}
        \item Para cada $A \in F$, si $A \subseteq B \subseteq X$, entonces $B \in F$, y
        \item Para cualesquiera $A,B \in F$ ocurre que $A \cap B \in F$.
    \end{enumerate}
\end{proposicion}

%\index[alph]{ultrafiltro}\index[alph]{conjunto!ultrafiltro}
%Un \textit{ultrafiltro} $\mathbb{P}$ es in filtro $\subseteq$-maximal del conjunto de filtros de $\mathbb{P}$. El siguiente es un teorema sumamente conocido, su demostración tiene por base el \cref{teo-PMO} \CTT.

%\begin{teorema}[Lema del Ultrafiltro]\phantomsection\label{lem-ultrafil}\index[alph]{lema!del ultrafiltro}\index[alph]{ultrafiltro!lema del}
    %Sean $\mathbb{P}$ un conjunto parcialmente ordenado y $F$ de $\mathbb{P}$ un filtro. Entonces existe un ultrafiltro $U \supseteq F$.
%\end{teorema}

\index[alph]{filtro!genérico}\index[alph]{filtro!$\ms{D}$-genérico}
Si $\ms{D} \subseteq \ms{P}(P)$ es una colección de subconjuntos densos y $G$ es un filtro de $\mathbb{P}$, diremos que $G$ es $\ms{D}$-\textit{genérico} cuando para cada $D \in \ms{D}$ ocurre $G \cap D \neq \emptyset$; si $\ms{D}$ es la colección de todos los densos de $\mathbb{P}$, se dice que $G$ es \textit{genérico}.

\index[alph]{axioma!de Martin en $\kappa$}\index[alph]{Martin!axioma de (en $\kappa$)}\index[sym]{$\Ma(\kappa)$}
En honor al matemático Donald A. Martin (1940–), para cada cardinal $\kappa$ se define el \textit{Axioma de Martin en} $\kappa$, denotado $\Ma(\kappa)$, como el enunciado: \textit{Para todo orden parcial $\mathbb{P}$ con la c.c.c. y toda colección $\ms{D}$, de densos en $\mathbb{P}$, de tamaño a lo más $\kappa$, existe un filtro $\ms{D}$-genérico}.

\begin{proposicion}
    Para todo par de cardinales $\lambda$ y $\kappa$:
    \begin{enumerate}
        \item Si $\kappa \leq \lambda$ y $\Ma(\lambda)$ es verdadero, entonces $\Ma(\kappa)$ es verdadero.
        \item Si $\kappa \leq \aleph_0$, entonces $\Ma(\kappa)$ es verdadero.
        \item Si $\mathfrak{c} \leq \lambda$, entonces $\Ma(\lambda)$ es falso.
    \end{enumerate}
\end{proposicion}
\begin{proof}
    El primer punto es claro.

    (ii) Sean $\mathbb{P}=(P,\leq)$ cualquier orden parcial, $p_0 \in D_0$ y supóngase que $\ms{D}$ es una colección de densos de $\mathbb{P}$ enumerada como $\{D_n \tq n \in \omega\}$. Utilizando recursión y el $\Ac$, para cada $n\in \omega$ fíjese un elemento $p_{n+1} \in D_{n+1}$ con $p_{n+1} \leq p_n$. Sea
    \[ G \mycoloneq \{ q \in P \tq \exists n \in \omega \, ( p_n \leq q ) \} \, .\]
    
    Claramente, si $p \in G$ y $q \geq p$, entonces $q \in G$. Además, si $p,q \in G$ entonces existen $n,m \in \omega$ tales que $p_n \leq p$ y $p_m \leq q$, de donde $p_k \leq p$ y $p_k \leq q$, si es que $k \mycoloneq \max\{n,m\}$. Por contrucción, $G$ es un filtro $\ms{D}$-genérico.

    (iiii) Considérese $\mathbb{T} \mycoloneq (2^{<\omega},\supseteq)$. Nótese que si $A \subseteq 2^{<\omega}$ es anticadena de $\mathbb{T}$ entonces $A$ es contable, pues $2^{<\omega}$ es numerable. Para cada $f \in 2^\omega$ defínase el conjunto $D_f  \mycoloneq  \{ p \in 2^{<\omega} \tq p \not\subseteq f \}$. Cada $D_f$ es denso, pues si $p \in 2^{<\omega}$, entonces $q  \mycoloneq  p \cup \{(\dom(p),1-f(\dom(p)))\} \in D$ y $q \supseteq p$. Como los $D_f$ son distintos dos a dos, la cantidad de densos en $\mathbb{T}$ es $\mathfrak{c}$.
    
    Supóngase, por contradicción, que existe un filtro $G$ de $\mathbb{T}$ tal que interseca a todos los densos de $\mathbb{T}$. Sea $g \mycoloneq \midcup G$, nótese que $g$ es función, pues para cualesquiera $h,k \in G$ existe $r \in G$ con $r \leq h,k$, de donde $r \supseteq h \cup k$, mostrando que $h \cup k$ es función. Similarmete a los conjuntos $D_n$, se tiene que para cada $n \in \omega$ el conjunto $E_n \mycoloneq \{ p \in 2^{<\omega} \tq n \in \dom(p) \} $ es denso. Así, para cada $n \in \omega$, existe $t \in G \cap E_n$, de donde $n \in \dom(t) \subseteq \dom(\midcup G)$. Luego, $\midcup G \colon  \omega \to 2$.

    Sin embargo, para cada $f \in 2^\omega$, existe cierto $s \in G \cap D_f$. Por lo cual, $s \not\subseteq f$ y con ello $f \neq \midcup G$. Esto implica que $\midcup G \notin 2^\omega$, lo que es una contradicción. Por lo tanto, la colección $\ms{D}$ de todos los densos de $\mathbb{T}$ es una colección de tamaño $\mathfrak{c}$ para la que no existe ningún filtro $\ms{D}$-genérico en $\mathbb{P}$.
\end{proof}

\phantomsection\label{MADef}\index[alph]{axioma!de Martin}\index[alph]{Martin!axioma de}\index[sym]{$\Ma$}\index[sym]{$\mathfrak{m}$}\index[alph]{cardinal!de Martin}\index[alph]{Martin!cardinal de}
Así, tiene sentido definir el \textit{cardinal de Martin}, $\mathfrak{m}$, como el menor cardinal $\kappa$ para el cual $\Ma(\kappa)$ es falso. El \textit{Axioma de Martin} es el enunciado: \textit{Para cualquier cardinal $\kappa$ menor que $\mathfrak{c}$, $\Ma(\kappa)$ es verdadero}; equivalentemente, $\mathfrak{m}=\mathfrak{c}$. $\HC$ y $\lnot\HC + \Ma$ son consistentes con $\zfc$ \cite[Teo.~ II.13.20, II.16.13]{jechSet}.