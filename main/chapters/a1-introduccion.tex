Corría el año de 1954 cuando Stanisław G. Mrówka (1933-2010), \enquote{hijo} doctoral de Kazimierz Kuratowski (1896-1980), expuso en su artículo \textit{On completely regular spaces} \cite{mrowkaContra} un método novedoso para la construcción de espacios topológicos de Tychonoff, pseudocompactos, pero no compactos. La construcción parte de una \textit{familia casi ajena}; esto es, en terminología moderna, un conjunto $\mathscr{A}$ compuesto por subconjuntos infinitos de $\omega$ que, dos a dos, tienen intersección finita. Se dice que $\mathscr{A}$ es \textit{maximal} si no existe una familia casi ajena $\mathscr{B}$ que contenga propiamente a $\mathscr{A}$. El espacio contraejemplo de Mrówka es $\Psi(\mathscr{A}) := \omega \cup \mathscr{A}$, donde un subconjunto $U \subseteq \Psi(\mathscr{A})$ es abierto si y sólo si para cada $x \in U \cap \mathscr{A}$, la diferencia $x \setminus U$ es finita. En $\Psi(\mathscr{A})$, el subespacio $\mathscr{A}$ es cerrado y discreto; por lo tanto, si $\mathscr{A}$ es infinita, $\Psi(\mathscr{A})$ no puede ser compacto. Por otro lado, si $\mathscr{A}$ es maximal, entonces $\Psi(\mathscr{A})$ resulta ser pseudocompacto. Aludiendo a la existencia de familias casi ajenas maximales e infinitas, se obtiene un esbozo de la demostración dada por Mrówka.

El aporte teórico recién mencionado constituye un antecedente temprano para el estudio de los \textit{espacios de Isbell-Mrówka}, aunque no es el primero, pues esta topología fue descrita por primera vez, al menos según la documentación reconocida, por Pavel Alexandroff (1896-1982) y Pavel Urysohn (1898-1924) en \cite{alexOrigen}. Pese a ello, su nombre rinde homenaje tanto a Mrówka como a su par profesional John R. Isbell (1930-2005), quienes de manera independiente de Alexandroff y Urysohn, desarrollaron el concepto entre las décadas de los cincuenta y los sesenta, mostrando por qué se trata de objetos dignos de investigación.

Todo espacio $\Psi(\mathscr{A})$, asociado a una familia casi ajena $\mathscr{A}$, es: de Tychonoff, cero-dimensional, disperso, separable y hereditariamente localmente compacto. De hecho, Kannan y Rajagopalan demostraron que los únicos espacios (infinitos, separables y de Hausdorff) hereditariamente localmente compactos son, precisamente, los espacios de Mrówka \cite{kannanHereditarily}. Sin duda, lo que ha colocado a estos objetos, a lo largo de los años, en un lugar privilegiado dentro de las matemáticas es su versatilidad; pues una amplia variedad de invariantes topológicos de $\Psi(\mathscr{A})$ pueden ser \enquote{codificados} mediante el comportamiento de la familia $\mathscr{A}$, considerada como conjunto. Como consecuencia, existen diversas aplicaciones relacionadas con estos espacios, que abarcan desde el estudio de compactaciones, selecciones continuas y espacios totalmente ordenados, hasta resultados en espacios de funciones continuas equipados con la topología de convergencia puntual.

El trabajo que aquí se propone pretende ofrecer una introducción asequible a los espacios de Isbell–Mrówka y puede ser concebido como un \enquote{manual}. Una motivación fundamental para la realización de este trabajo es el hecho de que el material disponible sobre este tema, especialmente en español, es relativamente limitado. La meta final es explicar, siempre de manera clara, cómo se van tendiendo \enquote{puentes} entre la topología y la teoría de conjuntos por medio de los espacios de Mrówka. Se asumirá que el lector cuenta con una formación elemental, equiparable a un par de cursos de nivel superior, en las dos ramas de las matemáticas anteriormente nombradas.

El texto se divide en dos grandes secciones: el estudio de las familias casi ajenas (Capítulo 1), que constituye la parte \enquote{conjuntista} del escrito, y su contraparte topológica, dedicada al estudio de los espacios de Mrówka (Capítulos 2 a 4). En el Capítulo 1 se presentarán construcciones clásicas de familias casi ajenas y se expondrá la teoría básica de su combinatoria infinita asociada, incluyendo el Teorema de Simon, las familias de Luzin y el Lema de Solovay. El Capítulo 2 tiene como objetivo presentar el resultado ya mencionado obtenido por Kannan y Rajagopalan en este contexto, el entendimiento del comportamiento esencial de estos espacios resulta clave y constituye un aspecto central del capítulo. Los Capítulos 3 y 4 abordan problemas específicos que ponen de manifiesto la versatilidad de los protagonistas de esta tesis. En el Capítulo 3 se exhibe la relación entre la propiedad de Fréchet y la compactación unipuntual de los espacios de Isbell–Mrówka (el compacto de Franklin). Finalmente, en el Capítulo 4 se estudian en detalle los aspectos fundamentales para poder \enquote{traducir} la propiedad de normalidad, presentando la conjetura de Moore, su restricción a la clase de espacios separables y los resultados de Silver y Tall al respecto.