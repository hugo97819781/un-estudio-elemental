%Colores
\definecolor{azul}{RGB}{0, 60, 113}
\definecolor{dorado}{RGB}{196, 151, 57}
\definecolor{morado}{RGB}{101, 43, 145}
\definecolor{rosa}{RGB}{146, 33, 209}
\definecolor{azulC}{RGB}{52, 90, 229}

%Setup
\mdfsetup{linewidth=0}
%\mdfsetup{
%    splittable=true,
%    usename=true,
%    skipabove=2pt,
%    skipbelow=2pt,
%    needspace=no
%}

%Estilos de Cajas
\mdfdefinestyle{caja1}{
	leftline=true,
	rightline=false,
	topline=false,
	bottomline=false,
	linecolor=morado!90,
	linewidth=2.5pt,
	backgroundcolor=morado!5,
}

\mdfdefinestyle{caja2}{
	roundcorner=5pt,
	backgroundcolor=dorado!8,
	linewidth=0,
}

\mdfdefinestyle{caja3}{
	topline=false,
	bottomline=true,
	leftline=true,
	rightline=false,
	linewidth=1.5pt,
	linecolor=azul!90,
}

\mdfdefinestyle{caja4}{
	linewidth=0,
}

\declaretheoremstyle[
    headfont=\bfseries,          % "Lema A.1" en negrita
    notefont=\bfseries,          % "(de Solovay)" en negrita
    notebraces={(}{)},           % Paréntesis automáticos
    bodyfont=\itshape,        % Texto del contenido (cámbialo a \itshape si lo quieres cursiva)
    %headpunct={.},               % Punto final tras el título
    postheadspace=0.5em,
    mdframed={style=caja2} % <--- AQUÍ SE CONECTAN
]{estiloLemaNegrita}

%Definicion de entornos
%\newmdtheoremenv{definicion}{Definición}[section]
%\newmdtheoremenv{proposicion}[definicion]{Proposición}
%\newmdtheoremenv{lema}[definicion]{Lema}
%\newmdtheoremenv[style=estiloLemaNegrita]{corolario}[definicion]{Corolario}
%\newmdtheoremenv{observacion}[definicion]{Observación}
%\newmdtheoremenv{ejemplo}[definicion]{Ejemplo}
%\newmdtheoremenv{consideracion}[definicion]{Consideración}

\declaretheorem[style=estiloLemaNegrita,name=Definición,numberwithin=section]{definicion}
\declaretheorem[style=estiloLemaNegrita,name=Proposición,numberwithin=section]{proposicion}
\declaretheorem[style=estiloLemaNegrita,name=Lema,numberwithin=section]{lema}
\declaretheorem[style=estiloLemaNegrita,name=Corolario,numberwithin=section]{corolario}
\declaretheorem[style=estiloLemaNegrita,name=Observación,numberwithin=section]{observacion}
\declaretheorem[style=estiloLemaNegrita,name=Ejemplo,numberwithin=section]{ejemplo}
\declaretheorem[style=estiloLemaNegrita,name=Consideración,numberwithin=section]{consideracion}
\declaretheorem[style=estiloLemaNegrita,name=Teorema,numberwithin=section]{teorema}

%Entorno (complicado) de teorema
%\newcounter{theot}
%\newenvironment{teorema}[1][]{%
	%\setcounter{theot}{\value{TemTheot}}
%	\stepcounter{definicion}
	%\setcounter{theot}{\value{definicion}}
%	\renewcommand{\thetheot}{\thedefinicion}
%	\refstepcounter{theot}
%	\ifstrempty{#1}%
%	{\mdfsetup{%
%			frametitle={%
%					\tikz[baseline=(current bounding box.east),outer sep=0pt]
%					\node[anchor=east,rectangle,fill=dorado!35]
%					{\strut Teorema~\thetheot};}}
%	}%
%	{\mdfsetup{%
%			frametitle={%
%					\tikz[baseline=(current bounding box.east),outer sep=0pt]
%					\node[anchor=east,rectangle,fill=dorado!35]
%					{\strut \textbf{Teorema~\thetheot~(#1)}};}}%
%	}%
%	\mdfsetup{innertopmargin=0pt,innerleftmargin=5pt,innerrightmargin=5pt,linecolor=dorado!35,%
%		linewidth=2pt,topline=true,
%		frametitleaboveskip=\dimexpr-\ht\strutbox\relax,}
%	\begin{mdframed}[]\relax%
%		}{\end{mdframed}}
%\providecommand*{\theotautorefname}{Teorema}

\makeatletter
\renewenvironment{proof}{%
	\par\pushQED{\qed}%
	\renewcommand{\qedsymbol}{$\blacksquare$}%
	\normalfont \topsep6\p@\@plus6\p@\relax
	\trivlist
	\item[\hskip\labelsep\bfseries\itshape Demostración.]\ignorespaces
}{%
	\popQED\endtrivlist\@endpefalse
}
\makeatother