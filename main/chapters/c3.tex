\chapter{El compacto de Franklin}
\emph{\small Los espacios conocidos como \textit{compactos de Franklin} son la extensión unipuntual de los espacios de Mrówka. En el presente capítulo se estudiarán sus cualidades, comenzando por dar una caracterización de los mismos en propiedades topológiacas. Se demostrará que los compactos de Franklin asociados a un espacio de Mrówka pseudocompacto es un espacio secuencial de orden $2$.}

\emph{\small Se probará que el carácter del punto al infinito en el compacto de Franklin asociado a $\Psi(\ms{A})$ es $|\ms{A}|$, y se demostrará que este espacio es de Fréchet si y sólo si su familia asociada es maximal en ninguna parte. Esto a su paso dará la negativa, dentro de $\zfc$, a una cuestión relevante que estuvo sin solución durante el siglo XX: ¿es la propiedad de Fréchet finitamente productiva?.}

\section{\texorpdfstring{Sucesiones en $\ms{F}(\ms{A})$}{Sucesiones en F(A)}}
\label{Subsec-sucesiones-Franklin}

\begin{definicion}\index[alph]{compacto!de Franklin}\index[alph]{Franklin! compacto de}\index[sym]{$\ms{F}(\ms{A})$}
	Sea $\ms{A}\subseteq [\omega]^\omega$ cualquiera. El \textbf{compacto de Franklin generado por $\ms{A}$} es la extensión unipuntual del $\Psi$-espacio generado por $\ms{A}$, se denota por $ \ms{F}(\ms{A}):=\Psi(\ms{A}) \cup \{\infty_\ms{A} \} $.

	Cuando el contexto así lo permita, se omitirá el subíndice ``$_\ms{A}$'' y se denotará el punto al infinito simplemente por $\infty$.
\end{definicion}

\index[alph]{familia!no compacta}
Dado que un espacio topológico, no compacto, admite compactaciones Hausdorff únicamente cuando es Tychonoff y localmente compacto (véase \cite[p.~ 221]{fidelElementos}), el compacto de Franklin resulta ser la compactación de Alexandroff de $\Psi(\ms{A})$ únicamente cuando $\ms{A}$ sea una familia casi ajena, lo que garantiza que $\Psi(\ms{A})$ sea de Tychonoff; y sea \textit{no compacta} (es decir, que no sea simultáneamente finita y maximal), lo cual obliga a que $\Psi(\ms{A})$ sea no compacto; a razón de la \autoref{prop-tra-compacidad}.

\begin{consideracion}
	Durante esta sección:
	\begin{enumerate}
		\item Cualquier familia casi ajena $\ms{A}$ que se considere será no compacta.
		\item Para cada subespacio compacto $K \subseteq \Psi(\ms{A})$, se denotará por $V(K)$ a la vecindad abierta de $\infty$: $\{\infty\} \cup \Psi(\ms{A}) \setminus K$ (como $\Psi(\ms{A})$ es Hausdorff, todos los abiertos al rededor de $\infty$ son de esta forma).
		\item Se utilizarán casi en exceso los resultados obtenidos en \ref{prop-Kcaract} y \ref{cor-IdealCompactosCarac}, así que no se referenciarán de ahora en más.
		\item Todas las convergencias y operadores que aparezcan sin subíndices, se asumirán en $\ms{F}(\ms{A})$.
	\end{enumerate}
\end{consideracion}

Lo primero a observar es lo siguiente: dado que $\Psi(\ms{A})$ no es compacto, al ser un espacio de Tychonoff y localmente compacto, se tiene efectivamente que $\ms{F}(\ms{A})$ es de Hausdorff, compacto (véase \textcolor{red}{(prelims)}; y en consecuencia, normal. Como es previsible, ciertas propiedades de $\Psi(\ms{A})$ \enquote{suben} a la topología de $\ms{F}(\ms{A})$; como ejemplo inmediato, la separabilidad se preserva.

\begin{observacion}
	Sea $\ms{A}$ una familia casi ajena. Entonces $\ms{F}(\ms{A})$ es de Hausdorff, compacto, normal, localmente compacto, separable y disperso.
\end{observacion}

Comparando con el \autoref{cor-MrwokaSiempre} con las observaciones recién hechas, vale mencionar que existen propiedades $\ms{F}(\ms{A})$ que tienen una dependencia más compleja con $\Psi(\ms{A})$. Iniciarmos mostrando estas disparidades con lo próximo:

\begin{proposicion}\label{prop-caracterFrechet}
	Para toda familia $\ms{A}$ se tiene que $\chi(\infty) = \aleph_0 + |\ms{A}|$.
\end{proposicion}

\begin{proof} Es evidente que $\aleph_0 \leq \chi(\infty)$. Ahora, sea $\mathcal{B}$ una base local de $\infty$ en $\ms{F}(\ms{A})$. Para cada $y \in \ms{A}$ fíjese ($\Ac$) un $B_y \in \mathcal{B}$ con $B_y \subseteq V(y \cup \{y\})$. Obsérvese que la asignación $y \mapsto B_y$ es inyectiva; pues, si $x,y \in \ms{A}$ son distintos, entonces $B_x \subseteq U(x \cup \{x\})$ y $B_y \subseteq U(y \cup \{y\})$, de donde $x \in B_y \setminus B_x$. Por lo tanto $ |\ms{A}| \leq |\mathcal{B}|$, y en consecuencia $\aleph_0 + |\ms{A}|\leq \chi(\infty)$.

Para la desigualdad recíproca defínase:
$$ \mathcal{B} = \big\{ V( B \cup \midcup h \cup h ) \tq (B,h) \in [\omega]^{<\omega} \times [\ms{A}]^{<\omega} \big\} $$
y nótese que $\mathcal{B}$ es un conjunto de vecindades de $\infty$ en $\ms{F}(\ms{A})$.

Ahora, si $K \subseteq \Psi(\ms{A})$ es compacto, $h:=K \cap \ms{A} \subseteq \ms{A}$ y $G:=(K \cap \omega) \setminus \midcup h \subseteq \omega$ son finitos, además $V\big( G \cup \midcup h \cup h \big) \subseteq V(K)$. Lo cual demuestra que $\mathcal{B}$ es base local para $\infty$ en $\ms{F}(\ms{A})$. Dado que $ |\mathcal{B}| \leq | [\omega]^{<\omega} \times [\ms{A}]^{<\omega} | = \aleph_0 + |\ms{A}| $, resulta que $\chi(\infty) \leq \aleph_0 + |\ms{A}|$.
\end{proof}

En seguida a lo anterior se concluye que, para toda familia no compacta e infinita, el espacio $\ms{F}(\ms{A})$ no es de primero numerable. 

El siguiente Corolario se puede enriquecer con \ref{prop-tra-numerable}.

\begin{corolario}\index[trad]{primero numerabilidad de $\ms{F}(\ms{A})$}
	Para toda familia no compacta $\ms{A}$, el espacio $\ms{F}(\ms{A})$ es primero numerable si y sólo si $|\ms{A}| \leq \aleph_0$.
\end{corolario}

El próximo Lema es clave por varios motivos; entre ellos, responde a una pregunta que sugiere la discusión previa al \autoref{prop-CaracMADPositiv} ¿qué distingue a los subconjuntos de $\omega$ casi ajenos con cada elemento de $\ms{A}$, con los elementos de $\ms{I}^+(\ms{A})$?, encontramos la solución a esta interrogante por medio de la topología.

\begin{lema}\label{lem-convClave}
	Sean $\ms{A}\in \Ad(\omega)$ y $B \subseteq \Psi(\ms{A})$ infinito, entonces:
	\begin{enumerate}[i)]
		\item $\infty \in \scl(B)$ si y sólo si $B \cap \ms{A}$ es infinito, o $B \cap \omega$ es infinito y casi ajeno con cada elemento de $\ms{A}$.
		\item $\infty \in \cla(B)$ si y sólo si $B \cap \ms{A}$ es infinito, o $B \cap \omega \in \ms{I}^+(\ms{A})$.
	\end{enumerate}
\end{lema}

\begin{proof}
	(i) Para la suficiencia supóngase por absurdo que $\infty \in \scl(B)$, esto es (véase \textcolor{red}{(prelim)}), existe $C \in [B]^\omega$ tal que $C \to \infty$; que $B \cap \ms{A}$ es finito; y que $a \in \ms{A}$ es un elemento tal que $a \cap (B \cap \omega)$ es infinito. Obsérvese que, necesariamente $a \cap C$ es infinito. Como $C \to \infty$ y $a \cap C \subseteq C$ es infinito, ocurre que $a \cap C \to \infty$. Sin embargo, $a \cap C \subseteq a$ y a razón del \autoref{lem-convObvia}, $a \cap C \to a$ en $\Psi(\ms{A})$; así mismo en $\ms{F}(\ms{A})$. Lo anterior es imposible, ya que $a \neq \infty$ y $\ms{F}(\ms{A})$ es de Hausdorff.

	Recíprocamente, si $B \cap \ms{A}$ es infinito, existe $C_0 \in [B \cap \ms{A}]^\omega$; y, si $K \subseteq \Psi(\ms{A})$ es cualquier compacto, entonces $C_0 \setminus V(K) = C_0 \cap K \subseteq K \cap \ms{A}$ es finito, mostrando que $C_0 \to \infty$ y que $\infty \in \scl(B)$.

	Ahora, supóngase que $B \cap \omega$ es infinito y casi ajeno con cada elemento de $\ms{A}$. Si $\infty \notin \scl(B)$, entonces $B \cap \omega \not\to \infty$ y existiría un compacto $K_0 \subseteq \Psi(\ms{A})$ tal que $S:=(B \cap \omega) \cap K_0$ es infinito. $S \subseteq K$ es un elemento en $\ms{I}(\ms{A})$; a consecuencia de esto y de \ref{cor-CasiajenoPartePositiva}, existe cierto $a \in A$ con $S \cap a$ infinito; de donde, $B \cap a$ es infinito, contradiciendo la suposición sobre $B$. Por tanto, $\infty \in \scl(B)$

	(ii) Para la suficiencia, supóngase que $\infty \in \cla(B)$ y que $B \cap \ms{A}$ es finito. Resulta necesario que $\infty \in \cla(B \cap \omega)$. Si $K \subseteq \Psi(\ms{A})$ es compacto, $(B \cap \omega) \cap V(K) \neq \emptyset$, prohibiendo que $B \cap \omega \subseteq K$, por tanto, $B \cap \omega \notin \ms{I}(\ms{A})$.

	Para la necesidad, sí $B \cap \ms{A}$ es infinito, existe $C \subseteq B \cap \ms{A}$ numerable y por el inciso anterior $C \to \infty$, de donde $\infty \in \cla(C) \subseteq \cla(B)$. Y, si $B \cap \omega \in \ms{I}^+(\ms{A})$ y $K \subseteq \Psi(\ms{A})$ es compacto, resulta que $B \cap \omega \not\subseteq K$ y con ello $B\cap V(K) \neq \emptyset$, mostrando que $\infty \in \cla(B)$.
\end{proof}

Si $\ms{A}$ es una familia maximal, la condición (i) del resultado anterior implica que ninguna sucesión contenida en $\omega$ es convergente a $\infty$. A continuación se establecen las únicas convergencias posibles en el compacto de Franklin.
\begin{corolario}\label{cor-convMaximal}
	Sean $\ms{A} \in \Mad(\omega)$ infinita y $X \subseteq \ms{F}(\ms{A})$ numerable.
	\begin{enumerate}
		\item $X$ es convergente si y sólo si $X \subseteq^* \ms{A}$, o para algún $a \in \ms{A}$, $X \subseteq^* a$.
		\item Si $X \subseteq \omega$ y $x \in \omega \cup \{\infty\}$, entonces $X \not\to x$.
		\item Si $B \subseteq \omega \cap \scl(X)$, entonces $B \subseteq X$.
	\end{enumerate}
\end{corolario}

\begin{proof}
	(i) El recíproco es inmediato a razón de \ref{lem-convObvia} y el Lema previo. Para la suficiencia asúmase que $X \to x$ en el compacto de Franklin. Si $x = \infty$, se sigue del Lema anterior y la maximalidad de $\ms{A}$ que $X \subseteq^* \ms{A}$. En otro caso, se puede suponer sin pérdida de generalidad que $X \subseteq \Psi(\ms{A})$, siguiéndose de \ref{lem-convObvia} que $X$ debe estar casi contenido en algún elemento de $\ms{A}$.

	(ii) y (iii) Se desprenden inmediatamente del inciso (i) y de que cada punto en $\omega$ es aislado (por lo que las sucesiones convergentes a puntos de $\omega$ son eventualmente constantes).
\end{proof}
\begin{corolario}
	Sea $\ms{A} \in \Mad(\omega)$ infinita, entonces $\Osq(\ms{F}(\ms{A}))=2$.
\end{corolario}
\begin{proof}
	Nótese que $\scl^2(\omega) \not\subseteq \scl(\omega)$. Efectivamente; dado el Corolario anterior, $\scl(\omega) = \omega \cup \ms{A}$. Más aún, como $\ms{A}$ es infinita, contiene cierto subconjunto numerable $B \subseteq \ms{A}$. Y se obtiene de \ref{lem-convClave} que $B \to \infty$, esto muestra que $\infty \in \scl^2(\omega) \setminus \scl(\omega)$, por lo que $\Osq(\ms{F}(\ms{A})) \geq 2$.

	Ahora, sea $X \subseteq \ms{F}(\ms{A})$ cualquiera y supóngase que $x \in \scl^3(X)$. Como $\Psi(\ms{A})$ tiene orden secuencial $1$ (pues es $1\AN$, consecuentemente de Fréchet), es requisito que $x=\infty$. Así, existe $A \subseteq \scl^2(X)$ numerable tal que $A \to \infty$ en $\ms{F}(\ms{A})$. Sin pérdida de generalidad, $A \subseteq \ms{A}$ (dado \ref{lem-convClave}). Para cada $a \in A$ fíjese un conjunto numerable $B_a \in [\scl(X)]^\omega$ de manera que $B_a \to a$.
	
	Ahora, si $\scl(X) \cap \ms{A}$ es infinito, entonces por \ref{lem-convClave}, $x=\infty \in \scl(\scl(X))$. Supóngase pues que $\scl(X) \subseteq^* \omega$. Como cada $B_a \subseteq \scl(X)$ es convergente, asúmase sin pérdida de generalidad que $B_a \subseteq \omega$ (de nuevo, debido a \ref{lem-convClave}). En consecuencia, $B_a \subseteq \omega \cap \scl(X)$ y se obtiene del inciso (iii) del Corolario previo que $B_a \subseteq X$. Así $A \subseteq \scl(X)$, ya que cada $B_a$ satisface $B_a \to a \in A$. Esto muestra que $x \in \scl(A) \subseteq \scl^2(X)$. Es decir, $\scl^3(X) \subseteq \scl^2(X)$ y $\Osq(\ms{F}(\ms{A})) \leq 2$.
\end{proof}

%El posterior Teorema sigue la línea del Teorema de Kannan y Rajagopalan (\ref{teo-HLCCaract}), es una caracterización en propiedades topológicas de ciertos compactos de Franklin.
\begin{teorema}\index[trad]{homeomorfismo con $\ms{F}(\ms{A})$ ($\ms{A}$ maximal)}
	Sea $X$ un espacio topológico infinito. $X$ es homeomorfo a un compacto de Franklin generado por una familia maximal infinita si y sólo si se satisfacen:
	\begin{enumerate}
		\item $X$ es compacto, de Hausdorff y separable, y
		\item Existe $x_0 \in X$ tal que $x_0$ es el único punto de acumulación de $\der(X)$, y, para cada $B \in [X \setminus \der(X)]^\omega$ se tiene $B \not\to x_0$.
	\end{enumerate}
	Claramente, en tal caso $x_0$ se identifica bajo algún homeomorfismo con el punto al infinito del compacto de Franklin.
\end{teorema}

\begin{proof}
	Para la suficiencia basta suponer que $X=\ms{F}(\ms{A})$; con $\ms{A} \in \Mad(\omega)$ infinita, y probar (ii). Sea $x_0:=\infty$. Por ser $X$ la compactación unipuntual de $\Psi(\ms{A})$, se tiene que $\Psi(\ms{A})$ es un denso de $X$ y por ello $\der(X) = \{\infty\} \cup \der_{\Psi(\ms{A})}(\Psi(\ms{A}))$. De lo anterior y \ref{lem-primerosSubs} se tiene que $\der(X) = \{\infty\} \cup \ms{A}$; y además que $\ms{A}$ es discreto. En consecuencia, $\der_{\der(X)}(\der(X)) \subseteq \{\infty\}$ y la contención recíproca ocurre; pues cada subespacio compacto de $\Psi(\ms{A})$ tiene intersección finita; particularmente no vacía, con el conjunto (infinito) $\ms{A}$. Por lo que $\der(X)$ sólo se acumula en $x_0=\infty$. Ahora, si $B \in [X \setminus \der(X)]^\omega$, entonces $B \subseteq \omega$ y por la maximalidad de $\ms{A}$, se obtiene que $B \not\to \infty$ (utilizando el \autoref{cor-convMaximal}).

	Véase ahora la necesidad; esto es, supóngase que es compacto, de Hausdorff, separable y que $x_0$ actúa tal cual dicta (ii). Defínase $Y:=X\setminus \{x_0\}$, se mostrará primero que $Y \cong \Psi(\ms{A})$ para alguna familia maximal $\ms{A}$. Efectivamente, nótese que $Y$ es infinito, de Hausdorff y separable (ya que $Y$ es abierto al ser $\{x_0\}$ cerrado); así que haciendo uso del \autoref{cor-HLCPseudoCaract}, es suficiente mostrar los siguientes tres puntos:

	($Y$ es regular) Dado que $X$ es compacto, de Hausdorff es normal y particularmente, regular. Esto prueba que $Y \subseteq X$ es regular.

	($\der_Y(Y)$ es discreto) Efectivamente, si $y \in \der_Y(Y)$ es cualquiera, entonces $y \in Y$ es punto de acumulación de $X$. Como $y \neq x_0$ y $x_0$ es el único punto de acumulación de $\der_X(X)$, $\{y\}$ es abierto en $\der_X(X)$; y por tanto, $\{y\}$ es abierto en $\der_Y(Y)$. Mostrando que $\der_Y(Y)$ es discreto.

	(Si $B \subseteq Y$ es discreto, abierto y cerrado a la vez, entonces $B$ es finito) Supóngase que $B \subseteq Y$ es discreto, abierto y cerrado a la vez. Por ser $B$ discreto y abierto, se da $B \subseteq X \setminus \der(X)$. Ahora, si $B$ es infinito (sin pérdida de generalidad, numerable) se tiene de la hipótesis que $B \not\to x_0$; así, existe una vecindad de $x_0$; a saber U, de modo que $B \setminus U$ es infinito. Sin embargo, $B \cap U$ es cerrado en vista de que $B$ es cerrado; por ello, tal conjunto es cerrado, discreto e infinito en $X$; lo que contradice que $X$ sea compacto y $\T_1$. Por ello, es necesario que $B$ sea finito. Concluyéndose de \ref{cor-HLCPseudoCaract}, la existencia de una familia $\ms{A}\in\Mad(\omega)$ de modo que $Y \cong \Psi(\ms{A})$.

	Para finalizar, obsérvese que $\{x_0\}$ no es abierto en $X$, pues de lo contrario no podría ser punto de acumulación de ninguno de sus subespacios. Así, $Y$ es un subespacio denso de $X$ y como $X$ es de Hausdorff, compacto, con $X \setminus Y = \{x_0\}$, resulta que $X$ es la compactación unipuntual de $Y \cong \Psi(\ms{A})$; esto es, $X \cong \ms{F}(\ms{A})$.
\end{proof}

\section{La propiedad de Fréchet}

Continuando con los frutos del \autoref{lem-convClave}, se extrae el siguiente Corolario; este relaciona las propiedades de combinatoria de las familias casi ajenas con propiedades de convergencia.

\begin{lema}\label{lem-TrazaMad}
	Si $\ms{A}$ es una familia no compacta, entonces para cada $X \subseteq \omega$ son equivalentes:
	\begin{enumerate}
		\item $\infty \in \scl(X)$.
		\item $\ms{A} \upharpoonright X \notin \Mad(X)$.
	\end{enumerate}
\end{lema}

\begin{proof}
	(i) $\to$ (ii) Si $\infty \in \scl(X)$, entonces existe $B \subseteq X \subseteq \omega$ de modo que $B \to x$ y de acuerdo al \autoref{lem-convClave} se tiene garantizado que $B$ es casi ajeno con cada elemento de $\ms{A}$ (pues $B \cap \ms{A}$ es finito, por ser vacío). Entonces $B \in [X]^\omega$ es casi ajeno con cada elemento de $\ms{A} \upharpoonright X$; efectivamente, si $a \cap X \in \ms{A} \upharpoonright X$ es cualquiera, entonces $B \cap (X \cap a) = X \cap (a \cap B) \subseteq a \cap B =* \emptyset$. Mostrando que $\ms{A} \upharpoonright X$ no es maximal en $X$.

	(ii) $\to$ (i) Si $\ms{A} \upharpoonright X$ no es maximal en $X$, existe $B \subseteq X$ infinito y casi ajeno con cada elemento de $\ms{A} \upharpoonright X$. Nótese que entonces $B \cap X$ es casi ajeno con cada elemento de $\ms{A}$; y por lo tanto, $B \to \infty$ (por \ref{lem-convClave}). Por ello $\infty \in \scl(B) \subseteq \scl(X)$.
\end{proof}

Del \autoref{prop-CaracMADPositiv} se desprende fácilmente la contención:
\[ \{X \in [\omega]^{\omega} \tq \forall A \in \ms{A} \: (A \cap X =^* \emptyset) \} \subseteq \ms{I}^+(\ms{A}) \]

Resulta que la contención recíproca encapsula la conexión que existe entre la combinatoria de $\ms{A}$ y la propiedad de Fréchet de su compacto de Franklin asociado.

\begin{corolario}\label{cor-TraFrechet}\index[trad]{propiedad de!Fréchet en $\ms{F}(\ms{A})$}
	Sea $\ms{A}$ una familia casi ajena no compacta. Son equivalentes:
	\begin{enumerate}
		\item $\ms{F}(\ms{A})$ es de Fréchet.
		\item $\{X \in [\omega]^{\omega} \tq \forall A \in \ms{A} \: (A \cap X =^* \emptyset) \} = \ms{I}^+(\ms{A})$.
		\item $\ms{A}$ es maximal en ninguna parte.
	\end{enumerate}
\end{corolario}

\begin{proof}
	(i) $\to$ (ii) Supóngase que $\ms{F}(\ms{A})$ es de Fréchet. Basta probar la contención recíproca de (ii). Y efectivamente, si $X \in \ms{I}^+(\ms{A})$, entonces $\infty \in \cla(X)$ debido \ref{lem-convClave}, pero como $\ms{F}(\ms{A})$ es de Fréchet, $\infty \in \scl(X)$; siguiéndose del mismo \autoref{lem-convClave}, que $X$ es casi ajeno con cada elemento de $\ms{A}$.

	(ii) $\to$ (iii) Supóngase (ii) y sea $X \in \ms{I}^+(\ms{A})$ cualquiera. Dada la hipótesis, $X$ es casi ajeno con cada elemento en $\ms{A}$, así que por \ref{lem-convClave}, $\infty \in \scl(X)$. Obteniéndose del \autoref{lem-TrazaMad}, que $\ms{A} \upharpoonright X \notin \Mad(X)$.

	(iii) $\to$ (i) Supóngase que $\ms{A}$ es maximal en ninguna parte. Como $\Psi(\ms{A})$ es de Fréchet (por ser primero numerable) basta verificar la propiedad de Fréchet en $\infty \in \ms{F}(\ms{A})$. Sea $X \subseteq \ms{F}(\ms{A})$ de modo que $\infty \in \cla(X)$, entonces por \ref{lem-convClave}, $X \cap \ms{A}$ es infinito o $X \cap \omega \in \ms{I}^+(\ms{A})$. Si ocurre lo primero, sea $B \subseteq X \cap \ms{A}$ numerable y nótese que entonces $B \to \infty$, lo cual basta para mostrar que $\infty \in \scl(X)$. Si ocurre el segundo caso, de la hipótesis se obtiene $A \upharpoonright (X \cap \omega) \notin \Mad(X \cap \omega)$, probando que $\infty \in \scl(X \cap \omega) \subseteq \scl$ (en virtud \ref{lem-TrazaMad}). En ambos casos, $\infty \in \scl(X)$; y por tanto $\scl(X)=\cla(X)$.
\end{proof}

El Corolario anterior puede ser empleado para solucionar un problema clásico en topología general; determinar si el producto de dos espacios de Fréchet es de Fréchet. Los espacios de Mrówka dejan ver su ``maleabilidad'' al momento de generar contraejemplos a través de la subsecuente ilación de ideas.

\begin{proposicion}
	Sea $\ms{A}$ una familia casi ajena, unión ajena de las familias no vacías $\ms{B}$ y $\ms{C}$. Si $\ms{A}$ es maximal en alguna parte, entonces $\ms{F}(\ms{B}) \times \ms{F}(\ms{C})$ no es de Fréchet.
\end{proposicion}

\begin{proof}
	Supóngase que existe $X \in\ms{I}^+(\ms{A})$ de modo que $\ms{A} \upharpoonright X \in \Mad(X)$ y sea $B:=\{ (n,n) \tq n \in X \}$.

	Como $X \in \ms{I}^+(\ms{A})$ y $\ms{B},\ms{C} \subseteq \ms{A}$, resulta que $X \in \ms{I}^+(\ms{B})$ y $X \in \ms{I}^+(\ms{C})$ (véase \ref{prop-TrazaBasicos}). Entonces se tiene que $\infty_\ms{B} \in \cla_{\ms{F}(\ms{B})}$ y $\infty_\ms{C} \in \cla_{\ms{F}(\ms{C})}$ a consecuencia del \autoref{lem-convClave}. De este modo:
	$$ (\infty_\ms{B},\infty_C) \in \cla_{\ms{F}(\ms{B}) \times \ms{F}(\ms{C})}(B) $$
	sin embargo $(\infty_\ms{B},\infty_\ms{C}) \notin \scl_{\ms{F}(\ms{B}) \times \ms{F}(\ms{C})}(B)$. Efectivamente, de lo contrario, existe $Y \in [X]^\omega$ de manera que $\{(n,n) \tq n \in Y\}$ converge a $ (\infty_\ms{B},\infty_C)$. De lo anterior, y la continuidad de las funciones proyección, se obtiene que $Y \to \infty_\ms{B}$ en $\ms{F}(\ms{B})$ y $Y \to \infty_\ms{C}$ en $\ms{F}(\ms{C})$. Sin embargo a consecuencia de ello; y por \ref{lem-convClave}, $Y \subseteq X$ es infinito y casi ajeno con cada elemento de $\ms{B}$ y $\ms{C}$; es decir, con cada elemento de $\ms{A}\ms{B} \cup \ms{C}$, siendo esto una contradicción a la maximalidad de $\ms{A} \upharpoonright X$ en $X$.

	Por lo tanto $(\infty_\ms{B},\infty_\ms{C}) \notin \scl_{\ms{F}(\ms{B}) \times \ms{F}(\ms{C})}(B)$ y el producto $\ms{F}(\ms{B}) \times \ms{F}(\ms{C})$ no tiene la propiedad de Fréchet.
\end{proof}

Combinando con el Teorema de Simon (\ref{Teo-Simon}), se tiene la siguiente fuente de contrajemplos: Cada vez que $\ms{A}$ sea una familia infinita y maximal (por ello, no compacta), se pueden dar dos familias $\ms{B} \subseteq \ms{C}$ no vacías, maximales en ninguna parte, de modo que $\ms{A}=\ms{B} \cup \ms{C}$. Y se desprende de la Proposición previa que $\ms{F}(\ms{B}) \times \ms{F}(\ms{C})$ no es de Fréchet; pues claramente $\ms{A}$ es maximal en alguna parte, ya que $\omega \in \ms{I}^+(\ms{A})$ (por \ref{cor-IdealPropioCaract}); y además, $\ms{F}(\ms{B})$ y $\ms{F}(\ms{C})$ son ambos de Fréchet (por \ref{cor-TraFrechet}). Esto implica:

\begin{corolario}\label{cor-FrechNoProd}
	Existen dos espacios de Mrówka cuyas compactaciones unipuntuales son de Fréchet, pero su producto no

	En particular, la propiedad de Fréchet no es finitamente productiva; ni siquiera en la clase de espacios compactos, de Hausdorff.
\end{corolario}

Dado el Teorema de Simon (\ref{Teo-Simon}), toda familia maximal de tamaño $\kappa$ contiene una familia maximal en ninguna parte de cardinalidad, también $\kappa$. De la caracterización dada en \ref{cor-TraFrechet} y la \autoref{prop-caracterFrechet}, se obtiene:

\begin{corolario}
	Si existe una familia maximal de tamaño $\kappa$, existe un espacio de Fréchet tal que uno de sus puntos tiene carácter $\kappa$.

	Particularmente, existe un espacio de Fréchet, que contiene un punto de carácter $\mathfrak{c}$.
\end{corolario}
