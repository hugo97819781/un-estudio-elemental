\chapter{Familias casi ajenas}
\emph{\small Las familias casi ajenas son objetos fascinantes de la teoría de conjuntos; como se verá a lo largo de este trabajo, sus aplicaciones no sólo se limitan a esta rama de las matemáticas, sino que sus repercusiones se extienden a la topología. Entre los pioneros de su estudio destacan figuras como Hausdorff, Sierpiński, Erdős y Rado.}

\emph{\small El presente capítulo tiene como meta presentar a este objeto matemático y exponer sus propiedades más inmediatas; algunos métodos para su construcción; y finalmente, un estudio básico sobre su combinatoria. En este última parte se abordarán resultados clásicos: los Lemas de Dočkálková y Solovay, el Teorema de Simon y la existencia de las familias de Luzin.}

\section{Observaciones inmediatas}

\begin{definicion}\label{def-casi-ajena}\index[alph]{casi!ajeno}\index[alph]{casi!ajena sobre $N$, familia}\index[alph]{familia!casi ajena}\index[alph]{casi!ajena, familia}\index[alph]{familia!casi ajena sobre $N$}\index[sym]{$\Ad(N)$}
	Sea $N$ un conjunto numerable. Una \textbf{familia casi ajena sobre $N$} es un subconjunto $\ms{A}\subseteq[N]^\omega$ cuyos elementos son casi ajenos por pares. $\Ad(N)$ denotará el conjunto de todas las familias casi ajenas sobre $N$.
	
	El término \textbf{familia casi ajena} (o simplemente \textbf{familia}) hará referencia a una familia casi ajena sobre $\omega$.
\end{definicion}

El concepto previo es fácilmente generalizable, el lector puede indagar al respecto en \cite[Def.~9.20, p.~118]{jechSet}. Sin embargo, la teoría asociada a las familias casi ajenas, definidas como en \ref{def-casi-ajena}, es suficientemente amplia y meritoria de un estudio dedicado.

Cualquier familia de subconjuntos ajenos por pares de $N$, es también una familia casi ajena sobre $N$; particularmente, $\emptyset$ y cualquier colección de la forma $\{A\}$, con $A \in [N]^\omega$. Además, resulta evidente que cada subconjunto de una familia casi ajena sobre $N$ es, a su vez, una familia casi ajena sobre $N$.

Es claro que toda familia casi ajena tiene tamaño menor o igual a $\mathfrak{c}$; así que en virtud de lo previamente observado, de existir alguna de ellas de tamaño el continuo, se garantizaría la existencia de familias ajenas de cualquier tamaño inferior a $\mathfrak{c}$.

\begin{ejemplo}
	\label{ej-ADfacil}
	Las colecciones $\{\omega\}$, $\{ \{ 2n \tq n \in \omega \}, \{ 2n+1 \tq n \in \omega \} \}$ y $\{ \{ p^n \tq n \in \omega \setminus \{0\} \} \tq p \text{ es primo} \}$ son familias casi ajenas sobre $\omega$.
\end{ejemplo}

No resulta muy difícil verificar que las primeras dos familias del ejemplo anterior son \enquote{grandes}, en el siguiente sentido:

\begin{definicion}\index[alph]{familia!casi ajena sobre $N$!maximal en $N$}\index[sym]{$\Mad(N)$}\index[alph]{familia!casi ajena!maximal}
	Sea $N$ conjunto numerable. Una familia $\ms{A}$ sobre $N$ es \textbf{maximal en} $N$ si es un elemento $\subseteq$-maximal del conjunto $\Ad(N)$. Se denotará por $\Mad(N)$ al conjunto de todas ellas.
	
	Cuando no haya riesgo de ambigüedad, el término \textbf{familia maximal} hará referencia a una familia maximal en $\omega$.
\end{definicion}

Dado que los elementos de toda familia casi ajena son infinitos, se tiene inmediatamente la siguiente observación:

\begin{observacion}
	Una familia $\ms{A} \in \Ad(N) $ es maximal en $N$ si y sólo si se cumple cualquiera de las siguientes condiciones equivalentes:
	\begin{enumerate}
		%\item Para toda $\ms{B} \in \Ad(N)$, si $\ms{A} \subseteq \ms{B}$, entonces $\ms{A} = \ms{B}$
		\item Para cualquier $\ms{B} \subseteq [N]^\omega$, si $\ms{A} \subsetneq \ms{B}$, entonces $\ms{B} \notin \Ad(N)$.
		\item Para cada $B \in [N]^\omega$ existe $A \in \ms{A}$ tal que $A \cap B$ es infinito.
	\end{enumerate}
\end{observacion}

Se advierte que las familias sobre $\omega$ parecerán deslucir a las construidas sobre otros conjuntos numerables; pero al no ser el estudio sobre estas últimas nulo, es menester considerar las propiedades que son transferibles entre estas dos clases de objetos.

\begin{definicion}\label{def-Biyecs-h}\index[sym]{$\Phi_h$}
	Sean $N,M$ conjuntos numerables y $h:N \to M$ cualquier biyección. Se define $\Phi_h : \ms{P}(\ms{P}(N)) \to \ms{P}(\ms{P}(M))$ como:
	\[ \Phi_h (\ms{A}) = \{ h[A] \tq A \in \ms{A} \} \, . \]
\end{definicion}

En términos de lo recién definido, se remarca que al ser $h$ biyección, $\Phi_h$ será una biyección. Más aún, estas biyecciones se comportan bien respecto a ciertas virtudes conjuntistas, tal y como se ilustra a continuación.

\begin{proposicion}\label{prop-ADbiyec}
	Sean $N,M$ son numerables y $h:N \to M$ una biyección cualquiera. Entonces:
	\begin{enumerate}
		\item $\ms{A} \subsetneq \ms{B}$ si y sólo si $\Phi_h(\ms{A}) \subsetneq \Phi_h(\ms{B})$.
		\item $\Phi_h(\ms{A} \cap \ms{B}) = \Phi_h(\ms{A}) \cap \Phi_h(\ms{B})$.
		\item $\Phi_h(\ms{A} \cup \ms{B}) = \Phi_h(\ms{A}) \cup \Phi_h(\ms{B})$.
		\item $|\ms{A}|=|\Phi_h(\ms{A})|$.		
		\item $\Phi_h[\Ad(N)]=\Ad(M)$.
		\item $\Phi_h[\Mad(N)]=\Mad(M)$
	\end{enumerate}
\end{proposicion}
\begin{proof}
	Se mostrarán únicamente (v) y (vi). En ambos basta probar la contención directa, pues al ser $h$ biyección, $\Phi_h^{-1} = \Phi_{h^{-1}}$.

	(v) Si $\ms{A} \in \Ad(N)$, entonces $\ms{A} \subseteq [N]^\omega$ y así $\Phi_h(\ms{A}) \subseteq [M]^\omega$. Ahora, si $h[A],h[B] \in \Phi_h(\ms{A})$ son distintos, es necesario que $A \neq B$ y por ello $A \cap B =^* \emptyset$. Se obtiene que $h[A] \cap h[B]=h[A \cap B]=^* \emptyset $, y con ello, $\Phi_h(\ms{A}) \in \Ad(M)$.

	(vi) Si $\ms{A} \in \Mad(\ms{A})$ y $B \subseteq M$ es infinito, entonces $h^{-1}[B] \subseteq N$ es infinito y existe $A \in \ms{A}$ tal que $A \cap h^{-1}[B]$ es infinito. Al ser $h$ biyección, $h[A \cap h^{-1}[B]]=h[A] \cap B$ es infinito, por ende $\Phi_h(\ms{A}) \in \Mad(M)$.
\end{proof}

A partir de este momento se consolida la usanza de hacer hincapié en qué propiedades, u objetos, basados en familias casi ajenas se preservan bajo las biyecciones $\Phi_h$.

Una aplicación superflua de lo anterior es el nacimiento de un método cómodo para generar familias casi ajenas; en especial, infinitas.

\begin{ejemplo}
	\label{ej-Bandas}
	Claramente $\ms{A}=\{ \{n\} \times \omega \tq n \in \omega \} \in \Ad(\omega \times \omega)$. Así que si $h:\omega \times \omega \to \omega$ es biyección, $\Phi_h(\ms{A}) \in \Ad(\omega)$. Más aún, tal familia es del mismo tamaño que $\ms{A}$ (todo gracias a \ref{prop-ADbiyec}).
\end{ejemplo}

A continuación se comenzarán a examinar las propiedades de las familias casi ajenas maximales; se tiene la intención de responder a las preguntas que surgen naturalmente como: ¿puede haber familias casi ajenas más que numerables?, o, ¿existen familias maximales infinitas?

\begin{lema}\label{lem-MADnecesarioUnion}
	Si $\ms{A}$ es familia casi ajena maximal, entonces $\omega \subseteq^* \midcup \ms{A}$.
\end{lema}

\begin{proof}
	Por contrapuesta, supóngase que $B:=\omega \setminus \midcup \ms{A}$ es infinito. Si $A \in \ms{A}$, entonces $A \subseteq \midcup \ms{A}$, y así $A \cap B \subseteq A \setminus \midcup \ms{A} \subseteq A \setminus A = \emptyset$. Por lo que $B \in [\omega]^\omega$ es casi ajeno con cada elemento de $\ms{A}$.
\end{proof}

El recíproco del Lema previo falla para familias infinitas (véase la familia $\Psi(\ms{A})$ del \autoref{ej-Bandas}); y de hecho, no se cuenta un resultado \enquote{amigable} para determinar cuándo estas resultan ser maximales (véase \ref{prop-CaracMADPositiv}). En contraparte a esto, se deduce rápidamente la siguiente equivalencia:

\begin{corolario}\label{cor-MADnecesarioUnion}
	Sea $\ms{A} \in \Ad(\omega)$. $\ms{A}$ es maximal si y sólo si $\omega \subseteq^* \midcup \ms{A}$.
\end{corolario}

\begin{proof}
	Por el Lema previo, basta demostrar la necesidad.

	Supóngase $\omega \subseteq^* \midcup \ms{A}$ y nótese que si $B \in [\omega]^\omega$, entonces $B \subseteq^*\midcup \ms{A}$ y con ello $\emptyset \neq^* B  \subseteq^* B \cap \midcup \ms{A} = \midcup\{B \cap A \tq A \in \ms{A}\}$. Como la última es una unión finita, $B$ debe tener intersección infinita con algún elemento de $\ms{A}$.
\end{proof}

El posterior resultado puede ser visto como un símil al Teorema del Ultrafiltro \textcolor{red}{\textbf{(ver prelims)}}, o bien, cualquier resultado afín en el que se haga uso del Principio de Maximalidad de Hausdorff o sus equivalentes \textcolor{red}{\textbf{(ver prelims)}}.

\begin{lema}\label{lem-MADs}
	Toda familia casi ajena está contenida en una familia maximal.
\end{lema}

\begin{proof}
	Sean $\ms{A} \in \Ad(\omega)$ y $X$ el conjunto de todos las familias casi ajenas que contienen a $\ms{A}$. Como $A \in X$, por el Principio de Maximalidad de Hausdorff ($\Ac$), existe $Y \subseteq X$, una cadena $\subseteq$-maximal de $(X,\subseteq)$.

	Defínase $\ms{B}:=\midcup Y$, como $Y \subseteq \ms{P}([\omega]^\omega)$, entonces $\ms{B} \subseteq [\omega]^{\omega}$. Además, si $C,D \in \ms{B}$, existen $\ms{C},\ms{D} \in Y \subseteq \Ad(\omega)$ con $C \in \ms{C}$ y $D\in \ms{D}$. Puesto que $Y$ es cadena de $(X,\subseteq)$, sin pérdida de generalidad, $C,D \in \ms{D} \supseteq \ms{C}$; y con ello, $C \cap D$ es finito, ya que $\mathscr{D} \in Y \subseteq X \subseteq \Ad(\omega)$. Luego $\ms{B} \in \Ad(\omega)$, y además, $\ms{A} \subseteq \ms{B}$.

	Finalmente, si $\ms{B}' \in \Ad(\omega)$ y $\ms{B} \subseteq \ms{B}'$, entonces $Y \cup \{\ms{B}'\}$ es una cadena de $(X,\subseteq)$; lo cual, junto a la maximalidad de $Y$, implica que $\mathscr{B}' \in Y$ y $\mathscr{B} = \mathscr{B}'$. Por lo tanto, $\ms{B}\in \Mad(\omega)$.
\end{proof}

El siguiente resultado revela un fenómeno interesante respecto al tamaño de las familias maximales. Este se le atribuye a Wacław Sierpinski (se desprende de \cite[Teo.~2, p.~458]{SierpinskiCardinal}).

\begin{lema}\label{prop-MADnoNum}
	Ninguna familia casi ajena numerable es maximal.
\end{lema}

\begin{proof}
	Sea $\ms{A} \in \Ad(\omega)$ enumerada por $\ms{A}=\{A_n \tq n \in \omega\}$. Si $n \in \omega$ es cualquiera, $A_n \cap \midcup \{A_m \tq m< n\} = \midcup \{A_n \cap A_m \tq m<n\}$ es finito, al ser unión finita de conjuntos finitos. Luego, por ser $A_n$ infinito, $A_n \setminus \midcup \{A_m \tq m<n\} = A_n \setminus \big( A_n \cap \midcup \{A_m \tq m<n\} \big)$ debe ser infinito; y particularmente, no no vacío.

	Considérese $f:\omega \to \omega$ definida por $f(n) = \min\{A_n \setminus \midcup \{A_m \tq m<n\}\}$ para cada $n$. Resulta que $f$ es inyectiva; si $m<n$, entonces $f(n) \notin A_m$, $f(m) \in A_m$ y $f(n) \neq f(m)$. Además, para cada $n \in \omega$ se tiene que $A_n \cap f[\omega] = \{f(n)\}$; y con ello $f[\omega] \subseteq \omega$ es infinito y casi ajeno con cada elemento de $\ms{A}$.
\end{proof}

Tomando cualquier familia infinita $\ms{A}$ y aplicando \autoref{lem-MADs}, se obtiene una familia maximal $\ms{B} \supseteq \ms{A}$ de tamaño infinito. Por el resultado anterior, tal infinito debe ser más que numerable. Esta consecuencia es tan inmediata como, quizás, poco satisfactoria; pues su naturaleza es \enquote{no constructiva}. Durante la posterior sección se mostrarán métodos para obtener estos últimos objetos de una manera más explícita.

\begin{observacion}\label{obs-ExisteNoNumMAD}
	Existe una familia maximal de tamaño al menos $\aleph_1$.
\end{observacion}

\section{Familias casi ajenas de tamaño \texorpdfstring{$\mathfrak{c}$}{c}}

%La presente sección tiene por meta exhibir dos de los métodos más típicos para la construcción de familias casi ajenas infinitas. El primero de ellos, se basa en las sucesiones convergentes de espacios topológicos de Hausdorff, primero numerables.

Al tomar un espacio de Fréchet $X$ y cualquier subespacio denso $D \subseteq X$, para cada $x \in X \setminus D$ ha de existir una sucesión en $D$ convergente a $x$. Si a $X$ le adicionamos la condición de ser $\T_1$, la imagen de tal sucesión es necesariamente un conjunto numerable $A_x \subseteq D$, convergente a $x$ en $X$ \textcolor{red}{\textbf{(revisar PRELIMS)}}.

\begin{proposicion}\label{prop-famSucesiones}
	Supóngase que $X$ es un espacio de Hausdorff, de Fréchet y que $D \subseteq X$ es denso y numerable. Para cada $A \subseteq X \setminus D$ existe una familia casi ajena sobre $D$ biyectable con $A$.
\end{proposicion}

\begin{proof}
	Con sustento en los comentarios previos, para cada $x \in A$ fíjese ($\Ac$) un conjunto $A_x \subseteq D$ numerable y convergente a $x$ en $X$. Defínase la colección $\ms{A}_{D,A}$ como $\{ A_x \subseteq D \tq x \in A \} \subseteq [D]^\omega$.

	Sean $x,y \in A$ con $x \neq y$, por ser $X$ de Hausdorff, hay abiertos ajenos $U,V$ tales que $x \in U$ y $y \in V$. Seguido de que $A_x \to x$ y $A_y \to y$, se tiene $A_x \subseteq^* U$ y $A_y \subseteq ^* V$; consecuentemente $A_x \cap A_y \subseteq^* U \cap V = \emptyset$. Lo cual prueba que $\ms{A}_{D,A} \in \Ad(D)$ y $|\ms{A}_{D,A}|=|A|$.
\end{proof}

\begin{definicion}\label{def-FamSucesiones}\index[alph]{familia!de!sucesiones en $D$ convergentes a $A$ en $X$}\index[sym]{$\ms{A}_{D,A}$}
	Si $X$, $D$ y $A$ son como en la Proposición anterior, a $\ms{A}_{D,A}$ se le denomina \textbf{familia de sucesiones en $D$ convergentes a $A$ en $X$}.
\end{definicion}

Como la recta real $\mathbb{R}$ es de Hausdorff, de Fréchet (por ser $1\AN$) y $\mathbb{Q} \subseteq \mathbb{R}$ es un subespacio denso numerable; de lo previamente establecido se obtiene que $\ms{A}_{\mathbb{Q},\mathbb{R}\setminus \mathbb{Q}}$ es una familia casi ajena sobre $\mathbb{Q}$ de tamaño $\mathfrak{c}=|\mathbb{R} \setminus \mathbb{Q}|$. Conviene destacar que la construcción recién mencionada no depende del $\Ac$; pues para cada $x \in \mathbb{R} \setminus \mathbb{Q}$ el conjunto $A_x$ de la \autoref{prop-famSucesiones} se puede construir de manera explícita. Si $q:\omega \to \mathbb{Q}$ es cualquier biyección, basta considerar:
\[ A_x = \Big\{ q \Big( \min \Big( q^{-1} \Big[ \mathbb{Q} \cap \Big( x-\frac{1}{n+1}, x-\frac{1}{n} \Big) \Big] \Big) \Big) \, \Big| \, n \in \omega \setminus \{0\} \Big\} \, . \]



El corazón de la próxima estrategia para la obtención explícita de familias casi ajenas de tamaño el continuo, son los árboles.

Comenzaremos observando que si $S$ es cualquier rama de un árbol $(T,\leq)$, entonces $S$ es cerrada bajo cotas inferiores. En efecto, sean $x \in S$ y $y \leq x$. Para cada $s \in S$, al ser $S$ cadena, se tiene que $x \leq s$ o $s < x$. En el primer caso, $y \leq x \leq s$ y $y$ es comparable con $s$. En el segundo $y,s < x$; y como $(\{ y \in T \tq y < x \},\leq)$ es buen orden, $y$ y $s$ son comparables. Por lo tanto, $S \cup \{y\}$ es una cadena; y seguido de que $S$ es rama, $y \in S$.

\begin{proposicion}
	Sea $(T,\leq)$ un árbol numerable de altura $\omega$ y $\ms{A} \subseteq [T]^\omega$ el conjunto de todas las ramas numerables de $(T,\leq)$. Entonces $\ms{A} \in \Ad(T)$.
\end{proposicion}

\begin{proof}
	Sean $R,S \in \ms{A}$ con $R \neq S$, sin pérdida de generalidad, existe $x_0 \in R \setminus S \neq \emptyset$. De existir $y \in R \cap S$ tal que $y \not < x_0$, resultaría que $x_0 \leq y$, en virtud de que $x,y \in R$ y $R$ es rama. Lo anterior y la discusión previa a esta Proposición implican que $x_0 \in S$, lo cual es imposible.
	
	Por lo tanto $R \cap S \subseteq \{ y \in T \tq y<z \}$; y como $T$ tiene altura $\omega$, el orden de $x_0$ es un natural; consecuentemente, $R \cap S$ es finito.
\end{proof}

Un ejemplo canónico de árbol numerable de altura $\omega$ es $2^{<\omega}$ \textcolor{red}{\textbf{(véase PRELIMS)}}; considerar la siguiente clase de familias desembocará en resultados sumamente notables (como se puede ver en la \autoref{Sec-PDM}).

\begin{proposicion}
	Para cada $f \in 2^\omega$ defínase $A_f:=\{ f \upharpoonright n \tq n \in \omega \} \subseteq 2^{<\omega}$; entonces:
	\begin{enumerate}
		\item Cada $A_f$ es una rama de $(2^{<\omega},\subseteq)$.
		\item Si $f\neq g$, entonces $A_f \neq A_g$.
	\end{enumerate}
\end{proposicion}
\begin{proof}
	(i) Sea $f \in 2^\omega$, inmediatamente, $A_f$ es cadena de $(2^{<\omega},\subseteq)$. Supóngase ahora que $S \subseteq 2^{<\omega}$ es una rama de $(2^{<\omega},\subseteq)$ y que $A_f \subseteq S$. 
	
	Sean $g \in S$ y $n=\dom(g)$; puesto que $S$ es cadena de $(2^{<\omega},\subseteq)$ y $f \in A_f \subseteq S$, entonces $f \upharpoonright n \subseteq g$ o $g \subseteq f \upharpoonright n$. Cualquiera de los casos anteriores implican que $f \upharpoonright n = g$, pues $\dom(g)=\dom(f \upharpoonright n)$. Luego, $g \in A_f$ y entonces $A_f = S$.

	(ii) Si $f \neq g$, entonces existe $m\in \omega$ tal que $f(m) \neq g(m)$. Así, se obtiene que $f \upharpoonright m+1 \neq g \upharpoonright m+1$ y $f \upharpoonright m+1 \in R_f \setminus R_g$, por lo que $R_f \neq R_g$.
\end{proof}

Las proposiciones anteriores permiten definir, de forma explícita y sin recurrir al $\Ac$, el siguiente tipo de familias casi ajenas.

\begin{definicion}\label{def-FamRamas}\index[sym]{$\ms{A}_X$}\index[alph]{familia!de!ramas de $X$ en $2^\omega$}
	Para cada $X \subseteq 2^\omega$ defínase $\ms{A}_X:=\{A_f \tq f \in X\}$ como en la Proposición previa.

	Esta familia será nombrada la \textbf{familia de las ramas de $X$ en $2^{<\omega}$}.
\end{definicion}

En paralelo a lo comentado después de \ref{def-FamSucesiones}, también se puede concluir vía la construcción recién expuesta, y el \autoref{lem-MADs}, lo siguiente:

\begin{corolario}\label{cor-famGrandes}
	Existe una familia maximal de cardinalidad $\mathfrak{c}$.

	En consecuencia, para cualquier cardinal $\lambda \leq \mathfrak{c}$ existe una familia casi ajena de cardinalidad $\lambda$.
\end{corolario}

Se concluirá esta sección comentando qué ocurre respecto una interrogante surge naturalmente tras todo lo realizado: ¿existen familias maximales de cualquier cardinalidad entre $\aleph_1$ y $\mathfrak{c}$?

\begin{definicion}\index[alph]{cardinal! de casi ajenidad}\index[alph]{casi!ajenidad, cardinal de}\index[sym]{$\mathfrak{a}$}
	El \textbf{cardinal de casi ajenidad}, $\mathfrak{a}$, es el mínimo cardinal infinito $\kappa$ tal que existe una familia maximal de tamaño $\kappa$.
\end{definicion}

Debido a \ref{prop-MADnoNum}, se tiene $\aleph_1 \leq \mathfrak{a} \leq \mathfrak{c}$ y claramente bajo $\HC$ se debe satisfacer que $\mathfrak{a}=\mathfrak{c}$; luego, es consistente con $\zfc$ que $\mathfrak{a}=\mathfrak{c}$. Comentar que la teoría en relación al cardinal $\mathfrak{a}$ es basta y existen resultados de consistencia como el enunciado en seguida:

\begin{teorema}\label{teo-stafa}
	Si $\aleph_1 \leq \kappa \leq \mathfrak{c}$ y $\kappa$ es cardinal regular, es consistente con $\zfc$ que $\mathfrak{a}=\kappa$.
\end{teorema}

El Teorema recién enunciado consecuencia de \cite[Teo.~5.1, p.~127]{kunenHandbook}; y si bien su demostración escapa a los propósitos del presente texto, sí se expondrán resultados en relación a la igualdad $\mathfrak{a}=\mathfrak{c}$ más adelante (véase \ref{cor-MaSimple}).

\section{El ideal generado y su comportamiento}
\label{Sec-IdealGenerado}
\index[alph]{ideal generado por $\ms{A}$}\index[sym]{$\ms{I}_N(\ms{A})$}\index[alph]{parte!positiva de $\ms{A}$}\index[sym]{$\ms{I}_N^+(\ms{A})$}\index[sym]{$\ms{I}(\ms{A})$}\index[sym]{$\ms{I}^+(\ms{A})$}
\begin{definicion}\label{def-ideal}
	Sean $N$ un conjunto numerable y $\ms{A} \in \Ad(N)$.
	\begin{enumerate}[i)]
		\item El \textbf{ideal generado por $\ms{A}$} es el conjunto $\ms{I}_N(\ms{A})$; definido como la colección $\{ B \subseteq N \tq \exists H \in [\ms{A}]^{<\omega} \: ( B \subseteq^* \midcup H ) \} $.
		\item La \textbf{parte positiva de $\ms{A}$} es $ \ms{I}_N^+(\ms{A}) := \ms{P}(N) \setminus \ms{I}_N(\ms{A})$.
	\end{enumerate}
	Si $N=\omega$, estos conjuntos se denotarán por $\ms{I}(\ms{A})$ y $\ms{I}^+(\ms{A})$, respectivamente.
\end{definicion}

El objeto introducido previamente es de vital importancia para el estudio de la combinatoria de las familias casi ajenas. Como se había advertido anteriormente; con el propósito de no perder generalidad en los resultados expuestos durante esta sección, es necesario notar lo siguiente:

\begin{proposicion}\label{prop-IdealBiyec}
	Sean $N,M$ conjuntos numerables y $h:N \to M$ biyectiva. Si $\ms{A} \in \Ad(N)$, entonces $\Phi_h (\ms{I}_N (\ms{A})) = \ms{I}_M (\Phi_h( \ms{A} )) $.
\end{proposicion}

\begin{proof}
	Como $h$ es biyección, $\Phi_h^{-1} = \Phi_{h^{-1}}$. Por lo cual, basta probar la contención directa de la igualdad.
	
	Sea $Y \in \ms{I}_N (\ms{A})$, entonces existe $H \subseteq \ms{A}$ finito tal que $Y \setminus \midcup H$ es finito. Como $h$ es biyectiva, $ h \big[ Y \setminus \midcup H \big] = h[Y] \setminus h\big[ \midcup H \big] = h[Y] \setminus \midcup \Phi_h(H)$ es finito. Además, de \ref{prop-ADbiyec}, $\Phi_h(H) \subseteq \Phi_h(\ms{A})$ es finito. Por ello, $h[Y] \in \ms{I}_M (\Phi_h( \ms{A}))$.
\end{proof}

Resulta sencillo constatar que el objeto definido en \ref{def-ideal} es; como su nombre indica, un ideal (no necesariamente propio) sobre $\ms{P}(\omega)$. Además, se destaca lo siguiente:

\begin{observacion}\label{obs-IdealPrevia}
	Sea $\ms{A}$ cualquier familia casi ajena.
	\begin{enumerate}[i)]
		\item Cualquier subconjunto finito de $\omega$, así como cualquier elemento de $\ms{A}$, es elemento de $\ms{I}(\ms{A})$. Por lo que $\emptyset \subsetneq [\omega]^{<\omega} \cup \ms{A} \subseteq \ms{I}(\ms{A})$.
		\item Si $\ms{B} \in \Ad(\omega)$ y $\ms{A} \subseteq \ms{B}$, entonces $\ms{I}(\ms{A}) \subseteq \ms{I}(\ms{B})$.
	\end{enumerate}
\end{observacion}

Obsérvese adempás que si $\ms{A} \in \Mad(\omega)$ es finita; en virtud del \autoref{lem-MADnecesarioUnion} se tendrá que $\omega \in \ms{I}(\ms{A})$, pues $\ms{A} \subseteq \ms{A}$ es finito y $\omega \subseteq^* \midcup \ms{A}$. El recíproco de esto también es cierto.

\begin{proposicion}
	Sea $\ms{A} \in \Ad(\omega)$ cualquiera. Si $\omega \in \ms{I}(\ms{A})$, entonces $\ms{A}$ es maximal y finita.
\end{proposicion}

\begin{proof}
	Supóngase que $\omega \in \ms{I}(\ms{A})$, entonces existe $H \subseteq \ms{A}$ finito con $\omega \subseteq^* \midcup H \subseteq \midcup \ms{A}$. Por \ref{cor-MADnecesarioUnion}, basta ver que $\ms{A}$ es finita.

	Supóngase que existe $B \in \ms{A} \setminus H$. Así, $B$ casi ajeno con cada elemento de $H \subseteq \ms{A}$, luego, $B \cap \midcup H = \midcup \{B \cap h \tq h \in H \}$ es finito. De lo anterior, $B = B \cap \omega \subseteq^* B \cap \midcup H$ es finito, lo cual es imposible. Por lo tanto, $\ms{A} \subseteq H$ y $\ms{A}$ es finita.
\end{proof}

\begin{corolario}\label{cor-IdealPropioCaract}
	Sean $N$ un conjunto numerable y $\ms{A} \in \Ad(N)$. Las siguientes condiciones son equivalentes:
	\begin{enumerate}[i)]
		\item El ideal $\ms{I}_N(\ms{A})$ no es propio, es decir, $N \in \ms{I}_N(\ms{A})$.
		\item $\ms{A}$ es finita y maximal en $N$.
	\end{enumerate}
\end{corolario}

Con relativa frecuencia aparecerán familias que, pese a no ser maximales, satisfacen la condición (ii) de lo subsecuente; esta puede ser tomada como un debilitamiento a la condición de maximalidad.

\begin{definicion}\label{def-MaxEnAlguna}\index[alph]{traza de $\ms{A}$ en $X$}\index[sym]{$\ms{A} \upharpoonright X$}\index[alph]{familia!maximal en alguna parte}\index[alph]{familia!maximal en ninguna parte}
	Sean $N$ un conjunto numerable y $\ms{A} \in \Ad(N)$.
	\begin{enumerate}[i)]
		\item Para cada $X \in [N]^\omega$, la \textbf{traza de $\ms{A}$ en $X$} es la colección $ \ms{A} \upharpoonright X$, definida como el conjunto $\{ A \cap X \in [X]^\omega \tq A \in \ms{A} \} $.
		\item $\ms{A}$ es \textbf{maximal en alguna parte} si y sólo si existe $X \in \ms{I}_N^+(\ms{A})$ tal que la familia $\ms{A} \upharpoonright X$ es maximal en $X$.
		\item $\ms{A}$ es \textbf{maximal en ninguna parte} si no es maximal en alguna parte.
	\end{enumerate}
\end{definicion}

Si $X \in [N]^\omega$, entonces $\ms{A} \upharpoonright X \in \Ad(X)$. Más aún, si $\ms{A}$ es maximal, para cada $B \subseteq X$ infinito existe $A \in \ms{A}$ tal que $A \cap B = (A \cap X) \cap (B \cap X)$ es infinito, mostrando la maximalidad de $\ms{A} \upharpoonright X$. Así que, efectivamente, la definición anterior es un debilitamiento de la condición de maximalidad.

Sin causa de asombro, los conceptos recién establecidos son respetados por las biyecciones $\Phi_h$.

\begin{proposicion}
	Sean $N,M$ conjuntos numerables, $h:N \to M$ biyección y $\ms{A} \in \Ad(N)$.
	\begin{enumerate}[i)]
		\item Para cada $X \in [N]^\omega$ se cumple $\Phi_h(\ms{A}) \upharpoonright h[X]=\Phi_h(\ms{A} \upharpoonright X)$.
		\item $\ms{A}$ es maximal en alguna parte si y sólo si $\Phi_h(\ms{A})$ también lo es.
	\end{enumerate}
\end{proposicion}

\begin{proof}
	(i) Como $h$ biyección, de la definición de $\Phi_h(\ms{A})$ se obtiene:
	\begin{align*}
		\Phi_h(\ms{A}) \upharpoonright h[X] & = \{ B \cap h[X] \in \big[ h[X] \big]^\omega \tq B \in \Phi_h(\ms{A}) \} \\
		                                    & = \{ h[A] \cap h[X] \in \big[ h[X] \big]^\omega \tq A \in \ms{A} \} \\
											& = \{ h[A \cap X] \tq A \cap X  \in [X]^\omega \land A \in \ms{A} \} \\
		                                    & = \Phi_h (\ms{A} \upharpoonright X) \, .
	\end{align*}
	(ii) Como $\Phi_h ^{-1} = \Phi_{h^{-1}}$, basta probar la suficiencia. Supóngase que $\ms{A}$ es maximal en alguna parte, entonces existe $X \in \ms{I}^+(\ms{A})$ tal que $\ms{A} \upharpoonright X$ es maximal en $X$. 
	
	Dada la igualdad de \ref{prop-IdealBiyec}, $h[X] \in \ms{I}^+(\Phi_h(\ms{A}))$; además, la función de restricción $g:= h \upharpoonright X : X \to h[X]$ es biyección y utilizando el inciso anterior se tiene que $\Phi_h(\ms{A}) \upharpoonright h[X]=\Phi_h (\ms{A} \upharpoonright X)=\Phi_g (\ms{A} \upharpoonright X)$. Por el último inciso de \ref{prop-ADbiyec}, $\Phi_h(\ms{A}) \upharpoonright h[X]$ es maximal en $h[X]$.
\end{proof}

Las siguientes propiedades son esperables, pero no por ello menos útiles. Es importante no desdeñar sus pruebas, pues en ellas, hay un par de sutilezas que deben ser atendidas.

\begin{proposicion}\label{prop-TrazaBasicos}
	Sean $\ms{A},\ms{B} \in \Ad(\omega)$ y $X,Y \in [\omega]^\omega$ arbitrarios.
	\begin{enumerate}[i)]
		\item Si $\ms{A} \subseteq \ms{B}$, entonces $\ms{A} \upharpoonright X \subseteq \ms{B} \upharpoonright X$.
		\item Se da la igualdad $(\ms{A} \upharpoonright Y) \upharpoonright X = \ms{A} \upharpoonright (Y \cap X)$.
		\item Si $X \subseteq Y$, entonces $\ms{I}_X(\ms{A} \upharpoonright X) \subseteq \ms{I}_Y(\ms{A} \upharpoonright Y)$.
	\end{enumerate}
\end{proposicion}
\begin{proof}
	El punto (i) es claro.

	(ii) Si $(A \cap Y) \cap X \in (\ms{A} \upharpoonright Y) \upharpoonright X$, entonces $A \cap Y \in \ms{A} \upharpoonright Y$ y $(A \cap Y) \cap X = A \cap (Y \cap X)$ es infinito; luego, $A \in \ms{A}$ y $(A \cap Y) \cap X \in \ms{A} \upharpoonright (Y \cap X)$. Recíprocamente, si $A \cap (Y \cap X) \in \ms{A} \upharpoonright (Y \cap X)$, entonces $A \in \ms{A}$ y $A \cap (Y \cap X) = (A \cap Y) \cap X$ es infinito. Además, $A \cap (Y \cap X) \subseteq A \cap Y$, entonces $A \cap Y$ es infinito, por lo que $A \cap Y \in \ms{A} \upharpoonright Y$ y así $A \cap (Y \cap X) = (A \cap Y) \cap X \in (\ms{A} \upharpoonright Y) \upharpoonright X$.

	(iii) Supóngase que $X \subseteq Y$ y sea $B \in \ms{I}_X(\ms{A} \upharpoonright X)$. Entonces $B \subseteq X \subseteq Y$ y existe $H \subseteq \ms{A} \upharpoonright X$ finito tal que $B \subseteq^* \midcup H$. Cada elemento $A \cap X \in H$ es infinito y está contenido en $A \cap Y$, luego $J:=\{A \cap Y \tq A \cap X \in H\} \subseteq \ms{A} \upharpoonright Y$ es finito y $B \subseteq^* \midcup J$ y por lo tanto $B \in \ms{I}_Y(\ms{A} \upharpoonright Y)$.
\end{proof}

Si bien se demostró, tras la \autoref{def-MaxEnAlguna}, que la maximalidad de una familia implica la de cada una de sus trazas. La próxima observación muestra que la maximalidad de aquellas trazas tomadas sobre elementos del ideal generado no constituye una condición suficiente para concluir la maximalidad de la familia dada.

\begin{proposicion}\label{prop-CaracMADIdeal}
	Sean $\ms{A} \in \Ad(\omega)$ y $X \in [\omega]^\omega$. Entonces $X \in \ms{I}(\ms{A})$ si y sólo si la familia $\ms{A} \upharpoonright X$ es finita y maximal en $X$.
\end{proposicion}

\begin{proof}
	Por el \autoref{cor-IdealPropioCaract}, basta mostrar que $X \in \ms{I}(\ms{A})$ si y sólo si $X \in \ms{I}_X(\ms{A} \upharpoonright X)$. Dado lo demostrado en \ref{prop-TrazaBasicos}, la necesidad de tal equivalencia es inmediata. Ahora, si $X \in \ms{I}(\ms{A})$, entonces $X \subseteq^* \midcup H$ para cierto $H \in [\ms{A}^{<\omega}]$. Luego $X \subseteq^* X \cap \midcup H$ y la finitud de $H$ implica que:
	\begin{align*}
		X & \subseteq^* \{ A \cap X \in [X]^{<\omega} \tq A \in H \} \cup \{ A \cap X \in [X]^\omega \tq A \in H \} \\
		& =^* \{ A \cap X \in [X]^\omega \tq A \in H \} \, ,
	\end{align*}
	probando que $X \in \ms{I}_X(\ms{A} \upharpoonright X)$.
\end{proof}

Como la familia vacía no es maximal, la Proposición anterior deja como reflexión que los únicos conjuntos que podrían ser testigos de que $\ms{A}$ no es maximal, son aquellos que pertenecen a la parte positiva de $\ms{A}$.

\begin{corolario}\label{cor-CasiajenoPartePositiva}
	Si $\ms{A} \in \Ad(\omega)$, $\{X \in [\omega]^{\omega} \tq \ms{A} \upharpoonright X= \emptyset \} \subseteq \ms{I}^+(\ms{A})$.
\end{corolario}

Si $X \in [\omega]^\omega$ y $\ms{A}$ es maximal, entonces $X$ debe tener intersección infinita con algún elemento de $\ms{A}$; los conjuntos en $\ms{I}^+(\ms{{A}})$ tienen un comportamiento más fuerte, cada uno de ellos tiene intersección infinita con una infinidad de elementos de $\ms{A}$.

\begin{corolario}\label{prop-CaracMADPositiv}
	Sea $\ms{A} \in \Ad(\omega)$. Entonces $\ms{A}$ es maximal si y sólo si para cada $X \in \ms{I}^+(\ms{A})$ la familia $\ms{A} \upharpoonright X$ es infinita.
\end{corolario}
\begin{proof}
	Supóngase que $\ms{A}$ es maximal y sea $X \in \ms{I}^+(\ms{A})$. Por la maximalidad de $\ms{A}$, se sigue de lo comentado tras la \autoref{def-MaxEnAlguna} que $\ms{A} \upharpoonright X$ es maximal en $X$. Por lo que, acorde a \ref{prop-CaracMADIdeal}, $\ms{A} \upharpoonright X$ debe ser infinita.

	Recíprocamente, si $\ms{A}$ no es maximal, existe $B \in [\omega]^\omega$ casi ajeno con cada elemento de $\ms{A}$. Así, $\ms{A} \upharpoonright B = \emptyset$ y se sigue de \ref{cor-CasiajenoPartePositiva}, que $B \in \ms{I}^+(\ms{A})$.
\end{proof}

Como ninguna familia maximal es numerable, del Corolario anterior se desprende la siguiente condensación de todo lo comentado y demostrado tras \ref{def-MaxEnAlguna}.

\begin{corolario}\label{cor-MADPositivCarac}
	Sean $N$ un conjunto numerable y $\ms{A} \in \Ad(N)$. Entonces ocurre $\{ X \in [N]^\omega \tq \ms{A} \upharpoonright X \in \Mad(X) \land \omega \leq |\ms{A} \upharpoonright X|\} \subseteq \ms{I}_N^+(\ms{A})$. La contención recíproca ocurre si y sólo si $\ms{A}$ es maximal, y en tal caso:
	\[ \{ X \in [N]^\omega \tq \omega < |\ms{A} \upharpoonright X| \} = \ms{I}_N^+(\ms{A}) \, . \]
\end{corolario}

\section{Resultados en combinatoria infinita}

\subsection{Teorema de Simon}\label{subsec-Simon}
\label{Sec-TeoSimon}

Durante esta subsección del escrito se abordará un análisis simple sobre la combinatoria inherente a las familias casi ajenas. Comenzaremos configurado la antesala para enunciar el primer resultado importante de la sección, el Teorema de Simon (\ref{Teo-Simon}).

\begin{lema}
	Sea $(X_n)_{n\in \omega} \subseteq [\omega]^\omega$ una sucesión $\subseteq$-decreciente. Entonces existe $Y \in [X_0]^\omega$ tal que para cada $k \in \omega$, ocurre $Y \subseteq^* X_k$.
\end{lema}
\begin{proof}
	Considérese la función $f:\omega \to X_0$, dada recursivamente como $f(n) = \min (X_n \setminus f[n])$. Nótese que $f$ es inyectiva, si $m < n$ entonces $f(n) \notin f[n]$ y particularmente $f(n) \neq f(m)$. Consecuentemente, $Y:= f[\omega] \in [X_0]^\omega$.

	Si $k \in \omega$, cada $x=f(n) \in Y \setminus X_k$ debe satisfacer que $n<k$, dada la hipótesis sobre $(X_n)_{n \in \omega}$. Esto prueba que $Y \setminus X_k \subseteq f[k] =^* \emptyset$; es decir, $Y \subseteq^* X_k$.
\end{proof}

Una posible interpretación al siguiente resultado es que, la parte positiva de cualquier familia maximal contiene tantos elementos como para acotar inferiormente a cualquier cadena numerable dentro de ella.
\begin{lema}[Dočkálková]\index[alph]{Dočkálková!Lema de}\index[alph]{Lema!de Dočkálková}\label{lem-PositivCadenaDecreciente}
	Sean $\ms{A} \in \Mad(\omega)$ y $(X_n)_{n\in\omega} \subseteq \ms{I}^+(\ms{A})$ una sucesión $\subseteq$-decreciente. Existe $Y \in \ms{I}^+(\ms{A})$ tal que si $n\in \omega$, entonces $Y \subseteq^* X_n$.
\end{lema}

\begin{proof}
	Por el \autoref{cor-MADPositivCarac}, si $n \in \omega$, $H_n:=\{ A \in \ms{A} \tq A \cap X_n \neq^* \emptyset \}$ debe ser infinito. Para cada $n \in \omega$ fijese ($\Ac$) una función $f_n : \ms{P}(H_n) \setminus \{\emptyset\} \to H_n$ de elección; y defínase recursivamente $B:\omega \to \ms{A}$ como $B(n) = f(H_n \setminus B[n] )$.

	Así, $(\midcup\{B(m) \tq m\geq n \} \cap X_n)_{n \in \omega}$ satisface las hipótesis del Lema previo. Efectivamente; si $n \in \omega$, entonces $B(n) \in H_n$ y $\midcup\{B(m) \tq m\geq n \} \cap X_n \supseteq B(n) \cap X_n$ es infinito. Esta sucesión es $\subseteq$-decreciente, debido a que $(X_n)_n \in \omega$ también lo es. Por lo tanto, existe $Y \in [\omega]^\omega$ de manera que para cada $k \in \omega$:
	\[ Y \subseteq^* \midcup\{B(m) \tq m\leq k \} \cap X_n \subseteq B(k) \cap X_k \subseteq X_k \, , \]
	al ser $Y$ infinito, se tiene que $Y \cap B(k) \supseteq X \cap (B(k) \cap X_k )$ es infinito. Por lo tanto $\ms{A} \upharpoonright Y$ es infinita; siguiéndose del \autoref{cor-MADPositivCarac} que $Y \in \ms{I}^+(\ms{A})$.
\end{proof}

El siguiente resultado fue demostrado en 1980 por Petr Simon \cite[p.~751]{SimonFrechet} y como se verá más adelante (\autoref{cor-FrechNoProd}), juega un papel fundamental en el estudio de la propiedad topológica de Fréchet.

\begin{teorema}[Simon]\index[alph]{Simon!Teorema de}\index[alph]{Teorema!de Simon}\label{Teo-Simon}
	Para toda familia maximal e infinita $\ms{A}$ existe un elemento $X \in \ms{I}^+(\ms{A})$ tal que $\mathscr{A} \upharpoonright X$ es unión ajena de dos familias maximales en ninguna parte.
\end{teorema}
\begin{proof}
	Procédase por contradicción supongiendo que $\ms{A} \in \Mad(\omega)$ es tal que para cada $X \in \ms{I}^+(\ms{A})$, si $\ms{A} \upharpoonright X = \ms{B} \cup \ms{C}$ y $\ms{B} \cap \ms{C} = \emptyset$, entonces $\ms{B}$ es maximal en alguna parte o $\ms{C}$ es maximal en alguna parte.

	Como $|\ms{A}| \leq |2^\omega|$, existe $F \subseteq 2^\omega$ tal que $\ms{A}=\{A_f \tq f \in F\}$; donde $f \neq g$ implica $A_f \neq A_g$. Para cada $(n,k) \in \omega \times 2$ defínase:
	\[ \ms{A}(n,k)=\{A_f \in \ms{A} \tq f(n)=k \} \, . \]
	
	Nótese que para cualesquiera $m \in \omega$ y $X \in \ms{I}^+(\ms{A})$ ocurren simultáneamente $\ms{A} \upharpoonright X = \ms{A}(m,0) \upharpoonright X \cup \ms{A}(m,1) \upharpoonright X$ y $\ms{A}(m,0) \upharpoonright X \cap \ms{A}(m,1) \upharpoonright X = \emptyset$. Se sigue de la hipótesis la existencia $(\Ac)$ de una función $k:\omega \times \ms{I}^+(\ms{A}) \to 2$ tal que si $(m,X) \in \dom(k)$, entonces $\ms{A}(m,k(m,X)) \upharpoonright X$ es maximal en alguna parte. Con ello, existe una función $j:\omega \times \ms{I}^+(\ms{A}) \to \ms{I}^+_X(\ms{A}(m,k(m,X)) \upharpoonright X)$ de manera que para cada $(m,X) \in \dom(j)$:
	\[ (\ms{A}(m,k(m,X)) \upharpoonright X) \upharpoonright j(m,X) = \ms{A}(m,k(m,X)) \upharpoonright j(m,X) \in \Mad(j(m,X)) \,. \]

	Defínase recursivamente la sucesión $(X_n)_{n \in \omega} \subseteq \ms{I}^+(\ms{A})$ como $X_0=\omega$; y para cada $n \in \omega$, $X_{n+1}=j(n,X_n)$. Para cada $n \in \omega$, por las propiedades de monotonía expuestas en \ref{prop-TrazaBasicos}, $X_{n+1} \in \ms{I}_{X_n}^+( \ms{A}(n,k(n,X_n)) \upharpoonright X_n ) \subseteq \ms{I}^+(\ms{A})$; y, por la definición de $j$:
	\[ \ms{A}(n,k(n,X_n)) \upharpoonright X_{n+1} \in \Mad(X_{n+1}) \, ; \]	
	a consecuencia del ello $X_{n+1} \subseteq X_n$. Se obtiene, dado el Lema de Dočkálková, un conjunto $Y \in \ms{I}^+ (\ms{A})$ casi contenido en cada $X_n$.

	Como $F$ es infinito, existen $g \in F$ y $m \in \omega$ tales que $f(m) \neq k(m,X_m)$. Además $Y \setminus X_{m+1} =^* \emptyset$ y $Y \cap A_g \neq^* \emptyset$, luego $A_g \cap X_{m+1} \subseteq X_m$ tiene que ser infinito. Puesto que $\ms{A}(m,k(m,X_m)) \upharpoonright X_{m+1} \in \Mad(X_{m+1})$, existe $A_f \in \ms{A}(m,k(m,X_m))$ tal que $(A_g \cap X_{m+1}) \cap (A_f \cap X_{m+1}) \neq ^* \emptyset$. Sin embargo, lo anterior conduce a una contradicción, pues $f(m)=k(m,X_m) \neq g(m)$ implica que $f \neq g$, y esto a su vez, que $A_f \cap A_g =^* \emptyset$, por ser $\ms{A}$ familia casi ajena.
\end{proof}

\newpage
\begin{proof}
	Por contradicción, supóngase que $\ms{A}$ es una familia maximal infinita la cual, sin perder generalidad, la podemos suponer definida sobre $\omega$, tal que para cada $X \in \ms{I}^+(\ms{A})$ o bien $\ms{A} \upharpoonright X$ no es maximal en $X$, o bien, si $\ms{A} \upharpoonright X$ es unión ajena de $\ms{B}$ y $\ms{C}$, entonces $\ms{B}$ o $\ms{C}$ es maximal en alguna parte.

	Como $|\ms{A}| \leq \mathfrak{c}$, existe $F \subseteq 2^\omega$ tal que $\ms{A}$ se puede enumerar inyectivamente como $\ms{A}=\{A_f \tq f \in F\}$. Dados $n \in \omega$ y $k \in 2$, defínase:
	$$ \ms{A}(n,k)=\{A_f \in \ms{A} \tq f(n)=k \} $$
	y nótese que para todo $m$ natural, $\ms{A}$ unión ajena de $\ms{A}(m,0)$ y $\ms{A}(m,1)$.

	Constrúyanse; por recursión en $\omega$, las sucesiones $(X_n)_{n\in\omega} \subseteq \ms{I}^+(\ms{A})$ y $(k_n)_{n\in \omega} \subseteq 2$, tales que si $n \in \omega$, se da $X_{n+1} \in \ms{I}_{X_n}^+ ( \ms{A}(0,k_n) \upharpoonright X_n )$ y $\ms{A}(n,k_n) \upharpoonright X_{n+1} \in \Mad(X_{n+1})$.

	Como $\ms{A}$ es familia maximal e infinita, defínase $X_0:=\omega \in \ms{I}^+(\ms{A})$ (véase \ref{cor-IdealPropioCaract}); así que $\ms{A} = \ms{A} \upharpoonright X_0$ es maximal sobre $X_0$. Como $\ms{A}$ es unión ajena de $\ms{A}(0,0)$ y $\ms{A}(0,1)$, se sigue de la hipótesis la existencia de un elemento $k_0 \in 2$ tal que $\ms{A}(0,k_0)$ es maximal en ninguna parte. Con $\Ac$, fíjese un elemento $X_1 \in \ms{I}^+(\ms{A}(0,k_0)) = \ms{I}_{X_0}^+(\ms{A}(0,k_0) \upharpoonright X_0)$ de forma tal que $(\ms{A}(0,k_0) \upharpoonright X_0) \upharpoonright X_1 = \ms{A}(0,k_0) \upharpoonright X_1$ sea maximal en $X_1$. Como $X_1 \notin \ms{I}_{X_0}(\ms{A}(0,k_0) \upharpoonright X_0)$ y $X_1 \subseteq X_0$, se desprende de \autoref{prop-TrazaBasicos} que $X_1 \in \ms{I}_{X_1}^+ (\ms{A}(0,k_0) \upharpoonright X_1)$, por lo que $\ms{A}(0,k_0) \upharpoonright X_1$ es infinita, en virtud de su maximalidad y del \autoref{cor-IdealPropioCaract}. Así, $\ms{A} \upharpoonright X_1$ es infinita, lo cual implica que $X_1 \in \ms{I}^+(\ms{A})$, pues $\ms{A}$ es maximal (véase \ref{cor-MADPositivCarac}).

	Supóngase ahora que $n \in \omega$ y que $X_n, X_{n+1} \in \ms{I}^+(\ms{A})$ y $k_n \in 2$ son tales que $X_{n+1} \in \ms{I}_{X_n}^+ ( \ms{A}(n,k_n) \upharpoonright X_n )$ de modo que $\ms{A}(n,k_n) \upharpoonright X_{n+1}$ es maximal en $X_{n+1}$. Como $\ms{A}$ unión ajena de $\ms{A}(n+1,0)$ y $\ms{A}(n+1,1)$, $\ms{A} \upharpoonright X_{n+1}$ es unión ajena de $\ms{A}(n+1,0) \upharpoonright X_{n+1}$ y $\ms{A}(n+1,1) \upharpoonright X_{n+1}$. De nuevo, con $\Ac$ fíjense $k_{n+1}\in 2$ y $X_{n+2} \in \ms{I}_{X_{n+1}}^+ ( \ms{A}(n+1,k_{n+1}) \upharpoonright X_{n+1} )$ de modo tal que $\ms{A}(n+1,k_{n+1}) \upharpoonright X_{n+2} \in \Mad(X_{n+2})$. Al igual que antes, se obtiene de \ref{cor-IdealPropioCaract} y \ref{cor-MADPositivCarac}, que $X_{n+2} \in \ms{I}^+(\ms{A})$, lo que finaliza la construcción recursiva.

	Por construcción, $(X_n)_{n\in \omega} \subseteq \ms{I}^+(\ms{A})$ es $\subseteq$-decreciente, por lo que del Lema de Dočkálková, existe un conjunto $Y \in \ms{I}^+(\ms{A})$ tal que para cada $n \in \omega$ se cumple $Y \subseteq^* X_n$. Puesto que $Y \in \ms{I}^+(\ms{A})$ y $\ms{A}$ es maximal, de \ref{cor-MADPositivCarac} se tiene que $\ms{A} \upharpoonright Y$ es infinita, y con ello, existe $g \in F \setminus \{(k_n)_{n \in \omega}\} \subseteq 2^\omega$ tal que $A_g \cap Y$ es infinito. Siendo $k$ distinta de $g$, hay un natural $m$ tal que $k_m \neq g(m)$.

	Como $Y \subseteq^* X_{m+1}$, entonces $Y \setminus X_{m+1}$ es finito. Luego, derivado de que $Y \cap A_g$ es infinito, se obtiene que $A_g \cap X_{m+1} \subseteq X_{m+1}$ es infinito. Así, por la maximalidad de $\ms{A}(m,k_m) \upharpoonright X_{m+1}$ en $X_{m+1}$ se obtiene un $A_f \in \ms{A}(m,k_m)$ tal que $(A_g \cap X_{m+1}) \cap (A_f \cap X_{m+1})$ es infinito. Sin embargo, lo anterior conduce a una contradicción, pues $f(m)=k_m \neq g(m)$ implica que $f \neq g$, y esto a su vez, que $A_f \cap A_g$ es finito por ser $\ms{A}$ familia casi ajena.
\end{proof}

\begin{corolario}
	Existe una familia maximal de tamaño $\mathfrak{c}$ que es unión ajena de dos familias maximales en ninguna parte.
\end{corolario}

\subsection{Familias y grietas de Luzin}
\label{Sec-Luzin}

\begin{definicion}\label{Def-particionador}\label{def-grieta}\index[alph]{particionador}\index[alph]{grieta}\index[alph]{grieta!separada}\index[alph]{grieta!contenida en una familia}\index[alph]{familia!que contiene a una grieta}
	Sea $N$ un conjunto numerable y $\ms{A},\ms{B} \in \Ad(N)$.
	\begin{enumerate}
		\item El par $(\ms{A},\ms{B})$ es una \textbf{grieta} si y sólo si $\ms{A} \cap \ms{B}=\emptyset$ y $\ms{A} \cup \ms{B} \in \Ad(N)$. Se suele decir que $(\ms{A},\ms{B})$ \textbf{está contenida} en $\ms{A} \cup \ms{B}$, o que $\ms{A} \cup \ms{B}$ \textbf{contiene a} $(\ms{A},\ms{B})$.
		\item Un subconjunto $D \subseteq N$ es \textbf{particionador} de $\ms{A}$ y $\ms{B}$ si y sólo si para cada $A \in \ms{A}$ y $B \in \ms{B}$ se tiene $A \subseteq^* D$ y $B \cap D =^* \emptyset$.
		\item Una grieta $(\ms{A},\ms{B})$ está \textbf{separada} si y sólo si existe un particionador de $\ms{A}$ y $\ms{B}$.
	\end{enumerate}
\end{definicion}

En términos de la definición anterior, no resulta complicado notar que $D$ es particionador de $\ms{A}$ y $\ms{B}$ si y sólo si $N \setminus D$ es particionador de $\ms{B}$ y $\ms{A}$. Además, en tal caso $\ms{A}=\{X \in \ms{A} \cup \ms{B} \tq X \subseteq ^* D\}$ y $\ms{B}=\{X \in \ms{A} \cup \ms{B} \tq X \cap D =^* \emptyset\}$.

Como ha resultado ser rutina a lo largo de todo el capítulo, se deberá hacer hincapié en el comportamiento de las grietas respecto a las biyecciones $\Phi_h$. La demostración de este hecho resulta estándar.

\begin{observacion}\label{prop-grietasBiyec}
	Sean $N$ y $M$ conjuntos numerables. Para toda biyección $h:N \to M$ y toda grieta $(\ms{A},\ms{B})$ en $N$ se cumple:
	\begin{enumerate}
		\item $(\Phi_h(\ms{A}),\Phi_h(\ms{B}))$ es grieta en $M$.
		\item $(\ms{A},\ms{B})$ está separada si y sólo si $(\Phi_h(\ms{A}),\Phi_h(\ms{B}))$ está separada.
	\end{enumerate}
\end{observacion}

En virtud de lo anterior, cada vez que $(\ms{A},\ms{B})$ sea grieta; y salvo que se diga lo contrario, se dará por sentado que $\ms{A},\ms{B} \in \Ad(\omega)$

\begin{observacion}\label{obs-GrietasSimple}
	Para cualesquiera grietas $(\ms{A},\ms{B})$ y $(\ms{A}',\ms{B}')$:
	\begin{enumerate}
		\item $\ms{A} \subseteq \ms{I}^+(\ms{B})$ y $\ms{B} \subseteq \ms{I}^+(\ms{A})$ (seguido de la \autoref{obs-IdealPrevia}).
		\item $(\ms{A},\ms{B})$ está separada si y sólo si $(\ms{B},\ms{A})$ está separada.
		\item Si $(\ms{A}',\ms{B}')$ está separada, $\ms{A} \subseteq \ms{A}'$ y $\ms{B} \subseteq \ms{B}'$, entonces $(\ms{A},\ms{B})$ está separada.
	\end{enumerate}
\end{observacion}

A continuación se dan dos hechos básicos sobre la separación de grietas; y sin bien se podrían exponer las correspondientes demostraciones disponiendo solamente de la artillería dada hasta el momento, se dejarán a modo de corolario de la teoría resultante de los $\Psi$-espacios (véase \ref{col-tra-interrelacion}).

\begin{ejemplo}\label{ej-interrelacion}
	Sea $\ms{C}\in \Ad(\omega)$, entonces:
	\begin{enumerate}
		\item Si $|\ms{C}|\leq \aleph_0$, entonces cualquier grieta contenida en $\ms{C}$ está separada.
		\item Si $\ms{C}$ es inifnita y, $|\ms{C}|=\mathfrak{c}$ o $\ms{C}\in \Mad(\omega)$; entonces $\ms{C}$ contiene una grieta que no está separada.
	\end{enumerate}
\end{ejemplo}

El siguiente tipo de familias poseen virtudes que las convierten en objetos de suma relevancia para la teoria  de conjuntos.

\begin{definicion}\label{def-LuzinFam}\index[alph]{familia!de Luzin}\index[alph]{Luzin!familia de}
	Una familia casi ajena $\ms{A}$ es \textbf{de Luzin} si existe $X \in [\omega_1]^{\omega_1}$ de forma que $\ms{A}$ se puede enumerar como $\{A_\alpha \tq \alpha \in X \}$ y se cumple que para todo $(\alpha, n) \in \omega_1 \times \omega$, el conjunto $ \{ \beta<\alpha \tq A_\alpha \cap A_\beta \subseteq n \} $ es finito.
\end{definicion}

Una de las ideas centrales detrás de que $\ms{A}=\{A_\alpha \tq \alpha \in \omega_1 \}$ sea de Luzin es que; fijando $\alpha \in \omega_1$, para cada $D \subseteq \alpha$ infinito, $A_\alpha \cap \midcup\{A_\beta \tq \beta \in D\}$ es infinito. Esto se debe a que si $n \in \omega$, entonces $D \setminus \{ \beta<\alpha \tq A_\alpha \cap A_\beta \subseteq n \} $ es infinito, particularmente no vacío.

\begin{proposicion}\label{pro-LuzinExisten}
	Toda familia casi ajena numerable se extiende a una familia de Luzin. Particularmente, existe una familia de Luzin.
\end{proposicion}
\begin{proof}
	Sea $\ms{B}=\{A_n \tq n \in \omega\}$ cualquier familia casi ajena numerable y nóntese que claramente para cualesquiera $m,n \in \omega$, el conjunto $ \{ k<m \tq A_m \cap A_k \subseteq n \} $ es finito.

	Por recursión sobre $\omega_1 \setminus \omega$, sea $\gamma \in \omega_1 \setminus \omega$ cualquiera y supóngase $\{A_\alpha \tq \alpha \in \gamma\}$ es una familia casi ajena tal que, si $\alpha<\gamma$ y $n \in \omega$, el conjunto $ \{ \beta<\alpha \tq A_\alpha \cap A_\beta \subseteq n \} $ es finito.

	Como $\gamma \in \omega \setminus \omega_1$, $\gamma$ es numerable y se puede enumerar $\{A_\alpha \tq \alpha \in \gamma\}$ como $\{B_n \tq n\in \omega\}$. Por ser tal, una familia casi ajena, cada conjunto $C_n:=B_n \setminus \midcup\{ B_j \tq j<n \}$ es infinito (corrobórese ésto en la demostración de \ref{prop-MADnoNum}). Para cada $n \in \omega$ fíjese $a_n \in [C_n]^n$ y defínase:
	$$ A_\gamma:=\midcup\{a_m \tq m \in \omega\} $$

	Nótese que si $n \neq m$, entonces $a_n \cap a_m = \emptyset$. De este modo, si $n \in \omega$ es cualquiera, resulta que $A_\gamma \cap B_n = a_n \cap B_n = a_n$ es finito. Más aún, como $a_n$ tiene exatamente $n$ elementos, $n \leq \max(A_n)$; y consecuentemente, si $m \in \omega$ y $A_\gamma \cap B_n \subseteq m$, entonces $n \leq m$.

	Lo anterior prueba, no sólo que $\{A_\alpha \tq \alpha \ \leq \gamma\}$ es familia casi ajena, sino que para cualesquiera $\alpha\leq \gamma$ y $n \in \omega$, el conjunto $ \{ \beta<\alpha \tq A_\alpha \cap A_\beta \subseteq n \} $ es finito. Lo cual finaliza la construcción por recursión de los conjuntos $A_\alpha$ (con $\omega \leq \alpha< \omega_1$); es claro que para $\ms{A}:=\{A_\alpha \tq \alpha \in \omega_1 \}$ es una familia Luzin que extiende a $\ms{B}$.
\end{proof}

\index[alph]{familia!inseparable}\label{def-FamInseparable}
Cualquier familia de Luizn cumplirá que ninguna grieta formada por sus subconjuntos más que numerables está separada; es decir, es una \textit{familia inseparable} (siguiendo la nomenclatura expuesta en \cite[\S~ 3.2]{hruAlmost}).

Obsérvese que si $\ms{B}$ y $\ms{C}$ son familias casi ajenas de modo que $\midcup \ms{B} \cap \midcup \ms{C}$ es finito, entonces $\midcup \ms{B}$ es separador de $\ms{B}$ y $\ms{C}$, así $(\ms{B},\ms{C})$ está separada. Pese a no ocurrir el recíproco de lo anterior, se configura la siguiente caracterización.

\begin{lema}\label{lem-InseparableCar}
	Sea $\ms{A}\in \Ad(\omega)$, entonces $\ms{A}$ es inseparable si y sólo si para cualesquiera $\ms{B},\ms{C} \in [\ms{A}]^{\omega_1}$ ajenos, $\midcup \ms{B} \cap \midcup \ms{C}$ es infinito.
\end{lema}
\begin{proof}
	Por la discusión previa, basta sólo probar la necesidad.

	Por contrapuesta, supóngase que $\ms{B},\ms{C} \in [\ms{A}]^{\omega_1}$ son tales que existe $D \subseteq \omega$, particionador de $\ms{B}$ y $\ms{C}$. Entonces las asignaciones $\ms{B} \to \omega$ y $\ms{C} \to \omega$; dadas por $b \mapsto \max(b \setminus D)$ y $c \mapsto \max(c \cap D)$ están bien definidas. Pero en vista de que $|\ms{B}|=|\ms{C}|=\omega_1$, estas no pueden ser inyectivas, y existen $m,n \in \omega$ de modo que $\ms{B}':=\{b \in \ms{B} \tq b \setminus D \subseteq m\}$ y $\ms{C}':=\{b \in \ms{C} \tq c \cap D \subseteq n\}$ tienen tamaño $\omega_1$.

	Además $\midcup \ms{B}' \setminus D \subseteq m =^* \emptyset$ y $\midcup \ms{C}' \cap D \subseteq n =^* \emptyset$, por lo que $\midcap \ms{B} \subseteq^* D$, $\midcap \ms{C} \subseteq^* \omega \setminus D$, y así, $\midcup \ms{B}' \cap \midcup \ms{C}' \subseteq^* D \cap (\omega \setminus D) = \emptyset$.
\end{proof}

\begin{proposicion}\label{prop-LuzinSeparadas}
	Cualquier familia Luzin es inseparable.
\end{proposicion}
\begin{proof}
	Sean $\ms{A}=\{A_\alpha \tq \alpha \in \omega_1\}$ cualquier familia de Luzin y $\ms{B}=\{A_\alpha \tq \alpha \in B\},\ms{C}=\{A_\alpha \tq \alpha \in C\} \subseteq \ms{A}$ no numerables y ajenos. Como $C$ es infinito, existe $\alpha \in \omega_1$ de manera que $C \cap \alpha$ es infinito. Nótese que $B$ es cofinal en $\omega_1$, por ser $\omega_1$ regular; así que sin pérdida de generalidad, supóngase $\alpha \in B$.

	En virtud de los comentarios posteriores a la \autoref{def-LuzinFam}, se tiene que $A_\alpha \cap \midcup \{A_\beta \tq \beta \in C \cap \alpha \}$ es infinito, demostrando que $\midcup \ms{B} \cap \midcup \ms{C}$ es infinito. Se concluye del lema previo que $\ms{A}$ es inseparable.
\end{proof}

	\index[alph]{familia!parcialmente separable}
	De manera casi dual al concepto de familia inseparable, existe el concepto de \textit{familia parcialmente separable}; esto es, una familia casi ajena $\ms{A}$ tal que si $\ms{B},\ms{C} \in [\ms{A}]^{\omega_1}$ son ajenas, entonces existen $\ms{B}' \subseteq \ms{B}$ y $\ms{C}' \subseteq \ms{C}$ tales que la grieta $(\ms{B}', \ms{C}')$ está separada. Los siguientes objetos surgen como una suerte de debilitamiento para la condición de familia de Luzin, pues jamás resultan ser parcialmente separables.

	\begin{definicion}\index[alph]{$n$-grieta! de Luzin}\index[alph]{Luzin!$n$-grieta de,}\index[alph]{grieta!de Luzin}\index[alph]{Luzin!grieta de,}\index[aplh]{testigos!de una $n$-grieta de Luzin}
		Sea $n \in \omega \setminus 2$. Una \textbf{$n$-grieta de Luzin} es un $n$-ada $(\ms{A}_i \tq i \in I)$ de familias casi ajenas, disjuntas entre sí, tales que existen $X \in [\omega_1]^{\omega_1}$ y $m \in \omega$, de modo que cada $\ms{A}_i$ se puede enumerar como $\{ A_\alpha^i \tq \alpha \in X \}$ y se cumple que si $\alpha, \beta \in X$:
		\begin{enumerate}
			\item Para cualesquiera $i,j \in n$ distintos, $A_\alpha^i \cap A_\alpha^j \subseteq m$.
			\item Si $\alpha \neq \beta$, entonces $\midcup\{ A_\alpha^i \cap A_\beta^j \tq i, j \in n \, \land \, i \neq j\} \not \subseteq m$.
		\end{enumerate}

		A los conjuntos $X$ y $m$ se les llama testigos de que $(\ms{A}_i \tq i \in I)$ es $n$-grieta de Luzin. Una $2$-grieta de Luzin se denomina simplemente \textbf{grieta de Luzin}.
	\end{definicion}

	Una observación pertiente es que las $n$-grietas de Luzin poseen un comportamiento en común a las familias Luzin, el comentado posteriormente a la \autoref{def-LuzinFam}. Inclusive, en según qué literatura se consulte, se suelen confundir los términos ``familia de Luzin'' y ``grieta de Luzin''.
	
	\begin{lema}
		Sea $(\ms{A}_i \tq i \in I)$ una $n$- grieta de Luzin con testigos $X$ y $m$. Para cada $Y \subseteq X$ no numerable, existe $Y' \subseteq Y$ no numerable tal que para cualesquiera $\alpha,\beta \in Y'$ distintos, $\midcup\{ A_\alpha^i \cap A_\beta^j \tq i, j \in n \, \land \, i \neq j\}$ es infinito.
	\end{lema}
	\begin{proof}
		Sea $k \in \omega \setminus m$. Para cada $i \in n$ sea $f_i:Y \to \omega$ dada por $f(\alpha)=A_\alpha ^ i \cap k$. Como $f_0$ no es inyectiva, existe $Y_0 \subseteq Y$ de modo que $f_0 \upharpoonright Y_0$ es constante. Sea ahora $j \in n \setminus \{n\}$ y supóngase que $Y_j \subseteq Y$ es no numerable y tal que para cada $l<j$ se cumple que $Y_j \subseteq Y_l$ y que $f_j \upharpoonright Y_l$ es constante. Como $f_{j+1} \upharpoonright Y_j$ no es ineyctiva, pues $Y_j$ no es numerable, fíjese $Y_{j+1} \subseteq Y_j \subseteq Y$ no numerable de modo que $f_{j+1}$ es constante en $Y_{j+1}$. Finalizando esta recursión se obtiene que $Y':=\midcap\{ Y_i \tq i \in n \}$ es un subconjunto no numerable de $Y$ tal que para cada $i \in n$, $f_i$ es constante en $Y'$.

		Sean $\alpha, \beta \in Y'$ distintos, como $(\ms{A}_i \tq i \in I)$ es $n$-grieta de Luzin, existen $i,j \in n$ diferentes de manera que $A_\alpha^i \cap A_\beta^j \not \subseteq m$. Como $f_i$ es constante en $Y'$, entonces $A_\alpha^i \cap k = A_\beta ^i \cap k $; y, como $A_\beta^i \cap A_\beta ^j \subseteq m \subseteq k$, resulta necesario que $A_\alpha^i \cap A_\beta^j \not \subseteq k$. Dada la arbitrariedad de $k$, se concluye que $\midcup\{ A_\alpha^i \cap A_\beta^j \tq i, j \in n \, \land \, i \neq j\}$ es infinito.
	\end{proof}

	\begin{corolario}\label{cor-ngrietasNoSep}
		Si una familia casi ajena es parcialmente separable, no puede contener ninguna $n$-grieta de Luzin.
	\end{corolario}
	\begin{proof}
		Sea $(\ms{A}_i \tq i \in I)$ una $n$- grieta de Luzin con testigos $X$ y $m$. Si $ \ms{A} \supseteq \midcup \{ \ms{A}_i \tq i \in n \}$, entonces dado el Lema previo, para cualesquiera $i,j \in n$ distintos, existen $\alpha,\beta \in X$ diferentes de modo que $\midcup\{ A_\alpha^i \cap A_\beta^j \tq i, j \in n \, \land \, i \neq j\}$ es infinito. Es decir, $\midcup \ms{A}_i \cap \midcup \ms{A}_j$ es infinito, siguiéndose la conclusión automáticamente de \ref{lem-InseparableCar}.
	\end{proof}

	La siguiente Proposición deja en claro la labor de ``debilitamiento'' que cumplen las $n$-grietas de Luzin respecto a las familias de Luzin; resulta que, cualquiera de estas últimas contiene siempre una ($2$-)grieta de Luzin. La prueba del hecho en cuestión requiere invocar el \textit{Lema de Fodor} \textcolor{red}{\textbf{(ver REFE)}} y el hecho de que siempre existen funciones ordinales como las solicitadas en la subsecuente prueba.

	\begin{proposicion}
		Toda familia de Luzin contiene una grieta de Luzin.
	\end{proposicion}
	\begin{proof}
		Sea $\ms{A}=\{A_\alpha \tq \alpha \in \omega_1\}$ una familia de Luzin y $f,g:\omega_1 \to \omega_1$ tales que si $ \beta < \alpha < \omega_1$, entonces $g(\beta) < f(\beta) < g(\alpha)$.

		Como $\omega_1 \setminus \omega$ es estacionario en $\omega_1$ y la asignación $j:\omega_1 \setminus \omega \to \omega_1$; dada por $j(\alpha) = \max(A_{f(\alpha)} \cap A_{g(\alpha)}) +1 $, es regresiva, en virtud del Lema de Fodor, existen $S \subseteq \omega_1 \setminus \omega$ estacionario en $\omega_1$ y un natural $m$ tales que $j[S] \subseteq \{m\}$.

        Ahora, si $\beta < \alpha$, entonces $f(\beta) < g(\alpha)$, de modo que $f[\{ \beta < \alpha \tq A_{f(\beta)} \cap A_{g(\alpha)} \subseteq m \}] \subseteq \{ \gamma < g(\alpha) \tq A_{g(\alpha)} \subseteq m \}$. Como $\ms{A}$ es de Luzin y $f$ es inyectiva (por ser estrictamente creciente), el conjunto $\{ \beta < \alpha \tq A_{f(\beta)} \cap A_{g(\alpha)} \subseteq m \}$ es finito, permitiendo la buena definición de $l:S \to \omega_1$ dada por:
        
        \[ l(\alpha) = \left\lbrace \begin{array}{l l}
            0 & ; \,\, \forall \gamma < \alpha ( A_{f(\gamma)} \cap A_{g(\alpha)} \not\subseteq m ) \\
            \max\{ \gamma < \alpha \tq A_{f(\gamma)} \cap A_{g(\alpha)} \subseteq m \} & ; \,\, \text{otro caso}
        \end{array} \right. \]
        
        Como $0 \notin S$, $l$ es regresiva; nuevamente, del Lema de Fodor, se sigue la existencia de un conjunto $X \subseteq S$, estacionario en $\omega_1$, tal que $l$ es constante en $X$. Así, para cualesquiera $\alpha,\beta \in X$:
        \begin{enumerate}
            \item Como $X \subseteq S$ y $j[S] \subseteq \{m\}$, entonces $m=\max(A_{f(\alpha)} \cap A_{f(\alpha)})+1$ y en consecuencia $A_{f(\alpha)} \cap A_{f(\alpha)} \subseteq m$.
            \item Supóngase que $\beta < \alpha$, como $l$ es constante en $X$, $l(\alpha) = l(\beta)$. Si este último valor es cero, entonces $A_{f(\beta)} \cap A_{g(\alpha)} \not\subseteq m$. En caso contrario:
            \[ 0 \neq l(\alpha) = l (\beta) < \beta < \alpha \]
            y como $l(\alpha)< \beta < \alpha$, la definición de $l$ obliga a que $A_{f(\beta)} \cap A_{g(\alpha)} \not\subseteq m$. 
        \end{enumerate} 
        Los puntos anteriores demuestran que $(\{A_{f(\alpha)} \tq \alpha \in X\},\{A_{g(\alpha)} \tq \alpha \in X\})$ es una grieta de Luzin (con testigos $X$ y $m$).
	\end{proof}
	
\subsection{Lema de Solovay}

\begin{definicion}\label{def-ordenBasado} \index[alph]{orden!basado en $\ms{A}$}\index[sym]{$\mathbb{P}_\ms{A}$}\index[sym]{$\leq_\ms{A}$}
	Sea $\ms{A}$ una familia casi ajena. El \textbf{orden basado en} $\ms{A}$ es el par:
	\[ \mathbb{P}_\ms{A}:=([\omega]^{<\omega} \times [\ms{A}]^{<\omega}, \leq_\ms{A}) \]
	donde $(p,P) \leq_\ms{A} (h,H)$ si y sólo si $h \subseteq p$, $H \subseteq P$ y $p \setminus h \subseteq \omega \setminus \midcup H$.
	
	Cuando el contexto sea claro, se escribirá $\leq$ en vez de $\leq_\ms{A}$.
\end{definicion}

Habrá de verificarse que el orden basado en $\ms{A}$ es, en efecto, un orden parcial. Claramente es una relación reflexiva y antisimétrica. Ahora, si $(p,P) \leq (h,H)$ y $(h,H) \leq (k,K)$, es inmediato a la definición de $\leq_\mathscr{A}$ que $k \subseteq h \subseteq p$ y $K \subseteq H \subseteq P$; en consecuencia $k \subseteq p$ y $K \subseteq P$. Además $p \setminus h \subseteq \omega \setminus \midcup H$ y $h \setminus k \subseteq \omega \setminus \midcup K$, resultando en que:
	\[ p \setminus k \subseteq (h \setminus k) \cup (p \setminus h) \subseteq \big( \omega \setminus \midcup K \big) \cup \big( \omega \setminus \midcup H \big) \]
mostrando $p \setminus k \subseteq \omega \setminus \midcup K$, por lo que $\leq$ es transitiva, y con ello, orden parcial.

En términos informales, $(p,P) \leq (h,H)$ significa que ``$h$ se extiende a $p$ y $H$ a $P$''. Conforme $H \subseteq \ms{A}$ crece, se aproxima a $\ms{A}$. Dado que, conforme $h$ crece, este se acerca a un subconjunto casi ajeno con $\midcup H$; eventualmente, se formará un subconjunto casi ajeno con $\midcup \ms{A}$.

\begin{consideracion}\index[sym]{$D_a$ (si $a \in \ms{A}$)}\index[sym]{$D_G$ (si $\mathcal{G} \subseteq \mathbb{P}_\ms{A}$)}
	En lo que resta de la subsección:
	\begin{enumerate}
		\item Para cada $a \in \ms{A}$, $ D_a:=\{ (p,P) \in \mathbb{P}_\ms{A} \tq a \in P\} $.
		\item Si $\mathcal{G} \subseteq \mathbb{P}_\ms{A}$, $ D_\mathcal{G}:=\midcup\{ h \subseteq \omega \tq \exists H \in [\omega]^{\omega} \: \big( (h,H) \in \mathcal{G} \big) \} $.
	\end{enumerate}
\end{consideracion}

\begin{lema}\label{lem-DgMagia}
	Sean $\ms{A}$ una familia casi ajena, y $\mathcal{G}$ un filtro de $\mathbb{P}_\ms{A}$, entonces para cada $a \in \ms{A}$:
	\begin{enumerate}
		\item $D_a$ es denso en $\mathbb{P}_\ms{A}$.
		\item Si $\mathcal{G} \cap D_a \neq \emptyset$; entonces, $D_\mathcal{G} \cap a$ es finito.
	\end{enumerate}
\end{lema}

\begin{proof}
	(i) Si $a \in \ms{A}$ y $(p,P) \in \mathbb{P}_\ms{A}$ son elementos arbitrarios, entonces $(p,P\cup\{a\}) \in D_a$ y además es inmediato a la \autoref{def-ordenBasado} que $(p,P\cup\{a\}) \leq (p,P)$.

	(ii) Supóngase que $(p,P) \in \mathcal{G} \cap D_a$ y sea $x \in D_\mathcal{G} \cap a$ cualquier elemento. Por definición de $D_\mathcal{G}$, existe $(h,H) \in \mathcal{G}$ de modo que $x \in h$. Y por ser $\mathcal{G}$ filtro, $(k,K) \leq (p,P),(h,H)$ para cierto $(k,K) \in \mathcal{G}$. De esto, particularmente se obtiene que $h \subseteq k$, $k \setminus p \subseteq \omega \setminus \midcup P$.

	Ahora, como $a \in P$ (pues $(p,P)\in D_a$), se tiene que $x \in \midcap P$. Además, $x \in h \subseteq k$, así que $x \in k \cap \midcup P$, lo cual obliga a que $x \in p$. Por tanto $D_\mathcal{G} \cap a \subseteq p =^* \emptyset$.
\end{proof}

\begin{corolario}\label{cor-SolovayDebil}
	Sean $\ms{A} \in \Ad(\omega)$ y $\ms{D}:=\{D_a \tq a \in \ms{A}\}$. Si existe un filtro $\ms{D}$-genérico, $\ms{A}$ no es maximal.
\end{corolario}

Debido a lo recién mostrado, de tener $\mathbb{P}_\ms{A}$ la \textit{c.c.c.} \textcolor{red}{\textbf{(ver PRELIMS)}}, se satisfaría que $\Ma(|\ms{A}|)$ implica $\ms{A} \notin \Mad(\omega)$.

Y en efecto, si $\mathcal{A} \subseteq \mathbb{P}_\ms{A}$ es anticadena y $(p,P),(h,H) \in \mathcal{A}$, se tiene $p\neq h$; sino $(p,P \cup H) \leq (p,P),(h,H)$ y $\mathcal{A}$ dejaría de ser anticadena. En consecuencia $|\mathcal{A}|\leq|[\omega]^{<\omega}|=\aleph_0$ y $\mathbb{P}_\ms{A}$ tiene la \textit{c.c.c.}

\begin{corolario}\label{cor-MaSimple}
	Si $\kappa$ un cardinal con $\omega \leq \kappa <\mathfrak{c}$; bajo $\Ma(\kappa)$, se tiene $\Mad(\omega) \subseteq \left[[\omega]^\omega\right]^{>\kappa}$; y por ello $\mathfrak{a}>\kappa$.
	Consecuentemente:
	\begin{enumerate}
		\item $\zfc \vdash \mathfrak{m} \leq \mathfrak{a}$ % (recuérdese \textbf{Def m}).
		\item $\zfc + \Ma \vdash \mathfrak{a}=\mathfrak{c}$.
		\item $\mathfrak{a} = \mathfrak{c}$ es estrictamente más débil que $\HC$.
	\end{enumerate}
\end{corolario}

\begin{proof}
	Únicamente falta verificar (iii). Basta tener en cuenta que $\Ma + \lnot \HC$ es consistente con $\zfc$ (consúltese \cite[p.~279-281]{kunenSet}); así que de (iii), se obtiene que $\zfc + \Ma + \lnot \HC \vdash \mathfrak{a}=\mathfrak{c}$. Por ende, $\zfc + \mathfrak{a}=\mathfrak{c} \not\vdash \HC$.
\end{proof}

El \autoref{cor-SolovayDebil} es una inmediatez, dada toda su discusión previa. Una versión bastante más fortalecida de este, es el siguiente resultado mostrado por Robert Solovay.

\begin{lema}[Solovay]\label{lem-Solovay}\index[alph]{Lema!de Solovay}\index[alph]{Solovay!Lema de}
	Sea $\kappa$ un cardinal de modo que $\omega \leq \kappa < \mathfrak{c}$. Bajo $\Ma(\kappa)$; para toda grieta $(\ms{A},\ms{B})$; con $|\ms{A}|,|\ms{B}|\leq \kappa$, existe $D\subseteq \ms{A}$ tal que para cada $A \in \ms{A}$ y $b \in \ms{B}$, $a \cap D=^*\emptyset$ y $b \cap D\neq ^*\emptyset$.
\end{lema}

\begin{proof}
	Supóngase $\Ma(\kappa)$ y sea $(\ms{A},\ms{B})$ una grieta de forma que $|\ms{A}|,|\ms{B}|\leq \kappa$. Para cualesquiera $b \in \ms{B}$ y $n \in \omega$, defínase el conjunto $D(b,n):=\{ (h,H) \in \mathbb{P}_\ms{A} \tq h \cap b \not\subseteq n \}$.

	Cada $D(b,n)$ es denso en $\mathbb{P}_\ms{A}$. Sea $(p,P) \in \mathbb{P}_\ms{A}$ cualquiera; por \ref{obs-GrietasSimple}, $b \notin \ms{I}(\ms{A}) $, luego $b \setminus \midcup P$ es infinito. Por ello, existe $m \in \omega$ de modo que $n+1 \in m$ y $m \in b \setminus \midcup P \subseteq \omega \setminus \midcup P$; así, $p \cup \{m\}$ es finito, $(p \cup \{m\}, P) \in D(b,n)$ y $(p \cup \{m\}, P) \leq_\ms{A} (p,P)$.

	Sea $\ms{D}=\{ D(b,n) \tq (b,n) \in \ms{B} \times \omega \} \cup \{D_a \tq a \in \ms{A} \}$ y obsérvese que $\ms{D}$ es una familia de densos de $\mathbb{P}_\ms{A}$ de cardinalidad menor o igual a $\kappa$. Como $\mathbb{P}_\ms{A}$ es \textit{c.c.c.}, de $\Ma(\kappa)$ se desprende la existencia de un filtro $\mathcal{G}$ en $\mathbb{P}_\ms{A}$, $\ms{D}$-genérico. Se afirma que $D_\mathcal{G}$ es el conjunto buscado.

	En efecto, por \ref{lem-DgMagia} se tiene que para cada $a \in \ms{A}$, el conjunto $D_\mathcal{G} \cap a$ es finito. Ahora, si $b \in \ms{B}$ es cualquiera, para cada $n \in \omega$ existe $(k,K) \in \mathcal{G} \cap D(b,n)$; y en consecuencia $D_\mathcal{G} \cap b \not \subseteq n$ (pues $h \cap b \not \subseteq n$). Por lo que el conujunto $D_\mathcal{G} \cap b$ no puede ser finito.
\end{proof}

Sería deseable que la conclusión del Lema de Solovay fuese que la grieta $(\ms{A},\ms{B})$ está separada; sin embargo tal formulación desemboca en un resultado falso.

\index[alph]{familia!débilmente separada}
Bajo $\Ma + \lnot\HC$, la existencia de una familia de Luzin (probada en \ref{prop-LuzinSeparadas}) sería testigo de tal falsedad. Se puede decir que si una grieta $(\ms{A},\ms{B})$ satisface la conclusión de \ref{lem-Solovay}, entonces está \textit{débilmente separada} (siguiendo la terminología de \cite[\S~ 3.2]{hruAlmost}), si $D$ es como en la conclusión de \ref{lem-Solovay}.

La siguiente es una aplicación típica quel lema de Solovay, una prueba sencilla de la consistencia, respecto a $\zfc$, del enunciado $2^{\aleph_0}=2^{\aleph_1}$. Como se podrá verificar en secciones futuras de este escrito, el enunciado $2^{\aleph_1} = 2^{\aleph_0}$ y su negación $2^{\aleph_1} > 2^{\aleph_0}$, tendrán repercusiones en el comportamiento topológico de los $\Psi$-espacios. Especialmente, se mostrarán sus efectos sobre la Conjetura de Moore (véase \autoref{Sec-PDM}).

\begin{corolario}
	Sea $\kappa$ un cardinal con $\omega \leq \kappa <\mathfrak{c}$, entonces bajo $\Ma(\kappa)$, se tiene $2^\kappa=\mathfrak{c}$.

	Consecuentemente, es consistente con $\zfc$ que $2^{\aleph_0}=2^{\aleph_1}$
\end{corolario}

\begin{proof}
	Sea $\kappa$ un cardinal con $\omega \leq \kappa <\mathfrak{c}$ y supóngase $\Ma(\kappa)$. Tomando en cuenta \ref{cor-famGrandes}, fíjese una familia casi ajena $\ms{A}$ con $|\ms{A}|=\kappa$ y defínase $f:\ms{P}(\omega) \to \ms{P}(\ms{A})$ como $ f(X)=\{ b \in \ms{A} \tq b \cap X =^* \emptyset \} $.

	Si $\ms{B} \subseteq \ms{A}$ es cualquiera, entonces $|\ms{A} \setminus \ms{B}|, |\ms{B}| \leq \kappa$ y por el Lema de Solovay (\ref{lem-Solovay}), existe un particionador $D \subseteq \omega$ para $\ms{A} \setminus \ms{B}$ y $\ms{B}$, resultando en que $ f(D)=\{b \in \ms{A} \tq b \cap X =^* \emptyset \} = \ms{B} $. Luego $f$ es sobreyectiva y $\mathfrak{c} \geq 2^\kappa $. Como además $\kappa \geq \aleph_0$, entonces $2^\kappa \geq 2^{\aleph_0}=\mathfrak{c}$.

	Para la segunda parte, $\lnot \HC + \Ma$ es consistente con $\zfc$. Y como $\lnot \HC$ y $\Ma$ implican $\Ma(\aleph_1)$ y $\omega \leq \aleph_1 < \mathfrak{c}$, se tiene por consiguiente que $\zfc+\lnot \HC + \Ma \vdash 2^{\aleph_0}=2^{\aleph_1}$.
\end{proof}


