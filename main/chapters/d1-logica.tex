\chapter{Lógica y Axioma de Martin}

\section{Consistencia relativa}

\index[alph]{fórmula de la teoría de conjuntos}\index[sym]{$\ms{L}_{\operatorname{TC}}$}\index[sym]{$\Sigma + \Lambda$}\index[sym]{$\Sigma + \lambda$}
A continuación se exponen, de forma sumamente laxa, los conceptos necesarios para definir la consistencia de un enunciado respecto a un conjunto de enunciados. Todo lo subsecuente será relativo al lenguaje de la teoría de conjutnos, entenderemos por \textit{fórmula} a cualquier sentencia, predicado, o \enquote{proposición} escrita con los símbolos $(,),\forall,\exists,\to,\lnot,\land,\lor,\leftrightarrow,\in,x,y,z,\dots$ y $\ms{L}_{\operatorname{TC}}$ será el conjunto de fórmulas de la teoría de conjuntos. Cuando $\Sigma, \Lambda \subseteq \ms{L}_{\operatorname{TC}}$, será común escribir $\Sigma + \Lambda$ para referirse a $\Sigma \cup \Lambda$, cuando $\Lambda=\{\lambda\}$, se escribirá únicamente $\Sigma + \lambda$.

\index[alph]{axioma!de Hilbert}\index[alph]{Hilbert!axioma de}
Un \textit{axioma de Hilbert} es cualquier fórmula de la forma:
\begin{enumerate}[ ]
    \item \text{Ax. } 1. $\alpha \to (\beta \to \alpha)$.
    \item \text{Ax. } 2. $(\alpha \to (\beta \to \gamma)) \to ((\alpha \to \beta) \to \gamma)$.
    \item \text{Ax. } 3. $(\lnot \alpha \to \lnot \beta) \to ((\lnot \alpha \to \beta) \to \alpha)$.
\end{enumerate}

\index[alph]{regla de inferencia}\index[sym]{$\Gen$}\index[sym]{$\MP$}
Una \textit{regla de inferencia} es una operación parcial sobre $\ms{L}_{\operatorname{TC}}$; y particularmente, nos interesarán las reglas de inferencia:
\begin{align*}
    \Gen : \varphi & \mapsto \forall x \, ( \varphi ) \\
    \MP : (\varphi,\varphi \to \sigma) & \mapsto \sigma
\end{align*}

\index[alph]{derivación formal}\index[alph]{prueba formal}\index[alph]{demostración formal}\index[sym]{$\Sigma \vdash \varphi$}
Dados un $\varphi \in \ms{L}_{\operatorname{TC}}$ y $\Sigma \subseteq \ms{L}_{\operatorname{TC}}$, diremos que $\Sigma$ \textit{deriva} a $\varphi$, denotado $\Sigma \vdash \varphi$, si y sólo si existe una sucesión finita de enunciados $\varphi_1,\dots,\varphi_n$ de manera que:
\begin{enumerate}
    \item $\varphi_n=\varphi$.
    \item Para cada $i \leq n$ ocurre alguna de las siguientes:
    \begin{enumerate}
        \item $\varphi_i \in \Sigma$,
        \item $\varphi_i$ es un axioma de Hilbert,
        \item Existen $j<i$ con $\varphi_i=\Gen(\varphi_j)$, o
        \item Existen $k,j<i$ con $\varphi_i=\MP(\varphi_j,\varphi_k)$.
    \end{enumerate}
\end{enumerate}
En tal caso, $(\varphi_i)_{i \leq n}$ es una \textit{prueba} (o \textit{demostración}) \textit{formal} de $\varphi$ a partir de $\Sigma$.

\index[alph]{enunciado}\index[alph]{variable libre}\index[alph]{teoría}
Un \textit{enunciado} es una fórmula que no tiene \textit{variables libres}, esto es, tal que todas sus variables están cuantificadas, universal o existencialmente. Una \textit{teoría} es un conjunto de enunciados. Por ejemplo, $\zfc \subseteq \ms{L}_{\operatorname{TC}}$ es la teoría que consta, como ya se ha mencionado, de $\zf \subseteq \ms{L}_{\operatorname{TC}}$ junto con el Axioma de Elección, $\Ac \in \ms{L}_{\operatorname{TC}}$; esto es $\zfc=\zf + \Ac$.

\index[alph]{Teorema}\index[alph]{teoría!consistente}\index[alph]{teoría!inconsistente}\index[alph]{consistente! con}\index[alph]{independiente! con}\index[alph]{conjunto! de axiomas}\index[alph]{axioma}
Sea $\Sigma \subseteq \ms{L}_{\operatorname{TC}}$ una teoría. Un \textit{teorema} de $\Sigma$ es una fórmula $\alpha$ tal que $\Sigma \vdash \alpha$. Se dice que $\Sigma$ es \textit{consistente} cuando para ningún $\alpha \in \ms{L}_{\operatorname{TC}}$ ocurre $\Sigma \vdash \alpha$ y $\Sigma \vdash \lnot \alpha$. Un enunciado $\sigma$ es \textit{consistente} con $\Sigma$ cuando $\Sigma \not\vdash \lnot\sigma$, y es \textit{independiente} de $\Sigma$ si tanto $\sigma$ como $\lnot \sigma$ son consistentes con $\Sigma$. Un subconjunto de enunciados $\Sigma \subseteq \ms{L}_{\operatorname{TC}}$ es un \textit{conjunto de axiomas} (y sus elementos son \textit{axiomas}) cuando:
\begin{enumerate}
    \item $\Sigma$ es consistente, y
    \item Cada $\sigma \in \Sigma$ es independiente de $\Sigma \setminus \{\sigma\}$.
\end{enumerate}

Es un hecho que si $\Sigma$ es consistente y $\sigma$ es consistente con $\Sigma$, entonces $\Sigma \cup \sigma$ sigue siendo consistente. Además, $\Sigma$ es inconsistente si y sólo si para cada $\alpha \in \ms{L}_{\operatorname{TC}}$ ocurre $\Sigma \vdash \alpha$.

En el mundo de la lógica existen diversas, muy diversas y sofisticadas, técnicas para demostrar la consistencia relativa de cierto enunciado respecto un conjunto de una teoría. Para este trabajo nos limitaremos a utilizar que: siempre que $\sigma$ sea consistente con $\Sigma$ y $\Sigma + \sigma \vdash \varphi$, se tiene que $\varphi$ es consistente con $\Sigma$.

Todas las demostraciones de consistencia, e independencia, que figuran en este texto aluden al párrafo anterior más la suposición de que $\zfc$ es consistente. Como $\HC$ y $\Ma$ son independientes de $\zfc$ \cite[algo]{kunenSet}, cualquier teorema $\sigma$ de $\zfc + \HC$, $\zfc + \lnot\HC$, $\zfc + \Ma$ o $\zfc + \lnot\Ma$, será consistente con $\zfc$.

\newpage

\index[alph]{función!de elección}\index[alph]{elección!función de}\index[alph]{elección!axioma de}
Si $X$ es un conjunto no vacío, se dice que $f \colon \ms{P}(X) \setminus \{\emptyset\} \to X$ es una \textit{función de elección} en $X$ cuando para cada $x \in X$ ocurre $f(x) \in x$. La formulación clásica del \textcolor{magenta}{\textit{Axioma de Elección} ($\Ac$)} dicta que todo conjunto no vacío admite una función de elección.