\setcounter{chapter}{-1}
\chapter{Preeliminares}

\section{Conjuntos}

\begin{enumerate}[1.]
    \item Convención sobre las notaciones ``no estándar'' (potencia, colecciones $[A]^{z \kappa}$, etcétera)
    \item Naturales, ordinales y cardinales.
    \item Inducción y Recursión más allá de $\omega$ (?)
    \item Órdenes parciales y sus elementos distinguidos, árboles.
    \item Casi contención y el comportamiento básico de la misma.
\end{enumerate}

\section{Topología}

\begin{enumerate}[1.]
    \item Espacios topológicos, bases, axiomas de numerabilidad. Definición sucinta de los términos: peso, carácter
    \item Convenciones sobre las notaciones para producto, suma, homeomorfismos, encajes, y  esclarecimiento de los los términos: propiedades topológicas, productivas, factorizables, etc...
    \item Axiomas de separación (desde $\T_0$ a $\T_4$, normalidad, regularidad, regularidad completa y normalidad).
    \item Espacios cero dimensionales y su caracterización.
    \item Convergencia de sucesiones, espacios de Fréchet y secuenciales. Convención de términos como: sucesiones convergentes, clausura secuencial, etcétera.
    \item Compacidad y sus ``variantes'': compacidad numerable, Lindelöf, compacidad secuencial, pseudocompacidad, compacidad local, etc...
    \item Metrizabilidad: se conviene de forma breve lo que es una métrica (y métrica completa ?), un espacio metrizable (y completamente metrizable ?). Se enuncian teoremas de equivalencia para: su separabilidad; y, su compacidad.
    \item Metrización:Se enuncian los teoremas de metrización de: Urysohn, Bing y Arhangel’skii.
    \item Categoría de Baire, se define el concepto de espacio de Baire y se enuncia el teorema de Categoría de Baire para: espacios de Hausdorff y localmente compactos; y, espacios completamente metrizables.
\end{enumerate}