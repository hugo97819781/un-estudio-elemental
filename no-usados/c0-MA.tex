\subsubsection{MA}

\phantomsection
\label{copo-ccc}

\index[alph]{elemento!compatible con otro}\index[alph]{elemento!incompatible con otro}\index[sym]{$p \parallel q$}\index[sym]{$p \perp q$}
Durante todo lo que resta del capítulo, $\mathbb{P}=(P,\leq)$ será un conjunto parcialmente ordenado. Dos elementos $p,q \in P$ son \textit{compatibles} cuando existe $r \in P$ de forma que $r\leq p$ y $r\leq p$, esta situación se denotará por $p \parallel q$; en caso de que $p \not \parallel q$, diremos que $p$ y $q$ son \textit{incompatibles}, lo que se denotará $p \perp q$.

\index[alph]{conjunto!anticadena}\index[alph]{conjunto!denso}\index[alph]{denso (orden parcial)}\index[alph]{anticadena}\index[alph]{propiedad!de anticadena contable}\index[alph]{propiedad!c.c.c.}\index[alph]{c.c.c.}
Una \textit{anticadena} de $\mathbb{P}$ es un subconjunto $A \subseteq P$ de elementos incompatibles dos a dos. Cuando cualquier anticadena de $\mathbb{P}$ sea contable, diremos que $\mathbb{P}$ tiene la \textit{propiedad de anticadena contable} o que tiene la \textit{c.c.c.} Un subconjunto $D \subseteq P$ es denso cuando para cualquier $p \in P$ existe $d \in D$ de manera que $d \leq a$.

\index[alph]{conjunto!filtro}\index[alph]{conjunto!ideal}\index[alph]{filtro}\index[alph]{ideal}
Un \textit{filtro} es un subconjunto $F \subseteq P$ tal que:
\begin{enumerate}
    \item Si $p \in F$ y $p \leq q$, entonces $q \in F$, y
    \item Si $p,q \in F$, entonces existe $r \in F$ de manera que $r \leq p$ y $r \leq q$.
\end{enumerate}

\index[alph]{filtro!propio}\index[alph]{ideal!propio}
Un \textit{ideal} es un filtro en $(P,\geq)$. Un filtro (ideal, respectivamente) es \textit{propio} cuando es diferente de $P$. No es difícil verificar lo siguiente:
\begin{proposicion}
    Sean $X$ un conjunto y $F \subseteq \ms{P}(X)$. Entonces $F$ es filtro si y sólo si:
    \item Si $A \in F$ y $A \leq B$, entonces $B \in F$, y
    \item Si $A,B \in F$, entonces $A \cap B \in F$.
\end{proposicion}
\begin{proof}
    Basta demostrar la suficiencia. Supóngase que $F$ es filtro, el punto (i) es inmediato a la definición de filtro. Para (ii) sean $A,B \in F$, como $F$ es filtro, existe $C \in F$ tal que $C \subseteq A$ y $C \subseteq B$. De esto, $C \subseteq A \cap B$, y por (i), resulta que $A \cap B \in F$.
\end{proof}

\index[alph]{ultrafiltro}\index[alph]{conjunto!ultrafiltro}
Un filtro de $\mathbb{P}$ es \textit{ultrafiltro} si es $\subseteq$-maximal del conjunto de todos los filtros de $\mathbb{P}$. El siguiente es un Teorema sumamente conocido, su demostración es una aplicación rutinaria del \cref{teo-PMO}.

\begin{teorema}[Lema del Ultrafiltro]\phantomsection\label{lem-ultrafil}\index[alph]{Lema!del ultrafiltro}\index[alph]{ultrafiltro!Lema del}
    Para todo filtro $F$ de $\mathbb{P}$, existe un ultrafiltro $U \supseteq F$.
\end{teorema}

\index[alph]{filtro!genérico}\index[alph]{filtro!$\ms{D}$-genérico}
Si $\ms{D} \subseteq \ms{P}(P)$ es una colección de subconjuntos densos y $G$ es un filtro de $\mathbb{P}$, diremos que $G$ es $\ms{D}$-\textit{genérico} cuando para cada $D \in \ms{D}$ ocurre $G \cap D \neq \emptyset$. Cuando $\ms{D}$ sea la coleción de todos los subconjuntos densos de $\mathcal{P}$, diremos simplemente que $G$ es \textit{genérico}. Para nada es inmediato probar la existencia, o imposibilidad de existencia, de filtros genéricos.

\index[alph]{axioma!de Martin en $\kappa$}\index[alph]{Martin!axioma de, en $\kappa$}\index[sym]{$\Ma(\kappa)$}
El Axioma de Martin es nombrado así en honor a Donald A. Martin (1940-$\rightarrow$), uno de sus precursores, y se formula de manera puntual, para cada cardinal infinito $\kappa$ como el enunciado: \enquote{Para todo orden parcial $\mathbb{P}$, con la c.c.c. y toda colección de densos $\ms{D}$ de tamaño a lo más $\kappa$, existe un filtro $\ms{D}$-genérico}.
\begin{proposicion}
    $\Ma(\aleph_0)$ es verdadero y $\Ma(\mathfrak{c})$ es falso.
\end{proposicion}
\begin{proof}
    Para la primera parte, sean $\mathbb{P}=(P,\leq)$ cualquier orden parcial (no necesariamente c.c.c.), $p_0 \in D_0$ y supóngase que $\ms{D}$ es una colección de densos de $\mathbb{P}$ enumerada como $\{D_n \tq n \in \omega\}$. Por recursión, para cada $n\in \omega$ fíjese ($\Ac$) $p_{n+1} \in D_{n+1}$ tal que $p_{n+1} \leq p_n$. Sea:
    \[ G \mycoloneq \{ q \in P \tq \exists n \in \omega \, ( p_n \leq q ) \} \, .\]
    
    Claramente, si $p \in G$ y $q \geq p$, entonces $q \in G$. Además, si $p,q \in F$ entonces existen $n,m \in \omega$ tales que $p_n \leq p$ y $p_m \leq q$, de donde $p_k \leq p$ y $p_k \leq q$, donde $k \mycoloneq \max\{n,m\}$. Por contrucción, $G$ es un filtro $\ms{D}$-genérico.

    Para la siguiente parte, considere el orden parcial $\mathbb{T} \mycoloneq (2^{<\omega},\supseteq)$. Nótese que si $A \subseteq 2^{<\omega}$ es anticadena de $\mathbb{T}$ entonces $A$ es contable, pues $2^{<\omega}$ es numerable. Para cada $f \in 2^\omega$ sea $D_f \mycoloneq \{ p \in 2^{<\omega} \tq p \not\subseteq f \}$. Nótese que cada $D_f$ es denso, pues si $p \in 2^{<\omega}$, entonces $q \mycoloneq p \cup \{(\dom(p)+1,1-f(\dom(p)+1))\} \in D$ y $q \supseteq p$. Como los $D_f$ son distintos dos a dos, la cantidad de densos en $\mathfrak{T}$ es $mathfrak{c}$.
    
    Supóngase, por contradicción, que existe un filtro $G$ de $\mathbb{T}$ tal que interseca a todos los densos de $\mathbb{T}$. Sea $g \mycoloneq \midcup G$, nótese que $g$ es función, pues para cualesquiera $h,k \in G$ existe $r \in G$ con $r \leq h,k$, de donde $r \supseteq h \cup k$, mostrando que $h \cup k$ es función. Ahora, si $n \in \omega$ es cualquiera, es fácil ver que el conjunto:
    \[ E_n \mycoloneq \{ p \in 2^{<\omega} \tq n \in \dom(p) \} \]
    es denso, por lo tanto, existe $t \in G \cap E_n$, mostrando que $p \in \dom(t) \subseteq \dom(\midcup G)$. Así que $\midcup G: \omega \to 2$. Sin embargo, dada $f \in 2^\omega$, existe $s \in G \cap D_f$ (pues $D_f$ es denso), luego $f \not\subseteq s$, y con ello, $f \neq \midcup G$, esto implica que $\midcup G \notin 2^\omega$, lo que es una contradicción.

    Por lo tanto, la colección $\ms{D}$ de todos los densos de $\mathbb{T}$ es una colección de tamaño $\mathfrak{c}$ y no existe ningún filtro $\ms{D}$-genérico, mostrando que $\Ma(\mathfrak{c})$ es falso.
\end{proof}

\phantomsection\label{MADef}\index[alph]{axioma!de Martin}\index[alph]{Martin!axioma de}\index[sym]{$\Ma$}\index[sym]{$\mathfrak{m}$}\index[alph]{cardinal!de Martin}\index[alph]{Martin!cardinal de}
A razón de lo anterior, tiene sentido definir el \textit{cardinal de Martin}, $\mathfrak{m}$, esto es, el menor cardinal para el cual $\Ma(\kappa)$ es verdadero. Se formula así, el \textit{Axioma de Martin} como el enunciado: para todo $\kappa < \mathfrak{c}$, $\Ma(\kappa)$ es verdadero; equivalentemente, $\mathfrak{m}=\mathfrak{c}$.