%Formato e idioma
    \documentclass[letterpaper,DIV=12,12pt]{scrbook}
    \usepackage[spanish,mexico,shorthands=off,es-lcroman]{babel}
%Utilidades
    \usepackage{array}
    \usepackage[x11names]{xcolor}
    \usepackage{lipsum}
    \usepackage{stix2}
%Matemáticos
    \usepackage{amsmath}
    \usepackage{amsthm}
    \usepackage{amssymb} % eliminar si se usa unicode-math
    \usepackage{mathrsfs} % agregar después de fuente unicode-math, si se usa
%Etiquetas de las enumeraciones
    \usepackage[shortlabels]{enumitem}
    \setenumerate[1]{label=\MakeLowercase{\roman*}), ref=\roman*}
    \setenumerate[2]{label=\MakeLowercase{\alph*}), ref=\alph*}
%Cajas de Teoremas
    \usepackage{thmtools}
    \usepackage[framemethod=TikZ]{mdframed}
    \newtheorem{definicion}{Definición}[section]
    \newtheorem{proposicion}[definicion]{Proposición}
    \newtheorem{lema}[definicion]{Lema}
    \newtheorem{corolario}[definicion]{Corolario}
    \newtheorem{observacion}[definicion]{Observación}
    \newtheorem{ejemplo}[definicion]{Ejemplo}
    \newtheorem{consideracion}[definicion]{Consideración}
    \newtheorem{teorema}[definicion]{Teorema}
%Comandos Creados
	%GENERALES
\newcommand{\tq}{\text{ $|$ }}
\newcommand{\midcup}{\mbox{$\bigcup$}\,}
\newcommand{\midcap}{\mbox{$\bigcap$}\,}
\newcommand{\ms}[1]{\mathscr{#1}}
%RENOVACIÓN DE COMANDOS
\renewcommand{\emptyset}{\varnothing}
\renewcommand{\tau}{\ms{T}}
%Nuevo "Setminus"
\newcommand{\mysetminusD}{\hbox{\tikz{\draw[line width=0.6pt,line cap=round] (3pt,0) -- (0,6pt);}}}
\newcommand{\mysetminusT}{\mysetminusD}
\newcommand{\mysetminusS}{\hbox{\tikz{\draw[line width=0.45pt,line cap=round] (2pt,0) -- (0,4pt);}}}
\newcommand{\mysetminusSS}{\hbox{\tikz{\draw[line width=0.4pt,line cap=round] (1.5pt,0) -- (0,3pt);}}}
\newcommand{\mysetminus}{\mathbin{\mathchoice{\mysetminusD}{\mysetminusT}{\mysetminusS}{\mysetminusSS}}}
\renewcommand{\setminus}{\mysetminus}
%OPERADORES
%\makeatletter
%	\renewcommand{\operator@font}{\opfont}
%\makeatother

\DeclareMathOperator{\inte}{int}
\DeclareMathOperator{\ext}{ext}
\DeclareMathOperator{\cla}{cl}
\DeclareMathOperator{\der}{der}
\DeclareMathOperator{\fron}{fr}
\DeclareMathOperator{\scl}{sqcl}
\DeclareMathOperator{\cf}{cf}
\DeclareMathOperator{\Id}{Id}
\DeclareMathOperator{\ima}{ima}
\DeclareMathOperator{\dom}{dom}
\DeclareMathOperator{\St}{St}
\DeclareMathOperator{\Osq}{O_{sq}}
%TEXTOS
\DeclareMathOperator{\T}{T}
\DeclareMathOperator{\AN}{AN}
\DeclareMathOperator{\OR}{OR}
\DeclareMathOperator{\CAR}{CAR}
\DeclareMathOperator{\zfc}{ZFC}
\DeclareMathOperator{\zf}{ZF}
\DeclareMathOperator{\HC}{CH}
\DeclareMathOperator{\Ma}{MA}
\DeclareMathOperator{\Ac}{AC}
\DeclareMathOperator{\Pm}{MC}
\DeclareMathOperator{\Pdm}{WMC}
\DeclareMathOperator{\Ad}{AD}
\DeclareMathOperator{\Mad}{MAD}
%COCIENTE
%\newcommand{\quot}[2]{{\raisebox{.2em}{$#1$}\left/\raisebox{-.2em}{$#2$}\right.}}

%Exclusivo del minimal
    \renewcommand{\ref}[1]{??}
    \renewcommand{\pageref}[1]{??}
    \newcommand{\autoref}[1]{??}
    \renewcommand{\cite}[2][]{??}
    \renewcommand{\index}[2][]{}
    \newcommand{\hyperref}[2][]{#2}
    \providecommand{\texorpdfstring}[2]{#1}

\begin{document}
    \frontmatter
        \pagestyle{empty}
        \cleardoublepage
\vspace*{3cm}
\begin{flushright}
	\textit{Dedicado a Pepe, María, Claris, Dionisio y Margarita}.
\end{flushright}
\cleardoublepage
        \chapter*{Agradecimientos}
        \input{../chapters/a0-agradecimientos.tex}
        \tableofcontents
        \chapter*{Introducción}
        \addcontentsline{toc}{chapter}{Introducción}
        Corría el año de 1954 cuando Stanisław G. Mrówka (1933-2010), \enquote{hijo} doctoral de Kazimierz Kuratowski (1896-1980), expuso en su artículo \textit{On completely regular spaces} \cite{mrowkaContra} un método novedoso para la construcción de espacios topológicos de Tychonoff, pseudocompactos, pero no compactos. La construcción parte de una \textit{familia casi ajena}; esto es, en terminología moderna, un conjunto $\mathscr{A}$ compuesto por subconjuntos infinitos de $\omega$ que, dos a dos, tienen intersección finita. Se dice que $\mathscr{A}$ es \textit{maximal} si no existe una familia casi ajena $\mathscr{B}$ que contenga propiamente a $\mathscr{A}$. El espacio contraejemplo de Mrówka es $\Psi(\mathscr{A}) := \omega \cup \mathscr{A}$, donde un subconjunto $U \subseteq \Psi(\mathscr{A})$ es abierto si y sólo si para cada $x \in U \cap \mathscr{A}$, la diferencia $x \setminus U$ es finita. En $\Psi(\mathscr{A})$, el subespacio $\mathscr{A}$ es cerrado y discreto; por lo tanto, si $\mathscr{A}$ es infinita, $\Psi(\mathscr{A})$ no puede ser compacto. Por otro lado, si $\mathscr{A}$ es maximal, entonces $\Psi(\mathscr{A})$ resulta ser pseudocompacto. Aludiendo a la existencia de familias casi ajenas maximales e infinitas, se obtiene un esbozo de la demostración dada por Mrówka.

El aporte teórico recién mencionado constituye un antecedente temprano para el estudio de los \textit{espacios de Isbell-Mrówka}, aunque no es el primero, pues esta topología fue descrita por primera vez, al menos según la documentación reconocida, por Pavel Alexandroff (1896-1982) y Pavel Urysohn (1898-1924) en \cite{alexOrigen}. Pese a ello, su nombre rinde homenaje tanto a Mrówka como a su par profesional John R. Isbell (1930-2005), quienes de manera independiente de Alexandroff y Urysohn, desarrollaron el concepto entre las décadas de los cincuenta y los sesenta, mostrando por qué se trata de objetos dignos de investigación.

Todo espacio $\Psi(\mathscr{A})$, asociado a una familia casi ajena $\mathscr{A}$, es: de Tychonoff, cero-dimensional, disperso, separable y hereditariamente localmente compacto. De hecho, Kannan y Rajagopalan demostraron que los únicos espacios (infinitos, separables y de Hausdorff) hereditariamente localmente compactos son, precisamente, los espacios de Mrówka \cite{kannanHereditarily}. Sin duda, lo que ha colocado a estos objetos, a lo largo de los años, en un lugar privilegiado dentro de las matemáticas es su versatilidad; pues una amplia variedad de invariantes topológicos de $\Psi(\mathscr{A})$ pueden ser \enquote{codificados} mediante el comportamiento de la familia $\mathscr{A}$, considerada como conjunto. Como consecuencia, existen diversas aplicaciones relacionadas con estos espacios, que abarcan desde el estudio de compactaciones, selecciones continuas y espacios totalmente ordenados, hasta resultados en espacios de funciones continuas equipados con la topología de convergencia puntual.

El trabajo que aquí se propone pretende ofrecer una introducción asequible a los espacios de Isbell–Mrówka y puede ser concebido como un \enquote{manual}. Una motivación fundamental para la realización de este trabajo es el hecho de que el material disponible sobre este tema, especialmente en español, es relativamente limitado. La meta final es explicar, siempre de manera clara, cómo se van tendiendo \enquote{puentes} entre la topología y la teoría de conjuntos por medio de los espacios de Mrówka. Se asumirá que el lector cuenta con una formación elemental, equiparable a un par de cursos de nivel superior, en las dos ramas de las matemáticas anteriormente nombradas.

El texto se divide en dos grandes secciones: el estudio de las familias casi ajenas (Capítulo 1), que constituye la parte \enquote{conjuntista} del escrito, y su contraparte topológica, dedicada al estudio de los espacios de Mrówka (Capítulos 2 a 4). En el Capítulo 1 se presentarán construcciones clásicas de familias casi ajenas y se expondrá la teoría básica de su combinatoria infinita asociada, incluyendo el Teorema de Simon, las familias de Luzin y el Lema de Solovay. El Capítulo 2 tiene como objetivo presentar el resultado ya mencionado obtenido por Kannan y Rajagopalan en este contexto, el entendimiento del comportamiento esencial de estos espacios resulta clave y constituye un aspecto central del capítulo. Los Capítulos 3 y 4 abordan problemas específicos que ponen de manifiesto la versatilidad de los protagonistas de esta tesis. En el Capítulo 3 se exhibe la relación entre la propiedad de Fréchet y la compactación unipuntual de los espacios de Isbell–Mrówka (el compacto de Franklin). Finalmente, en el Capítulo 4 se estudian en detalle los aspectos fundamentales para poder \enquote{traducir} la propiedad de normalidad, presentando la conjetura de Moore, su restricción a la clase de espacios separables y los resultados de Silver y Tall al respecto.
    \mainmatter
        \setcounter{chapter}{-1}
\chapter{Preeliminares}

\section{Conjuntos}

\index[sym]{$\zfc$}\index[sym]{$\zf$}\index[sym]{$\Ac$}\index[sym]{$\HC$}\index[sym]{$\Ma$}
El presente trabajo tiene su base en el sistema axiomático usual para la teoría de conjuntos; es decir $\zfc$, que es el marco $\zf$ junto con el axioma de elección $\Ac$. Ocasionalmente haremos referencia a axiomas adicionales, como lo son la hipótesis del continuo $\HC$, o el axioma de Martin $\Ma$. Los marcos axiomáticos previamente mencionados pueden ser consultados en \cite[p.~118]{jechSet}.

\index[sym]{$\ms{P}(X)$}\index[sym]{$f \upharpoonright B$}
Se utilizará la notación convencional para los predicados lógicos y las operaciones conjuntistas. Cuando $X$ sea un conjunto, se denotará por $\ms{P}(X)$ a su conjunto potencia. Si $f:A \to X$ es una función y $B \subseteq A$, se escribirá la restricción de $f$ hacia $B$ como $f \upharpoonright B$.

Como es estándar, trabajaremos sobre el universo de Von Neumann para la teoría de conjuntos. En este tenor, un conjunto $\alpha$ se dice \textit{número ordinal} cuando $\alpha \subseteq \ms{P}(\alpha)$ y $(\alpha,\in)$ es un conjunto bien ordenado \textcolor{blue}{(ver def 2)}. Cualquier conjunto no vacío de ordinales tiene $\in$-mínimo. Cuando $\alpha$ y $\beta$ sean ordinales, $\alpha < \beta$ significará $\alpha \in \beta$. Se denotará por $\omega$ al primer ordinal infinito. Es un hecho que $\omega=\mathbb{N}$, nótese que si $n$ es cualquier natural, $n=\{m \in \omega \tq m<n\}$. Un ordinal $\kappa$ es \textit{cardinal} cuando no es biyectable con ninguno de sus elementos. Utilizaremos la enumeración habitual para los cardinales $\aleph_\alpha=\omega_\alpha$. Es un hecho que cualquier conjunto $X$ es biyectable con un ordinal, el mínimo ordinal biyectable con $X$ es, de hecho, un cardinal y se denotará $|X|$; tal cardinal es \textit{la cardinalidad} (o el \textit{tamaño}) de $X$. La letra $\mathfrak{c}$ es el cardinal del \textit{continuo}, es decir, $\mathfrak{c}=|\mathbb{R}|=2^\aleph_0$ De manera puntual, se aludirá a la aritmética ordinal, la notación será la convencional para la suma, producto y exponenciación entre ordinales y cardinales. Se sugiere revisar \cite{jechSet} para un entendimiento más profundo del tema.

Dado un conjunto $X$ y un cardinal $\kappa$, se denotarán por $[X]^\kappa$ y $[X]^{<\kappa}$ a las colecciones de subconjuntos de $X$ de tamaño exactamente $\kappa$, y menor estricto que $\kappa$, respectivamente. De forma similar se definen: $[X]^{\leq \kappa}$, $[X]^{>\kappa}$ y $[X]^{\geq \kappa}$. Además, $X^\kappa$ es el conjunto de todas las funciones de $\kappa$ en $X$; y, $X^{<\kappa}$ es el conjunto de funciones $f$ para las que existe $\alpha < \kappa$ de manera que $f:\alpha \to X$.

Un subconjunto $C$ de un ordinal límite $\gamma$ es \textit{cerrado} cuando para cada $\alpha < \gamma$, si $\midcup (C \cap \alpha)=\alpha$, entonces $\alpha \in C$. Si $C$ es cerrado y no acotado en $\gamma$ \textcolor{blue}{(ver abajo)}, diremos que es un \textit{club} de $\gamma$. Un conjunto $S \subseteq \gamma$ es estacionario si y sólo si tiene intersección no vacía con cada club de $\gamma$. Sobre esta terminología se desenvuelve el \textit{Lema de Fodor}: Si $\kappa$ es un ordinal regular, $S$ es estacionario en $\kappa$ y $f:S \to \kappa$ es tal que para cada $\alpha \in S \setminus \{0\}$, $f(\alpha) < \alpha$; entonces, existe $T \subseteq \kappa$ estacionario tal que $f \upharpoonright T$ es constante.

\newpage

El presente trabajo se desarrolla dentro del sistema axiomático usual de la teoría de conjuntos, es decir, $\zfc$, entendido como el marco $\zf$ junto con el axioma de elección $\Ac$. Ocasionalmente haremos referencia a axiomas adicionales, tales como la hipótesis del continuo $\HC$ o el axioma de Martin $\Ma$. Los marcos axiomáticos previamente mencionados pueden ser consultados en \cite[p.~118]{jechSet}.

Se utilizará la notación convencional para los predicados lógicos y las operaciones conjuntistas. Dado un conjunto $X$, se denotará por $\ms{P}(X)$ a su conjunto potencia. Si $f:A \to X$ es una función y $B \subseteq A$, se escribirá la restricción de $f$ a $B$ como $f \upharpoonright B$.

Como es estándar, trabajaremos sobre el universo de Von Neumann para la teoría de conjuntos. En este contexto, un conjunto $\alpha$ se dice \textit{número ordinal} cuando $\alpha \subseteq \ms{P}(\alpha)$ y $(\alpha,\in)$ es un conjunto bien ordenado \textcolor{blue}{(ver def 2)}. Todo conjunto no vacío de ordinales tiene un $\in$-mínimo. Si $\alpha$ y $\beta$ son ordinales, escribiremos $\alpha < \beta$ para indicar que $\alpha \in \beta$. Denotaremos por $\omega$ al primer ordinal infinito; es un hecho que $\omega=\mathbb{N}$, y nótese que, si $n$ es cualquier número natural, entonces
\[
n=\{m \in \omega \tq m<n\}.
\]

Un ordinal $\kappa$ se dice \textit{cardinal} cuando no es biyectable con ninguno de sus elementos. Utilizaremos la enumeración habitual de los cardinales, denotando $\aleph_\alpha=\omega_\alpha$. Es un hecho que cualquier conjunto $X$ es biyectable con algún ordinal; el mínimo ordinal biyectable con $X$ es, en efecto, un cardinal, el cual se denotará por $|X|$ y se llamará la \textit{cardinalidad} (o el \textit{tamaño}) de $X$. La letra $\mathfrak{c}$ denota el cardinal del \textit{continuo}, es decir,
\[
\mathfrak{c}=|\mathbb{R}|=2^{\aleph_0}.
\]
De manera puntual, se hará referencia a la aritmética ordinal; la notación será la convencional para la suma, el producto y la exponenciación entre ordinales y cardinales. Para un tratamiento más detallado del tema, se sugiere consultar \cite{jechSet}.

Dado un conjunto $X$ y un cardinal $\kappa$, se denotará por $[X]^\kappa$ a la colección de todos los subconjuntos de $X$ de cardinalidad exactamente $\kappa$, y por $[X]^{<\kappa}$ a la colección de aquellos de cardinalidad estrictamente menor que $\kappa$. De forma análoga se definen $[X]^{\leq \kappa}$, $[X]^{>\kappa}$ y $[X]^{\geq \kappa}$. Además, $X^\kappa$ denota el conjunto de todas las funciones de $\kappa$ en $X$, mientras que $X^{<\kappa}$ es el conjunto de todas las funciones $f$ para las cuales existe algún $\alpha<\kappa$ tal que $f:\alpha \to X$.

Sea $\gamma$ un ordinal límite. Un subconjunto $C \subseteq \gamma$ se dice \textit{cerrado} si para todo $\alpha<\gamma$, cuando $\midcup (C \cap \alpha)=\alpha$, se tiene $\alpha \in C$. Si, además, $C$ es cerrado y no acotado en $\gamma$ \textcolor{blue}{(ver abajo)}, diremos que $C$ es un \textit{club} de $\gamma$. Un conjunto $S \subseteq \gamma$ se dice \textit{estacionario} si y sólo si tiene intersección no vacía con todo club de $\gamma$.

Con esta terminología se enuncia el \textit{Lema de Fodor}: si $\kappa$ es un ordinal regular, $S$ es un subconjunto estacionario de $\kappa$ y $f:S \to \kappa$ es tal que, para cada $\alpha \in S \setminus \{0\}$, se cumple $f(\alpha)<\alpha$, entonces existe un conjunto estacionario $T \subseteq \kappa$ tal que la restricción $f \upharpoonright T$ es constante.

\newpage

El presente trabajo se desarrolla dentro del sistema axiomático usual de la teoría de conjuntos, a saber, $\zfc$, entendido como el marco $\zf$ junto con el axioma de elección $\Ac$. Ocasionalmente haremos referencia a axiomas adicionales, tales como la hipótesis del continuo $\HC$ o el axioma de Martin $\Ma$. Los marcos axiomáticos previamente mencionados pueden consultarse en \cite[p.~118]{jechSet}.

A lo largo del texto se empleará la notación convencional para los predicados lógicos y las operaciones conjuntistas. Dado un conjunto $X$, se denotará por $\ms{P}(X)$ a su conjunto potencia. Si $f:A \to X$ es una función y $B \subseteq A$, se escribirá la restricción de $f$ a $B$ como $f \upharpoonright B$. Identificaremos cualquier fórmula $\varphi$ de la teoría de conjuntos con una \textit{clase}, esto es, una colección $\{ x \tq \varphi \}$.

Como es estándar, trabajaremos sobre el universo de Von Neumann para la teoría de conjuntos. En este contexto, un conjunto $\alpha$ se denomina \textit{número ordinal} cuando $\alpha \subseteq \ms{P}(\alpha)$ y $(\alpha,\in)$ es un conjunto bien ordenado \textcolor{blue}{(ver def 2)}. Todo conjunto no vacío de ordinales posee un $\in$-mínimo. Dados ordinales $\alpha$ y $\beta$, escribiremos $\alpha < \beta$ para indicar que $\alpha \in \beta$. Denotaremos por $\omega$ al primer ordinal infinito; es un hecho que $\omega=\mathbb{N}$. Nótese, además, que si $n$ es cualquier número natural, entonces
\[
n=\{m \in \omega \tq m<n\}.
\]

Un ordinal $\kappa$ se dice \textit{cardinal} cuando no es biyectable con ninguno de sus elementos. Utilizaremos la enumeración habitual de los cardinales, denotando $\aleph_\alpha=\omega_\alpha$. Es un hecho que todo conjunto $X$ es biyectable con algún ordinal; el mínimo ordinal biyectable con $X$ es, de hecho, un cardinal, el cual se denotará por $|X|$ y se llamará la \textit{cardinalidad} (o el \textit{tamaño}) de $X$. La letra $\mathfrak{c}$ denota el cardinal del \textit{continuo}, es decir,
\[
\mathfrak{c}=|\mathbb{R}|=2^{\aleph_0}.
\]
De manera puntual, se hará referencia a la aritmética ordinal, utilizando la notación convencional para la suma, el producto y la exponenciación entre ordinales y cardinales. Para un tratamiento más detallado del tema se sugiere consultar \cite{jechSet}.

Dado un conjunto $X$ y un cardinal $\kappa$, se denotará por $[X]^\kappa$ la colección de todos los subconjuntos de $X$ de cardinalidad exactamente $\kappa$, y por $[X]^{<\kappa}$ la colección de aquellos cuya cardinalidad es estrictamente menor que $\kappa$. De forma análoga se definen las colecciones $[X]^{\leq \kappa}$, $[X]^{>\kappa}$ y $[X]^{\geq \kappa}$. Además, $X^\kappa$ denota el conjunto de todas las funciones de $\kappa$ en $X$, mientras que $X^{<\kappa}$ es el conjunto de todas las funciones $f$ para las cuales existe algún $\alpha<\kappa$ tal que $f:\alpha \to X$.

Sea $\gamma$ un ordinal límite. Un subconjunto $C \subseteq \gamma$ se dice \textit{cerrado} si, para todo $\alpha<\gamma$, se cumple que, cuando $\midcup (C \cap \alpha)=\alpha$, entonces $\alpha \in C$. Si $C$ es cerrado y no acotado en $\gamma$ \textcolor{blue}{(ver abajo)}, diremos que $C$ es un \textit{club} de $\gamma$. Un conjunto $S \subseteq \gamma$ se denomina \textit{estacionario} si y sólo si tiene intersección no vacía con todo club de $\gamma$.

Con esta terminología se enuncia el \textit{Lema de Fodor}: si $\kappa$ es un ordinal regular, $S$ es un subconjunto estacionario de $\kappa$ y $f:S \to \kappa$ es tal que, para cada $\alpha \in S \setminus \{0\}$, se cumple $f(\alpha)<\alpha$, entonces existe un conjunto estacionario $T \subseteq \kappa$ tal que la restricción $f \upharpoonright T$ es constante.

        \chapter{Familias casi ajenas}
\emph{\small Las familias casi ajenas (\enquote{almost disjoint families}, en inglés) son objetos fascinantes de la teoría de conjuntos; como se verá a lo largo de este trabajo, sus aplicaciones no sólo se limitan a esta rama de las matemáticas, sino que sus repercusiones se extienden a la topología. Entre los pioneros de su estudio destacan enormes personajes, entre ellos Felix Hausdorff, Wacław Sierpiński, Kazimierz Kuratowski, Eduard Čech y Stefan Banach.}

\emph{\small En lo que sigue, se presentarán estos objetos y se expondrán sus propiedades más inmediatas; algunos métodos para su construcción; y finalmente, un estudio básico sobre su combinatoria. A lo largo de esta última sección, se abordarán las pruebas de algunos resultados clásicos, principalmente: los Lemas de Dočkálková y de Solovay, el Teorema de Simon y la existencia de las familias de Luzin. Por último, se darán algunas aplicaciones de esta teoría sobre el comportamiento de los cardinales $\mathfrak{c}$, $\mathfrak{m}$ y $\mathfrak{a}$.}

\section{Observaciones inmediatas}

\begin{definicion}\phantomsection\label{def-casi-ajena}\index[alph]{casi!ajeno}\index[alph]{casi!ajena sobre $N$, familia}\index[alph]{familia!casi ajena}\index[alph]{casi!ajena, familia}\index[alph]{familia!casi ajena sobre $N$}\index[sym]{$\Ad(N)$}
	Sea $N$ un conjunto numerable. Una \textbf{familia casi ajena sobre $N$} es un subconjunto $\ms{A}\subseteq[N]^\omega$ cuyos elementos son casi ajenos por pares. $\Ad(N)$ es el conjunto de todas las familias casi ajenas sobre $N$.
	
	El término \textbf{familia casi ajena} (o simplemente \textbf{familia}) hará referencia a una familia casi ajena sobre $\omega$.
\end{definicion}

El concepto previo es fácilmente generalizable, el lector puede indagar al respecto en \cite[Def.~9.20, p.~118]{jechSet}. Sin embargo, la teoría asociada a las familias casi ajenas, definidas como en \ref{def-casi-ajena}, es suficientemente amplia y meritoria de un estudio dedicado.

Cualquier familia de subconjuntos ajenos por pares de $N$, es también una familia casi ajena sobre $N$; particularmente, $\emptyset$ y cualquier colección de la forma $\{A\}$, con $A \in [N]^\omega$. Además, resulta evidente que cada subconjunto de una familia casi ajena sobre $N$ es, a su vez, una familia casi ajena sobre $N$.

Es claro que toda familia casi ajena tiene tamaño menor o igual a $\mathfrak{c}$; así que en virtud de lo previamente observado, de existir alguna de ellas de tamaño el continuo, se garantizaría la existencia de familias ajenas de cualquier tamaño inferior a $\mathfrak{c}$.

\begin{ejemplo}
	\phantomsection\label{ej-ADfacil}
	Las siguientes colecciones: $\{\omega\}$, $\{ \{ 2n \tq n \in \omega \}, \{ 2n+1 \tq n \in \omega \} \}$ y $\{ \{ p^n \tq n \in \omega \setminus \{0\} \} \tq p \text{ es primo} \}$ son familias casi ajenas sobre $\omega$.
\end{ejemplo}

No resulta muy difícil verificar que las primeras dos familias del ejemplo anterior son \enquote{grandes}, en el siguiente sentido:

\begin{definicion}\index[alph]{familia!casi ajena sobre $N$!maximal en $N$}\index[sym]{$\Mad(N)$}\index[alph]{familia!casi ajena!maximal}
	Sea $N$ conjunto numerable. Una familia $\ms{A}$ sobre $N$ es \textbf{maximal en} $N$ si es $\subseteq$-maximal del conjunto $\Ad(N)$. Se denotará por $\Mad(N)$ al conjunto de todas ellas.
	
	Cuando no haya riesgo de ambigüedad, el término \textbf{familia maximal} hará referencia a una familia maximal en $\omega$.
\end{definicion}

Dado que los elementos de toda familia casi ajena son infinitos, se tiene inmediatamente la siguiente observación:

\begin{observacion}
	Una familia $\ms{A} \in \Ad(N) $ es maximal en $N$ si y sólo si se cumple cualquiera de las siguientes condiciones equivalentes:
	\begin{enumerate}
		%\item Para toda $\ms{B} \in \Ad(N)$, si $\ms{A} \subseteq \ms{B}$, entonces $\ms{A} = \ms{B}$
		\item Para cualquier $\ms{B} \subseteq [N]^\omega$, si $\ms{A} \subsetneq \ms{B}$, entonces $\ms{B} \notin \Ad(N)$.
		\item Para cada $B \in [N]^\omega$ existe $A \in \ms{A}$ tal que $A \cap B$ es infinito.
	\end{enumerate}
\end{observacion}

Se advierte que las familias sobre $\omega$ parecerán deslucir a las construidas sobre otros conjuntos numerables; pero al no ser el estudio sobre estas últimas nulo, es menester considerar las propiedades que son transferibles entre estas dos clases de objetos.

\begin{definicion}\phantomsection\label{def-Biyecs-h}\index[sym]{$\Phi_h$}
	Sean $N,M$ conjuntos numerables y $h \colon N \to M$ cualquier biyección. Se define $\Phi_h  \colon  \ms{P}(\ms{P}(N)) \to \ms{P}(\ms{P}(M))$ como:
	\[ \Phi_h (\ms{A}) = \{ h[A] \tq A \in \ms{A} \} \, . \]
\end{definicion}

En términos de lo recién definido, se remarca que al ser $h$ biyección, $\Phi_h$ será una biyección. Más aún, estas biyecciones se comportan bien respecto a ciertas virtudes conjuntistas, tal y como se ilustra a continuación.

\begin{proposicion}\phantomsection\label{prop-ADbiyec}
	Sean $N,M$ son numerables y $h \colon N \to M$ una biyección cualquiera. Entonces:
	\begin{enumerate}
		\item $\ms{A} \subsetneq \ms{B}$ si y sólo si $\Phi_h(\ms{A}) \subsetneq \Phi_h(\ms{B})$.
		\item $\Phi_h(\ms{A} \cap \ms{B}) = \Phi_h(\ms{A}) \cap \Phi_h(\ms{B})$.
		\item $\Phi_h(\ms{A} \cup \ms{B}) = \Phi_h(\ms{A}) \cup \Phi_h(\ms{B})$.
		\item $|\ms{A}|=|\Phi_h(\ms{A})|$.		
		\item $\Phi_h[\Ad(N)]=\Ad(M)$.
		\item $\Phi_h[\Mad(N)]=\Mad(M)$
	\end{enumerate}
\end{proposicion}
\begin{proof}
	Se mostrarán únicamente (v) y (vi). En ambos basta probar la contención directa, pues al ser $h$ biyección, $\Phi_h^{-1} = \Phi_{h^{-1}}$.

	(v) Si $\ms{A} \in \Ad(N)$, entonces $\ms{A} \subseteq [N]^\omega$ y así $\Phi_h(\ms{A}) \subseteq [M]^\omega$. Ahora, si $h[A],h[B] \in \Phi_h(\ms{A})$ son distintos, es necesario que $A \neq B$ y por ello $A \cap B =^* \emptyset$. Se obtiene que $h[A] \cap h[B]=h[A \cap B]=^* \emptyset $, y con ello, $\Phi_h(\ms{A}) \in \Ad(M)$.

	(vi) Si $\ms{A} \in \Mad(\ms{A})$ y $B \subseteq M$ es infinito, entonces $h^{-1}[B] \subseteq N$ es infinito y existe $A \in \ms{A}$ tal que $A \cap h^{-1}[B]$ es infinito. Al ser $h$ biyección, $h[A \cap h^{-1}[B]]=h[A] \cap B$ es infinito, por ende $\Phi_h(\ms{A}) \in \Mad(M)$.
\end{proof}

A partir de este momento se consolida la usanza de hacer hincapié en qué propiedades, u objetos, basados en familias casi ajenas se preservan bajo las biyecciones $\Phi_h$.

Una aplicación superflua de lo anterior es el nacimiento de un método cómodo para generar familias casi ajenas; en especial, infinitas.

\begin{ejemplo}
	\phantomsection\label{ej-Bandas}
	Claramente $\ms{A}=\{ \{n\} \times \omega \tq n \in \omega \} \in \Ad(\omega \times \omega)$. Así que si $h \colon \omega \times \omega \to \omega$ es biyección, $\Phi_h(\ms{A}) \in \Ad(\omega)$. Más aún, tal familia es del mismo tamaño que $\ms{A}$ (todo gracias a \ref{prop-ADbiyec}).
\end{ejemplo}

A continuación se comenzarán a examinar las propiedades de las familias casi ajenas maximales; se tiene la intención de responder a las preguntas que surgen naturalmente como: ¿puede haber familias casi ajenas más que numerables?, o, ¿existen familias maximales infinitas?

\begin{lema}\phantomsection\label{lem-MADnecesarioUnion}
	Si $\ms{A}$ es familia casi ajena maximal, entonces $\omega \subseteq^* \midcup \ms{A}$.
\end{lema}

\begin{proof}
	Por contrapuesta, supóngase que $B \mycoloneq \omega \setminus \midcup \ms{A}$ es infinito. Si $A \in \ms{A}$, entonces $A \subseteq \midcup \ms{A}$, y así $A \cap B \subseteq A \setminus \midcup \ms{A} \subseteq A \setminus A = \emptyset$. Por lo que $B \in [\omega]^\omega$ es casi ajeno con cada elemento de $\ms{A}$.
\end{proof}

El recíproco del Lema previo falla para familias infinitas (véase la familia $\Psi(\ms{A})$ del \cref{ej-Bandas}); y de hecho, no se cuenta un resultado \enquote{amigable} para determinar cuándo estas resultan ser maximales (véase \ref{prop-CaracMADPositiv}). En contraparte a esto, se deduce rápidamente la siguiente equivalencia:

\begin{corolario}\phantomsection\label{cor-MADnecesarioUnion}
	Sea $\ms{A} \in \Ad(\omega)$. $\ms{A}$ es maximal si y sólo si $\omega \subseteq^* \midcup \ms{A}$.
\end{corolario}

\begin{proof}
	Por el Lema previo, basta demostrar la necesidad.

	Supóngase $\omega \subseteq^* \midcup \ms{A}$ y nótese que si $B \in [\omega]^\omega$, entonces $B \subseteq^*\midcup \ms{A}$ y con ello $\emptyset \neq^* B  \subseteq^* B \cap \midcup \ms{A} = \midcup\{B \cap A \tq A \in \ms{A}\}$. Como la última es una unión finita, $B$ debe tener intersección infinita con algún elemento de $\ms{A}$.
\end{proof}

%El posterior resultado es un ejemplo típico de aplicación del Principio de Maximalidad de Hausdorff (\ref{teo-PMO}).  o sus equivalentes.

\begin{lema}\phantomsection\label{lem-MADs}
	Toda familia casi ajena se extiende a una familia maximal.
\end{lema}

\begin{proof}
	Sean $\ms{A} \in \Ad(\omega)$ y $X$ el conjunto de todos las familias casi ajenas que contienen a $\ms{A}$. Como $A \in X$, por el Principio de Maximalidad de Hausdorff ($\Ac$), existe $Y \subseteq X$, una cadena $\subseteq$-maximal de $(X,\subseteq)$.

	Defínase $\ms{B} \mycoloneq \midcup Y$, como $Y \subseteq \ms{P}([\omega]^\omega)$, entonces $\ms{B} \subseteq [\omega]^{\omega}$. Además, si $C,D \in \ms{B}$, existen $\ms{C},\ms{D} \in Y \subseteq \Ad(\omega)$ con $C \in \ms{C}$ y $D\in \ms{D}$. Puesto que $Y$ es cadena de $(X,\subseteq)$, sin pérdida de generalidad, $C,D \in \ms{D} \supseteq \ms{C}$; y con ello, $C \cap D$ es finito, ya que $\mathscr{D} \in Y \subseteq X \subseteq \Ad(\omega)$. Luego $\ms{B} \in \Ad(\omega)$, y además, $\ms{A} \subseteq \ms{B}$.

	Finalmente, si $\ms{B}' \in \Ad(\omega)$ y $\ms{B} \subseteq \ms{B}'$, entonces $Y \cup \{\ms{B}'\}$ es una cadena de $(X,\subseteq)$; lo cual, junto a la maximalidad de $Y$, implica que $\mathscr{B}' \in Y$ y $\mathscr{B} = \mathscr{B}'$. Por lo tanto, $\ms{B}\in \Mad(\omega)$.
\end{proof}

El siguiente resultado revela un fenómeno interesante respecto al tamaño de las familias maximales. Este se le atribuye a Wacław Sierpiński (se desprende de \cite[Teo.~2, p.~458]{SierpinskiCardinal}).

\begin{lema}\phantomsection\label{prop-MADnoNum}
	Ninguna familia casi ajena numerable es maximal.
\end{lema}

\begin{proof}
	Sea $\ms{A} \in \Ad(\omega)$ enumerada por $\ms{A}=\{A_n \tq n \in \omega\}$. Si $n \in \omega$ es cualquiera, $A_n \cap \midcup \{A_m \tq m< n\} = \midcup \{A_n \cap A_m \tq m<n\}$ es finito, al ser unión finita de conjuntos finitos. Luego, por ser $A_n$ infinito, $A_n \setminus \midcup \{A_m \tq m<n\} = A_n \setminus \big( A_n \cap \midcup \{A_m \tq m<n\} \big)$ debe ser infinito; y particularmente, no no vacío.

	Considérese $f \colon \omega \to \omega$ definida por $f(n) = \min\{A_n \setminus \midcup \{A_m \tq m<n\}\}$ para cada $n$. Resulta que $f$ es inyectiva; si $m<n$, entonces $f(n) \notin A_m$, $f(m) \in A_m$ y $f(n) \neq f(m)$. Además, para cada $n \in \omega$ se tiene que $A_n \cap f[\omega] = \{f(n)\}$; y con ello $f[\omega] \subseteq \omega$ es infinito y casi ajeno con cada elemento de $\ms{A}$.
\end{proof}

Tomando cualquier familia infinita $\ms{A}$ y aplicando \cref{lem-MADs}, se obtiene una familia maximal $\ms{B} \supseteq \ms{A}$ de tamaño infinito. Por el resultado anterior, tal infinito debe ser más que numerable. Esta consecuencia es tan inmediata como, quizás, poco satisfactoria; pues su naturaleza es \enquote{no constructiva}. Durante la posterior sección se mostrarán métodos para obtener estos últimos objetos de una manera más explícita.

\begin{observacion}\phantomsection\label{obs-ExisteNoNumMAD}
	Existe una familia maximal de tamaño al menos $\aleph_1$.
\end{observacion}

\section{Familias casi ajenas de tamaño \texorpdfstring{$\mathfrak{c}$}{c}}

%La presente sección tiene por meta exhibir dos de los métodos más típicos para la construcción de familias casi ajenas infinitas. El primero de ellos, se basa en las sucesiones convergentes de espacios topológicos de Hausdorff, primero numerables.

Al tomar un espacio de Fréchet $X$ y cualquier subespacio denso $D \subseteq X$, para cada $x \in X \setminus D$ ha de existir una sucesión en $D$ convergente a $x$. Si a $X$ le adicionamos la condición de ser $\T_1$, la imagen de tal sucesión es necesariamente un conjunto numerable $A_x \subseteq D$, convergente a $x$ en $X$ (véase \ref{prop-sucesionesConvergentes}).

\begin{proposicion}\phantomsection\label{prop-famSucesiones}
	Supóngase que $X$ es un espacio de Hausdorff, de Fréchet y que $D \subseteq X$ es denso y numerable. Para cada $A \subseteq X \setminus D$ existe una familia casi ajena sobre $D$ biyectable con $A$.
\end{proposicion}

\begin{proof}
	Con sustento en los comentarios previos, para cada $x \in A$ fíjese ($\Ac$) un conjunto $A_x \subseteq D$ numerable y convergente a $x$ en $X$. Defínase la colección $\ms{A}_{D,A}$ como $\{ A_x \subseteq D \tq x \in A \} \subseteq [D]^\omega$.

	Sean $x,y \in A$ con $x \neq y$, por ser $X$ de Hausdorff, hay abiertos ajenos $U,V$ tales que $x \in U$ y $y \in V$. Seguido de que $A_x \to x$ y $A_y \to y$, se tiene $A_x \subseteq^* U$ y $A_y \subseteq ^* V$; consecuentemente $A_x \cap A_y \subseteq^* U \cap V = \emptyset$. Lo cual prueba que $\ms{A}_{D,A} \in \Ad(D)$ y $|\ms{A}_{D,A}|=|A|$.
\end{proof}

\begin{definicion}\phantomsection\label{def-FamSucesiones}\index[alph]{familia!de!sucesiones en $D$ convergentes a $A$ en $X$}\index[sym]{$\ms{A}_{D,A}$}
	Si $X$, $D$ y $A$ son como en la Proposición anterior, a $\ms{A}_{D,A}$ se le denomina \textbf{familia de sucesiones en $D$ convergentes a $A$ en $X$}.
\end{definicion}

Como la recta real $\mathbb{R}$ es de Hausdorff, de Fréchet (por ser $1\AN$) y $\mathbb{Q} \subseteq \mathbb{R}$ es un subespacio denso numerable; de lo previamente establecido se obtiene que $\ms{A}_{\mathbb{Q},\mathbb{R}\setminus \mathbb{Q}}$ es una familia casi ajena sobre $\mathbb{Q}$ de tamaño $\mathfrak{c}=|\mathbb{R} \setminus \mathbb{Q}|$. Conviene destacar que la construcción recién mencionada no depende del $\Ac$; pues para cada $x \in \mathbb{R} \setminus \mathbb{Q}$, el conjunto $A_x$ de la \cref{prop-famSucesiones} se puede construir de manera explícita. Si $q \colon \omega \to \mathbb{Q}$ es cualquier biyección, basta considerar:
\[ A_x = \Big\{ q \Big( \min \Big( q^{-1} \Big[ \mathbb{Q} \cap \Big( x-\frac{1}{n+1}, x-\frac{1}{n} \Big) \Big] \Big) \Big) \, \Big| \, n \in \omega \setminus \{0\} \Big\} \, . \]



El corazón de la próxima estrategia para la obtención explícita de familias casi ajenas de tamaño el continuo, son los árboles.

Comenzaremos observando que si $S$ es cualquier rama de un árbol $(T,\leq)$, entonces $S$ es cerrada bajo cotas inferiores. En efecto, sean $x \in S$ y $y \leq x$. Para cada $s \in S$, al ser $S$ cadena, se tiene que $x \leq s$ o $s < x$. En el primer caso, $y \leq x \leq s$ y $y$ es comparable con $s$. En el segundo $y,s < x$; y como $(\{ y \in T \tq y < x \},\leq)$ es buen orden, $y$ y $s$ son comparables. Por lo tanto, $S \cup \{y\}$ es una cadena; y seguido de que $S$ es rama, $y \in S$.

\begin{proposicion}
	Sea $(T,\leq)$ un árbol numerable de altura $\omega$ y $\ms{A} \subseteq [T]^\omega$ el conjunto de todas las ramas numerables de $(T,\leq)$. Entonces $\ms{A} \in \Ad(T)$.
\end{proposicion}

\begin{proof}
	Sean $R,S \in \ms{A}$ con $R \neq S$, sin pérdida de generalidad, existe $x_0 \in R \setminus S \neq \emptyset$. De existir $y \in R \cap S$ tal que $y \not < x_0$, resultaría que $x_0 \leq y$, en virtud de que $x,y \in R$ y $R$ es rama. Lo anterior y la discusión previa a esta Proposición implican que $x_0 \in S$, lo cual es imposible.
	
	Por lo tanto $R \cap S \subseteq \{ y \in T \tq y<z \}$; y como $T$ tiene altura $\omega$, el orden de $x_0$ es un natural; consecuentemente, $R \cap S$ es finito.
\end{proof}

Un ejemplo canónico de árbol numerable de altura $\omega$ es $2^{<\omega}$ (\cref{2fin-es-arbol}); considerar la siguiente clase de familias desembocará en resultados sumamente notables (como se puede ver en la \cref{Sec-PDM}).

\begin{proposicion}
	Para cada $f \in 2^\omega$ defínase $A_f \mycoloneq \{ f \upharpoonright n \tq n \in \omega \} \subseteq 2^{<\omega}$; entonces:
	\begin{enumerate}
		\item Cada $A_f$ es una rama de $(2^{<\omega},\subseteq)$.
		\item Si $f\neq g$, entonces $A_f \neq A_g$.
	\end{enumerate}
\end{proposicion}
\begin{proof}
	(i) Sea $f \in 2^\omega$, inmediatamente, $A_f$ es cadena de $(2^{<\omega},\subseteq)$. Supóngase ahora que $S \subseteq 2^{<\omega}$ es una rama de $(2^{<\omega},\subseteq)$ y que $A_f \subseteq S$. 
	
	Sean $g \in S$ y $n=\dom(g)$; puesto que $S$ es cadena de $(2^{<\omega},\subseteq)$ y $f \in A_f \subseteq S$, entonces $f \upharpoonright n \subseteq g$ o $g \subseteq f \upharpoonright n$. Cualquiera de los casos anteriores implican que $f \upharpoonright n = g$, pues $\dom(g)=\dom(f \upharpoonright n)$. Luego, $g \in A_f$ y entonces $A_f = S$.

	(ii) Si $f \neq g$, entonces existe $m\in \omega$ tal que $f(m) \neq g(m)$. Así, se obtiene que $f \upharpoonright m+1 \neq g \upharpoonright m+1$ y $f \upharpoonright m+1 \in R_f \setminus R_g$, por lo que $R_f \neq R_g$.
\end{proof}

Las proposiciones anteriores permiten definir, de forma explícita y sin recurrir al $\Ac$, el siguiente tipo de familias casi ajenas.

\begin{definicion}\phantomsection\label{def-FamRamas}\index[sym]{$\ms{A}_X$}\index[alph]{familia!de!ramas de $X$ en $2^\omega$}
	Para cada $X \subseteq 2^\omega$ defínase $\ms{A}_X \mycoloneq \{A_f \tq f \in X\}$ como en la Proposición previa.

	Esta familia será nombrada la \textbf{familia de las ramas de $X$ en $2^{<\omega}$}.
\end{definicion}

En paralelo a lo comentado después de \ref{def-FamSucesiones}, también se puede concluir vía la construcción recién expuesta, y el \cref{lem-MADs}, lo siguiente:

\begin{corolario}\phantomsection\label{cor-famGrandes}
	Existe una familia maximal de cardinalidad $\mathfrak{c}$.

	En consecuencia, para cualquier cardinal $\lambda \leq \mathfrak{c}$ existe una familia casi ajena de cardinalidad $\lambda$.
\end{corolario}

Se concluirá esta sección comentando qué ocurre respecto una interrogante surge naturalmente tras todo lo realizado: ¿existen familias maximales de cualquier cardinalidad entre $\aleph_1$ y $\mathfrak{c}$?

\begin{definicion}\index[alph]{cardinal! de casi ajenidad}\index[alph]{casi!ajenidad, cardinal de}\index[sym]{$\mathfrak{a}$}
	El \textbf{cardinal de casi ajenidad}, $\mathfrak{a}$, es el mínimo cardinal infinito $\kappa$ tal que existe una familia maximal de tamaño $\kappa$.
\end{definicion}

Debido a \textbf{\ref{prop-MADnoNum}}, se tiene $\aleph_1 \leq \mathfrak{a} \leq \mathfrak{c}$ y claramente bajo $\HC$ se debe satisfacer que $\mathfrak{a}=\mathfrak{c}$; luego, es consistente con $\zfc$ que $\mathfrak{a}=\mathfrak{c}$. Comentar que la teoría en relación al cardinal $\mathfrak{a}$ es basta y existen resultados de consistencia como el enunciado en seguida:

\begin{teorema}\phantomsection\label{teo-stafa}
	Si $\aleph_1 \leq \kappa \leq \mathfrak{c}$ y $\kappa$ es cardinal regular, es consistente con $\zfc$ que $\mathfrak{a}=\kappa$.
\end{teorema}

El Teorema recién enunciado consecuencia de \cite[Teo.~5.1, p.~127]{kunenHandbook}; y si bien su demostración escapa a los propósitos del presente texto, sí se expondrán resultados en relación a la igualdad $\mathfrak{a}=\mathfrak{c}$ más adelante (véase \ref{cor-MaSimple}).

\section{El ideal generado y su comportamiento}
\phantomsection\label{Sec-IdealGenerado}
\index[alph]{ideal!generado por $\ms{A}$}\index[sym]{$\ms{I}_N(\ms{A})$}\index[alph]{parte!positiva de $\ms{A}$}\index[sym]{$\ms{I}_N^+(\ms{A})$}\index[sym]{$\ms{I}(\ms{A})$}\index[sym]{$\ms{I}^+(\ms{A})$}
\begin{definicion}\phantomsection\label{def-ideal}
	Sean $N$ un conjunto numerable y $\ms{A} \in \Ad(N)$.
	\begin{enumerate}[i)]
		\item El \textbf{ideal generado por $\ms{A}$} es el conjunto $\ms{I}_N(\ms{A})$; definido como la colección $\{ B \subseteq N \tq \exists H \in [\ms{A}]^{<\omega} \: ( B \subseteq^* \midcup H ) \} $.
		\item La \textbf{parte positiva de $\ms{A}$} es $ \ms{I}_N^+(\ms{A})  \mycoloneq  \ms{P}(N) \setminus \ms{I}_N(\ms{A})$.
	\end{enumerate}
	Si $N=\omega$, estos conjuntos se denotarán por $\ms{I}(\ms{A})$ y $\ms{I}^+(\ms{A})$, respectivamente.
\end{definicion}

El objeto introducido previamente es de vital importancia para el estudio de la combinatoria de las familias casi ajenas. Como se había advertido anteriormente; con el propósito de no perder generalidad en los resultados expuestos durante esta sección, es necesario notar lo siguiente:

\begin{proposicion}\phantomsection\label{prop-IdealBiyec}
	Sean $N,M$ conjuntos numerables y $h \colon N \to M$ biyectiva. Si $\ms{A} \in \Ad(N)$, entonces $\Phi_h (\ms{I}_N (\ms{A})) = \ms{I}_M (\Phi_h( \ms{A} )) $.
\end{proposicion}

\begin{proof}
	Como $h$ es biyección, $\Phi_h^{-1} = \Phi_{h^{-1}}$. Por lo cual, basta probar la contención directa de la igualdad.
	
	Sea $Y \in \ms{I}_N (\ms{A})$, entonces existe $H \subseteq \ms{A}$ finito tal que $Y \setminus \midcup H$ es finito. Como $h$ es biyectiva, $ h \big[ Y \setminus \midcup H \big] = h[Y] \setminus h\big[ \midcup H \big] = h[Y] \setminus \midcup \Phi_h(H)$ es finito. Además, de \ref{prop-ADbiyec}, $\Phi_h(H) \subseteq \Phi_h(\ms{A})$ es finito. Por ello, $h[Y] \in \ms{I}_M (\Phi_h( \ms{A}))$.
\end{proof}

Resulta sencillo constatar que el objeto definido en \ref{def-ideal} es; como su nombre indica, un ideal sobre $\ms{P}(\omega)$. Además, se destaca lo siguiente:

\begin{observacion}\phantomsection\label{obs-IdealPrevia}
	Sea $\ms{A}$ cualquier familia casi ajena.
	\begin{enumerate}[i)]
		\item Cualquier subconjunto finito de $\omega$, así como cualquier elemento de $\ms{A}$, es elemento de $\ms{I}(\ms{A})$. Por lo que $\emptyset \subsetneq [\omega]^{<\omega} \cup \ms{A} \subseteq \ms{I}(\ms{A})$.
		\item Si $\ms{B} \in \Ad(\omega)$ y $\ms{A} \subseteq \ms{B}$, entonces $\ms{I}(\ms{A}) \subseteq \ms{I}(\ms{B})$.
	\end{enumerate}
\end{observacion}

Obsérvese adempás que si $\ms{A} \in \Mad(\omega)$ es finita; en virtud del \cref{lem-MADnecesarioUnion} se tendrá que $\omega \in \ms{I}(\ms{A})$, pues $\ms{A} \subseteq \ms{A}$ es finito y $\omega \subseteq^* \midcup \ms{A}$. El recíproco de esto también es cierto.

\begin{proposicion}
	Sea $\ms{A} \in \Ad(\omega)$ cualquiera. Si $\omega \in \ms{I}(\ms{A})$, entonces $\ms{A}$ es maximal y finita.
\end{proposicion}

\begin{proof}
	Supóngase que $\omega \in \ms{I}(\ms{A})$, entonces existe $H \subseteq \ms{A}$ finito con $\omega \subseteq^* \midcup H \subseteq \midcup \ms{A}$. Por \ref{cor-MADnecesarioUnion}, basta ver que $\ms{A}$ es finita.

	Supóngase que existe $B \in \ms{A} \setminus H$. Así, $B$ casi ajeno con cada elemento de $H \subseteq \ms{A}$, luego, $B \cap \midcup H = \midcup \{B \cap h \tq h \in H \}$ es finito. De lo anterior, $B = B \cap \omega \subseteq^* B \cap \midcup H$ es finito, lo cual es imposible. Por lo tanto, $\ms{A} \subseteq H$ y $\ms{A}$ es finita.
\end{proof}

\begin{corolario}\phantomsection\label{cor-IdealPropioCaract}
	Sean $N$ un conjunto numerable y $\ms{A} \in \Ad(N)$. Las siguientes condiciones son equivalentes:
	\begin{enumerate}[i)]
		\item El ideal $\ms{I}_N(\ms{A})$ no es propio, es decir, $N \in \ms{I}_N(\ms{A})$.
		\item $\ms{A}$ es finita y maximal en $N$.
	\end{enumerate}
\end{corolario}

Con relativa frecuencia aparecerán familias que, pese a no ser maximales, satisfacen la condición (ii) de lo subsecuente; esta puede ser tomada como un debilitamiento a la condición de maximalidad.

\begin{definicion}\phantomsection\label{def-MaxEnAlguna}\index[alph]{traza de $\ms{A}$ en $X$}\index[sym]{$\ms{A} \upharpoonright X$}\index[alph]{familia!maximal en alguna parte}\index[alph]{familia!maximal en ninguna parte}
	Sean $N$ un conjunto numerable y $\ms{A} \in \Ad(N)$.
	\begin{enumerate}[i)]
		\item Para cada $X \in [N]^\omega$, la \textbf{traza de $\ms{A}$ en $X$} es la colección $ \ms{A} \upharpoonright X$, definida como el conjunto $\{ A \cap X \in [X]^\omega \tq A \in \ms{A} \} $.
		\item $\ms{A}$ es \textbf{maximal en alguna parte} si y sólo si existe $X \in \ms{I}_N^+(\ms{A})$ tal que la familia $\ms{A} \upharpoonright X$ es maximal en $X$.
		\item $\ms{A}$ es \textbf{maximal en ninguna parte} si no es maximal en alguna parte.
	\end{enumerate}
\end{definicion}

Si $X \in [N]^\omega$, entonces $\ms{A} \upharpoonright X \in \Ad(X)$. Más aún, si $\ms{A}$ es maximal, para cada $B \subseteq X$ infinito existe $A \in \ms{A}$ tal que $A \cap B = (A \cap X) \cap (B \cap X)$ es infinito, mostrando la maximalidad de $\ms{A} \upharpoonright X$. Así que, efectivamente, la definición anterior es un debilitamiento de la condición de maximalidad.

Sin causa de asombro, los conceptos recién establecidos son respetados por las biyecciones $\Phi_h$.

\begin{proposicion}
	Sean $N,M$ conjuntos numerables, $h \colon N \to M$ biyección y $\ms{A} \in \Ad(N)$.
	\begin{enumerate}[i)]
		\item Para cada $X \in [N]^\omega$ se cumple $\Phi_h(\ms{A}) \upharpoonright h[X]=\Phi_h(\ms{A} \upharpoonright X)$.
		\item $\ms{A}$ es maximal en alguna parte si y sólo si $\Phi_h(\ms{A})$ también lo es.
	\end{enumerate}
\end{proposicion}

\begin{proof}
	(i) Como $h$ biyección, de la definición de $\Phi_h(\ms{A})$ se obtiene:
	\begin{align*}
		\Phi_h(\ms{A}) \upharpoonright h[X] & = \{ B \cap h[X] \in \big[ h[X] \big]^\omega \tq B \in \Phi_h(\ms{A}) \} \\
		                                    & = \{ h[A] \cap h[X] \in \big[ h[X] \big]^\omega \tq A \in \ms{A} \} \\
											& = \{ h[A \cap X] \tq A \cap X  \in [X]^\omega \land A \in \ms{A} \} \\
		                                    & = \Phi_h (\ms{A} \upharpoonright X) \, .
	\end{align*}
	(ii) Como $\Phi_h ^{-1} = \Phi_{h^{-1}}$, basta probar la suficiencia. Supóngase que $\ms{A}$ es maximal en alguna parte, entonces existe $X \in \ms{I}^+(\ms{A})$ tal que $\ms{A} \upharpoonright X$ es maximal en $X$. 
	
	Dada la igualdad de \ref{prop-IdealBiyec}, $h[X] \in \ms{I}^+(\Phi_h(\ms{A}))$; además, la función de restricción $g \colon = h \upharpoonright X : X \to h[X]$ es biyección y utilizando el inciso anterior se tiene que $\Phi_h(\ms{A}) \upharpoonright h[X]=\Phi_h (\ms{A} \upharpoonright X)=\Phi_g (\ms{A} \upharpoonright X)$. Por el último inciso de \ref{prop-ADbiyec}, $\Phi_h(\ms{A}) \upharpoonright h[X]$ es maximal en $h[X]$.
\end{proof}

Las siguientes propiedades son esperables, pero no por ello menos útiles. Es importante no desdeñar sus pruebas, pues en ellas, hay un par de sutilezas que deben ser atendidas.

\begin{proposicion}\phantomsection\label{prop-TrazaBasicos}
	Sean $\ms{A},\ms{B} \in \Ad(\omega)$ y $X,Y \in [\omega]^\omega$ arbitrarios.
	\begin{enumerate}[i)]
		\item Si $\ms{A} \subseteq \ms{B}$, entonces $\ms{A} \upharpoonright X \subseteq \ms{B} \upharpoonright X$.
		\item Se da la igualdad $(\ms{A} \upharpoonright Y) \upharpoonright X = \ms{A} \upharpoonright (Y \cap X)$.
		\item Si $X \subseteq Y$, entonces $\ms{I}_X(\ms{A} \upharpoonright X) \subseteq \ms{I}_Y(\ms{A} \upharpoonright Y)$.
	\end{enumerate}
\end{proposicion}
\begin{proof}
	El punto (i) es claro.

	(ii) Si $(A \cap Y) \cap X \in (\ms{A} \upharpoonright Y) \upharpoonright X$, entonces $A \cap Y \in \ms{A} \upharpoonright Y$ y $(A \cap Y) \cap X = A \cap (Y \cap X)$ es infinito; luego, $A \in \ms{A}$ y $(A \cap Y) \cap X \in \ms{A} \upharpoonright (Y \cap X)$. Recíprocamente, si $A \cap (Y \cap X) \in \ms{A} \upharpoonright (Y \cap X)$, entonces $A \in \ms{A}$ y $A \cap (Y \cap X) = (A \cap Y) \cap X$ es infinito. Además, $A \cap (Y \cap X) \subseteq A \cap Y$, entonces $A \cap Y$ es infinito, por lo que $A \cap Y \in \ms{A} \upharpoonright Y$ y así $A \cap (Y \cap X) = (A \cap Y) \cap X \in (\ms{A} \upharpoonright Y) \upharpoonright X$.

	(iii) Supóngase que $X \subseteq Y$ y sea $B \in \ms{I}_X(\ms{A} \upharpoonright X)$. Entonces $B \subseteq X \subseteq Y$ y existe $H \subseteq \ms{A} \upharpoonright X$ finito tal que $B \subseteq^* \midcup H$. Cada elemento $A \cap X \in H$ es infinito y está contenido en $A \cap Y$, luego $J \mycoloneq \{A \cap Y \tq A \cap X \in H\} \subseteq \ms{A} \upharpoonright Y$ es finito y $B \subseteq^* \midcup J$ y por lo tanto $B \in \ms{I}_Y(\ms{A} \upharpoonright Y)$.
\end{proof}

Si bien se demostró, tras la \cref{def-MaxEnAlguna}, que la maximalidad de una familia implica la de cada una de sus trazas. La próxima observación muestra que la maximalidad de aquellas trazas tomadas sobre elementos del ideal generado no constituye una condición suficiente para concluir la maximalidad de la familia dada.

\begin{proposicion}\phantomsection\label{prop-CaracMADIdeal}
	Sean $\ms{A} \in \Ad(\omega)$ y $X \in [\omega]^\omega$. Entonces $X \in \ms{I}(\ms{A})$ si y sólo si la familia $\ms{A} \upharpoonright X$ es finita y maximal en $X$.
\end{proposicion}

\begin{proof}
	Por el \cref{cor-IdealPropioCaract}, basta mostrar que $X \in \ms{I}(\ms{A})$ si y sólo si $X \in \ms{I}_X(\ms{A} \upharpoonright X)$. Dado lo demostrado en \ref{prop-TrazaBasicos}, la necesidad de tal equivalencia es inmediata. Ahora, si $X \in \ms{I}(\ms{A})$, entonces $X \subseteq^* \midcup H$ para cierto $H \in [\ms{A}^{<\omega}]$. Luego $X \subseteq^* X \cap \midcup H$ y la finitud de $H$ implica que:
	\begin{align*}
		X & \subseteq^* \{ A \cap X \in [X]^{<\omega} \tq A \in H \} \cup \{ A \cap X \in [X]^\omega \tq A \in H \} \\
		& =^* \{ A \cap X \in [X]^\omega \tq A \in H \} \, ,
	\end{align*}
	probando que $X \in \ms{I}_X(\ms{A} \upharpoonright X)$.
\end{proof}

Como la familia vacía no es maximal, la Proposición anterior deja como reflexión que los únicos conjuntos que podrían ser testigos de que $\ms{A}$ no es maximal, son aquellos que pertenecen a la parte positiva de $\ms{A}$.

\begin{corolario}\phantomsection\label{cor-CasiajenoPartePositiva}
	Si $\ms{A} \in \Ad(\omega)$, $\{X \in [\omega]^{\omega} \tq \ms{A} \upharpoonright X= \emptyset \} \subseteq \ms{I}^+(\ms{A})$.
\end{corolario}

Si $X \in [\omega]^\omega$ y $\ms{A}$ es maximal, entonces $X$ debe tener intersección infinita con algún elemento de $\ms{A}$; los conjuntos en $\ms{I}^+(\ms{{A}})$ tienen un comportamiento más fuerte, cada uno de ellos tiene intersección infinita con una infinidad de elementos de $\ms{A}$.

\begin{corolario}\phantomsection\label{prop-CaracMADPositiv}
	Sea $\ms{A} \in \Ad(\omega)$. Entonces $\ms{A}$ es maximal si y sólo si para cada $X \in \ms{I}^+(\ms{A})$ la familia $\ms{A} \upharpoonright X$ es infinita.
\end{corolario}
\begin{proof}
	Supóngase que $\ms{A}$ es maximal y sea $X \in \ms{I}^+(\ms{A})$. Por la maximalidad de $\ms{A}$, se sigue de lo comentado tras la \cref{def-MaxEnAlguna} que $\ms{A} \upharpoonright X$ es maximal en $X$. Por lo que, acorde a \ref{prop-CaracMADIdeal}, $\ms{A} \upharpoonright X$ debe ser infinita.

	Recíprocamente, si $\ms{A}$ no es maximal, existe $B \in [\omega]^\omega$ casi ajeno con cada elemento de $\ms{A}$. Así, $\ms{A} \upharpoonright B = \emptyset$ y se sigue de \ref{cor-CasiajenoPartePositiva}, que $B \in \ms{I}^+(\ms{A})$.
\end{proof}

Como ninguna familia maximal es numerable, del Corolario anterior se desprende la siguiente condensación de todo lo comentado y demostrado tras \ref{def-MaxEnAlguna}.

\begin{corolario}\phantomsection\label{cor-MADPositivCarac}
	Sean $N$ un conjunto numerable y $\ms{A} \in \Ad(N)$. Entonces $\{ X \in [N]^\omega \tq \ms{A} \upharpoonright X \notin \Mad(X) \lor \omega \leq |\ms{A} \upharpoonright X|\} = \ms{I}_N^+(\ms{A})$, y en caso de ser $\ms{A}$ maximal, $ \{ X \in [N]^\omega \tq \omega < |\ms{A} \upharpoonright X| \} = \ms{I}_N^+(\ms{A})$ y más aún
\end{corolario}

\section{Resultados en combinatoria infinita}

\subsection{Teorema de Simon}\phantomsection\label{subsec-Simon}
\phantomsection\label{Sec-TeoSimon}

Se abordará en lo subsecuente un análisis simple sobre la combinatoria de las familias casi ajenas. Los próximos lemas configuran la antesala para enunciar el primer resultado importante de la sección, el Teorema de Simon (\ref{Teo-Simon}).

\begin{lema} 
	Sea $(X_n)_{n\in \omega} \subseteq [\omega]^\omega$ una sucesión $\subseteq$-decreciente. Entonces existe $Y \in [X_0]^\omega$ tal que para cada $k \in \omega$, ocurre $Y \subseteq^* X_k$.
\end{lema}
\begin{proof}
	Considérese la función $f \colon \omega \to X_0$, dada recursivamente por $f(n) = \min (X_n \setminus f[n])$. Nótese que $f$ es inyectiva, si $m < n$ entonces $f(n) \notin f[n]$ y particularmente $f(n) \neq f(m)$. En consecuencia, $Y \mycoloneq  f[\omega] \in [X_0]^\omega$.

	Si $k \in \omega$, cada $x=f(n) \in Y \setminus X_k$ debe satisfacer que $n<k$, dada la hipótesis sobre $(X_n)_{n \in \omega}$. Esto prueba que $Y \setminus X_k \subseteq f[k] =^* \emptyset$; es decir, $Y \subseteq^* X_k$.
\end{proof}

El siguiente resultado constata que la parte positiva de una familia es grande, en el sentido de que esta contiene la suficiente cantidad de elementos como para acotar inferiormente a cualquier $\subseteq$-cadena numerable contenida en ella.

\begin{lema}[Dočkálková]\index[alph]{Dočkálková!Lema de}\index[alph]{Lema!de Dočkálková}\phantomsection\label{lem-PositivCadenaDecreciente}
	Sean $\ms{A} \in \Mad(\omega)$ y $(X_n)_{n\in\omega} \subseteq \ms{I}^+(\ms{A})$ una sucesión $\subseteq$-decreciente. Existe $Y \in \ms{I}^+(\ms{A})$ tal que si $n\in \omega$, entonces $Y \subseteq^* X_n$.
\end{lema}

\begin{proof}
	Por el \cref{cor-MADPositivCarac}, si $n \in \omega$, $H_n \mycoloneq \{ A \in \ms{A} \tq A \cap X_n \neq^* \emptyset \}$ debe ser infinito. Para cada $n \in \omega$ fijese ($\Ac$) una función $f_n  \colon  \ms{P}(H_n) \setminus \{\emptyset\} \to H_n$ de elección; y defínase recursivamente $B \colon \omega \to \ms{A}$ como $B(n) = f(H_n \setminus B[n] )$.

	Así, $(\midcup\{B(m) \tq m\geq n \} \cap X_n)_{n \in \omega}$ satisface las hipótesis del Lema previo. Efectivamente; si $n \in \omega$, entonces $B(n) \in H_n$ y $\midcup\{B(m) \tq m\geq n \} \cap X_n \supseteq B(n) \cap X_n$ es infinito. Esta sucesión es $\subseteq$-decreciente, debido a que $(X_n)_n \in \omega$ también lo es. Por lo tanto, existe $Y \in [\omega]^\omega$ de manera que para cada $k \in \omega$:
	\[ Y \subseteq^* \midcup\{B(m) \tq m\leq k \} \cap X_n \subseteq B(k) \cap X_k \subseteq X_k \, . \]
	Y, al ser $Y$ infinito, se tiene que $Y \cap B(k) \supseteq X \cap (B(k) \cap X_k )$ es infinito. Por lo tanto $\ms{A} \upharpoonright Y$ es infinita; siguiéndose del \cref{cor-MADPositivCarac} que $Y \in \ms{I}^+(\ms{A})$.
\end{proof}

El siguiente resultado fue demostrado en 1980 por Petr Simon \cite[p.~751]{SimonFrechet} y como se verá más adelante (\cref{cor-FrechNoProd}), juega un papel fundamental en el estudio de la propiedad topológica de Fréchet.

\begin{teorema}[Simon]\index[alph]{Simon!Teorema de}\index[alph]{Teorema!de Simon}\phantomsection\label{Teo-Simon}
	Para toda familia maximal e infinita $\ms{A}$ existe $Y \in \ms{I}^+(\ms{A})$ tal que $\mathscr{A} \upharpoonright Y$ es unión ajena de dos familias maximales en ninguna parte.
\end{teorema}
\begin{proof}
	Procédase por contradicción suponiendo que $\ms{A} \in \Mad(\omega)$ es una familia infinita tal que si $X \in \ms{I}^+(\ms{A})$, $\ms{A} \upharpoonright X = \ms{B} \cup \ms{C}$ y $\ms{B} \cap \ms{C} = \emptyset$, entonces alguna de las familias $\ms{B}, \ms{C} \in \Ad(X)$ es maximal en alguna parte.

	Como $\omega < |\ms{A}| \leq |2^\omega|$, existe $F \subseteq 2^\omega$ inifnito y tal que $\ms{A}=\{A_f \tq f \in F\}$; donde $f \neq g$ implica $A_f \neq A_g$. Para cada $(n,k) \in \omega \times 2$ defínase:
	\[ \ms{A}(n,k)=\{A_f \in \ms{A} \tq f(n)=k \} \, . \]
	
	Nótese que para cualesquiera $m \in \omega$ y $X \in \ms{I}^+(\ms{A})$ ocurren simultáneamente $\ms{A} \upharpoonright X = \ms{A}(m,0) \upharpoonright X \cup \ms{A}(m,1) \upharpoonright X$ y $\ms{A}(m,0) \upharpoonright X \cap \ms{A}(m,1) \upharpoonright X = \emptyset$. Se sigue de la hipótesis la existencia $(\Ac)$ de una función $k \colon \omega \times \ms{I}^+(\ms{A}) \to 2$ tal que si $(m,X) \in \dom(k)$, entonces $\ms{A}(m,k(m,X)) \upharpoonright X$ es maximal en alguna parte. Así mismo, existe una función $j \colon \omega \times \ms{I}^+(\ms{A}) \to \ms{I}^+_X(\ms{A}(m,k(m,X)) \upharpoonright X)$ de manera que para cada $(m,X) \in \dom(j)$:
	\[ (\ms{A}(m,k(m,X)) \upharpoonright X) \upharpoonright j(m,X) = \ms{A}(m,k(m,X)) \upharpoonright j(m,X) \in \Mad(j(m,X)) \,. \]

	Defínase recursivamente la sucesión $(X_n)_{n \in \omega}$ como $X_0=\omega$; y para cada $n \in \omega$, $X_{n+1}=j(n,X_n)$. Si $n \in \omega$, se da $X_{n+1} \in \ms{I}_{X_n}^+( \ms{A}(n,k(n,X_n)) \upharpoonright X_n ) \subseteq \ms{I}^+(\ms{A})$ (véase \ref{prop-TrazaBasicos}); y además, por la definición de $j$:
	\[ \ms{A}(n,k(n,X_n)) \upharpoonright X_{n+1} \in \Mad(X_{n+1}) \, ; \]	
	a consecuencia del ello $X_{n+1} \subseteq X_n$. Aplicando el Lema de Dočkálková, se obtiene un conjunto $Y \in \ms{I}^+ (\ms{A})$ casi contenido en cada $X_n$.

	Por \ref{cor-MADPositivCarac}, $\ms{A} \upharpoonright Y$ es infinita; y por ello, existen $g \in F$ y $m \in \omega$ tales que $g(m) \neq k(m,X_m)$ y $A_g \cap Y$ es infinito. Puesto que $Y \setminus X_{m+1} =^* \emptyset$, es necesario que $A_g \cap X_{m+1} \subseteq X_{m+1}$ sea infinito y como $\ms{A}(m,k(m,X_m)) \upharpoonright X_{m+1} \in \Mad(X_{m+1})$, existe $A_f \in \ms{A}(m,k(m,X_m))$ tal que $(A_g \cap X_{m+1}) \cap (A_f \cap X_{m+1}) \neq ^* \emptyset$. Pero, a razón de que $f(m)=k(m,X_m) \neq g(m)$, se tiene que $f \neq g$ y $A_f \cap A_g =^* \emptyset$, lo cual es un absurdo a lo previamente establecido.
\end{proof}

La familia $\ms{A} \upharpoonright Y$ provista por el Teorema de Simon es maximal e infinita necesariamente. Así que, tomando en cualquier $\ms{A} \in \Mad(\omega)$ infinita se tiene:

\begin{corolario}
	Existe una familia maximal infinita que es unión ajena de dos familias maximales en ninguna parte.
\end{corolario}

\subsection{Grietas y familias de Luzin}
\phantomsection\label{Sec-Luzin}

La teoría de grietas (\enquote{gaps}, en inglés) es extensa, y su estudio se remonta a los trabajos de Felix Hausdorff. En lo que sigue nos restringiremos a analizar grietas en $\omega$. No obstante, a partir de la \cref{prop-ADbiyec} es inmediato comprobar que la definición que presentamos a continuación se extiende sin dificultad a grietas sobre cualquier conjunto numerable $N$.

\begin{definicion}\phantomsection\label{Def-separador}\phantomsection\label{def-grieta}\index[alph]{separador}\index[alph]{grieta}\index[alph]{grieta!separada}\index[alph]{grieta!contenida en una familia}\index[alph]{familia!que contiene a una grieta}
	Una \textbf{grieta} es un par ordenado $(\ms{A},\ms{B}) \in \Ad(\omega) \times \Ad(\omega)$ tal que $\ms{A} \cup \ms{B} \in \Ad(\omega)$.
	\begin{enumerate}
		\item Se suele decir que $\ms{A} \cup \ms{B}$ \textbf{contiene a} $(\ms{A},\ms{B})$.
		\item Un subconjunto $D \subseteq \omega$ es \textbf{separador} de $\ms{A}$ y $\ms{B}$ si y sólo si para cualesquiera $A \in \ms{A}$ y $B \in \ms{B}$ se tiene que $A \subseteq^* D$ y $B \cap D =^* \emptyset$.
		\item $(\ms{A},\ms{B})$ está \textbf{separada} cuando existe un separador de $\ms{A}$ y $\ms{B}$.
	\end{enumerate}
\end{definicion}

En términos de la definición anterior, es inmediato que si $(\ms{A}',\ms{B}')$ es una grieta separada, $\ms{A} \subseteq \ms{A}'$ y $\ms{B} \subseteq \ms{B}'$, entonces la grieta $(\ms{A},\ms{B})$ está separada. Además, cuando $D$ es separador de $\ms{A}$ y $\ms{B}$, el conjunto $\omega \setminus D$ es separador de $\ms{B}$ y $\ms{A}$; por ello, $(\ms{A},\ms{B})$ está separada si y sólo si $(\ms{B},\ms{A})$ está separada. Obsérvese además que seguido del \cref{cor-CasiajenoPartePositiva}:

\begin{proposicion}\phantomsection\label{obs-GrietasSimple}
	Para toda grieta $(\ms{A},\ms{B})$, $\ms{A} \subseteq \ms{I}^+(\ms{B})$ y $\ms{B} \subseteq \ms{I}^+(\ms{A})$.
\end{proposicion}

Al considerar una familia casi ajena, surgen preguntas en relación al comportamiento de las grietas contenidas en ella; en tal aspecto se mueve la siguiente definición.

%Las familias de Luzin tienen una propiedad destacable, cualquiera de ellas es \textit{inseparable}, en el siguiente sentido (siguiendo la terminología de \cite[\S~ 3.2]{hruAlmost}):

\begin{definicion}\index[alph]{familia!separable}\index[alph]{familia!normal}\index[alph]{familia!inseparable}\index[alph]{familia!parcialemnte separable}\phantomsection\label{def-FamInseparable}\phantomsection\label{def-FamParcialmenteSeparable}
	Una familia $\ms{A} \in \Ad(\omega)$ se dice:
	\begin{enumerate}
		\item \textbf{Separable} (o \textbf{normal}) cuando todas las grietas $(\ms{B},\ms{C}) \in \ms{P}(\ms{A})^2$ están separadas.
		\item \textbf{Parcialmente separable} cuando para toda grieta $(\ms{B},\ms{C}) \in ([\ms{A}]^{\omega_1})^2$ existen $\ms{B}' \subseteq \ms{B}$ y $\ms{C}' \subseteq \ms{C}$ tales que $(\ms{B}',\ms{C}')$ está separada.
		\item \textbf{Inseparable} cuando ninguna grieta $(\ms{B},\ms{C}) \in ([\ms{A}]^{\omega_1})^2$ está separada.
	\end{enumerate}	
\end{definicion}

Se obtiene el siguiente comportamiento; la demostración del mismo puede ser realizada con lo desarrollado hasta el momento, sin embargo, dejaremos este resultado como una consecuencia del estudio de los $\Psi$-espacios (véase \ref{col-tra-interrelacion})

\begin{ejemplo}\phantomsection\label{ej-interrelacion}
	Sea $\ms{C}\in \Ad(\omega)$. Si $|\ms{C}|\leq \omega$, entonces $\ms{C}$ es separable. Y, si $\ms{C}$ más que numerable y separable, entonces $\omega<|\ms{C}|<\mathfrak{c}$ y $\ms{C} \notin \Mad(\omega)$.
\end{ejemplo}

Obsérvese ahora que si $(\ms{B},\ms{C})$ es una grieta y $\midcup \ms{B} \cap \midcup \ms{C}$ es finito, entonces el conjunto $\midcup \ms{B} \subseteq \omega$ es separador de $\ms{B}$ y $\ms{C}$. Se puede caracterizar la inseparabilidad de una familia mediante lo siguiente:

\begin{lema}\phantomsection\label{lem-InseparableCar}
	Sea $\ms{A}\in \Ad(\omega)$, entonces $\ms{A}$ es inseparable si y sólo si para cualesquiera $\ms{B},\ms{C} \in [\ms{A}]^{\omega_1}$ ajenos, $\midcup \ms{B} \cap \midcup \ms{C}$ es infinito.
\end{lema}
\begin{proof}
	Por la discusión previa, basta sólo probar la necesidad.

	Por contrapuesta, supóngase que $\ms{B},\ms{C} \in [\ms{A}]^{\omega_1}$ son tales que existe $D \subseteq \omega$, separador de $\ms{B}$ y $\ms{C}$. Entonces las asignaciones $\ms{B} \to \omega$ y $\ms{C} \to \omega$; dadas por $b \mapsto \max(b \setminus D)+1$ y $c \mapsto \max(c \cap D)+1$ están bien definidas. Pero en vista de que $|\ms{B}|=|\ms{C}|=\omega_1$, estas no pueden ser inyectivas, y existen $m,n \in \omega$ de modo que $\ms{B}' \mycoloneq \{b \in \ms{B} \tq b \setminus D \subseteq m\}$ y $\ms{C}' \mycoloneq \{b \in \ms{C} \tq c \cap D \subseteq n\}$ tienen tamaño $\omega_1$.

	Además $\midcup \ms{B}' \setminus D \subseteq m =^* \emptyset$ y $\midcup \ms{C}' \cap D \subseteq n =^* \emptyset$, por lo que $\midcup \ms{B} \subseteq^* D$, $\midcup \ms{C} \subseteq^* \omega \setminus D$, y así, $\midcup \ms{B}' \cap \midcup \ms{C}' \subseteq^* D \cap (\omega \setminus D) = \emptyset$.
\end{proof}

En la década de 1910, el analista Nikolai Luzin introdujo ciertos subconjuntos especiales de $\mathbb{R}$, mismos que hoy llevan su nombre. Y pese a que el estudio de aquellos objetos es relativo a la teoría descriptiva de conjuntos, las analogías entre sus propiedades y las de las familias que a continuación presentamos, son suficiente razón como para que estas lleven el subfijo \enquote{de Luzin}.

\begin{definicion}\phantomsection\label{def-LuzinFam}\index[alph]{familia!de!Luzin}\index[alph]{Luzin!familia de}
	Una familia $\ms{A}$ es \textbf{de Luzin} cuando existen $X \in [\omega_1]^{\omega_1}$ y una enumeración $\ms{A} = \{A_\alpha \tq \alpha \in X \}$ tales que para todo $(\alpha, n) \in \omega_1 \times \omega$, el conjunto $ \{ \beta<\alpha \tq A_\alpha \cap A_\beta \subseteq n \} $ es finito.
\end{definicion}

Una de las ideas centrales detrás de que $\ms{A}=\{A_\alpha \tq \alpha \in \omega_1 \}$ sea de Luzin es que; fijando $\alpha \in \omega_1$, para cada $D \subseteq \alpha$ infinito, $A_\alpha \cap \midcup\{A_\beta \tq \beta \in D\}$ es infinito. Esto se debe a que si $n \in \omega$, entonces $D \setminus \{ \beta<\alpha \tq A_\alpha \cap A_\beta \subseteq n \} $ es infinito, particularmente no vacío. Como consecuencia:

\begin{proposicion}\phantomsection\label{prop-LuzinSeparadas}
	Cualquier familia Luzin es inseparable.
\end{proposicion}
\begin{proof}
	Sean $\ms{A}=\{A_\alpha \tq \alpha \in \omega_1\}$ una familia de Luzin y supógnase que $\ms{B}=\{A_\alpha \tq \alpha \in B\},\ms{C}=\{A_\alpha \tq \alpha \in C\} \subseteq \ms{A}$ son no numerables y ajenos. Como $C$ es infinito, existe $\alpha \in \omega_1$ de manera que $C \cap \alpha$ es infinito. Nótese que $B$ es cofinal en $\omega_1$, por ser $\omega_1$ regular; así que sin pérdida de generalidad, supóngase $\alpha \in B$.

	Dados los comentarios previos a esta demostración, $A_\alpha \cap \midcup \{A_\beta \tq \beta \in C \cap \alpha \}$ es infinito, demostrando que $\midcup \ms{B} \cap \midcup \ms{C}$ es infinito, y se sigue del \cref{lem-InseparableCar}, que $\ms{A}$ es inseparable.
\end{proof}

La existencia de familias de Luzin no es tan inucitada, resulta que cualquier familia infinita trae consigo una extensión que es de Luzin.

\begin{proposicion}\phantomsection\label{pro-LuzinExisten}
	Toda familia casi ajena numerable se extiende a una familia de Luzin. Particularmente, existe una familia de Luzin.
\end{proposicion}
\begin{proof}
	Sea $\ms{B}=\{A_n \tq n \in \omega\} \in \Ad(\omega)$ numerable y nótese que para cualesquiera $m,n \in \omega$, el conjunto $ \{ k<m \tq A_m \cap A_k \subseteq n \} $ es finito.

	Por recursión sobre $\omega_1 \setminus \omega$, sea $\gamma \in \omega_1 \setminus \omega$ y supóngase $\{A_\alpha \tq \alpha \in \gamma\}$ es una familia tal que, si $\alpha<\gamma$ y $n \in \omega$, el conjunto $ \{ \beta<\alpha \tq A_\alpha \cap A_\beta \subseteq n \} $ es finito.

	Como $\gamma \in \omega \setminus \omega_1$, $\gamma$ es numerable y se puede enumerar $\{A_\alpha \tq \alpha \in \gamma\}$ como $\{B_n \tq n\in \omega\}$. Por ser tal, una familia casi ajena, cada conjunto $C_n \mycoloneq B_n \setminus \midcup\{ B_j \tq j<n \}$ es infinito (corrobórese ésto en la demostración de \ref{prop-MADnoNum}). Para cada $n \in \omega$ fíjese $a_n \in [C_n]^n$ y defínase:
	$$ A_\gamma \mycoloneq \midcup\{a_m \tq m \in \omega\} $$

	Nótese que si $n \neq m$, entonces $a_n \cap a_m = \emptyset$. De este modo, si $n \in \omega$ es cualquiera, resulta que $A_\gamma \cap B_n = a_n \cap B_n = a_n$ es finito. Más aún, como $a_n$ tiene exatamente $n$ elementos, $n \leq \max(A_n)$; y consecuentemente, si $m \in \omega$ y $A_\gamma \cap B_n \subseteq m$, entonces $n \leq m$.

	Lo anterior prueba, no sólo que $\{A_\alpha \tq \alpha \ \leq \gamma\}$ es familia casi ajena, sino que para cualesquiera $\alpha\leq \gamma$ y $n \in \omega$, el conjunto $ \{ \beta<\alpha \tq A_\alpha \cap A_\beta \subseteq n \} $ es finito. Lo cual finaliza la construcción por recursión de los conjuntos $A_\alpha$ (con $\omega \leq \alpha< \omega_1$); es claro que para $\ms{A} \mycoloneq \{A_\alpha \tq \alpha \in \omega_1 \}$ es una familia Luzin que extiende a $\ms{B}$.
\end{proof}

Los siguientes objetos son una suerte de modificación a la \cref{def-LuzinFam}; mientras que las familias de Luzin son \enquote{unidimensionales}, en el sentido de que están indexadas por $\omega_1$; las grietas de Luzin son \enquote{$n$-dimensionales}:

	\begin{definicion}\index[alph]{$n$-grieta! de Luzin}\index[alph]{Luzin!$n$-grieta de}\index[alph]{grieta!de Luzin}\index[alph]{Luzin!grieta de}\index[alph]{testigos!de una $n$-grieta de Luzin}
		Sea $n \in \omega \setminus 2$. Una \textbf{$n$-grieta de Luzin} es una sucesión finita $(\ms{A}_i)_{i \in I} \subseteq \Ad(\omega)$ cuya union es una familia casi ajena; y, tales que existen $X \in [\omega_1]^{\omega_1}$ y $m \in \omega$, de modo que cada $\ms{A}_i$ se puede enumerar como $\{ A_\alpha^i \tq \alpha \in X \}$ de manera que:
		\begin{enumerate}
			\item Para cualesquiera $i,j \in n$ distintos y $\alpha \in X$, se da $A_\alpha^i \cap A_\alpha^j \subseteq m$.
			\item Si $\alpha,\beta \in X$ y $\alpha \neq \beta$, entonces $\midcup\{ A_\alpha^i \cap A_\beta^j \tq i, j \in n \, \land \, i \neq j\} \not \subseteq m$.
		\end{enumerate}

		A los conjuntos $X$ y $m$ se les llama testigos de que $(\ms{A}_i \tq i \in I)$ es $n$-grieta de Luzin. Una $2$-grieta de Luzin se denomina simplemente \textbf{grieta de Luzin}.
	\end{definicion}

	Las $n$-grietas de Luzin poseen un comportamiento en común a las familias Luzin, el comentado posteriormente a la \cref{def-LuzinFam}. Este comportamiento similar es la razón de que, según la literatura que se consulte, a veces se confunden los términos \enquote{familia de Luzin} y \enquote{grieta de Luzin}.
	
	\begin{lema}
		Sea $(\ms{A}_i \tq i \in I)$ una $n$- grieta de Luzin con testigos $X$ y $m$. Para cada $Y \subseteq X$ no numerable, existe $Y' \subseteq Y$ no numerable tal que para cualesquiera $\alpha,\beta \in Y'$ distintos, $\midcup\{ A_\alpha^i \cap A_\beta^j \tq i, j \in n \, \land \, i \neq j\}$ es infinito.
	\end{lema}
	\begin{proof}
		Sea $k \in \omega \setminus m$. Para cada $i \in n$ sea $f_i \colon Y \to \omega$ dada por $f(\alpha)=A_\alpha ^ i \cap k$. Para cada $S \in [X]^{>\omega}$ y $g \colon Z \to \omega$, $g$ no es inyectiva; fíjese entonces ($\Ac$) un conjunto $S(g) \subseteq S$ no numerable para el cual $g \upharpoonright A_g$ es constante. Defínase la sucesión finita $(Y_i)_{i \in n}$ de manera recursiva como $Y_0 \mycoloneq Y$; y, para cada $j \in n$, $Y_{j+1} \mycoloneq S(f \upharpoonright Y_j) \subseteq Y_j$.
		
		Entonces el conjunto $Y' \mycoloneq \midcap\{ Y_i \tq i \in n \} = Y_{\cup n}$ es un subconjunto no numerable de $Y$ tal que para cada $i \in n$, $f_i$ es constante en $Y'$. Sean $\alpha, \beta \in Y'$ distintos, como $(\ms{A}_i)_{i \in I}$ es $n$-grieta de Luzin, existen $i,j \in n$ diferentes de manera que $A_\alpha^i \cap A_\beta^j \not \subseteq m$. Como $f_i$ es constante en $Y'$, entonces $A_\alpha^i \cap k = A_\beta ^i \cap k$.


		Como además $A_\beta^i \cap A_\beta ^j \subseteq m \subseteq k$, resulta necesario que $A_\alpha^i \cap A_\beta^j \not \subseteq k$. Así, en virtud de la arbitrariedad del natural $k$, se concluye que el siguiente conjunto debe ser infinito: $\midcup\{ A_\alpha^i \cap A_\beta^j \tq i, j \in n \, \land \, i \neq j\}$.
	\end{proof}

	A razón de la proposición anterior se desprende que; así como ninguna familia parcialmente separable puede ser de Luzin, resulta que:

	\begin{corolario}\phantomsection\label{cor-ngrietasNoSep}
		Si una familia casi ajena es parcialmente separable, no puede contener ninguna $n$-grieta de Luzin.
	\end{corolario}
	\begin{proof}
		Sea $(\ms{A}_i \tq i \in I)$ una $n$- grieta de Luzin con testigos $X$ y $m$. Si $ \ms{A} \supseteq \midcup \{ \ms{A}_i \tq i \in n \}$, entonces dado el Lema previo a esta demostración, para cualesquiera $i,j \in n$ distintos, existen $\alpha,\beta \in X$ diferentes de modo que el conjunto $\midcup\{ A_\alpha^i \cap A_\beta^j \tq i, j \in n \, \land \, i \neq j\}$ es infinito. Es decir, $\midcup \ms{A}_i \cap \midcup \ms{A}_j$ es infinito, siguiéndose la conclusión automáticamente de \ref{lem-InseparableCar}.
	\end{proof}

	La siguiente Proposición deja en claro la labor de \enquote{debilitamiento} que cumplen las $n$-grietas de Luzin respecto a las familias de Luzin; cualquiera de estas últimas contiene siempre una $2$-grieta de Luzin. La prueba del hecho en cuestión requiere invocar el \textit{Lema de Fodor} (\cref{fodor-lem}) y el hecho de que siempre existen funciones ordinales como las solicitadas en la subsecuente prueba.

	\begin{proposicion}
		Toda familia de Luzin contiene una grieta de Luzin. En consecuencia, existe una grieta de Luzin.
	\end{proposicion}
	\begin{proof}
		Sea $\ms{A}=\{A_\alpha \tq \alpha \in \omega_1\}$ una familia de Luzin y $f,g \colon \omega_1 \to \omega_1$ tales que si $ \beta < \alpha < \omega_1$, entonces $g(\beta) < f(\beta) < g(\alpha)$.

		Como $\omega_1 \setminus \omega$ es estacionario en $\omega_1$ y la asignación $j \colon \omega_1 \setminus \omega \to \omega_1$; dada por $j(\alpha) = \max(A_{f(\alpha)} \cap A_{g(\alpha)}) +1 $, es regresiva, en virtud del Lema de Fodor, existen $S \subseteq \omega_1 \setminus \omega$ estacionario en $\omega_1$ y un natural $m$ tales que $j[S] \subseteq \{m\}$.

        Ahora, si $\beta < \alpha$, entonces debido a la hipótesis $f(\beta) < g(\alpha)$, de modo que $f[\{ \beta < \alpha \tq A_{f(\beta)} \cap A_{g(\alpha)} \subseteq m \}] \subseteq \{ \gamma < g(\alpha) \tq A_{g(\alpha)} \subseteq m \}$. Como $\ms{A}$ es de Luzin y $f$ es inyectiva (por ser estrictamente creciente), $\{ \beta < \alpha \tq A_{f(\beta)} \cap A_{g(\alpha)} \subseteq m \}$ es finito, permitiendo la buena definición de $l \colon S \to \omega_1$, definida por medio de la siguiente regla de correspondencia:
        \[ l(\alpha) = \left\lbrace \begin{array}{l l}
            0 & ; \,\, \forall \gamma < \alpha ( A_{f(\gamma)} \cap A_{g(\alpha)} \not\subseteq m ) \\
            \max\{ \gamma < \alpha \tq A_{f(\gamma)} \cap A_{g(\alpha)} \subseteq m \} & ; \,\, \text{otro caso}
        \end{array} \right. \]
        
        Como $0 \notin S$, $l$ es regresiva; nuevamente, del Lema de Fodor, se sigue la existencia de un conjunto $X \subseteq S$, estacionario en $\omega_1$, tal que $l$ es constante en $X$. Así, para cada $\alpha \in X$ se tiene que $m=\max(A_{f(\alpha)} \cap A_{f(\alpha)})+1$ y en consecuencia $A_{f(\alpha)} \cap A_{f(\alpha)} \subseteq m$, pues $X \subseteq S$ y $j[S] \subseteq \{m\}$.
		
		Ahora, supóngase que $\beta, \alpha \in X$ y $\beta < \alpha$, como $l$ es constante en $X$, $l(\alpha) = l(\beta)$. Si este último valor es cero, entonces $A_{f(\beta)} \cap A_{g(\alpha)} \not\subseteq m$. En caso contrario:
            \[ 0 \neq l(\alpha) = l (\beta) < \beta < \alpha \, , \]
        y como $l(\alpha)< \beta < \alpha$, la definición de $l$ obliga a que $A_{f(\beta)} \cap A_{g(\alpha)} \not\subseteq m$. Lo anterior demuestra que $(\{A_{f(\alpha)} \tq \alpha \in X\},\{A_{g(\alpha)} \tq \alpha \in X\})$ es una grieta de Luzin (con testigos $X$ y $m$) contenida en $\ms{A}$.
	\end{proof}
	
\subsection{Lema de Solovay}

\begin{definicion}\phantomsection\label{def-ordenBasado} \index[alph]{orden!basado en $\ms{A}$}\index[sym]{$\mathbb{P}_\ms{A}$}\index[sym]{$\leq_\ms{A}$}
	Sea $\ms{A}$ una familia casi ajena. El \textbf{orden basado en} $\ms{A}$ es el par $\mathbb{P}_\ms{A} \mycoloneq ([\omega]^{<\omega} \times [\ms{A}]^{<\omega}, \leq_\ms{A})$; donde: $(p,P) \leq_\ms{A} (h,H)$ si y sólo si $h \subseteq p$, $H \subseteq P$ y $p \setminus h \subseteq \omega \setminus \midcup H$.
	
	Cuando el contexto sea claro, se escribirá $\leq$ en vez de $\leq_\ms{A}$.
\end{definicion}

Habrá de verificarse que el orden basado en $\ms{A}$ es, en efecto, un orden parcial. Claramente es una relación reflexiva y antisimétrica. Ahora, si $(p,P) \leq (h,H)$ y $(h,H) \leq (k,K)$, es inmediato a la definición de $\leq_\mathscr{A}$ que $k \subseteq h \subseteq p$ y $K \subseteq H \subseteq P$; en consecuencia $k \subseteq p$ y $K \subseteq P$. Además $p \setminus h \subseteq \omega \setminus \midcup H$ y $h \setminus k \subseteq \omega \setminus \midcup K$, resultando en que:
	\[ p \setminus k \subseteq (h \setminus k) \cup (p \setminus h) \subseteq \big( \omega \setminus \midcup K \big) \cup \big( \omega \setminus \midcup H \big) \, ; \]
lo cual demuestra $p \setminus k \subseteq \omega \setminus \midcup K$. Luego $\leq$ es transitiva, y con ello, orden parcial.

En términos sumamente intuitivos, $(p,P) \leq (h,H)$ significa que \enquote{$h$ se extiende a $p$ y $H$ a $P$}. Así, conforme $H \subseteq \ms{A}$ \enquote{crece}, se aproxima a $\ms{A}$. Dado que, conforme $h$ \enquote{crece}, este se \enquote{aproxima} a un subconjunto casi ajeno con $\midcup H$; y eventualmente, se formará un subconjunto casi ajeno con $\midcup \ms{A}$.

Para dar una formalización de la intuición recien dada, comenzaremos definiendo los siguientes objetos para todo lo que resta de la sección.
\index[sym]{$D_a$ (si $a \in \ms{A}$)}\index[sym]{$D_G$ (si $G \subseteq \mathbb{P}_\ms{A}$)}
\begin{enumerate}
		\item Para cada $a \in \ms{A}$, $ D_a \mycoloneq \{ (p,P) \in \mathbb{P}_\ms{A} \tq a \in P\} $.
		\item Si $G \subseteq \mathbb{P}_\ms{A}$, se define $ D_G \mycoloneq \midcup\{ h \subseteq \omega \tq \exists H \in [\omega]^{\omega} \: \big( (h,H) \in G \big) \} $.
\end{enumerate}

El siguiente Lema indica por qué la existencia de ciertos filtros en $\mathbb{P}_{\mathscr{A}}$ genera conjuntos casi ajenos con cada elemento de $\ms{A}$; es decir, se implica bajo esta condición, la no maximalidad de la familia.

\begin{lema}\phantomsection\label{lem-DgMagia}
	Sean $\ms{A}$ una familia casi ajena, y $G$ un filtro de $\mathbb{P}_\ms{A}$, entonces para cada $a \in \ms{A}$:
	\begin{enumerate}
		\item $D_a$ es denso en $\mathbb{P}_\ms{A}$.
		\item Si $G \cap D_a \neq \emptyset$; entonces, $D_G \cap a$ es finito.
	\end{enumerate}
\end{lema}

\begin{proof}
	(i) Si $a \in \ms{A}$ y $(p,P) \in \mathbb{P}_\ms{A}$ son elementos arbitrarios, entonces $(p,P\cup\{a\}) \in D_a$ y además, por \ref{def-ordenBasado}, $(p,P\cup\{a\}) \leq (p,P)$.

	(ii) Supóngase que $(p,P) \in G \cap D_a$ y sea $x \in D_G \cap a$ cualquier elemento. Por definición de $D_G$, existe $(h,H) \in G$ de modo que $x \in h$. Y por ser $G$ filtro, $(k,K) \leq (p,P),(h,H)$ para cierto $(k,K) \in G$. De esto, particularmente se obtiene que $h \subseteq k$, $k \setminus p \subseteq \omega \setminus \midcup P$.

	Ahora, como $(p,P)\in D_a$, entonces como $a \in P$. En consecuencia, se tiene que $x \in \midcup P$. Finalmente, $x \in h \subseteq k$, así que $x \in k \cap \midcup P$, lo cual obliga a que $x \in p$. Por tanto $D_G \cap a \subseteq p =^* \emptyset$.
\end{proof}

Se tiene entonces lo subsecuente de forma inmediata:

\begin{corolario}\phantomsection\label{cor-SolovayDebil}
	Sean $\ms{A} \in \Ad(\omega)$ y $\ms{D} \mycoloneq \{D_a \tq a \in \ms{A}\}$. Si existe un filtro $\ms{D}$-genérico, $\ms{A}$ no es maximal.
\end{corolario}

Debido a lo recién mostrado, de tener $\mathbb{P}_\ms{A}$ la c.c.c. (propiedad definida en la \cref{copo-ccc}), se satisfaría que $\Ma(|\ms{A}|)$ implica $\ms{A} \notin \Mad(\omega)$. En tal caso, se estaría probando que si $\Ma(\kappa)$ es cierto, entonces $\kappa < \mathfrak{a}$; por ello, $\mathfrak{m} \leq \mathfrak{a}$.

Y en efecto, si $\mathcal{A} \subseteq \mathbb{P}_\ms{A}$ es anticadena y $(p,P),(h,H) \in \mathcal{A}$, se tiene $p\neq h$; sino $(p,P \cup H) \leq (p,P),(h,H)$ y $\mathcal{A}$ dejaría de ser anticadena. En consecuencia $|\mathcal{A}|\leq|[\omega]^{<\omega}|=\aleph_0$ y $\mathbb{P}_\ms{A}$ tiene la \textit{c.c.c.}

\begin{corolario}\phantomsection\label{cor-MaSimple}
	Si $\kappa \geq \omega$ y $\Ma(\kappa)$, entonces $\kappa < \mathfrak{a}$; así que:
	\begin{enumerate}
		\item $\mathfrak{m} \leq \mathfrak{a}$.
		\item Bajo $\Ma + \lnot\HC$ se tiene $\mathfrak{m}=\mathfrak{a}=\mathfrak{c}$.
		\item $\zfc$ no demuestra la implicación $\mathfrak{a}=\mathfrak{c} \to \HC$.
	\end{enumerate}
\end{corolario}

\begin{proof}
	Los puntos (i) y (ii) son inmediatos. Para (iii) únicamente basta recordar que $\Ma + \lnot \HC$ es consistente con $\zfc$ (consúltese \cite[p.~279-281]{kunenSet}).
\end{proof}

El \cref{cor-SolovayDebil} es una inmediatez, dada toda su discusión previa. Una versión bastante más fortalecida de este, es el siguiente resultado mostrado por Robert Solovay.

\begin{lema}[Solovay]\phantomsection\label{lem-Solovay}\index[alph]{Lema!de Solovay}\index[alph]{Solovay!Lema de}
	Sean $\kappa$ un cardinal con $\omega \leq \kappa < \mathfrak{c}$ y $(\ms{A},\ms{B})$ una grieta tal que $|\ms{A}|,|\ms{B}|\leq \kappa$. Bajo $\Ma(\kappa)$; existe $D\subseteq \ms{A}$ de modo que para cualesquiera $A \in \ms{A}$ y $b \in \ms{B}$ ocurre $a \cap D=^*\emptyset$ y $b \cap D\neq ^*\emptyset$.
\end{lema}

\begin{proof}
	Supóngase $\Ma(\kappa)$ y para cada $(b,n) \in \ms{B} \times \omega$, defínase el conjunto $D(b,n) \mycoloneq \{ (h,H) \in \mathbb{P}_\ms{A} \tq h \cap b \not\subseteq n \}$.

	Sea $(p,P) \in \mathbb{P}_\ms{A}$; por la \cref{obs-GrietasSimple} $b \notin \ms{I}(\ms{A}) $, luego $b \setminus \midcup P$ es infinito. Por ello, existe $m \in \omega$ de modo que $n+1 \in m$ y $m \in b \setminus \midcup P \subseteq \omega \setminus \midcup P$; así, $p \cup \{m\}$ es finito, $(p \cup \{m\}, P) \in D(b,n)$ y $(p \cup \{m\}, P) \leq_\ms{A} (p,P)$. Por lo tanto, cada $D(b,n)$ es denso en $\mathbb{P}_\ms{A}$.

	Sea $\ms{D}=\{ D(b,n) \tq (b,n) \in \ms{B} \times \omega \} \cup \{D_a \tq a \in \ms{A} \}$ y obsérvese que $\ms{D}$ es una familia de densos de $\mathbb{P}_\ms{A}$ de cardinalidad menor o igual a $\kappa$. Como $\mathbb{P}_\ms{A}$ es \textit{c.c.c.}, de $\Ma(\kappa)$ se desprende la existencia de un filtro $G$ en $\mathbb{P}_\ms{A}$, $\ms{D}$-genérico. Se afirma que $D_G$ es el conjunto buscado.

	En efecto, por \ref{lem-DgMagia} se tiene que para cada $a \in \ms{A}$, el conjunto $D_G \cap a$ es finito. Ahora, si $b \in \ms{B}$ es cualquiera, para cada $n \in \omega$ existe $(k,K) \in G \cap D(b,n)$; y en consecuencia $D_G \cap b \not \subseteq n$ (pues $h \cap b \not \subseteq n$). Por lo que el conujunto $D_G \cap b$ no puede ser finito.
\end{proof}

Es deseable que la conclusión del Lema de Solovay fuese que la grieta $(\ms{A},\ms{B})$ está separada; sin embargo tal formulación desemboca en un resultado falaz. Suponiendo $\Ma + \lnot\HC$, la existencia de una familia de Luzin (véase \ref{prop-LuzinSeparadas}) sería testigo de tal falsedad; pues esta tiene tamaño $\aleph_1 < \mathfrak{m}$ y es inseparable.

\index[alph]{familia!débilmente separada}
Como \enquote{breviario cultural}, cuando una grieta $(\ms{A},\ms{B})$ satisface la conclusión de \ref{lem-Solovay} para cierto subconjunto $D \subseteq \omega$, se dirá que está \textit{débilmente separada} (siguiendo la terminología de \cite[\S~ 3.2]{hruAlmost}). 

\index[alph]{hipótesis!de Jones}\index[alph]{Jones! hipótesis de}
La siguiente es una aplicación típica del lema de Solovay, una prueba sencilla de la consistencia, respecto a $\zfc$, del enunciado $2^{\aleph_0}=2^{\aleph_1}$ (la moderadamente famosa \textit{Hipótesis de Jones}). Como se podrá verificar en secciones futuras de este escrito, tanto la hipótesis de Jones, como su negación, tendrán repercusiones en el comportamiento topológico de los $\Psi$-espacios; especialmente, en la relación que estos guardan con la Conjetura de Moore (véase la \cref{Sec-PDM}).

Aprovechamos para mencionar una última aplicación \enquote{conjuntista} de la combinatoria de las familias casi ajenas; a través de ellas, se puede mostrar la consistencia de la Hipótesis de Jones con $\zfc$.

\begin{corolario}
	Sea $\kappa$ un cardinal con $\omega \leq \kappa <\mathfrak{c}$, entonces bajo $\Ma(\kappa)$, se tiene $2^\kappa=\mathfrak{c}$.

	Consecuentemente, es consistente con $\zfc$ que $2^{\aleph_0}=2^{\aleph_1}$
\end{corolario}

\begin{proof}
	Sea $\kappa$ un cardinal con $\omega \leq \kappa <\mathfrak{c}$ y supóngase $\Ma(\kappa)$. Tomando en cuenta \ref{cor-famGrandes}, fíjese una familia casi ajena $\ms{A}$ con $|\ms{A}|=\kappa$ y defínase la función $f \colon \ms{P}(\omega) \to \ms{P}(\ms{A})$ como $ f(X)=\{ b \in \ms{A} \tq b \cap X =^* \emptyset \} $.

	Si $\ms{B} \subseteq \ms{A}$ es cualquiera, entonces $|\ms{A} \setminus \ms{B}|, |\ms{B}| \leq \kappa$ y por el Lema de Solovay (\ref{lem-Solovay}) se concluye la existencia de un separador $D \subseteq \omega$ para $\ms{A} \setminus \ms{B}$ y $\ms{B}$. Esto resulta en que $ f(D)=\{b \in \ms{A} \tq b \cap X =^* \emptyset \} = \ms{B} $. Luego $f$ es sobreyectiva y $\mathfrak{c} \geq 2^\kappa $. Como además $\kappa \geq \aleph_0$, entonces $2^\kappa \geq 2^{\aleph_0}=\mathfrak{c}$.

	Para la segunda parte, es suficiente notar que $\Ma + \lnot \HC$ es consistente con $\zfc$, y que a partir de ellos se concluye $\Ma(\aleph_1)$ y $\omega \leq \aleph_1 < \mathfrak{c}$.
\end{proof}
        \chapter{Espacios de Isbell-Mrówka}\label{chap-mrowkas}
\emph{\small Los $\Psi$-espacios; o espacios de Mrówka, cuentan con un lugar privilegiado en la topología de conjuntos; esto se debe a que son, entre otras cosas, espacios idóneos para la búsqueda de ejemplos. Esta virtud tiene por motivo las múltiples caracterizaciones que existen para sus propiedades topológicas.}

\emph{\small La intención primordial del presente capítulo es presentar aspectos; en primer lugar, cubrir la definición de los $\Psi$-espacios y exhibir sus propiedades topológicas elementales; y en segundo lugar, dar una caracterización para los espacios de Mrówka en términos de propiedades topológicas, el hoy conocido como Teorema de Kannan y Rajagopalan.}

\section{\texorpdfstring{$\Psi$-espacios y caracterizaciones elementales}{Psi-espacios y caracterizaciones elementales}}

Dada $\ms{A} \subseteq [\omega]^\omega$, se satisface que $\omega \cap \ms{A} = \emptyset$. Por cómo se define la topología de Isbell-Mrókwa (en un conjunto $N$, dada $\ms{A}\subseteq[N]^\omega$), es conveniente establecer lo siguiente.

\begin{consideracion}\label{cons-ajenosNyA}
	A partir de ahora, siempre que $N$ sea un conjunto numerable y $\ms{A} \subseteq [N]^\omega$, se asumirá que $N \cap \ms{A} = \emptyset$.
\end{consideracion}

En el espacio (de ordinales) $X=\omega+1$, cada punto de $\omega$ es aislado, pero el punto $\omega \in X$; situado ``en la periferia'' de $X$, se mantiene cercano al subconjunto $\omega \subseteq X$ del espacio.

Los $\Psi$-espacios tienen por conjunto subyacente a $\omega \cup \ms{A}$; y pueden ser vistos como una forma generalizada de $\omega+1$. Se configura su topología de modo que $\omega$ es una masa de puntos aislados, y cada punto $\omega \cup \ms{A}$ ``en la periferia del espacio'' permanece cercano al subconjunto $a \subseteq \omega \cup \ms{A}$ del espacio.

\begin{proposicion}
	Sean $N$ un conjunto numerable y $\ms{A} \subseteq [N]^\omega$. La siguiente colección es una topología para $N \cup \ms{A}$.\index[sym]{$\tau_{N,\ms{A}}$}
	$$ \tau_{N,\ms{A}} := \{ U \subseteq N \cup \ms{A} \tq \forall x \in U \cap \ms{A} \: ( x \subseteq^* U ) \} $$
\end{proposicion}
\begin{proof}
	Resulta evidente que $\emptyset, N \cup \ms{A} \in \tau_{N,\ms{A}}$. Ahora, dados $U,V \in \tau_{N,\ms{A}}$ y $x \in (U \cap V) \cap \ms{A}$ cualquiera, $x \subseteq^* U$ y $x \subseteq^* V$, de donde $x \subseteq^* U \cap V$ y $U \cap V \in \tau_{N,\ms{A}}$. Finalmente, dados $\mathcal{U}\subseteq \tau_{N,\ms{A}}$ y $x \in \midcup \mathcal{U} \cap \ms{A}$ arbitrarios, existe $U_0 \in \mathcal{U}$ con $x \in U_0$; así que $x \subseteq^* U_0$ y consecuentemente $x \setminus U_0$ es finito. Como $x \setminus \midcup \mathcal{U} \subseteq x \setminus U_0$, resulta que $x \subseteq^* \midcup \mathcal{U}$ y así $\midcup \mathcal{U} \in \tau_{N,\ms{A}}$.
\end{proof}

\begin{definicion}\label{Def-Mrowka}\index[alph]{topología!de Mrówka}\index[alph]{topología!de Isbell-Mrówka}\index[alph]{Mrówka! topología de}\index[alph]{Isbell-Mrówka!topología de}\index[sym]{$\tau_\ms{A}$}\index[alph]{$\Psi$-espacio}\index[alph]{espacio!,$\Psi$}\index[sym]{$\Psi_N(\ms{A})$}\index[sym]{$\Psi(\ms{A})$}
	Sean $N$ un conjunto numerable y $\ms{A} \subseteq [N]^\omega$.
	\begin{enumerate}
		\item La colección $\tau_{N,\ms{A}}$ de la Proposición anterior es la \textbf{Topología de Mrówka (de Isbell-Mrókwa)} \textbf{generada por $\ms{A}$}.
		\item El \textbf{$\Psi$-espacio}\index[alph]{$\Psi$-espacio}\index[alph]{espacio!,$\Psi$} \textbf{generado por $\ms{A}$} se denota por $\Psi_N(\ms{A})$\index[sym]{$\Psi_N(\ms{A})$}, y consta del conjunto $N \cup \ms{A}$ dotado con su topología $\tau_{N,\ms{A}}$.
	\end{enumerate}

	Si $N=\omega$, se denotarán $\tau_\ms{A}:=\tau_{N,\ms{A}}$ y $\Psi(\ms{A})=\Psi_N(\ms{A})$.
\end{definicion}

Previo a abordar otros temas, se mostrará por qué; a efectos topológicos, bastará considerar familias de subconjuntos de $\omega$.% para estudiar las propiedades topológicas de los $\Psi$-espacios.

\begin{proposicion}\label{prop-MrowHomeoBiyec}
	Sean $N,M$ conjuntos numerables, $\ms{A} \subseteq [N]^\omega$ arbitraria y $h:N \to M$ una biyección. Entonces $\Psi_N(\ms{A}) \cong \Psi_M(\Phi_h(\ms{A}))$.
\end{proposicion}
\begin{proof}
	Sea $f:\Psi_N(\ms{A}) \to \Psi_M(\Phi_h(\ms{A}))$ definida por medio de $f(x)=h(x)$ si $x \in N$ y $f(x)=h[x]$ si $x \in \ms{A}$. Nótese que; por la \autoref{cons-ajenosNyA}, $f$ es biyectiva. Además, por definición de $f$, y como $\Psi_h^{-1}=\Phi_{h^{-1}}$, basta verificar únicamente la continuidad de $f$.

	Sea $U$ abierto en $\Psi_M(\Phi_h(\ms{A}))$ y supóngase que $x \in f^{-1}[U] \cap \ms{A}$. Entonces $f(x) = h[x] \in U \cap \Phi_h(\ms{A})$. Como $U$ es abierto en $\Psi_M(\Phi_h(\ms{A}))$, entonces $f(x) \setminus U$ es finito. Así que $f^{-1}[f(x) \setminus U] = h^{-1}[h(x)] \setminus f^{-1}[U] = x \setminus f^{-1}[U]$ es finito y así $x \subseteq^* f^{-1}[U]$, probando $f^{-1}[U]$ es abierto en $\Psi_N(\ms{A})$.
\end{proof}

La siguiente manera de describir la topología de Mrówka es la más común en la literatura (como ejemplo están \cite{hruMrowka} o \cite{hruAlmost}).

\begin{proposicion}\label{prop-BaseLocMrowka}\index[sym]{$\mathcal{B}_x$}\index[alph]{base!local!estándar de $x$ en $\Psi_N(\ms{A})$}
	Sea $\ms{A} \subseteq [\omega]^\omega$, entonces:
	\begin{enumerate}[i)]
		\item Cada $B \subseteq \omega$ es abierto en $\Psi(\ms{A})$, en particular, cada $n \in \omega$ es punto aislado.
		\item Si $x \in \ms{A}$, el conjunto $\mathcal{B}_x:=\{ \{x\} \cup x \setminus F \tq F \in[x]^{<\omega} \}$ es base local de $x$ en $\Psi(\ms{A})$. $\mathcal{B}_x$ es la \textbf{base local estándar de $x$ en $\Psi_N(\ms{A})$}.
	\end{enumerate}
\end{proposicion}
\begin{proof}
	(i) Si $B \subseteq \omega$, es vacuo que $B \in \tau_\ms{A}$, pues $B \cap \ms{A} = \emptyset$.

	(ii) Sea $x \in \ms{A}$, entonces $\mathcal{B}_x \subseteq \tau_\ms{A}$. En efecto, si $G \subseteq x$ es finito y $y \in \big( \{x\} \cup x \setminus G \big) \cap \ms{A}$, necesariamente $y=x$, de donde $y \subseteq^* \{x\} \cup x \setminus G$ pues $G$ es finito, así $\{x\} \cup x \setminus G \in \tau_{\ms{A}}$. Ahora, si $U \subseteq \Psi(\ms{A)}$ es abierto y $x \in U$, $F:= x \setminus U \subseteq x$ es finito y $x \in \{x\} \cup x \setminus F \subseteq U$.
\end{proof}

\begin{corolario}\label{cor-NumAxMrowka}\index[sym]{$\mathcal{B}_\ms{A}$}\index[alph]{base!estándar de $\Psi_N(\ms{A})$}
	Si $N$ es numerable y $\ms{A} \subseteq [N]^\omega$, entonces:
	\begin{enumerate}[i)]
		\item $\Psi(\ms{A})$ es $1\AN$.
		\item $\mathcal{B}_{\ms{A}} := \midcup \{ \mathcal{B}_x \tq x \in \ms{A} \} \cup \big\{ \{n\} \tq n \in N \big\}$; denominado la \textbf{base estándar de} $\Psi_N(\ms{A})$, es una base de $\Psi_N(\ms{A})$ de tamaño $\aleph_0+|\ms{A}|$.
		\item $w(\Psi_N(\ms{A})) = \aleph_0+|\ms{A}|$. Por ello, $\Psi(\ms{A})$ es $2\AN$ si y sólo si $|\ms{A}|\leq \aleph_0$.
	\end{enumerate}
\end{corolario}
\begin{proof}
	(i), (ii) y $w(\Psi_N(\ms{A})) \leq \aleph_0+|\ms{A}|$ son claros.

	Para $\aleph_0+|\ms{A}| \leq w(\Psi_N(\ms{A}))$ basta observar que $\omega,\ms{A} \subseteq \Psi(\omega)$ son subespacios discretos de tamaño (y por tanto, peso) $\aleph_0$ y $|\ms{A}|$, respectivamente. Por consiguiente, el peso de $\Psi(\omega)$ debe ser mayor o igual que ambos.
\end{proof}

Si $\ms{A}\subseteq [\omega]^\omega$ y $X \subseteq \Psi(\ms{A})$, dado que cada punto de $\omega$
es aislado, se tiene que $\der(X) \subseteq \ms{A}$. Por otra parte, si $a \in \ms{A}$, la única forma de que cada $a \setminus F$ (con $F \in [a]^{<\omega}$) tenga intersección no vacía con $X$ es que $X \cap a$ sea infinito.

Debido a \ref{prop-BaseLocMrowka}, la discusión recién dada prueba el primer inciso (y con ello todos los restantes) del siguiente útil Lema.

\begin{lema}\label{lem-primerosSubs}
	Sea $\ms{A} \subseteq [\omega]^\omega$, entonces:
	\begin{enumerate}[i)]
		\item Si $X \subseteq \Psi(\ms{A})$ , entonces $ \der(X)=\{ y \in \ms{A} \tq X \cap y \neq^* \emptyset \} $.
		\item $\ms{A}=\der( \Psi(\ms{A}) )$ y $\omega$ es discreto, denso en $\Psi(\ms{A})$.
		\item Cada $B \subseteq \ms{A}$ es un subespacio cerrado y discreto de $\Psi(\ms{A})$.
		\item $B \subseteq \omega$ es cerrado en $\Psi(\ms{A})$ sólo si es casi ajeno con cada elemento de $\ms{A}$.
	\end{enumerate}
\end{lema}

\begin{proposicion}\label{prop-PsiSiempre}
	Todo $\Psi$-espacio es separable, primero numerable, $\T_1$, disperso y desarrollable.
\end{proposicion}

\begin{proof}
	Sea $\ms{A} \subseteq [\omega]^\omega$ cualquiera. El $\Psi$-espacio generado por $\ms{A}$ es separable pues $\omega$ es denso en $\Psi(\ms{A})$ y numerable; además, éste espacio es primero numerable debido al \autoref{cor-NumAxMrowka}.

	(Axioma $\T_1$) Si $x \in \ms{A}$, del \autoref{lem-primerosSubs} se desprende la igualdad $\der(\{x\})=\{y \in \ms{A} \tq \{x\} \cap y \neq^* \emptyset \}=\emptyset$, lo cual implica que $\{x\}$ es cerrado.

	(Dispersión) Supóngase que $X \subseteq \Psi(\ms{A})$ es cualquier subconjunto no vacío. Si $X \subseteq \ms{A}$, cualquier $x \in X$ es aislado en $X$, pues $X$ es discreto (véase \ref{lem-primerosSubs}). En caso contrario, existe un elemento $x \in X \cap \omega$ y $x$ es aislado en $X$, pues $\{x\}$ es abierto al ser $x$ elemento de $\omega$.

	(Desarrollabilidad) Defínase $\mathcal{U}_n:=\{ \{a\} \cup a \setminus n \tq a \in \ms{A} \} \cup \{ \{y\} \tq y \in \omega \}$ para cada $n \in \omega$. Resulta claro que cada colección $\mathcal{U}_n$ es cubierta abierta de $\Psi(\ms{A})$. Sean $x \in \Psi(\ms{A})$ y $U$ un abierto tal que $x \in U$.

	Si $x \in \omega$, entonces $\{x\} = \St(x, \mathcal{U}_{x+1})$; en efecto, sea $V \in \mathcal{U}_{x+1}$ con $x \in V$, entonces $V=\{x\}$; pues de lo contrario $V =\{a\} \cup a \setminus (x+1)$ para cierto $a \in \ms{A}$, implicando esto que $x \notin x+1$, lo cual es imposible ya que $x \in \omega$. Por tanto, $x \in \{x\} = \St(x,\mathcal{U}_{x+1}) \subseteq U$.

	Si $x \in \ms{A}$, entonces $x \subseteq^* U$ y $x \setminus U \subseteq \omega$ es finito y por ello existe $n_0 \in \omega$ tal que $x \setminus U \subseteq n_0$. Como $\{x\} \cup x \setminus n_0 \in \mathcal{U}_{n_0}$ es el único abierto de $\mathcal{U}_{n_0}$ al cual $x$ pertenece, $x \in \{x\} \cup x \setminus n_0 = \St(x, \mathcal{U}_{n_0}) \subseteq U$.

	Así pues, para cada $n \in \omega$, la colección $\{\St(x, \mathcal{U}_n) \tq n \in \omega\}$ es base local de $x$. Así que $\{\mathcal{U}_n \tq n \in \omega \}$ es un desarrollo para $\Psi(\ms{A})$.
\end{proof}

Como se probó recién, todo $\Psi$-espacio es $\T_1$, sin embargo, cuando la familia $\ms{A} \subseteq [\omega]^\omega$ no es casi ajena, el espacio $\Psi(\ms{A})$ no satisface el axioma de separación $\T_2$. Por esta razón, en la literatura se suele dar la \autoref{Def-Mrowka} partiendo directamente de una familia casi ajena (el lector podrá corroborar esto en textos como \cite{hruMrowka,hruAlmost,kannanHereditarily}).

\begin{proposicion}\label{prop-tra-casiAjenidad}\index[trad]{cero-dimensionalidad de $\Psi(\ms{A})$}\index[trad]{propiedad de! Tychonoff en $\Psi(\ms{A})$}\index[trad]{propiedad de!Hausdorff en $\Psi(\ms{A})$}
	Para cualquier $\ms{A}\subseteq [\omega]^\omega$ son equivalentes las siguientes condiciones:
	\begin{enumerate}[i)]
		\item $\ms{A}$ es familia casi ajena.
		\item $\Psi(\ms{A})$ es cero-dimensional.
		\item $\Psi(\ms{A})$ es de Tychonoff.
		\item $\Psi(\ms{A})$ es de Hausdorff.
	\end{enumerate}
\end{proposicion}

\begin{proof}
	(i) $\rightarrow$ (ii) Si $\ms{A}$ es familia casi ajena, como $\Psi(\ms{A})$ es $\T_1$, basta verificar que cada elemento de la base estándar $\ms{B}_\ms{A}$ (definida en el \autoref{cor-NumAxMrowka}) es cerrado. En efecto, cada $\{n\}$ con $n \in \omega$ es cerrado pues $\Psi(\ms{A})$ es $\T_1$. Y dados $x \in \ms{A}$ y $F \subseteq x$ finto, haciendo uso de \ref{lem-primerosSubs} se tiene que por ser $\ms{A}$ familia casi ajena, $\der(\{x\} \cup x \setminus F)=\{x\} \subseteq \{x\} \cup x \setminus F$. Así que $\{x\} \cup x \setminus F$ es cerrado, mostrando que $\Psi(\ms{A})$ es cero-dimensional, pues es $\T_1$ y contiene una base de abiertos y cerrados (\textbf{resultado R)}.

	(ii) $\rightarrow$ (iii) $\rightarrow$ (iv) Si $\Psi(\ms{A})$ es cero-dimensional, al ser espacio $\T_1$, resulta que entonces es espacio de Tychonoff (\textbf{resultado R)}. Por su parte, si $\Psi(\ms{A})$ es de Tychonoff, entonces es de Hausdorff.

	(iv) $\rightarrow$ (i) Si $\Psi(\ms{A})$ es de Hausdorff y $x,y \in \ms{A}$ son distintos, existen abiertos ajenos $U,V \subseteq \Psi(\ms{A})$ abiertos tales que $x \in U$ y $y \in V$. De donde $x \subseteq^* U$, $y \subseteq^* V$ y por consiguiente $x \cap y \subseteq^* U \cap V = \emptyset$.
\end{proof}

La Proposición anterior es el motivo por el cual el presente trabajo se enfocará únicamente la siguiente clase de espacios:

\begin{definicion}
	Un \textbf{espacio de Mrówka (o, de Isbell-Mrówka)}\index[alph]{espacio!de Mrówka}\index[alph]{Mrówka!espacio de}\index[alph]{espacio!de Isbell-Mrówka}\index[alph]{Isbell-Mrówka!espacio de} es un $\Psi$-espacio generado por una familia casi ajena.
\end{definicion}

\begin{corolario}\label{cor-MrwokaSiempre}
	Todo espacio de Mrókwa es separable, primero numerable, de Tychonoff, cero-dimensional, disperso y de Moore.
\end{corolario}

La siguiente es sólo una de las múltiples relaciones importantes que existen entre los espacios de Mrówka y el conjunto de Cantor. Su demostración se basa en un hecho conocido en topología general; todo espacio cero-dimensional de peso $\kappa$ se encaja en $2^\kappa$ (véase \cite[Teo.~8.5.11, p.~299]{fidelElementos}).

\begin{corolario}\label{cor-EncajeMrowkaCantor}
	Todo espacio de Mrówka $\Psi(\ms{A})$ se encaja en $2^{\aleph_0+|\ms{A}|}$. Particularmente, si $|\ms{A}|\leq \aleph_0$, el espacio $\Psi(\ms{A})$ se encaja en $2^\omega$ y es metrizable.
\end{corolario}

\section{Compacidad y local compacidad}

Como el lector puede advertir, cada vez surgen más traducciones con las cuales maniobrar al momento de estudiar los $\Psi$-espacios. El ideal generado por cierta $\ms{A} \in \Ad(\omega)$ es clave para distinguir cuáles subespacios de $\Psi(\ms{A})$ son compactos, y cuales no.

\begin{proposicion}\label{prop-Kcaract}\index[trad]{compacidad de los subespacios de $\Psi(\ms{A})$}
	Sean $\ms{A}\in \Ad(\omega)$ y $K \subseteq \Psi(\ms{A})$. Entonces $K$ es compacto si y sólo si $K \cap \omega \subseteq^* \midcup (K \cap \ms{A})$ y $K \cap \ms{A}$ es finito.
\end{proposicion}
\begin{proof}
	Supóngase que $K \subseteq \Psi(\ms{A})$ es subespacio compacto, como la colección $\mathcal{U}:=\{ \{n\} \tq n \in K \cap \omega \} \cup \{ \{x\} \cup x \tq x \in K \cap \ms{A} \}$ es cubierta abierta para $K$ en $\Psi(\ms{A})$, existen $F \subseteq K \cap \omega$ y $G \subseteq K \cap \ms{A}$ finitos tales que $\{ \{n\} \tq n \in F\} \cup \{ \{x\} \cup x \tq x \in G\}$ es subcubierta de $\mathcal{U}$. Luego, es necesario que $K \cap \ms{A} =G$, así que $K \cap \ms{A}$ es finito. Además $(K \cap \omega) \setminus \midcup G = K \setminus \midcup G \subseteq F$ es finito y con ello $K \cap \omega \subseteq^* \midcup (K \cap \ms{A})$.

	Conversamente, supóngase que $K \cap \omega \subseteq^* \midcup (K \cap \ms{A})$ y que $K \cap \ms{A}$ es finito. Resulta claro que; si $y \in \ms{A}$, entonces $\{y\} \cup y$ es un subespacio compacto de $\Psi(\ms{A})$; consecuentemente $L:=\midcup\{ \{y\} \cup y \tq y \in K \cap \ms{A} \}$ es un subespacio compacto de $\Psi(\ms{A})$.

	Nótese que $K \cap L$ es cerrado en $L$; pues $L \setminus K \subseteq \omega$; consecuentemente $K \setminus L$ es compacto. Como $K \setminus L = (K \cap \omega) \setminus \midcup (K \cap \ms{A})$ es finito por hipótesis, $K \setminus L$ es compacto. Así, $K=(K \setminus L) \cup (K \cap L)$ es unión de subsespacios compactos de $\Psi(\ms{A})$; por tanto, es compacto.
\end{proof}

Así, los los subespacios compactos de $\Psi(\ms{A})$ son únicamente aquellos de la forma $M \cup H$; donde $H \subseteq \ms{A}$ es finito y $M \subseteq^* \midcup H$. Esto es, si $\mathcal{K}$ el conjunto de los subespacios compactos de $\Psi(\ms{A})$:
$$ \mathcal{K}=\bigcup_{H \in [\ms{A}]^{<\omega}} \{ F \cup M \cup H \tq (F,M) \in [\omega]^{<\omega} \times \ms{P}(H) \} $$

Por ello $|\ms{A}| \cdot \aleph_0 \leq |\mathcal{K}| \leq \sum \{ (\aleph_0 \cdot \mathfrak{c} ) \tq H \in [\ms{A}]^{<\omega} \} \leq |\ms{A}| \cdot \mathfrak{c} \leq \mathfrak{c} $; asi que todo espacio de Mrówka tiene; a lo sumo, $\mathfrak{c}$ subespacios compactos.

La discusión sobre cuántos subespacios compactos \textit{importantes} (esto es, los que determinan el carácter topológico de su extensión unipuntual) tiene $\Psi(\ms{A})$ se retomará en la \autoref{Subsec-sucesiones-Franklin}.

\begin{corolario}\label{cor-IdealCompactosCarac}
	Sean $\ms{A}$una familia casi ajena y $A \subseteq \omega$ cualquiera. Entonces son equivalentes las siguientes condiciones:
	\begin{enumerate}[i)]
		\item $A \in \ms{I}(\ms{A})$
		\item Existe $K \subseteq \Psi(\ms{A})$ compacto tal que $A \subseteq K$.
		\item Existe $K \subseteq \Psi(\ms{A})$ compacto tal que $A \subseteq^* K$.
	\end{enumerate}
\end{corolario}

\begin{proof}
	(i) $\to$ (ii) Si $A \in \ms{I}(\ms{A})$, existe $H \subseteq \ms{A}$ finito tal que $A \subseteq^* \midcup H$. De la Proposición anterior se desprende que $K:=A \cup H$ es un subespacio compacto de $\Psi(\ms{A})$ tal que $A \subseteq K$.

	La implicación (ii) $\to$ (iii) es clara, procédase con la restante.

	(iii) $\to$ (i) Supóngase que $K$ es un subespacio compacto de $\Psi(\ms{A})$ tal que $A \subseteq^* K$. Consecuentemente $A \setminus \midcup (K \cap \ms{A}) \subseteq^* A \setminus (K \cap \omega) = A \setminus K =^* \emptyset$, en virtud de la Proposición previa. Lo anterior; dado que $K \cap \ms{A}$ es finito, muestra que $A \in \ms{I}(\ms{A})$.
\end{proof}

\begin{proposicion}\label{prop-tra-compacidad}\index[trad]{compacidad de $\Psi(\ms{A})$}\index[trad]{compacidad numerable de $\Psi(\ms{A})$}
	Sea $\ms{A}\in \Ad(\omega)$, entonces son equivalentes:
	\begin{enumerate}[i)]
		\item $\Psi(\ms{A})$ es compacto.
		\item $\Psi(\ms{A})$ es numerablemente compacto.
		\item $\ms{A}$ es finita y maximal.
	\end{enumerate}
\end{proposicion}

\begin{proof}
	La implicación (i) $\rightarrow$ (ii) es evidente.

	(ii) $\rightarrow$ (iii) Supóngase que $\Psi(\ms{A})$ es numerablemente compacto. Dado que $\ms{A} \subseteq \Psi(\ms{A})$ es subespacio cerrado y discreto de $\Psi(\ms{A})$ (véase \ref{lem-primerosSubs}), entonces $\ms{A}$ es numerablemente compacto y discreto; por ello, es finito. De esta forma, $\mathcal{U}:=\big\{ \{n\} \tq n\in \omega \big\} \cup \big\{ \{x\} \cup x \tq x \in \ms{A} \big\}$ es una cubierta numerable para $\Psi(\ms{A})$ y en consecuencia, existe $F \subseteq \omega$ finito de tal modo que la colección $\big \{ \{n\} \tq n \in F \big\} \cup \big\{ \{x\} \cup x \tq x \in \ms{A}\}$ es subcubierta de $\mathcal{U}$. Por ende $\omega \subseteq^* \midcup \ms{A}$, al ser $\ms{A}$ y $F$ finitos. Así, $\ms{A}$ es maximal en virtud del \autoref{cor-MADnecesarioUnion}.

	(iii) $\rightarrow$ (i) Si $\ms{A}$ es finita y maximal, se desprende del \autoref{cor-MADnecesarioUnion} que $\omega \subseteq^* \midcup \ms{A}$. Así, $\Psi(\ms{A}) \cap \omega \subseteq^* \midcup (\Psi(\ms{A}) \cap \ms{A} )$ y $\Psi(\ms{A}) \cap \ms{A} = \ms{A}$ es finito, siguiéndose dela \autoref{prop-Kcaract} la compacidad de $\Psi(\ms{A})$.
\end{proof}

La siguiente Proposición para nada carece de importancia, pues los espacios de Isbell-Mrówka son los únicos (dentro de cierta clase) con tal propiedad.

\begin{proposicion}\label{prop-MrwokaHLC}
	Todo espacio de Mrówka es hereditariamente localmente compacto, y en consecuencia, es espacio de Baire.
\end{proposicion}

\begin{proof}
	Supóngase que $\ms{A}\in \Ad(\omega)$ y sea $X \subseteq \Psi(\ms{A})$ cualquiera. Como $\Psi(\ms{A})$ es de Hausdorff (recuérdese \ref{cor-MrwokaSiempre}), $X$ es de Hausdorff y basta verificar que cada punto de $X$ tiene una vecindad en $X$ compacta.

	Sea $x \in X$ arbitrario. Si $x \in \omega$, entonces $\{x\}$ es vecindad compacta de $x$ en $X$. Ahora, si $x \in \ms{A}$, entonces $K:=X \cap (\{x\} \cup x) \subseteq \{x\} \cup x$ es vecindad de $x$ en $X$. Además $K$ es compacto, en virtud del \autoref{cor-IdealCompactosCarac}, pues $K \cap \ms{A} = \{x\}$ es finito y $K \cap \omega \subseteq x \subseteq^* \midcup \{x\} = \midcup (K \cap \ms{A})$. Así, $X$ es localmente compacto y $\Psi(\ms{A})$ hereditariamente localmente compacto.

	Consecuentemente, $\Psi(\ms{A})$ es localmente compacto y de Hausdorff, siendo esto suficiente para ser de Baire \textbf{(Teorema de Categoría de Baire)}.
\end{proof}

\section{Metrizabilidad y Pseudocompacidad}

El \autoref{cor-EncajeMrowkaCantor} evidencía que la numerabilidad de una familia casi ajena $\ms{A}$ es suficiente para concluir la metrizabilidad de su espacio de Mrówka asociado, no resulta difícil notar que el recíproco también ocurre (dados \ref{cor-NumAxMrowka} y que $\Psi(\ms{A})$ es separable); sin embargo, se tienen más equivalencias:

\begin{proposicion}\label{prop-tra-numerable}\index[trad]{metrizabilidad de $\Psi(\ms{A})$}\index[trad]{segundo numerabilidad de $\Psi(\ms{A})$}\index[trad]{$\sigma$-compacidad de $\Psi(\ms{A})$}\index[trad]{propiedad de!Lindelöf en $\Psi(\ms{A})$}
	Sea $\ms{A}\in \Ad(\omega)$, entonces son equivalentes:
	\begin{enumerate}[i)]
		\item $\ms{A}$ es a lo más numerable
		\item $\Psi(\ms{A})$ es metrizable.
		\item $\Psi(\ms{A})$ es segundo numerable.
		\item $\Psi(\ms{A})$ es $\sigma$-compacto.
		\item $\Psi(\ms{A})$ es de Lindelöf.
	\end{enumerate}
\end{proposicion}

\begin{proof}
	(i) $\rightarrow$ (ii) $\rightarrow$ (iii) Si $|\ms{A}| \leq \omega$, se obtiene de \ref{cor-EncajeMrowkaCantor} que $\Psi(\ms{A})$ es metrizable. Por otro lado, si $\Psi(\ms{A})$ es metrizable, al ser éste un espacio separable, se tiene garantizado que es $2\AN$ \textbf{(MtzEq)}.

	(iii) $\rightarrow$ (iv) $\to$ (v) Si $\Psi(\ms{A})$ es $2\AN$, entonces al localmente compacto, resulta que es $\sigma$-compacto \textbf{(Resultado R)}. Además; todo espacio $\sigma$-compacto, es también de Lindelöf. \textbf{(Resultado R)}

	(v) $\rightarrow$ (i) Por último, supóngase que $\Psi(\ms{A})$ es de Lindelöf y sea $\mathcal{B}_\ms{A}$ la base estándar de $\Psi(\ms{A})$ (definida en \ref{cor-NumAxMrowka}). Luego $\mathcal{B}_ \ms{A}$ es una cubierta abierta de $\Psi(\ms{A})$, y deben existir $\ms{A}' \subseteq \ms{A}$ y $N \subseteq \omega$ a lo más numerables tales que $\midcup \big\{ \big\{ \{x\} \cup x \setminus F \tq F \in [x]^{<\omega} \big\} \tq x \in \ms{A}' \big\} \cup \big\{ \{n\} \tq n \in N \big\}$ es subcubierta de $\mathcal{B}_ \ms{A}$. Resulta así que $\ms{A} \subseteq \ms{A}'$ y $|\ms{A}| \leq \aleph_0$.
\end{proof}

Como fue mostrado en \ref{prop-MADnoNum}, ninguna familia casi ajena numerable es maximal. Así que si $\ms{A}$ es una familia casi ajena numerable, por \ref{prop-tra-compacidad}, $\Psi(\ms{A})$ no es compacto. Consecuentemente (por \ref{prop-tra-numerable}), si $\ms{A}$ es numerable, $\Psi(\ms{A})$ es Lindelöf y $\sigma$-compacto, pero no compacto.

\begin{observacion}
	Si $\Psi(\ms{A})$ es metrizable (o cualquiera de sus equivalentes planteados en \ref{prop-tra-numerable}) y no compacto, no necesariamente $\ms{A}$ es numerable. Esto responde al sencillo motivo de que $\ms{A}$ podría ser maximal o no; la \autoref{prop-tra-numerable} no toma en cuenta este aspecto.
\end{observacion}

La observación recién hecha da constancia de que falta establecer una relación entre $\Psi(\ms{A})$ y la maximalidad de la familia $\ms{A}$. En la \autoref{Subsec-sucesiones-Franklin} se ahondará con mucha más profundidad en el estudio de las sucesiones convergentes; pero de momento, es necesario considerar el siguiente Lema, en orden de dar una caracterización completa para $\ms{A} \in \Mad(\omega)$.

\begin{lema}\label{lem-convObvia}
	Sean $\ms{A} \in \Ad(\omega) $, $x \in \ms{A}$ y $B \subseteq [\omega]^\omega$ cualesquiera. Entonces $B \to x$ en $\Psi(\ms{A})$ si y sólo si $B \subseteq ^* x$.
\end{lema}

\begin{proof}
	Supóngase que $B \to x$ en $\Psi(\ms{A})$. Entonces, como $x \cup \{x\}$ es un abierto de $\Psi(\ms{A})$ que contiene a $x$, se tiene que $B \subseteq^* x \cup \{x\}$, mostrando que $B \subseteq^* x$. Y recíprocamente, si $B \subseteq^* x$ y $U \subseteq \Psi(\ms{A})$ es cualquier abierto con $x \in U$, entonces $x \subseteq^* U$, y por tanto, $B \subseteq^* U$.
\end{proof}

\begin{proposicion}\label{prop-tra-pseudoCaract}\index[trad]{pseudocompacidad de $\Psi(\ms{A})$}
	Sea $\ms{A} \in \Ad(\omega)$, son equivalentes:
	\begin{enumerate}[i)]
		\item $\Psi(\ms{A})$ es pseudocompacto.
		\item $\ms{A}$ es maximal.
		\item Todo subespacio discreto, abierto y cerrado de $\Psi(\ms{A})$ es finito.
		\item Toda sucesión en $\omega$ tiene una subsucesión convergente.
	\end{enumerate}
\end{proposicion}

\begin{proof}
	(i) $\rightarrow$ (ii). Si $\ms{A}$ no es maximal, existe $B \subseteq \omega$ infinito y casi ajeno con cada elemento de $\ms{A}$. Por \ref{prop-BaseLocMrowka} y \ref{lem-primerosSubs}, $B$ es discreto, abierto y cerrado, y de \textbf{(Ree A)} se sigue que $\Psi(\ms{A})$ no es pseudocompacto.

	(ii) $\rightarrow$ (iii) Por contrapuesta, supóngase que $B \subseteq \Psi(\ms{A})$ es infinito, discreto, abierto y cerrado de $\Psi(\ms{A})$. Sin pérdida de generalidad $B \subseteq \omega$ (de lo contrario cada $a \in B \cap \ms{A}$ cumple que $a \cap B = B \cap (\{a\} \cup a) \subseteq \omega$ es infinito, cerrado, abierto y discreto). Luego, de \ref{lem-primerosSubs} se desprende que $B$ casi ajeno con cada elemento de $\ms{A}$.

	(iii) $\rightarrow$ (iv) Supóngase (iii) y sea $B \in [\omega]^\omega $. Así, $B$ es discreto, infinito y abierto. Por hipótesis, debe existir $x \in \der(B) \setminus B$ y por \ref{cor-MrwokaSiempre}, $x \in \ms{A}$ y $B \cap x$ es infinito. Siguiéndose del \autoref{lem-primerosSubs} que $B \cap x \to x$.

	(iv) $\rightarrow$ (i) Por contrapuesta, supóngase que $f:\Psi(\ms{A}) \to \mathbb{R}$ es continua y no acotada. Entonces; por densidad de $\omega$, para cada $n \in \omega$ se puede fijar $m_n \in \omega \cap f^{-1}[(n,\infty)]$. Así, $B=\{m_n \tq n \in \omega\}$ es infinito, y no admite subsucesiones convergentes en $\Psi(\ms{A})$, pues ningún $C \in [B]^\omega$ tiene imagen no acotada bajo $f$.
\end{proof}

Combinando \ref{prop-tra-compacidad}, \ref{prop-tra-numerable} y \ref{prop-tra-pseudoCaract} se obtienen ejemplos muy concretos. Por ejemplo, si un espacio de Mrówka $\Psi(\ms{A})$ no es pseudocompacto pero sí es metrizable, necesariamente $\ms{A}$ es numerable. Otro ejemplo responde con una negativa a lo que en su momento fue un problema popular: ¿la pseudocompacidad equivale a la compacidad numerable en espacios Tychonoff?, resultado se sabía cierto en la clase de espacios $\T_4$ (\textbf{Reee R}) y falso dentro de la clase de espacios que no son $\T_1$. En virtud de \ref{prop-tra-compacidad}, y considerando cualquier familia maximal infinita, se obtiene:

\begin{corolario}\label{cor-EjmPseudoNoNumC}
	Existe un espacio de Tychonoff, que es pseudocompacto pero no numerablemente compacto.
\end{corolario}

La siguiente es una caracterización conocida (véase \cite[p.~39,45]{GeorginaTesis}) y; entre tanto, desvela que el comportamiento sumamente organizado y \textit{amigable} de $\Psi(\ms{A})$ se rompe bruscamente cuando $\ms{A}$ deja de ser numerable. Por tal motivo, no suelen ser de tanto interés los $\Psi$-espacios generados por familias casi ajenas a lo más numerables.

\begin{proposicion}\label{prop-alomasNumCaract}\index[trad]{ordenabilidad lineal de $\Psi(\ms{A})$}
	Sea $\ms{A}$ una familia casi ajena con cardinalidad $\kappa$, entonces\footnotemark se satisface:
	\begin{enumerate}[i)]
		\item Si $\kappa=0$, entonces $\Psi(\ms{A}) \cong \omega$.
		\item Si $\kappa \in \omega$ y $\ms{A}$ no es maximal, $\Psi(\ms{A}) \cong \omega \cdot (\kappa+1)$.
		\item Si $\kappa \in \omega$ y $\ms{A}$ es maximal, $\Psi(\ms{A}) \cong \omega \cdot (\kappa+1)+1$.
		\item Si $\kappa=\omega$, entonces $\Psi(\ms{A}) \cong \omega^2$.
		\item Si $\kappa>\omega$, entonces $\Psi(\ms{A})$ no homeomorfo a ningún espacio de ordinales; más aún, $\Psi(\ms{A})$ no es linealmente ordenable.
	\end{enumerate}
\end{proposicion}

\footnotetext{En los incisos (i)-(iv), los espacios homeomorfos a $\Psi(\ms{A})$ están escritos en aritmética ordinal y dotados de su topología de orden.}

Se derivan conclusiones de interés moderado, como puede ser que $\omega^2$ (como producto ordinal) es el único espacio de Mrówka metrizable, no compacto. Una consecuencia \textit{curiosa} en relación a éste espacio; y que además, surge como fruto del Teorema principal de la \autoref{sec-KRTeo}, es el \autoref{cor-omegaCuadra}.
\label{Dif-esencial}
La peculiaridad recién comentada, sugiere que todas las familias casi ajenas numerables son muy \textit{esencialmente iguales} (conviniendo que $\ms{A}$ y $\ms{B}$ son \textit{esencialmente iguales} cuando $\Psi(\ms{A})$ y $\Psi(\ms{B})$ son homeomorfos).

\section{Teorema de Kannan y Rajagopalan}
\label{sec-KRTeo}
La meta primordial en lo que resta del capítulo será caracterizar aquellos espacios que son homeomorfos a algún espacio de Mrówka. Como fue mostrado en la \autoref{prop-MrwokaHLC}, todos los espacios de Mrówka son hereditariamente localmente compactos, una propiedad cuanto menos peculiar. Tal propiedad será la que los caracterizará dentro de la clase de espacios infinitos, de Hausdorff y separables.

\begin{lema}\label{lem-TKR-Baire}
	Sea $X$ un espacio de Hausdorff y localmente compacto. Si $X$ contiene un denso $D$, abierto y a lo más numerable, entonces $N:=X \setminus \der(X) \subseteq X$ discreto y denso en $X$.
\end{lema}

\begin{proof}
	Claramente $N$ es discreto. Por el Teorema de Categoría De Baire (\textbf{TCB}), resulta que $X$ es un espacio de Baire.

	Ahora, si $x \in D$ es aislado en $D$, entonces $\{x\}=D \cap U$ para cierto abierto $U$ de $X$ y dado que $D$ es denso y $X$ es un espacio $\T_1$, es necesario que $U=\{x\}$. Lo anterior prueba que $X \setminus \der_D(D) \subseteq N$.

	Por otra parte, si $x \in \der_D(D) \subseteq \der(X)$, entonces $X \setminus \{x\}$ es abierto y denso en $X$. Luego $X \setminus \der_D(D)=\midcap\{ X \setminus \{x\} \tq x \in \der_D(D) \}$ es denso, debido a que $X$ es de Baire. Lo cual basta para mostrar que $N$ es denso.
\end{proof}

\begin{lema}\label{lem-TKR-DerX}
	Sean $X$ un espacio topológico y $N:=X \setminus \der(X)$. Las siguientes condiciones son equivalentes:
	\begin{enumerate}[i)]
		\item $N$ es denso y para cada $y \in \der(X)$, $N \cup \{y\}$ es abierto.
		\item $\der(X)$ es discreto.
	\end{enumerate}
\end{lema}

\begin{proof}
	(i) $\rightarrow$ (ii) Supóngase (i) y sea $y \in \der(X)$ cualquier elemento. $N \cup \{y\}$ es abierto en $X$, en consecuencia $y \in U \subseteq N \cup \{y\}$, para cierto abierto $U$. Seguido de lo anterior, ${y} = U \setminus N = U \cap \der(X)$. Mostrando que $\der(X)$ es discreto.

	(ii) $\rightarrow$ (i) Supóngase que $\der(X)$ es discreto. Si $N$ no es denso, existen $x \in X$ y un abierto $U$ de modo tal que $x \in U \subseteq \der(X)$. Pero al ser $\der(X)$ discreto, $\{x\}=W \cap \der(X)$ para cierto abierto $W$, de donde $U \cap W = \{x\}$ y $x \in N$, esto es imposible. Así que $N$ es denso en $X$.

	Ahora, si $y \in \der(X)$ es arbitrario, existe un abierto $U$ de modo que se da $\{y\} = U \cap \der(X)$, pues $\der(X)$ es discreto. De lo anterior se obtiene que $N \cup \{y\} = (N \cup U) \cap (N \cup \der(X)) = N \cup U$ es abierto en $X$.
\end{proof}


La siguiente caracterización es debida a Varadachariar Kannan y a Minakshisundaram Rajagopalan, quienes en 1970 (consúltese \cite{kannanHereditarily}) dieron con el resultado.

\begin{teorema}[Kannan, Rajagopalan]\label{teo-HLCCaract}\index[alph]{Kannan!Teorema de Rajagopalan y}\index[alph]{Rajagopalan!Teorema de Kannan y}\index[alph]{Teorema! de Kannan y Rajagopalan}\index[trad]{compacidad local hereditaria de cualquier espacio infinito, separable, de Hausdorff}
	Para cualquier espacio topológico $X$ infinito, de Hausdorff y separable son equivalentes:
	\begin{enumerate}[i)]
		\item $X$ es hereditariamente localmente compacto.
		\item $X$ es localmente compacto $\der(X)$ es discreto.
		\item $X$ es homeomorfo a un espacio de Mrówka.
	\end{enumerate}
\end{teorema}

\begin{proof}
	Supóngase que $X$ es cualquier espacio infinito, de Hausdorff, separable y sea $N:=X \setminus \der(X)$.

	(i) $\rightarrow$ (ii) Supóngase que $X$ es hereditariamente localmente compacto. Por separabilidad de $X$, existe $D \subseteq X$ denso y a lo más numerable. Se sigue de la hipótesis que $D$ es localmente compacto y por ello, es abierto en su cerradura, $X$. Debido al \autoref{lem-TKR-Baire}, $N$ es denso en $X$.

	Por otro lado, si $y \in \der(X)$ es cualquiera, $N \cup \{y\} \subseteq X$ es localmente compacto, y por ende, es abierto en su cerradura. Pero $N$ es denso, así que $N \cup \{y\}$ es abierto en $X=\cla(N \cup \{y\})$. Por lo tanto, de \ref{lem-TKR-DerX} se obtiene que $\der(X)$ es discreto.

	(ii) $\rightarrow$ (iii) Supóngase que $X$ es localmente compacto y que $\der(X)$ es discreto. Por el \autoref{lem-TKR-DerX} resulta que $N$ es denso en $X$ y que $N \cup \{y\}$ es abierto siempre que $y \in \der(X)$. Por ser $X$ infinito y separable, se tiene que $N$ es numerable. Utilizando la compacidad local de $X$, para cada $x \in \der(X)$ fíjese (utilizando $\Ac$) una vecindad compacta $V_x$ de $x$ en $X$ contenida en $N \cup \{x\}$. Se afirma que $ \ms{A} = \{ V_x \setminus \{x\} \subseteq N \tq x \in \der(X) \} \in \Ad(N) $.

	En efecto, si $x \in \der(X)$ es cualquiera, entonces $V_x \setminus \{x\}$ no es finito. De lo contrario, $\{x\}= (N \cup \{x\}) \setminus (V_x \setminus \{x\})$ sería abierto en $X$ (que es espacio $\T_1$) y se contradiría que $x \in \der(X)$. Por tanto, $\ms{A} \subseteq [N]^\omega$. Además, si $x,y \in \der(X)$ son distintos, se tiene que $V_x \cap V_y \subseteq N$. Así $V_x \cap V_y$ es subespacio compacto del discreto $N$, lo cual obliga a que sea finito. Como consecuencia, $\ms{A}$ es familia casi ajena en $N$.

	Defínase $f:X \to \Psi_N(\ms{A})$ por medio de $f(n)=n$ si $n \in N$ y $f(x)=V_x$ si $x \in \der(X)$. Claramente $f$ es función biyectiva; además, como $N$ es el conjunto de puntos aislados de $X$, para verificar que $f$ es homeomorfismo basta verificar lo siguiente.
	\begin{enumerate}[\hspace{1.5 cm}, listparindent=1.5em]
		\item \textit{Afirmación.} Un subconjunto $U \subseteq X$ es abierto si y sólo si para cada $x \in U \cap \der(X)$ se tiene $V_x \setminus \{x\} \subseteq^* U$.

		\item \textit{Demostración.} Sea $U \subseteq X$. Si $U$ es abierto y $x \in U \cap \der(X)$ es cualquiera, entonces $V_x \setminus U \subseteq N$ es cerrado en $X$, así en $V_x$ y como $V_x$ es compacto; $V_x \setminus U$ es subespacio compacto del discreto $N$, por tanto finito. Así que $V_x \setminus \{x\} \subseteq^* U$.

		      Recíprocamente, supóngase que para cada $x \in U \cap \der(X)$ se tiene que $V_x \setminus \{x\} \subseteq^* U$, es decir, que $V_x \setminus U$ es finito. Sea $y \in U$ cualquiera, si $y \in N$ entonces $\{y\}$ es abierto en $X$ y $U$ es vecindad de $y$. Ahora, si $y \in \der(X)$ entonces $V_y \setminus U$ es finito y con ello $V_y \setminus (V_y \setminus U) \subseteq U$, de donde $U$ es vecindad de $y$ (usando que $X$ es espacio $\T_1$). Luego, $U$ es vecindad de todos sus puntos, y por tanto, es abierto. \hfill$\boxtimes$
	\end{enumerate}

	(iii) $\rightarrow$ (i) Si $X$ es homeomorfo a un espacio de Mrówka, las propiedades topológicas del último se satisfacen en $X$, siguiéndose de \ref{prop-MrwokaHLC} que $X$ es hereditariamente localmente compacto.
\end{proof}

Del resultado anterior es casi inmediata la obtención de las siguientes condiciones equivalentes.

\begin{corolario}\label{cor-HLCPseudoCaract}\index[trad]{compacidad local hereditaria y pseudocompacidad de cualquier espacio infinito, separable, de Hausdorff}
	Sea $X$ cualquier espacio infinito, de Hausdorff y separable. Entonces las siguientes condiciones son equivalentes:
	\begin{enumerate}[i)]
		\item $X$ es pseudocompacto y hereditariamente localmente compacto.
		\item $X$ es regular, $\der(X)$ es subespacio discreto de $X$ y cualquier subespacio discreto, abierto y cerrado a la vez en $X$ es finito.
		\item $X$ es homeomorfo a un espacio de Mrówka generado por una familia casi ajena maximal.
	\end{enumerate}
\end{corolario}
\begin{proof}
	Por el Teorema de Kannan y Rajagopalan, lo demostrado en \ref{prop-tra-pseudoCaract} y como todo espacio de Mrówka es de Tychonoff (véase \ref{cor-MrwokaSiempre}); particularmente regular, bastará demostrar que si $X$ satisface (ii) entonces $X$ es localmente compacto. Supóngase (ii), claramente cada punto aislado de $X$ tiene una vecindad compacta en $X$.

	Sea $x \in \der(X)$ arbitrario, como $\der(X)$ es discreto, existe $U \subseteq X$ abierto con $\{x\} = U \cap \der(X)$. Por regularidad de $X$, fíjese un abierto $V$ tal que $x \in V \subseteq \cla(V) \subseteq U$ y nótese que entonces $\{x\}=\cla(V) \cap \der(X)$.

	Si $W$ es una vecindad abierta de $x$, entonces $\cla(V) \setminus W$ es discreto y abierto (por ser subespacio de $X \setminus \der(X)$) y cerrado (por ser intersección de cerrados). De (ii) se sigue la finitud de $\cla(V) \setminus W$, y de esto, la compacidad de $\cla(V)$, siendo tal subespacio, una vecindad compacta de $x$ en $X$.
\end{proof}

\begin{corolario}
	Sea $X$ un espacio topológico infinito, entonces $X$ es homeomorfo a un espacio de Mrówka si y sólo si es homeomorfo a un subespacio abierto de un espacio de Mrówka.
\end{corolario}
\begin{proof}
	Basta probar la necesidad. Supóngase que $\ms{A}$ es una familia casi ajena y que $U \subseteq \Psi(\ms{A})$ es un abierto tal que $X \cong U$. Como $X$ es infinito, $U$ es infinito, además por ser $\Psi(\ms{A})$ de Hausdorff y hereditariamente localmente compacto, se tiene que $U$ es de Hausdorff y hereditariamente localmente compacto. Por último, como $\omega$ es denso en $\Psi(\ms{A})$ y $U$ es abierto en $\Psi(\ms{A})$, se tiene que $U \cap \omega$ es denso en $U$; así que $U$ es separable. De lo anterior $U$, y por tanto $X$, es homeomorfo a un espacio de Mrówka; a saber $\Psi_{U \cap \omega} (U \cap \ms{A})$.
\end{proof}

\begin{corolario}
	Sea $\{X_\alpha \tq \alpha \in \kappa \}$ una familia no vacía de espacios topológicos infinitos; sin pérdida de generalidad ajenos dos a dos, entonces son equivalentes:
	\begin{enumerate}[i)]
		\item $\displaystyle Y:=\coprod_{\alpha \in \kappa} X_\alpha$ es homeomorfo a un espacio de Mrówka.
		\item $\kappa$ es contable y cada $X_\alpha$ es homeomorfo a un espacio de Mrówka.
	\end{enumerate}
\end{corolario}
\begin{proof}
	(i) $\to$ (ii) Supóngase que $Y$ es espacio de Mrówka. Como cada $X_\alpha \subseteq Y$ es infinito y abierto en $Y$, se sigue del Corolario anterior que $X_\alpha$ es de Mrówka. Por otro lado, si $\kappa$ fuese más que numerable, $Y$ no podía ser separable, pues es la suma de $\kappa$ espacios no vacíos; así que $\kappa$ es a lo más numerable.

	(ii) $\to$ (i) Supóngase que $\kappa$ es a lo más numerable y para cada $\alpha \in \kappa$, el espacio $X_\alpha$ es homeomorfo a un espacio de Mrówka. Entonces, del \autoref{sec-KRTeo}, cada $X_\alpha$ es (infinito) de Hausdorff, separable, localmente compacto y además el subespacio $\der_{X_\alpha}(X_\alpha)\subseteq X_\alpha$ es discreto.

	La suma de espacios de Hausdorff (localmente compactos, respectivamente) es de Hausdorff (localmente compacta, respectivamente); además, por ser cada $X_\alpha$ separable y $\kappa$ a lo más numerable, resulta que $Y$ es infinito, de Hausdorff, localmente compacto y separable.

	Sea $y \in \der_Y(Y)$ cualquiera, por definición de $Y$, para el único elemento $\alpha \in \kappa$ tal que $y \in X_\alpha$, se tiene $y \in \der_{X_\alpha}(X_\alpha)$. Y como tal subespacio de $X_\alpha$ es discreto, existe $V \subseteq X_\alpha$ abierto tal que $\{y\}=U \cap \der_{X_\alpha}(X_\alpha)$, pero $U$ es abierto también en $Y$ y además $\{y\}=U \cap \der_Y(Y)$. De lo contrario, existe $x \in V \cap \der_Y(Y) \setminus \{y\}$ y consecuentemente $x \notin \der_{X_\alpha}(X_\alpha)$, mostrando que $\{x\}$ es abierto en $X_\alpha$ y por tanto en $Y$, lo cual es absurdo dada la elección de $X$. Lo anterior prueba que $\der_Y(Y)$ es discreto, finalizando la prueba en virtud del \autoref{teo-HLCCaract}.
\end{proof}

Se explotará mucho la siguiente observación durante el subsecuente Corolario, pues nuevamente, se hará uso del inciso (ii) del \autoref{teo-HLCCaract}.
\begin{observacion}
	Sea $X$ un espacio topológico, $\der(X)$ es discreto si y sólo si $\der^2(X):=\der(\der(X)) = \emptyset$.

	Efectivamente; como $X\setminus \der(X)$ es abierto, $\der(X)$ es discreto si y sólo si es discreto y cerrado. Esto último sucede únicamente cuando $\der_{\der(X)}(\der(X))=\der(X) \cap \der^2(X) =\emptyset$. Sin embargo, cualquier punto aislado en $X$, es aislado en $\der(X)$, así que $\der^2(X) \subseteq \der(X)$; por lo tanto, $\der(X)$ es discreto si y sólo si $\der^2(X)=\emptyset$.
\end{observacion}

\begin{lema}
	Sean $X$ y $Y$ espacios topológicos infinitos, entonces $X \times Y$ es homeomorfo a un espacio de Mrówka si y sólo si $X$ y $Y$ son de Mrówka y además $X \cong \omega$ o $Y \cong \omega$
\end{lema}

\begin{proof}
	Obsérvese la igualdad:
	\begin{align*}
		\der^2_{X \times Y} (X \times Y) & = \der_{X \times Y} \Big( \der_X(X) \times \cla_Y(Y) \cup \cla_X(X) \times \der_Y(Y) \Big)               \\
		                                 & = \der_{X \times Y} \Big( \der_X(X) \times Y \cup X \times \der_Y(Y) \Big)                               \\
		                                 & = \der_{X \times Y} \Big( \der_X(X) \times Y \Big) \cup \der_{X \times Y} \Big( X \times \der_Y(Y) \Big) \\
		                                 & = \der_X(\der_X(X)) \times \cla_Y(Y) \cup \cla_X(\der_X(X)) \times \der_Y(Y) \: \cup                     \\
		                                 & \cup \der_X(X) \times \cla_Y(\der_Y(Y)) \cup \cla_X(X) \times \der_Y(\der_Y(Y))                          \\
		                                 & = \der^2_X(X) \times Y \cup \der_X(X) \times \der_Y(Y) \cup X \times \der^2_Y(Y)
	\end{align*}

	Puesto que $X,Y \neq \emptyset$, resulta que $\der^2_{X \times Y} (X \times Y)$ es vacío si y sólo si $\der^2_X(X) = \der^2_Y(Y) = \der_X(X) \times \der_Y(Y) = \emptyset$. Esto es, el subespacio $\der_{X \times Y}(X \times Y) \subseteq X \times Y$ es discreto si y sólo si los subespacios $\der_X(X)$ de $X$ y $\der_Y(Y)$ de $Y$ son discretos y además $X$ es discreto o $Y$ es discreto.

	Como $X,Y$ son infinitos, $X \times Y$ es infinito, además las propiedades de separabilidad, axioma de separación de Hausdorff y local compacidad son propiedades finitamente productivas y finitamente factorizables. De esto último, lo comentado en el párrafo anterior, el hecho de que el único espacio de Mrówka discreto es $\omega$ y el inciso (ii) del \autoref{teo-HLCCaract}, se obtiene el resultado.
\end{proof}

\begin{corolario}
	Sea $\{X_\alpha \tq \alpha \in \kappa \}$ una familia no vacía de espacios topológicos infinitos; sin pérdida de generalidad ajenos dos a dos, entonces son equivalentes:
	\begin{enumerate}[i)]
		\item $\displaystyle Y:=\prod_{\alpha \in \kappa} X_\alpha$ es homeomorfo a un espacio de Mrówka.
		\item $\kappa$ es finito, cada $X_\alpha$ es homeomorfo a un espacio de Mrówka y existe $\beta_0 \in \kappa$ tal que si $\alpha \in \kappa \setminus \{\beta_0\}$, se tiene $X_\alpha \cong \omega$.
	\end{enumerate}
\end{corolario}

\begin{proof}
	Sin perder generalidad, tómese $\kappa$ como un cardinal.

	(i) $\to$ (ii) Supóngase que $Y$ es homeomorfo a un espacio de Mrówka, entonces $Y$ es de Hausdorff, Separable y hereditariamente localmente compacto. Todas las propiedades anteriores son factorizables, así que por el por el Teorema de Kannan y Rajagopalan (\autoref{teo-HLCCaract}), cada $X_\alpha$ es homeomorfo a un espacio de Mrówka.

	Ahora, por contradicción, supóngase $\kappa \geq \omega$. Entonces, existen $P,Q \subseteq \kappa$ ajenos e infinitos, de donde:
	$$ Y = \prod_{\alpha \in \kappa} X_\alpha \cong \prod_{\alpha \in P} X_\alpha \times \prod_{\alpha \in Q} X_\alpha $$
	siguiéndose del Lema previo que; sin pérdida de generalidad, $\prod_{\alpha \in P} X_\alpha \cong \omega$. Lo anterior conduce a un absurdo, pues como $P$ es infinito y cada $X_\alpha$ también, resulta que:
	$$ \Bigg| \prod_{\alpha \in P} X_\alpha \Bigg| = \prod_{\alpha \in P} |X_\alpha| \geq \prod_{\alpha \in P} \aleph_0 = \aleph_0^{|P|} \geq \aleph_0^{\aleph_0} > \aleph_0 $$
	imposibilitando que $\prod_{\alpha \in P} X_\alpha \cong \omega$ sea biyectable con $\omega$. Así, $\kappa < \omega$.

	Finalmente, si cada $X_\alpha$ es homeomorfo a $\omega$, o $\kappa=1$, (ii) se satisface. Supóngase pues que $\kappa \geq 2$ y que existe $\beta_0 \in \kappa$ con $X_{\beta_0} \not\cong \omega$. Dado que:
	$$ Y = \prod_{\alpha \in \kappa} X_\alpha \cong X_{\beta_0} \times \prod_{\alpha \in \kappa \setminus \{\beta_0\}} X_\alpha $$
	se sigue del Lema Previo que $\prod_{\alpha \in \kappa \setminus \{\beta_0\}} X_\alpha \cong \omega$. Siendo así, cada $X_\alpha$ (con $\alpha \in \kappa \setminus \{\beta_0\}$) infinito, numerable y discreto; esto es, homeomorfo a $\omega$.

	(ii) $\to$ (i) Supóngase que $\kappa$ es finito, que cada $X_\alpha$ es homeomorfo a un espacio de Mrówka y que $\beta_0 \in \kappa$ es un elemento tal que si $\alpha \in \kappa \setminus \{\beta_0\}$, entonces $X_\alpha \cong \omega$. Como $\kappa \setminus \{\beta\}$ es finito, entonces:
	$$ Y = \prod_{\alpha \in \kappa} X_\alpha \cong X_\beta \times \prod_{\alpha \in \kappa \setminus \{\beta\}} X_\alpha \cong X_\beta \times \prod_{\alpha \in \kappa \setminus \{\beta\}} \omega = X_\beta \times \omega $$
	y a consecuencia del Lema previo, $Y$ es de Mrówka.
\end{proof}

El siguiente Corolario del Teorema de Kannan y Rajagopalan (\autoref{teo-HLCCaract}), es un resultado sencillo (y sumamente particular) de metrización.

\begin{corolario}\label{cor-omegaCuadra}\index[trad]{separabilidad hereditaria de cualquier espacio infinito, separable, de Hausdorff, hereditariamente localmente compacto}
	Si $X$ es infinito, separable, de Hausdorff y hereditariamente localmente compacto. Entonces son equivalentes:
	\begin{enumerate}[i)]
		\item $X$ es hereditariamente separable.
		\item $X$ es metrizable.
	\end{enumerate}
\end{corolario}

\begin{proof}
	Dado el \autoref{teo-HLCCaract} y la caracterización \ref{prop-tra-numerable}, basta ver que si $\ms{A}\in \Ad(\omega)$, entonces $\Psi(\ms{A})$ es hereditariamente separable si y sólo si $\ms{A}$ es a lo más numerable.

	Para la suficiencia procédase por contrapuesta suponiendo que $\ms{A}$ es más que numerable, entonces $\ms{A}$ es un subespacio de $\Psi(\ms{A})$ discreto y más que numerable, con lo que, no puede ser separable. Para la necesidad, si $\ms{A}$ es a lo más numerable, cada subespacio de $\Psi(\ms{A})$ es a lo más numerable, y con ello, separable.
\end{proof}

La \textit{curiosidad} (comentada posteriormente a \ref{prop-alomasNumCaract}) en relación al espacio de ordinales $\omega^2$ tiene su justificación en el anterior Corolario.

Se finalizará la sección; y con ello el actual capítulo, dando un Corolario importante en relación a las imágenes continuas de los espacios de Mrówka pseudocompactos.

\begin{corolario}
	Sea $X$ infinito y de Hausdorff. Son equivalentes:
	\begin{enumerate}[i)]
		\item Existe un denso $D \subseteq X$ de $X$ numerable tal que cada sucesión en $D$ tiene una subsucesión convergente en $X$.
		\item $X$ es imagen continua de un espacio de Mrówka generado por una familia maximal.
	\end{enumerate}
\end{corolario}

\begin{proof}
	(i) $\rightarrow$ (ii) Supóngase (ii) y sea $S \subseteq \Ad(D)$ el conjunto de familias casi ajenas en $D$ tales que para cada $\ms{B} \in S$, cada elemento de $\ms{B}$ es imagen de una sucesión en $D$ convergente en $X$. Como $D$ es numerable, existe una biyección $f_0:\omega \to D$ biyectiva, misma que admite una subsucesión convergente, a saber $g_0:\omega \to D$ convergente en $X$. Se desprende que $\{\ima(g_0)\} \in S$ y por tanto $S$ es no vacío, siguiéndose de una aplicación del Principio de Maximalidad de Hausdorff (similar al utilizado en \ref{lem-MADs}) la existencia de una familia casi ajena en $D$, $\ms{A} \subseteq \midcup S$ tal que si $\ms{B} \in S$ y $\ms{A} \subseteq \ms{B}$, entonces $\ms{A} = \ms{B}$.
	\begin{enumerate}[\hspace{1.5 cm}, listparindent=1.5em]
		\item \textit{Afirmación.} $\ms{A}$ es familia casi ajena maximal sobre $D$.

		\item \textit{Demostración.} Obsérvese primero que si $A\in \ms{A}$, existe $\ms{C} \in S$ tal que $A \in \ms{C}$ tal que $A \in \ms{C}$; consecuentemente $A$ es imagen de una sucesión en $D$ convergente en $X$; es decir $A \in \ms{A}$. Ahora, si $B \subseteq D$ infinito, entonces existe una biyección $f:\omega \to B$ y, por hipótesis, existe $g:\omega \to B \subseteq D$ subsucesión de $f$, convergente en $X$ y con ello $\{\ima(g)\} \in S$.

		      Por un lado, si $\ms{A} \cup \{\ima(g)\}$ no es casi ajena, existe $A \in \ms{A}$ de modo que $A \cap \ima(g)$ es infinito, y con ello $A \cap B$ es infinito. De otro modo, $\ms{A} \cup \{\ima(g)\} \in S$ y por la construcción de $A$ se tiene $\ms{A} \cup \{\ima(g)\} = \ms{A}$, siendo $A:=\ima(g) \in \ms{A}$ tal que $A \cap B$ es infinito (pues $g$ es subsucesión de $f$). Lo anterior prueba que $\ms{A}$ es maximal sobre $D$. \hfill $\boxtimes$
	\end{enumerate}

	Para cada $A \in \ms{A}$ fíjese ($\Ac$) una sucesión $f_A:\omega \to D$ convergente a $x_A$ en $X$ tal que $A=\ima(f_A)$. Nótese que, como $X$ es de Hausdorff tal elemento $x_A$ es el único al cual $f_A$ converge. Además, dado que los elementos de $\ms{A}$ son casi ajenos dos a dos, y de nuevo por ser $X$ de Hausdorff, cada vez que $A,B \in \ms{A}$ sean distintos, se tendrá que $f_A \neq f_B$ y $x_A \neq x_B$. Defínase la función $p:\Psi_D(\ms{A}) \to X$ como $p(d)=d$ si $d \in D$ y $p(A)=x_A$ si $A \in \ms{A}$, veamos que $p$ es continua y sobreyectiva.

	Sea $U \subseteq X$ abierto en $X$ y supóngase que $A \in p^{-1}[U] \cap \ms{A}$ es cualquiera, entonces $p(A)=x_A \in U$ y $A=\ima(f_A)$. Como $U$ es un abierto de $X$ y $f_A$ converge a $x_A$ en $X$, resulta que $\ima(f_A) \subseteq^* U$ y con ello $A \subseteq^* p^{-1}[U]$; así que $p^{-1}[U]$ es abierto en $\Psi_D(\ms{A})$, y por tanto $p$ es continua.

	Ahora, sea $x \in X \setminus D$ cualquier elemento. Por contradicción, supóngase que $x \notin \ima(p)$, entonces si $s:\omega \to D$ es cualquiera, $s$ no puede converger a $a$ en $X$; de lo contrario, existe $A \in \ms{A}$ tal que $A \cap \ima(s)$ es infinito y con ello $f_A$ converge a $x$ en $X$, con lo que $x=p(A)$. Sin embargo
	$$ \text{AQUÍ ESTO YA NO SALE} $$

	(ii) $\to$ (i) SALE FÁCIL
\end{proof}
% Supóngase que $\ms{A}$ es una familia maximal tal que $X$ es imagen continua de $\Psi(\ms{A})$ por medio de $f:\Psi(\ms{A}) \to X$. Como $f$ es continua y sobreyectiva, $D:=f[\omega]$ es denso en $X$. Supóngase que $(a_n)_{n \in \omega} \subseteq D$ es cualquier sucesión. Para cada $n \in \omega$ fíjese $b_n \in \omega$ de modo tal que $f(b_n)=a_n$ (esto no requiere de elección debido al buen orden de $\omega$). Así, $(b_n)_{n \in \omega}$ es una sucesión en $\omega$ y como $\ms{A}$ es maximal, se sigue de la Proposición \ref{prop-tra-pseudoCaract}, que $(b_n)_{n \in \omega}$ contiene una subsucesión convergente en $\Psi(\ms{A})$. Así que por continuidad de $f$, la sucesión $(a_n)_{n \in \omega}$ admite también una subsucesión convergente en $X$.

\begin{corolario}
	Todo espacio metrizable, separable y compacto es imagen continua de un espacio de Mrówka; en particular, el cubo de Hilbert $[0,1]^\omega$ y el conjunto de Cantor $2^\omega$.
\end{corolario}
        \chapter{El compacto de Franklin}
\emph{\small Los espacios conocidos como \textit{compactos de Franklin} son la extensión unipuntual de los espacios de Mrówka. En el presente capítulo se estudiarán sus cualidades, comenzando por dar una caracterización de los mismos en propiedades topológiacas. Se demostrará que los compactos de Franklin asociados a un espacio de Mrówka pseudocompacto es un espacio secuencial de orden $2$.}

\emph{\small Se probará que el carácter del punto al infinito en el compacto de Franklin asociado a $\Psi(\ms{A})$ es $|\ms{A}|$, y se demostrará que este espacio es de Fréchet si y sólo si su familia asociada es maximal en ninguna parte. Esto a su paso dará la negativa, dentro de $\zfc$, a una cuestión relevante que estuvo sin solución durante el siglo XX: ¿es la propiedad de Fréchet finitamente productiva?.}

\section{\texorpdfstring{Sucesiones en $\ms{F}(\ms{A})$}{Sucesiones en F(A)}}
\phantomsection\phantomsection\label{Subsec-sucesiones-Franklin}

\begin{definicion}\index[alph]{compacto!de Franklin}\index[alph]{Franklin! compacto de}\index[sym]{$\ms{F}(\ms{A})$}
	Sea $\ms{A}\subseteq [\omega]^\omega$ cualquiera. El \textbf{compacto de Franklin generado por $\ms{A}$} es la extensión unipuntual del $\Psi$-espacio generado por $\ms{A}$, se denota por $ \ms{F}(\ms{A}):=\Psi(\ms{A}) \cup \{\infty_\ms{A} \} $.

	Cuando el contexto así lo permita, se omitirá el subíndice ``$_\ms{A}$'' y se denotará el punto al infinito simplemente por $\infty$.
\end{definicion}

\index[alph]{familia!no compacta}
Dado lo demostrado en relación a la compactación de Alexandroff en \ref{admiAlex}, el compacto de Franklin resulta ser la compactación de Alexandroff de $\Psi(\ms{A})$ únicamente cuando $\ms{A}$ sea una familia casi ajena (lo que garantiza que $\Psi(\ms{A})$ sea de Hausdorff) \textit{no compacta}, esto es, que no sea simultáneamente finita y maximal (lo cual obliga a que $\Psi(\ms{A})$ sea no compacto); a razón de la \cref{prop-tra-compacidad}.

\begin{consideracion}
	Durante esta sección:
	\begin{enumerate}
		\item Cualquier familia casi ajena que se considere, será no compacta.
		\item Para cada subespacio compacto $K \subseteq \Psi(\ms{A})$, se denotará por $V(K)$ a la vecindad abierta de $\infty$: $\{\infty\} \cup \Psi(\ms{A}) \setminus K$ (como $\Psi(\ms{A})$ es Hausdorff, todos los abiertos al rededor de $\infty$ son de esta forma).
		\item Se utilizarán casi en exceso los resultados obtenidos en \ref{prop-Kcaract} y \ref{cor-IdealCompactosCarac}, así que no se referenciarán de ahora en más.
		\item Todas las convergencias y operadores que aparezcan sin subíndices, se asumirán en $\ms{F}(\ms{A})$.
	\end{enumerate}
\end{consideracion}

Lo primero a observar es lo siguiente: dado que $\Psi(\ms{A})$ no es compacto, al ser un espacio de Tychonoff y localmente compacto, se tiene efectivamente que $\ms{F}(\ms{A})$ es de Hausdorff, compacto (véase \textcolor{red}{(prelim)}; y en consecuencia, normal. Como es previsible, ciertas propiedades de $\Psi(\ms{A})$ \enquote{suben} a la topología de $\ms{F}(\ms{A})$; como ejemplo inmediato, la separabilidad se preserva.

\begin{observacion}
	Sea $\ms{A}$ una familia casi ajena. Entonces $\ms{F}(\ms{A})$ es de Hausdorff, compacto, normal, localmente compacto, separable y disperso.
\end{observacion}

Comparando con el \cref{cor-MrwokaSiempre} con las observaciones recién hechas, vale mencionar que existen propiedades $\ms{F}(\ms{A})$ que tienen una dependencia más compleja con $\Psi(\ms{A})$. Iniciarmos mostrando estas disparidades con lo próximo:

\begin{proposicion}\phantomsection\phantomsection\label{prop-caracterFrechet}
	Para toda familia $\ms{A}$ se tiene que $\chi(\infty) = \aleph_0 + |\ms{A}|$.
\end{proposicion}

\begin{proof} Es evidente que $\aleph_0 \leq \chi(\infty)$. Ahora, sea $\mathcal{B}$ una base local de $\infty$ en $\ms{F}(\ms{A})$. Para cada $y \in \ms{A}$ fíjese ($\Ac$) un $B_y \in \mathcal{B}$ con $B_y \subseteq V(y \cup \{y\})$. Obsérvese que la asignación $y \mapsto B_y$ es inyectiva; pues, si $x,y \in \ms{A}$ son distintos, entonces $B_x \subseteq U(x \cup \{x\})$ y $B_y \subseteq U(y \cup \{y\})$, de donde $x \in B_y \setminus B_x$. Por lo tanto $ |\ms{A}| \leq |\mathcal{B}|$, y en consecuencia $\aleph_0 + |\ms{A}|\leq \chi(\infty)$.

Para la desigualdad recíproca defínase:
$$ \mathcal{B} = \big\{ V( B \cup \midcup h \cup h ) \tq (B,h) \in [\omega]^{<\omega} \times [\ms{A}]^{<\omega} \big\} $$
y nótese que $\mathcal{B}$ es un conjunto de vecindades de $\infty$ en $\ms{F}(\ms{A})$.

Ahora, si $K \subseteq \Psi(\ms{A})$ es compacto, $h:=K \cap \ms{A} \subseteq \ms{A}$ y $G:=(K \cap \omega) \setminus \midcup h \subseteq \omega$ son finitos, además $V\big( G \cup \midcup h \cup h \big) \subseteq V(K)$. Lo cual demuestra que $\mathcal{B}$ es base local para $\infty$ en $\ms{F}(\ms{A})$. Dado que $ |\mathcal{B}| \leq | [\omega]^{<\omega} \times [\ms{A}]^{<\omega} | = \aleph_0 + |\ms{A}| $, resulta que $\chi(\infty) \leq \aleph_0 + |\ms{A}|$.
\end{proof}

En seguida a lo anterior se concluye que, para toda familia no compacta e infinita, el espacio $\ms{F}(\ms{A})$ no es de primero numerable. 

El siguiente Corolario se puede enriquecer con \ref{prop-tra-numerable}.

\begin{corolario}\index[trad]{primero numerabilidad de $\ms{F}(\ms{A})$}
	Para toda familia no compacta $\ms{A}$, el espacio $\ms{F}(\ms{A})$ es primero numerable si y sólo si $|\ms{A}| \leq \aleph_0$.
\end{corolario}

El próximo Lema es clave por varios motivos; entre ellos, responde a una pregunta que sugiere la discusión previa al \cref{prop-CaracMADPositiv} ¿qué distingue a los subconjuntos de $\omega$ casi ajenos con cada elemento de $\ms{A}$, con los elementos de $\ms{I}^+(\ms{A})$?, encontramos la solución a esta interrogante por medio de la topología.

\begin{lema}\phantomsection\phantomsection\label{lem-convClave}
	Sean $\ms{A}\in \Ad(\omega)$ y $B \subseteq \Psi(\ms{A})$ infinito, entonces:
	\begin{enumerate}[i)]
		\item $\infty \in \scl(B)$ si y sólo si $|B \cap \ms{A}| \geq \omega$ o $|B \cap \omega| \geq \omega$ y $\ms{A} \upharpoonright (B \cap \omega
		) = \emptyset$.
		\item $\infty \in \cla(B)$ si y sólo si $|B \cap \ms{A}| \geq \omega$ o $B \cap \omega \in \ms{I}^+(\ms{A})$.
	\end{enumerate}
\end{lema}

\begin{proof}
	(i) Para la suficiencia supóngase por absurdo que $\infty \in \scl(B)$, esto es (véase \ref{sqcl-en-T1}), existe $C \in [B]^\omega$ tal que $C \to \infty$; que $B \cap \ms{A}$ es finito; y que $a \in \ms{A}$ es un elemento tal que $a \cap (B \cap \omega)$ es infinito. Obsérvese que, necesariamente $a \cap C$ es infinito. Como $C \to \infty$ y $a \cap C \subseteq C$ es infinito, ocurre que $a \cap C \to \infty$. Sin embargo, $a \cap C \subseteq a$ y a razón del \cref{lem-convObvia}, $a \cap C \to a$ en $\Psi(\ms{A})$; así mismo en $\ms{F}(\ms{A})$. Lo anterior es imposible, ya que $a \neq \infty$ y $\ms{F}(\ms{A})$ es de Hausdorff.

	Recíprocamente, si $B \cap \ms{A}$ es infinito, existe $C_0 \in [B \cap \ms{A}]^\omega$; y, si $K \subseteq \Psi(\ms{A})$ es cualquier compacto, entonces $C_0 \setminus V(K) = C_0 \cap K \subseteq K \cap \ms{A}$ es finito, mostrando que $C_0 \to \infty$ y que $\infty \in \scl(B)$.

	Ahora, supóngase que $B \cap \omega$ es infinito y casi ajeno con cada elemento de $\ms{A}$. Si $\infty \notin \scl(B)$, entonces $B \cap \omega \not\to \infty$ y existiría un compacto $K_0 \subseteq \Psi(\ms{A})$ tal que $S:=(B \cap \omega) \cap K_0$ es infinito. $S \subseteq K$ es un elemento en $\ms{I}(\ms{A})$; a consecuencia de esto y de \ref{cor-CasiajenoPartePositiva}, existe cierto $a \in A$ con $S \cap a$ infinito; de donde, $B \cap a$ es infinito, contradiciendo la suposición sobre $B$. Por tanto, $\infty \in \scl(B)$

	(ii) Para la suficiencia, supóngase que $\infty \in \cla(B)$ y que $B \cap \ms{A}$ es finito. Resulta necesario que $\infty \in \cla(B \cap \omega)$. Si $K \subseteq \Psi(\ms{A})$ es compacto, $(B \cap \omega) \cap V(K) \neq \emptyset$, prohibiendo que $B \cap \omega \subseteq K$, por tanto, $B \cap \omega \notin \ms{I}(\ms{A})$.

	Para la necesidad, sí $B \cap \ms{A}$ es infinito, existe $C \subseteq B \cap \ms{A}$ numerable y por el inciso anterior $C \to \infty$, de donde $\infty \in \cla(C) \subseteq \cla(B)$. Y, si $B \cap \omega \in \ms{I}^+(\ms{A})$ y $K \subseteq \Psi(\ms{A})$ es compacto, resulta que $B \cap \omega \not\subseteq K$ y con ello $B\cap V(K) \neq \emptyset$, mostrando que $\infty \in \cla(B)$.
\end{proof}

Si $\ms{A}$ es una familia maximal, la condición (i) del resultado anterior implica que ninguna sucesión contenida en $\omega$ es convergente a $\infty$. A continuación se establecen las únicas convergencias posibles en el compacto de Franklin.
\begin{corolario}\phantomsection\phantomsection\label{cor-convMaximal}
	Sean $\ms{A} \in \Mad(\omega)$ infinita y $X \subseteq \ms{F}(\ms{A})$ numerable. Entonces $X$ es convergente si y sólo si $X \subseteq^* \ms{A}$, o para algún $a \in \ms{A}$, $X \subseteq^* a$, en cuyo caso: $X \to \infty$, o $X \to a$, respectivamente.
\end{corolario}

\begin{proof}
	El recíproco es inmediato a razón de \ref{lem-convObvia} y el Lema previo. Para la suficiencia asúmase que $X \to x$ en el compacto de Franklin. Si $x = \infty$, se sigue del Lema anterior y la maximalidad de $\ms{A}$ que $X \subseteq^* \ms{A}$. En otro caso, se puede suponer sin pérdida de generalidad que $X \subseteq \Psi(\ms{A})$, siguiéndose de \ref{lem-convObvia} que $X$ debe estar casi contenido en algún elemento de $\ms{A}$.
\end{proof}

El siguiente comportamiento es propio de las familias casi ajenas maximales, muestra que el compacto de Franklin asociado a una familia de este tipo tiene orden secuencial $2$. Además se puede verificar fácilmente la diferencia entre las distintas clausuras secuenciales de, por ejemplo, el denso $\omega \subseteq \ms{F}(\ms{A})$, donde $\scl^1(\omega)=\Psi(\ms{A})$ y $\scl^2(\omega) \setminus \scl^1(\omega)=\{\infty\}$. Se configura también de esta manera, el ejemplo de un espacio secuencial, compacto, de Hausdorff y separable, que no es de Fréchet (recuérdese que un espacio es de Fréchet si y sólo si su orden secuencial es $1$).

\begin{corolario}
	Sea $\ms{A} \in \Mad(\omega)$ infinita, entonces $\Osq(\ms{F}(\ms{A}))=2$.
\end{corolario}
\begin{proof}
	Nótese que $\scl(\omega) \not\subseteq \cla(\omega)$. Efectivamente; como $\omega$ es denso en $\Psi(\ms{A})$, lo es también en $\ms{F}(\ms{A})$; por lo que $\infty \in \cla(\omega)$. De darse $\infty \in \scl(\omega)$, por el Corolario previo, debería existir $B \subseteq \omega \cap \ms{A}$ numerable, lo cual es imposible. Por lo tanto, $\Osq(\ms{F}(\ms{A})) \geq 2$.

	Ahora, sea $X \subseteq \ms{F}(\ms{A})$, sin pérdida de generalidad infinito, y sea $x \in \cla(X)$. Dado que $\omega$ es discreto, de ocurrir $x \in \omega$, sería necesario que $x \in X \subseteq \scl^2(X)$. Por otro lado, si $x \in \ms{A}$, como $\{x\} \cup x$ es una vecindad abierta de $x$, por ser $\ms{F}(\ms{A})$ un espacio $\T_1$, debe existir $A \subseteq x \cap X$ numerable, concluyéndose a partir del \cref{lem-convObvia} que $A \to x$; y por ende, $x \in \scl(X) \subseteq \scl^2(X)$.

	Finalmente, si $x=\infty$, entonces por \ref{lem-convClave}, $B:=X \cap \omega \in \ms{I}^+(\ms{A})$; y como $\ms{A}$ es maximal, por el \cref{cor-MADPositivCarac}, existe $\ms{B} \subseteq \ms{A}$ infinito, sin pérdida de generalidad numerable, tal que para cada $b \in \ms{B}$, el conjunto $B \cap b$ es infinito. Por ello, y \ref{lem-convObvia}, $\ms{B} \subseteq \scl(X)$, desprendiéndose de \ref{lem-convObvia}, que $x \in \scl^2(X)$. En todo caso, se ha mostrado que $\cla(X) \subseteq \scl^2(X)$ y por ende, $\Osq(\ms{F}(\ms{A})) \leq 2$.
	%
	%Ahora, sea $X \subseteq \ms{F}(\ms{A})$ cualquiera y supóngase que $x \in \scl^3(X)$. Como $\Psi(\ms{A})$ tiene orden secuencial $1$ (pues es $1\AN$, consecuentemente de Fréchet), es requisito que $x=\infty$. Así, existe $A \subseteq \scl^2(X)$ numerable tal que $A \to \infty$ en $\ms{F}(\ms{A})$. Sin pérdida de generalidad, $A \subseteq \ms{A}$ (dado \ref{lem-convClave}). Para cada $a \in A$ fíjese un conjunto numerable $B_a \in [\scl(X)]^\omega$ de manera que $B_a \to a$.
	%
	%Ahora, si $\scl(X) \cap \ms{A}$ es infinito, entonces por \ref{lem-convClave}, $x=\infty \in \scl(\scl(X))$. Supóngase pues que $\scl(X) \subseteq^* \omega$. Como cada $B_a \subseteq \scl(X)$ es convergente, asúmase sin pérdida de generalidad que $B_a \subseteq \omega$ (de nuevo, debido a \ref{lem-convClave}). En consecuencia, $B_a \subseteq \omega \cap \scl(X)$ y se obtiene del inciso (iii) del Corolario previo que $B_a \subseteq X$. Así $A \subseteq \scl(X)$, ya que cada $B_a$ satisface $B_a \to a \in A$. Esto muestra que $x \in \scl(A) \subseteq \scl^2(X)$. Es decir, $\scl^3(X) \subseteq \scl^2(X)$ y $\Osq(\ms{F}(\ms{A})) \leq 2$.
\end{proof}

\begin{corolario}
	Para cada $\ms{A} \in \Ad(\omega)$ se tiene que $\Osq(\ms{F}(\ms{A})) \leq 2$.
\end{corolario}

%El posterior Teorema sigue la línea del Teorema de Kannan y Rajagopalan (\ref{teo-HLCCaract}), es una caracterización en propiedades topológicas de ciertos compactos de Franklin en la clase de espacios infinitos, de Hausdorff, separables.

\begin{teorema}\index[trad]{homeomorfismo con $\ms{F}(\ms{A})$ ($\ms{A}$ maximal)}
	Sea $X$ un espacio topológico infinito, de Hausdorff y separable. $X$ es homeomorfo a un compacto de Franklin generado por una familia maximal infinita si y sólo si existe $x_0 \in X$ tal que: $x_0$ es el único punto de acumulación de $\der(X)$ y $x_0 \notin \scl(X \setminus \der(X))$.
\end{teorema}

\begin{proof}
	Para la suficiencia basta suponer que $X=\ms{F}(\ms{A})$. Como $\Psi(\ms{A})$ es denso en $X$, resulta que $\der(X) = \{\infty\} \cup \der_{\Psi(\ms{A})}(\Psi(\ms{A})) = \{\infty\} \cup \ms{A}$. Consecuentemente $\der_{\der(X)}(\der(X)) \subseteq \{\infty\}$, pues $\ms{A}$ es un subespacio discreto de $\der(X)$. La contención recíproca ocurre; pues cada subespacio compacto de $\Psi(\ms{A})$ tiene intersección finita; particularmente no vacía, con el conjunto (infinito) $\ms{A}$. Así que $\infty$ es el único punto de acumulación de $\der(X)$. Además, $X \setminus \der(X) = \omega$, y dada la prueba del resultado anterior, $\infty \notin \scl(\omega)$.

	Para la necesitadad, sea $x_0 \in X$ el punto que satisface las condiciones del enunciado. Defínase $Y:=X\setminus \{x_0\}$, se mostrará primero que $Y \cong \Psi(\ms{A})$ para alguna familia maximal $\ms{A}$. Efectivamente, nótese que $Y$ es infinito, de Hausdorff y separable (ya que $Y$ es abierto en $X$). Haciendo uso del \cref{cor-HLCPseudoCaract}, es suficiente mostrar lo subsecuente:

	\begin{enumerate}
		\item ($Y$ es regular) Dado que $X$ es compacto, de Hausdorff es normal y particularmente, regular. Esto prueba que $Y \subseteq X$ es regular.

		\item ($\der_Y(Y)$ es discreto) Efectivamente, si $y \in \der_Y(Y)$ es cualquiera, entonces $y \in Y$ es punto de acumulación de $X$. Como $y \neq x_0$ y $x_0$ es el único punto de acumulación de $\der_X(X)$, $\{y\}$ es abierto en $\der_X(X)$; y por tanto, $\{y\}$ es abierto en $\der_Y(Y)$. Mostrando que $\der_Y(Y)$ es discreto.

		\item (Si $B \subseteq Y$ es discreto, abierto y cerrado a la vez, entonces $B$ es finito) Supóngase que $B \subseteq Y$ es discreto, abierto y cerrado a la vez. Por ser $B$ discreto y abierto, se da $B \subseteq X \setminus \der(X)$. Ahora, si $B$ es infinito (sin pérdida de generalidad, numerable) se tiene de la hipótesis que $B \not\to x_0$; así, existe una vecindad de $x_0$; a saber U, de modo que $B \setminus U$ es infinito. Sin embargo, $B \cap U$ es cerrado en vista de que $B$ es cerrado; por ello, tal conjunto es cerrado, discreto e infinito en $X$; lo que contradice que $X$ sea compacto y $\T_1$. Por ello, es necesario que $B$ sea finito. Concluyéndose de \ref{cor-HLCPseudoCaract}, la existencia de una familia $\ms{A}\in\Mad(\omega)$ de modo que $Y \cong \Psi(\ms{A})$.
	\end{enumerate}

	Para finalizar, obsérvese que $\{x_0\}$ no es abierto en $X$, pues de lo contrario no podría ser punto de acumulación de ninguno de sus subespacios. Así, $Y$ es denso en $X$ y como $X$ es de Hausdorff, compacto, con $X \setminus Y = \{x_0\}$, resulta que $X$ es la compactación de Alexandroff de $Y \cong \Psi(\ms{A})$; esto es, $X \cong \ms{F}(\ms{A})$.
\end{proof}

\section{La propiedad de Fréchet}

Continuando con los frutos del \cref{lem-convClave}, se extrae el siguiente Corolario; este relaciona las propiedades de combinatoria de las familias casi ajenas con propiedades de convergencia.

\begin{lema}\phantomsection\phantomsection\label{lem-TrazaMad}
	Sean $\ms{A} \in \Ad(\omega)$ y $X \subseteq \omega$. Entonces $\infty \in \scl(X)$ si y sólo si $\ms{A} \upharpoonright X \notin \Mad(X)$.
\end{lema}

\begin{proof}
	Si $\infty \in \scl(X)$, entonces po el \cref{lem-convClave}, $\ms{A} \upharpoonright X = \emptyset$, suiguiéndose en automático de \ref{cor-CasiajenoPartePositiva} y \ref{cor-MADPositivCarac} que $\ms{A} \upharpoonright X \notin \Mad(X)$.

	De forma recíproca, si $\ms{A} \upharpoonright X \notin \Mad(X)$, entonces existe $B \in [X]^\omega$ casi ajeno con cada elemento de $\ms{A} \upharpoonright X$. Nótese que entonces $B \cap X$ es casi ajeno con cada elemento de $\ms{A}$; y por lo tanto, $B \to \infty$, dado el \cref{lem-convClave}.
\end{proof}

La contención recíproca de \ref{cor-CasiajenoPartePositiva} encapsula exáctamente la conexión que existe entre la combinatoria de $\ms{A}$ y la propiedad de Fréchet en su compacto de Franklin asociado.
\begin{corolario}\phantomsection\phantomsection\label{cor-TraFrechet}\index[trad]{propiedad de!Fréchet en $\ms{F}(\ms{A})$}
	Para cada $\ms{A} \in \Ad{A}$, son equivalentes:
	\begin{enumerate}
		\item $\ms{F}(\ms{A})$ es de Fréchet.
		\item $\{X \in [\omega]^{\omega} \tq \ms{A} \upharpoonright X = \emptyset \} = \ms{I}^+(\ms{A})$.
		\item $\ms{A}$ es maximal en ninguna parte.
	\end{enumerate}
\end{corolario}

\begin{proof}
	(i) $\to$ (ii) Supóngase que $\ms{F}(\ms{A})$ es de Fréchet, basta probar la contención recíproca de (ii). Si $X \in \ms{I}^+(\ms{A})$, entonces $\infty \in \cla(X)=\scl(X)$ por \ref{lem-convClave}, obteniéndose del mismo resultado que $\ms{A} \upharpoonright X = \emptyset$.

	(ii) $\to$ (iii) Supóngase (ii) y sea $X \in \ms{I}^+(\ms{A})$ cualquiera. Dada la hipótesis, \ref{lem-convClave} y el Lema previo, resulta que $\ms{A} \upharpoonright X \notin \Mad(X)$.

	(iii) $\to$ (i) Supóngase que $\ms{A}$ es maximal en ninguna parte. Como $\Psi(\ms{A})$ es de Fréchet (por ser primero numerable) basta verificar la propiedad de Fréchet en $\infty \in \ms{F}(\ms{A})$. Sea $X \subseteq \ms{F}(\ms{A})$ de modo que $\infty \in \cla(X)$, entonces por \ref{lem-convClave}, $X \cap \ms{A}$ es infinito o $X \cap \omega \in \ms{I}^+(\ms{A})$. Si ocurre lo primero, sea $B \subseteq X \cap \ms{A}$ numerable y nótese que entonces $B \to \infty$, lo cual basta para mostrar que $\infty \in \scl(X)$. Si ocurre el segundo caso, de la hipótesis se obtiene $A \upharpoonright (X \cap \omega) \notin \Mad(X \cap \omega)$, probando que $\infty \in \scl(X \cap \omega) \subseteq \scl$ (en virtud \ref{lem-TrazaMad}). En ambos casos, $\infty \in \scl(X)$; y por tanto $\scl(X)=\cla(X)$.
\end{proof}

Por el Teorema de Simon (\ref{Teo-Simon}), cualquier familia maximal de tamaño $\kappa$ contiene una familia maximal en ninguna parte del mismo tamaño. Considerando entonces cualquier familia maximal de tamaño $\mathfrak{c}$, la \cref{prop-caracterFrechet} concede el siguiente ejemplo, testigo de lo \enquote{lejos} que están entre sí las propiedades $1\AN$ y de Fréchet, incluso en la clase de espacios compactos, de Hausdorff.

\begin{ejemplo}
	Existe un espacio compacto, de Hausdorff, separable y de Fréchet que contiene un punto de carácter $\mathfrak{c}$.
\end{ejemplo}

La \enquote{traducción} a la propiedad de Fréchet dada en \ref{cor-TraFrechet} esconde la capacidad de solucionar un problema relevante en topología general; determinar si el producto de dos espacios de Fréchet es, nuevamente, de Fréchet, la respuesta es negativa.

\begin{proposicion}
	Sea $(\ms{B},\ms{C})$ una grieta. Si $\ms{A}:=\ms{B} \cup \ms{C}$ es maximal en alguna parte, entonces $M:=\ms{F}(\ms{B}) \times \ms{F}(\ms{C})$ no es de Fréchet.
\end{proposicion}

\begin{proof}
	Supóngase que $X \in\ms{I}^+(\ms{A})$ es un conjunto tal que se satisface $\ms{A} \upharpoonright X \in \Mad(X)$ y sea $B:=\{ (n,n) \tq n \in X \}$.

	Como $X \in \ms{I}^+(\ms{A})$ y $\ms{B},\ms{C} \subseteq \ms{A}$, resulta inmediato a \ref{prop-TrazaBasicos} que $X \in \ms{I}^+(\ms{B})$ y $X \in \ms{I}^+(\ms{C})$. De aquí que $\infty_\ms{B} \in \cla_{\ms{F}(\ms{B})}$ y $\infty_\ms{C} \in \cla_{\ms{F}(\ms{C})}$, a consecuencia del \cref{lem-convClave}. Por lo anterior, $ (\infty_\ms{B},\infty_C) \in \cla_M (B) $.

	Ahora, si ocurriese $(\infty_\ms{B},\infty_\ms{C}) \notin \scl_M(B)$, en virtud de la continuidad de las funciones proyección, resultaría que $Y \to \infty_\ms{B}$ en $\ms{F}(\ms{B})$ y $Y \to \infty_\ms{C}$ en $\ms{F}(\ms{C})$. Gracias a \ref{lem-convClave}, lo anterior desembocaría en que $Y \subseteq X$ es infinito y casi ajeno con cada elemento de $\ms{B}$ y $\ms{C}$, así pues, con cada elemento de $\ms{A} \upharpoonright X$, contradiciendo la maximalidad de $\ms{A} \upharpoonright X$ en $X$. Así que $\scl_M(B) \subsetneq \cla_M(B)$, y $M$ no es de Fréchet.
\end{proof}

Combinando con el Teorema de Simon (\ref{Teo-Simon}), se tiene la siguiente fuente de contrajemplos: Cada vez que $\ms{A}$ sea una familia infinita y maximal (por ello, no compacta), se pueden dar dos familias $\ms{B} \subseteq \ms{C}$ no vacías, maximales en ninguna parte, de modo que $\ms{A}=\ms{B} \cup \ms{C}$. Y se desprende de la Proposición previa que $\ms{F}(\ms{B}) \times \ms{F}(\ms{C})$ no es de Fréchet; pues claramente $\ms{A}$ es maximal en alguna parte, ya que $\omega \in \ms{I}^+(\ms{A})$ (véase el \cref{cor-IdealPropioCaract}). Además, $\ms{F}(\ms{B})$ y $\ms{F}(\ms{C})$ son ambos de Fréchet, en virtud del \cref{cor-TraFrechet}. Esta discusión es la prueba del siguiente Corolario:

\begin{corolario}\phantomsection\phantomsection\label{cor-FrechNoProd}
	Existen dos espacios compactos, de Hausdorff, separables y de Fréchet cuyo producto no es de Fréchet.
\end{corolario}

Es decir, la propiedad de Fréchet no es (finitamente) productiva; ni siquiera en una clase \enquote{tan buena} como lo es la de los espacios compactos, de Hausdorff, separables.
        \chapter{Normalidad en los espacios de Mrówka}

\index[alph]{Conjetura!de Moore}\index[alph]{Conjetura!débil de Moore}\index[alph]{Moore!conjetura de}\index[alph]{Moore!conjetura débil de}\index[sym]{$\Pm$}\index[sym]{$\Pdm$}
\emph{\small La Conjetura de Moore }{\small ($\Pm$)}\emph{\small es el enunciado: todo espacio normal y de Moore es metrizable. Se trata de un problema lanzado a la comunidad matemática por Jones en 1933 que atiende a la cuestión ¿qué requiere un espacio de Moore para ser metrizable?. Este problema marcó un antes y un después para la topología general, consolidándose como uno de los grandes problemas, probablemente de los más importantes, en la topología general del sigo XX. $\Pm$ es, presumiblemente, independiente de $\zfc$ }{\small \cite[p.~429-435]{nyikosMoore}}\emph{\small. Lo que haremos en este capitulo será mostrar la independencia de una versión débil de $\Pm$, la que llamaremos \enquote{Conjetura débil de Moore} }{\small ($\Pdm$)}\emph{\small: todo espacio separable, normal y de Moore es metrizable.}

\emph{\small En el año 1937 }{\small \cite[Teo~5, p.~ 676]{jonesCM}}\emph{\small, el propio Jones muestra la consistencia de $\Pdm$, pero no sería sino hasta 1969 cuando Tall, en su tesis doctoral }{\small \cite{tallTesis}}\emph{\small, logra establecer una equivalencia para $\Pdm$ en términos de la existencia de $Q$-sets no numerables. Silver probó que la existencia de tales espacios es consistente con $\zfc$. La meta de este capítulo es exponer las contribuciones de Jones, Silver y Tall, lo cual culmina con la independencia de $\Pdm$ de la axiomática $\zfc$.}

\section{Independencia de $\Pdm$}

Todo espacio de Mrówka $\Psi(\ms{A})$ es de Moore y separable, algo demostrado en \ref{cor-MrwokaSiempre}; además, cuando $\ms{A}$ es más que numerable, resulta que $\Psi(\ms{A})$ no es metrizable (véase \ref{prop-tra-numerable}). Así que, si ocurre $\Pdm$, entonces ningún espacio $\Psi(\ms{A})$, con $|\ms{A}|>\omega$, puede ser normal. Lo que ataña a la presente sección presenta una dificultad mayor, se mostrará el recíproco de la anterior implicación.

\begin{proposicion}\label{pro-ObsNormalidad}
	Sea $\ms{A} \in \Ad(\omega)$, se cumple:
	\begin{enumerate}
		\item Si $|\ms{A}| \leq \omega$, entonces $\Psi(\ms{A})$ es normal.
		\item Si $|\ms{A}| > \omega$ y $\Psi(\ms{A})$ es normal, entonces $\ms{A} \notin \Mad(\omega)$ y $|\ms{A}|< \mathfrak{c}$.
	\end{enumerate}
\end{proposicion}
\begin{proof}
	(i) Por la caracterización \ref{prop-tra-numerable}, todo espacio de Mrówka numerable es metrizable, particularmente normal \textcolor{red}{(prelim)}.

	(ii) Supóngase que $\ms{A}$ es infinita y que $\Psi(\ms{A})$ es normal. Si $\ms{A}$ fuera maximal, entonces por \ref{prop-tra-pseudoCaract} y \ref{prop-tra-compacidad}, se tiene que $\Psi(\ms{A})$ es pseudocompacto pero no numerablemente compacto. Lo cual (por \textcolor{red}{(prelim)}) imposibilita que $\Psi(\ms{A})$ sea normal. Por tanto, $\ms{A}$ no es maximal.

	Nótese que $\ms{A}$ es un subespacio cerrado y discreto de $\Psi(\ms{A})$ de tamaño $\mathfrak{c}$ (en virtud de \ref{lem-primerosSubs}). Así, de la separabilidad y normalidad de $\Psi(\ms{A})$, junto con el Lema de Jones \textcolor{red}{(prelim)}, se obtiene que $2^{|\ms{A}|} \leq 2^{\aleph_0} = \mathfrak{c}$, lo cual obliga a que $|\ms{A}|< \mathfrak{c}$.
\end{proof}

La proposición anterior dicta que, para encontrar espacios de Mrówka que atestiguen la falsedad de $\Pdm$, es necesario buscar entre aquellos generados por familias casi ajenas de tamaño estrictamente entre $\aleph_0$ y $\mathfrak{c}$, consecuentemente:

\begin{corolario}\label{cor-HCnoMrowkasNormales}
	Si $\HC$ ocurre, entonces ningún espacio de Mrówka más que numerable puede ser normal.
\end{corolario}

\begin{proposicion}\label{pro-SeparadorCerrados}
	Sean $\ms{A}$ una familia casi ajena y $F,G \subseteq \Psi(\omega)$ cerrados ajenos. Son equivalentes:
	\begin{enumerate}
		\item $F$ y $G$ se separan por abiertos ajenos de $\Psi(\ms{A})$.
		\item $F \cap \ms{A}$ y $G \cap \ms{A}$ se separan por abiertos ajenos de $\Psi(\ms{A})$.
		\item La grieta $(F \cap \ms{A},G \cap \ms{A})$ está separada.
	\end{enumerate}
\end{proposicion}

\begin{proof}
	La implicación (i) $\to$ (ii) es inmediata.

	(ii) $\to$ (iii) Supóngase que $U,V \subseteq \Psi(\ms{A})$ son abiertos ajenos con $F \cap \ms{A} \subseteq U$ y $G \cap \ms{A} \subseteq U$. Para cada $a \in F \cap \ms{A} \subseteq U$ se tiene que $a \subseteq^* U$. Consecuentemente, para cada $b \in G \cap \ms{A} \subseteq V$ ocurre que $b \cap U =^* V \cap U = \emptyset$. Por lo tanto, $U$ es separador de $F \cap \ms{A}$ y $G \cap \ms{A}$.

	(iii) $\to$ (i) Supóngase que $D \subseteq \omega$ es separador de $F \cap \ms{A}$ y $G \cap \ms{A}$. Nótese que $F \subseteq U:=F \cup D\setminus G$ y que $U$ es abierto. En efecto, si $a \in U \cap \ms{A} \subseteq F \cap \ms{A}$, entonces $a \subseteq^* D$ y, por ser $G$ cerrado y ajeno a $F$, $a \subseteq^* \Psi(\ms{A}) \setminus G$. Luego, $a \subseteq^* D\setminus G \subseteq U$.

	Como $D$ es separador de $F \cap \ms{A}$ y $G \cap \ms{A}$; $\omega \setminus D$ es separador de $G\cap \ms{A}$ y $F \cap \ms{A}$, y resulta análogo que $G \subseteq V:=G \cup (\omega \setminus D)\setminus F$ y $V$ es abierto. Lo anterior demuestra que $F$ y $G$ se separan por los abiertos ajenos $U$ y $V$.
\end{proof}

Del \autoref{lem-primerosSubs} y la \autoref{obs-GrietasSimple} se desprende la posterior equivalencia:

\begin{corolario}\label{cor-tra-NormalSeparador}\index[trad]{Normalidad de $\Psi(\ms{A})$ (con grietas separables)}
	Para cada familia casi ajena $\ms{A}$ son equivalentes:
	\begin{enumerate}
		\item $\Psi(\ms{A})$ es normal.
		\item Para cada $\ms{B} \subseteq \ms{A}$, la grieta $(\ms{B},\ms{A} \setminus \ms{B})$ está separada.
	\end{enumerate}
\end{corolario}

La traducción de la \autoref{pro-ObsNormalidad} en términos \enquote{combinatorios} es el siguiente Corolario; mismo que por cierto, prueba el \autoref{ej-interrelacion} de la \autoref{Sec-Luzin}.

\begin{corolario}\label{col-tra-interrelacion}
	Sea $\ms{C}\in \Ad(\omega)$, entonces:
	\begin{enumerate}
		\item Si $|\ms{C}|\leq \aleph_0$, toda grieta contenida en $\ms{C}$ está separada.
		\item Si $\ms{C}$ es infinita y, $|\ms{C}|=\mathfrak{c}$ o $\ms{C}\in \Mad(\omega)$; entonces $\ms{C}$ contiene una grieta que no está separada.
	\end{enumerate}
\end{corolario}

Hay una diferencia notable entre los conceptos de familia no separable y familia inseparable, en términos  topológicos, las familias inseparables (definidas en \ref{def-FamInseparable}) generan espacios de Mrówka más que no normales. Cualquier familia de Luzin (\autoref{def-LuzinFam}) es inseparable, lo cual regala un ejemplo \enquote{canónico} dentro de $\zfc$ de espacio separable, de Moore y no normal de tamaño $\mathfrak{c}$.

\begin{corolario}\label{cor-MrowkaLuzin}
	%Si $\ms{A}$ es una familia que contiene una $n$-grieta de Luzin, ningún par de cerrados ajenos no numerables de $\Psi(\ms{A})$ se pueden separar por abiertos ajenos.
	Particularmente, existe un espacio de Mrówka con tamaño $\aleph_1$, no es normal; a saber, el generado por una familia de Luzin.
\end{corolario}

Ahora, por lo demostrado en el \autoref{cor-ngrietasNoSep}, se tiene el siguiente corolario:

\begin{corolario}
	Sea $\ms{A} \in \Ad(\omega)$. Si $\Psi(\ms{A})$ es normal, entonces $\ms{A}$ no contiene $n$-grietas de Luzin.
\end{corolario}

Dado todo lo realizado hasta el momento, el corolario anterior es relativamente inmediato; en contraste, su recíproco logra caracterizar por completo la normalidad de los espacios de Mrówka, en $\zfc +  \Ma$, \cite[Teo.~9]{hruAlmost}. Tal resultado es meritorio de un estudio mucho más profundo y dedicado, mismo que escapa a los propósitos de este trabajo.

\subsection{Consistencia de \textsf{WMC}}
\label{Sec-PDM}

Siguiendo la técnica de Tall, probaremos la independencia de $\Pdm$ de $\zfc$ utilizando, a modo de intermediario, los siguientes espacios:

\begin{definicion}\label{def-Qset}\index[alph]{$Q$-set}
	Un $Q$-set es un espacio metrizable, separable y tal que todos sus subespacios son de tipo $G_\delta$ (equivalentemente, todos sus subespacios son de tipo $F_\sigma$).
\end{definicion}

Como primer familia de ejemplos de espacios $Q$-set se tiene:

\begin{ejemplo}\label{ej-QsetFacil}
	Cualquier espacio $X$ metrizable, a lo más numerable, es un $Q$-set. Efectivamente, nótese que $X$ es separable. Y además, si $A \subseteq X$ es cualquiera, entonces $A=\midcup\big\{\{a\} \tq a \in X\big\}$ es de tipo $F_\sigma$.
\end{ejemplo}

Se comenzará por observar que todo espacio $Q$-set es, salvo homeomorfismos, un subespacio de $\mathbb{R}$ (y del conjunto de Cantor, $2^\omega$). El siguiente Lema, incluido en \cite[Teo.~1, p.~286]{kuratowskiTopology} por Kuratowski; se enunciará y demostrará con terminología moderna.
\begin{lema}
	Sea $X$ un espacio metrizable por la métrica $d$. Si $|X|<\mathfrak{c}$, entonces $X$ es cero-dimensional.
\end{lema}
\begin{proof}
	Supóngase que $|X|<\mathfrak{c}$. Basta corroborar que cada $x \in X$ admite una base local de abiertos y cerrados. Sean $x \in X$ y $\varepsilon>0$.

	Supóngase ahora que para cada $\delta \in (0,\varepsilon)$, el conjunto $\fron(B(x,\delta))$ es no vacío, y fíjese ($\Ac$) un elemento $x_\delta \in \fron(B(x,\delta)) \subseteq X$. Como $|(0,\varepsilon)|=\mathfrak{c}$, la asignación $\delta \to x_\delta$ no puede ser inyectiva. Consecuentemente, existen distintos $\delta,\delta'\in (0,\varepsilon)$ de modo que $\fron(B(x,\delta)) \cap \fron(B(x,\delta')) \neq \emptyset$. Pero esto es imposible, dado que $\delta \neq \delta'$. Por lo tanto, para cada $\varepsilon>0$ se puede fijar ($\Ac$) cierto $\delta_\varepsilon \in (0,\varepsilon)$ tal que $\fron(B(x,\delta_\varepsilon))=\emptyset$; esto es, $B(x,\delta_\varepsilon)$ es abierto y cerrado a la vez.
	
	Por construcción, $\{B(x,\delta_\varepsilon) \tq \varepsilon>0\}$ es una base local para $x$ en $X$.
\end{proof}

\begin{proposicion}\label{prop-QsetEquivs}
	Para todo espacio $X$ son equivalentes:
	\begin{enumerate}
		\item $X$ es un $Q$-set.
		\item $X$ se encaja en $2^\omega$ y todos sus subespacios son de tipo $G_\delta$.
		\item $X$ se encaja en $\mathbb{R}$ y todos sus subespacios son de tipo $G_\delta$.
	\end{enumerate}
\end{proposicion}
\begin{proof}
	Todo espacio metrizable y separable tiene peso numerable \textcolor{red}{(prelim)}, así que cualquier espacio metrizable, separable y cero-dimensional se encaja en $2^\omega$ (véase \cite[Teo.~8.5.11, p.~299]{fidelElementos}). Además, $2^\omega$ es subespacio de $\mathbb{R}$, y todo subespacio de $\mathbb{R}$ es metrizable y separable. Así que, basta probar que todo $Q$-set es cero-dimensional.

	Seas $X$ un $Q$-set, como $X$ es metrizable y separable, entonces es $2\AN$. Sea $\mathcal{B}$ una base numerable para $X$. Como $X$ es $Q$-set, para cada $A \subseteq X$ fíjese ($\Ac$) una colección a lo más numerable de abiertos $\mathcal{U}_A$ de modo que $A=\midcap \mathcal{U}_A$. Y como $\mathcal{B}$ es base; de nuevo haciendo uso de $\Ac$, para cada abierto $U \in \mathcal{U}_A$ fíjese $\mathcal{B}_U \subseteq \mathcal{B}$ de modo que $U=\midcup \mathcal{B}_U$.

	Lo anterior permite definir $\ms{P}(X) \to [\ms{P}(\mathcal{B})]^{\leq \omega}$ por medio de la correspondencia: $A \mapsto \{\mathcal{B}_U \tq U \in \mathcal{U}_A\}$. Tal asignación es inyectiva; en efecto, si $A,B \subseteq X$ y $\{\mathcal{B}_U \tq U \in \ms{A}\}=\{\mathcal{B}_U \tq U \in B\}$, resulta que:
	\[ \mathcal{U}_A = \{\midcup \mathcal{B}_U \tq U \in \mathcal{U}_A\} = \{\midcup \mathcal{B}_U \tq U \in \mathcal{U}_B\} = \mathcal{U}_A \]
	y con ello $A=\midcap \mathcal{U}_A = \midcap \mathcal{U}_B = B$. De esta manera:
	\[ 2^{|X|} \leq \left| [\ms{P}(\mathcal{B})]^{\leq \omega} \right| \leq \left( 2^{|B|} \right)^{\aleph_0} \leq \left( 2^{\aleph_0} \right) ^{\aleph_0} = 2^{\aleph_0 \cdot \aleph_0} = 2^{\aleph_0} = \mathfrak{c} \]

	Por ello $|X| < \mathfrak{c}$ la cero-dimensionalidad de $X$ se obtiene del Lema previo.
\end{proof}

\begin{observacion}\label{obs-HCNoQset}
	Todo $Q$-set tiene tamaño menor que $\mathfrak{c}$. Consecuentemente, $\HC$ implica que todos los espacios $Q$-set son a lo más numerables.
\end{observacion}

El paralelismo del resultado anterior con el \autoref{cor-HCnoMrowkasNormales} no es coincidencia. La meta ahora es mostrar que la existencia de $Q$-sets no numerables es equivalente a la existencia de espacios de Mrówka no numerables y normales; más aún, si estos espacios no existen, entonces $\Pdm$ se satisface.

\begin{lema}\label{coro-Bing}
	Supóngase que $X$ es un espacio normal, de Moore, no metrizable y que $D \subseteq X$ es denso a lo más numerable; entonces existe $A \subseteq X \setminus D$ más que numerable, discreto y cerrado en $X$.
\end{lema}

\begin{proof}
	A consecuencia del Tereoma de Bing \textcolor{red}{(prelim)}, $X$ no es colectivamente normal. Sea $\mathcal{A}$ una familia discreta de cerrados de $X$, cuyos elementos no se pueden separar por abiertos ajenos, como $X$ es normal, $\mathcal{A}$ es infinita. De igual forma, por normalidad de $X$, para cada par de cerrados ajenos, $F$ y $G$, elíjanse ($\Ac$) abiertos ajenos $W(F,G),S(F,G)$ con $F\subseteq W(F,G)$ y $G \subseteq S(F,G)$.

	\begin{enumerate}[\hspace{1.5 cm}, listparindent=1.5em]
		\item \textit{Afirmación.} $\mathcal{A}$ es más que numerable.

		\item \textit{Demostración.} Supóngase que $\mathcal{A}$ está enumerado como $\{A_n \tq n \in \omega\}$. Por ser $\mathcal{A}$ una familia discreta de cerrados de $X$, si $n \in \omega$, entonces el conjunto $B_n:=\midcup\{A_m \tq m>n\}$ es cerrado.

		      Por recursión defínanse $U_0:=W(A_0,B_0)$ y $V_0:=S(A_0,B_0)$; y, para cada $n \in \omega$, $U_{n+1}:=W(A_{n+1},B_{n+1}) \cap V_n$ y $V_{n+1}= S(A_{n+1},B_{n+1}) \cap V_n$.

		      Por construcción, $\{U_n \tq n\in \omega\}$ es una familia de abiertos, ajenos por pares tales que para cada $n \in \omega$ se tiene $A_n \subseteq U_n$. Así, los elementos de $\mathcal{A}$ se separan por abiertos ajenos; contradiciendo su elección. \hfill$\boxtimes$
	\end{enumerate}

	Dada la afirmación anterior, y fijando para cada $a \in \mathcal{A}$ un elemento $x_a \in \mathcal{A}$, se obtiene un conjunto mas que numerable $B:=\{x_a \tq a \in \mathcal{A}\}$; mismo que por ser $\ms{A}$ familia discreta y $X$ de Hausdorff, resulta ser cerrado y discreto.

	Por último nótese que cada subespacio de $B$ es discreto y cerrado en $X$; pues $B$ es discreto y cerrado en $X$. Particularmente, $A:=B \setminus D$ es discreto, discreto en $X$ y no numerable (pues $B$ es más que numerable y $D$ es numerable).
\end{proof}

El siguiente teorema aparece en la tesis doctoral de Franklin David Tall (ver \cite{tallTesis}), y es la pieza clave para atacar la Conjetura Débil de Moore.

\begin{teorema}[Tall]\label{teo-EquivPDM}
	Si $\kappa$ es un cardinal infinito, son equivalentes:
	\begin{enumerate}
		\item Existe un espacio de Moore, normal, no metrizable de tamaño $\kappa$.
		\item Existe un espacio de Mrówka normal de tamaño $\kappa$.
		\item Existe un $Q$-set de tamaño $\kappa$.
	\end{enumerate}
\end{teorema}
\begin{proof}
	(i) $\to$ (ii) Supóngase que $X$ es un espacio, normal, de Moore y no metrizable de tamaño $\kappa$ y sea fíjese $D\subseteq X$ denso numerable de $X$. Por \ref{coro-Bing}, $D$ es infinito y existe un subespacio $A \subseteq X \setminus D$ más que numerable; discreto y cerrado de $X$. Como $X$ es de Hausdorff y primero numerable, considérese $ \ms{A}_{D,A}=\{A_x \in [D]^\omega \tq x \in A\} $; la familia de sucesiones en $D$ convergentes a $A$ (definida en \ref{def-FamSucesiones}), donde cada $A_x$ converge a $x$. Por la \autoref{prop-famSucesiones}, $|\ms{A}_{D,A}|=\kappa$ y así mismo, $\Psi_D(\ms{A}_{D,A})=\kappa$.

	Sea $\ms{B}:=\{A_x \tq x \in F\} \subseteq \ms{A}_{D,A}$ cualquiera. Como $A$ es discreto y cerrado en $X$, cualquiera de sus subespacios es cerrado en $X$; en consecuencia y por normalidad de $X$, existen abiertos $U,V \subseteq X$ ajenos, de modo que $F \subseteq U$ y $A \setminus F \subseteq V$. Si $x \in F$, entonces $A_x \to x$ y por ello $A_x \subseteq^* U$, similarmente, si $y \in A \setminus F$, entonces $A_y \subseteq^* V \subseteq X \setminus U$; de donde $A_y \cap U^* = \emptyset$.

	Por tanto $(\ms{B}, \ms{A}_{D,A} \setminus \ms{B})$ está separada, obteniéndose de \ref{cor-tra-NormalSeparador} la normalidad de $\Psi_D(\ms{A}_{D,A})$.

	(ii) $\to$ (iii) Si $\kappa = \omega$, la implicación resulta vacua; pues todo subespacio numerable de $2^\omega$ es un $Q$-set (\autoref{ej-QsetFacil}). Supóngase pues, que $\Psi(\ms{A})$ es un espacio normal de tamaño $\kappa>\omega$; siendo necesario que $|\ms{A}|=\kappa$.

	Para cada $C \subseteq \omega$ denótese por $\varphi_C \in 2^ \omega$ a la función característica de $C$ y sea $X:=\{\varphi_A \in 2^\omega \tq A \in \ms{A} \}$. Obsérvese que $X$ es un espacio metrizable, separable (por ser subespacio del metrizable, separable, $2^\omega$) de tamaño $\kappa$.

	Sea $Y = \{ \varphi_A \in X \tq A \in \ms{B} \} \subseteq X$ cualquiera. Dado el \autoref{cor-tra-NormalSeparador}, la normalidad de $\Psi(\ms{A})$ implica la existencia de un separador de $\ms{A}\setminus \ms{B}$ y $\ms{B}$; de este modo:
	\begin{align*}
		Y & = \{ \varphi_A \in X \tq A \in \ms{B} \}                                        \\
		  & = \{ \varphi_A \in X \tq A \cap D \neq^* \emptyset \}                           \\
		  & = \{ \varphi_A \in X \tq \forall n \in \omega \: ( A \cap D \not\subseteq n) \} \\
		  & = \bigcap_{n \in \omega} \{\varphi_A \in X \tq A \cap D \not\subseteq n\}
	\end{align*}

	Ahora, si $n \in \omega$ y $\varphi \in U_n:=\{\varphi_A \in X \tq A \cap D \not\subseteq n\}$ es cualquiera, existe cierto $k \in (A \cap D) \setminus n$. Por ello, si $x=\varphi_B \in X$ es tal que $x(k)=1$, entonces $k \in (B \cap D) \setminus n$ y $B \cap D \not \subseteq n$; es decir $f \in U_n$. Mostrando así que $\{ x \in X \tq x\upharpoonright \{k\} = \varphi \upharpoonright \{k\} \} \subseteq U_n$. Por lo tanto, cada $U_n$ es abierto en $X$. De esta manera, cualquier $Y \subseteq X$ es $G_\delta$ en $X$ y $X$ es un $Q$-set.

	(iii) $\to$ (i) Supóngase que $X$ es un $Q$-set de tamaño $\kappa$. En virtud del \autoref{prop-QsetEquivs}, supóngase sin pérdida de generalidad que $X \subseteq 2^\omega$. Para cada $E \subseteq X$ considérese $ \ms{A}_E:=\{ A_x \in [N]^\omega \tq x \in E \} $, la familia de las ramas de $E$ en $N$ (definida en \ref{def-FamRamas}); donde $N:=2^{<\omega}$ y cada $A_x$ es el conjunto $\{x \upharpoonright n \tq n \in \omega\} \subseteq N$. Puesto que $|X|=\kappa$, del \autoref{def-FamRamas} se sigue que $|\ms{A}_X|=\kappa$, y con ello $|\Psi_N(\ms{A}_X)|=\kappa$.

	Sea $Y \subseteq X$ cualquiera. Como $X$ es un $Q$-set, $Y=\midcup\{F_n \tq n \in \omega\}$ y $X \setminus Y=\midcup\{G_n \tq n\in \omega\}$; donde cada conjunto $F_n$ y $G_n$ es cerrado en $X\subseteq 2^\omega$. Para cada $n \in \omega$ defínanse los conjuntos:
	\begin{align*}
		D_n := \left( \midcup \ms{A}_{F_n} \right) \setminus \bigcup_{m<n} \left( \midcup \ms{A}_{G_m} \right) \\
		L_n := \left( \midcup \ms{A}_{G_n} \right) \setminus \bigcup_{m\leq n} \left( \midcup \ms{A}_{F_m} \right)
	\end{align*}
	y sea $D:=\midcup\{D_n \tq n \in \omega\}$. Nótese que por construcción, si $m,n \in \omega$, se tiene $D_n \cap L_m = \emptyset$; consecuentemente, cada $L_n$ es ajeno con $D$.

	Sea $y \in A_Y$; entonces existe $n \in \omega$ de modo que $y \in F_n$. Por otra parte, cada $G_m\subseteq X\setminus Y$ (con $m<n$) es cerrado en $X$, por lo que existe $s \in \omega$ de modo que:
	$$ \{ x \in X \tq x \upharpoonright s = y \upharpoonright s \} \subseteq X \setminus \bigcup_{m<n} G_m $$

	Por ello, si $v \in A_y \setminus D_n\subseteq F_n$, existen $x \in F_n$ y $k \in \omega$ de modo que $v = x \upharpoonright k$. Así que $x \in \midcup \ms{A}_{F_n}$; y como $v \notin D_n$, existen $m<n$ y $g \in G_m$ de manera que $v = y \upharpoonright k = g \upharpoonright k$. Y a razón de ello, no puede ocurrir $s \subseteq k$. Por lo tanto $k<s$ y $A_y \setminus D_n \subseteq 2^{<s} =^* \emptyset$; esto es, $A_y \subseteq^* D_n \subseteq D$.

	Similarmente, para cada $y \in X \setminus Y$ existe un $n \in \omega$ tal que $A_y \subseteq^* L_n \subseteq N \setminus D$; de donde, $A_y \cap D ^= \emptyset$. Así que $D$ es separador de $\ms{B}$ y $\ms{A} \setminus \ms{B}$; probando por el \autoref{cor-tra-NormalSeparador} la normalidad de $\Psi_N(\ms{A}_X)$.
\end{proof}

Es inmediato al \autoref{teo-EquivPDM} (y a \ref{cor-HCnoMrowkasNormales}, o bien, \ref{obs-HCNoQset}) la siguiente consecuencia:

\begin{corolario}\label{cor-PdmConsistente}
	Bajo $\HC$ se cumple $\Pdm$, y con ello:
	\begin{enumerate}
		\item Ningún espacio de Mrówka no numerable es normal.
		\item Ningún $Q$-set es más que numerable.
	\end{enumerate}
	A razón de lo anterior, $\Pdm$ es consistente con $\zfc$.
\end{corolario}

\subsection{Consistencia de \texorpdfstring{$\lnot$}\textsf{WMC}}

Para la segunda parte de la prueba de independencia de $\Pdm$ se hará uso; como es previsible desde anteriores capítulos, de la negación de $\HC$ con el Axioma de Martin. Se comenzará observando cómo se pueden caracterizar a todos los $Q$-sets haciendo uso del Lema de Solovay (\autoref{lem-Solovay}) y $\Ma$.

\begin{lema}
	Sea $X$ un espacio metrizable y separable. Entonces existe una base $\mathcal{B}=\{B_n \tq n \in \omega\}$ para $X$ de modo que $\ms{A} = \{A_x \tq x \in X\}$ es familia casi ajena; donde para cada $x \in X$, $A_x=\{n \in \omega \tq x \in B_n\}$.
\end{lema}
\begin{proof}
	Por el teorema de metrización de Arhangel’skii \textcolor{red}{(prelim)}, $X$ admite una base regular \textcolor{red}{(prelim)} $\mathcal{C}$. Como $X$ es $2\AN$ por ser metrizable y separable \textcolor{red}{(prelim)}, existe una base $\mathcal{B}=\{B_n \tq n \in \omega\} \subseteq \mathcal{C}$, obsérvese que $\mathcal{B}$ sigue siendo una base regular.

	Dados $x,y \in X$ son distintos, sean $U,V$ abiertos ajenos que separan a $x$ y $y$. Por regularidad de $\mathcal{B}$, existe $W \subseteq U$ abierto de manera que $x\in W$ y
	\[ \mathcal{B}_W:=\{n \in \omega \tq B_n \cap W \neq \emptyset \land B_n \setminus W \neq \emptyset\} =^* \emptyset \, . \]
	
	Luego, el conjunto $A_x \cap A_y \subseteq \mathcal{B}_W$ es finito.
\end{proof}

\begin{proposicion}[Tall, Silver]\label{pro-MaQsetChar}
	Bajo $\Ma$, un espacio $X$ es $Q$-set si y sólo si es homeomorfo a un subespacio de $\mathbb{R}$ de tamaño menor que $\mathfrak{c}$.
\end{proposicion}
\begin{proof}
	Supóngase $\Ma$. La suficiencia viene dada por \ref{teo-EquivPDM} y \ref{prop-QsetEquivs}.
	
	Para la necesidad supóngase que $X \in [\mathbb{R}]^{<\mathfrak{c}}$, sin pérdida de generalidad más que numerable (véase \autoref{ej-QsetFacil}). Como $X$ es metrizable y separable, tómense $\mathcal{B}=\{B_n \tq n \in \omega \}$ y $\ms{A}=\{A_x \tq x \in X\}$ como en el Lema previo. Sean $Y \subseteq X$ y $\ms{B}:=\{A_y \in \ms{A} \tq y \in Y\}$.

	Como $|\ms{A} \setminus \ms{B}|,|\ms{B}|<\mathfrak{c}$ y se cumple $\Ma$, del Lema de Solovay (\ref{lem-Solovay}) se desprende la existencia de un subconjunto $D \subseteq \omega$  tal que, si $y \in Y$ y $x \in X \setminus Y$, entonces $A_y \cap D\neq ^*\emptyset$ y $A_x \cap D= ^*\emptyset$. Para cada $n \in \omega$ defínase el conjunto $U_n:=\midcup\{ B_m \in \mathcal{B} \tq m \in D \setminus n \}$.
	
	Nótese que $Y=\midcap\{U_n \tq n \in \omega\}$. En efecto, si $y \in X \setminus Y$ y $n \in \omega$ son cualesquiera, $A_y \cap D$ es infinito, y por ello, existe $m \in D \setminus n$ tal que $y \in B_m$. En consecuencia $y \in U_n$, y así $Y \subseteq \midcap\{U_n \tq n \in \omega\}$.

	De manera similar, si $x \in X \setminus Y$, $A_x \cap D$ es finito y existe $n \in \omega$ de modo que $A_x \cap D \subseteq n$. Por lo que para cada $m > n$ se tiene que $x \notin B_m$; luego, $x \notin U_n$. Lo anterior muestra que $X \setminus Y \subseteq X \setminus \midcap\{U_n \tq n \in \omega\}$, y así $Y=\midcap\{U_n \tq n \in \omega\}$ es de tipo $G_\delta$. Por lo que $X$ es un $Q$-set.
\end{proof}

%Nótese que la influencia de $\Ma$ en la previa caracterización radica únicamente en la necesidad, cuando $\aleph_1 \leq |X| < \mathfrak{c}$.

De la proposición recién mostrada, el \autoref{teo-EquivPDM} y el \autoref{cor-PdmConsistente} surge el resultado que pone punto final a la Conjetura Débil de Moore.

\begin{corolario}\label{cor-PdmIndependiente}
	Bajo $\Ma$; para cada cardinal infinito $\kappa<\mathfrak{c}$ existe un espacio de Mrówka normal, de tamaño $\kappa$. Consecuentemente:
	\begin{enumerate}
		\item Bajo $\Ma+\lnot \HC$; existen tales espacios.
		\item $\lnot\Pdm$ (y por ello, $\lnot\Pm$) es consistente con $\zfc$.
		\item $\Pdm$ es independiente de $\zfc$.
	\end{enumerate}
\end{corolario}

Contrastable con \ref{pro-MaQsetChar} es el hecho de que aun no se ha dado una caracterización para la normalidad de los espacios de Mrówka. Resulta seductor conjeturar que cualquier espacio de Isbell-Mrówka de tamaño menor al continuo es normal. Sin embargo, el \autoref{cor-MrowkaLuzin} muestra que; bajo $\Ma+\lnot \HC$, existe un espacio de Mrówka, no normal y de tamaño menor al continuo.

El comentario anterior deja como consecuencia la falsedad de que cualquier familia casi ajena sea esencialmente igual a alguna de las definidas en \ref{def-FamRamas} (\autoref{Dif-esencial}); de lo contrario, cualquier espacio espacio de Isbell-Mrówka de tamaño menor al continuo sería es normal, cosa que es falsa, en $\zfc$ únicamente. Se abre también una discusión interesante, el estudio del enunciado \enquote{para cada cardinal $\kappa$ entre $\aleph_1$ y $\mathfrak{c}$ existen espacios de Mrówka no normales de tamaño $\kappa$}.
\end{document}
