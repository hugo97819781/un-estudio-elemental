%Colores
\definecolor{azul}{RGB}{0, 60, 113}
\definecolor{dorado}{RGB}{196, 151, 57}
\definecolor{morado}{RGB}{101, 43, 145}
\definecolor{rosa}{RGB}{146, 33, 209}
\definecolor{azulC}{RGB}{52, 90, 229}

%Setup
\mdfsetup{linewidth=0}
%\mdfsetup{
%    splittable=true,
%    usename=true,
%    skipabove=2pt,
%    skipbelow=2pt,
%    needspace=no
%}

%Estilos de Cajas
\mdfdefinestyle{caja1}{
	roundcorner=5pt,
	backgroundcolor=azul!8,
	linewidth=0,
}

\mdfdefinestyle{caja2}{
	roundcorner=5pt,
	backgroundcolor=dorado!8,
	linewidth=0,
}

\mdfdefinestyle{caja3}{
	roundcorner=5pt,
	backgroundcolor=morado!8,
	linewidth=0,
}
\mdfdefinestyle{caja4}{
	roundcorner=5pt,
	backgroundcolor=gray!10,
	linewidth=0,
}

%Estilos de Teoremas
\declaretheoremstyle[
    headfont=\bfseries,         % "Lema A.1" en negrita
    notefont=\bfseries,         % "(de Solovay)" en negrita
    notebraces={(}{)},          % Paréntesis automáticos
    bodyfont=\itshape,        	% Texto del contenido (cámbialo a \itshape si lo quieres cursiva)
    %headpunct={.},             % Punto final tras el título
    postheadspace=0.5em,
]{myTeoStyle}

%Declaración de cajas.
\declaretheorem[style=myTeoStyle,mdframed={style=caja1},name=Definición,numberwithin=section]{definicion}
\declaretheorem[style=myTeoStyle,mdframed={style=caja2},name=Proposición,sharenumber=definicion]{proposicion}
\declaretheorem[style=myTeoStyle,mdframed={style=caja2},name=Lema,sharenumber=definicion]{lema}
\declaretheorem[style=myTeoStyle,mdframed={style=caja3},name=Corolario,sharenumber=definicion]{corolario}
\declaretheorem[style=myTeoStyle,mdframed={style=caja4},name=Observación,sharenumber=definicion]{observacion}
\declaretheorem[style=myTeoStyle,mdframed={style=caja4},name=Ejemplo,sharenumber=definicion]{ejemplo}
\declaretheorem[style=myTeoStyle,mdframed={style=caja4},name=Consideración,sharenumber=definicion]{consideracion}
\declaretheorem[style=myTeoStyle,mdframed={style=caja2},name=Teorema,sharenumber=definicion]{teorema}



%Entorno "proof"
\makeatletter
\renewenvironment{proof}{%
	\par\pushQED{\qed}%
	\renewcommand{\qedsymbol}{$\blacksquare$}%
	\normalfont \topsep6\p@\@plus6\p@\relax
	\trivlist
	\item[\hskip\labelsep\bfseries\itshape Demostración.]\ignorespaces
}{%
	\popQED\endtrivlist\@endpefalse
}
\makeatother