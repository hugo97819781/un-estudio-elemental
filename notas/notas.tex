%Formato e idioma
    \documentclass[letterpaper,DIV=12,12pt]{scrbook}
    \usepackage[spanish,mexico,shorthands=off,es-lcroman]{babel}
%Utilidades
    \usepackage{array}
    \usepackage[x11names]{xcolor}
    \usepackage{lipsum}
    \usepackage{stix2}
%Matemáticos
    \usepackage{amsmath}
    \usepackage{amsthm}
    \usepackage{amssymb} % eliminar si se usa unicode-math
    %\usepackage{mathrsfs} % agregar después de fuente unicode-math, si se usa
%Etiquetas de las enumeraciones
%    \usepackage[shortlabels]{enumitem}
%    \setenumerate[1]{label=\MakeLowercase{\roman*}), ref=\roman*}
%    \setenumerate[2]{label=\MakeLowercase{\alph*}), ref=\alph*}

\begin{document}
    \chapter*{Introducción}

    \section*{Conjuntos}

    \begin{enumerate}
        \item Axiomas y AC
        \item Axiomas adicionales como HC y \textcolor{green}{referenciar MA}.
        \item \textcolor{blue}{Consistencia, referenciar a lógica ??}
        \item Notación para contención, contención propia, diferencia, unión e intersección, potencia, producto, exponenciales, restricción.
        \item \textcolor{blue}{sucesiones, notación, crecencia y decrecencia}.
        \item Ordinales, cardinales (notaciones $\omega$, $\aleph$ y carindales $\mathfrak{c}$, $\mathfrak{m}$).
        \item Cardinalidad, finito, numerables, no numerable.
        \item Notación de corchetes y $X^{<\kappa}$.
        \item Conjuntos club, estacionarios y lema de Fodor.
        \item OPRDENES PARCIALES
            \begin{enumerate}
                \item \textcolor{blue}{segmentos iniciales ??}
                \item maximalidad y elementos distinguidos
                \item cadenas, ordenes totales y buenos ordenes.
                \item \textcolor{red}{AC sii LZ y PMO}
                \item Filtros, ideales (propios y no) ultras
                \item \textcolor{red}{teorema del ultrafiltro}
                \item CASI CONTENCION
                \begin{enumerate}
                    \item $\subseteq^*$ y $=^*$.
                    \item Propiedades de la $\subseteq^*$ respecto a: preorden, monotonía, finitud, uniones e intersecciones y SOBRE IMAGENES E IMAGENES INVERSAS.
                    \item \textcolor{blue}{Cocientes??}
                \end{enumerate}
                \item ARBOLES
                \begin{enumerate}
                    \item Definición
                    \item $2^{<\omega}$ es un arbol.
                \end{enumerate}
                \item COSAS MARTIN
                \begin{enumerate}
                    \item densos, genéricos
                    \item anticadenas y C.C.C.
                    \item compatibiliadad, paralelismo, extensiones :vvv
                \end{enumerate}
            \end{enumerate}
        \item 
    \end{enumerate}


    \section*{Topología}
    \begin{enumerate}
        \item Definición de espacio.
        \item Base, base local, vecindad, etc..
        \item Funciones continuas, encaje y homeomorfismo.
        \item Propiedades topológicas.
        \item Axiomas de separación.
        \item CONVERGENCIA
        \begin{enumerate}
            \item Sucesiones y conjuntos convergentes.
            \item Cerradura secuencial.
            \item \textcolor{red}{sucesión conv sii cjto conv en $T_1$}
            \item Espacios Secuenciales y de Fréchet.
            \item Implicaciones $1AN$ $\to$ Fréchet $\to$ Secuencial.
        \end{enumerate}
        \item 
    \end{enumerate}

\end{document}