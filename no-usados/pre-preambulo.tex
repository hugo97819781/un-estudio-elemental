\documentclass[letterpaper,DIV=12,12pt]{scrbook}
\usepackage[spanish,mexico,shorthands=off,es-lcroman]{babel}
\usepackage{scrlayer-scrpage}

\renewcommand{\chaptermark}[1]{\markboth{#1}{}}
\renewcommand{\sectionmark}[1]{\markright{#1}}

\KOMAoptions{twoside=true,open=left}

\newpairofpagestyles{chapstyle}{
	\clearpairofpagestyles
	\lehead{\textbf{\chaptername~\thechapter} \leftmark}
	\rohead{\textbf{\thesection} \rightmark}
	\rehead{\pagemark}
	\lohead{\pagemark}
	\addtokomafont{pagehead}{\normalfont\footnotesize}
	\KOMAoptions{headsepline=true}
}

\newpairofpagestyles{beginstyle}{
	\clearpairofpagestyles
	\KOMAoptions{headsepline=false}
	\cfoot{\footnotesize \pagemark}
}

\renewcommand*{\chapterpagestyle}{beginstyle}

\addtokomafont{chapter}{\rmfamily}
\addtokomafont{section}{\rmfamily}
\addtokomafont{subsection}{\raggedleft\rmfamily}
\addtokomafont{subsubsection}{\raggedleft\rmfamily}
\addtokomafont{chapterentry}{\rmfamily}

\setlength{\headsep}{5pt}
\setlength{\footskip}{25pt}
\setlength{\textheight}{500pt}
\setlength{\textwidth}{400pt}

\usepackage{array}
\usepackage[x11names]{xcolor}
\usepackage{lipsum}
\usepackage[font=footnotesize, labelfont=bf]{caption}
\usepackage[shortlabels]{enumitem}
\setenumerate[1]{label=\MakeLowercase{\roman*}), ref=\roman*}
\setenumerate[2]{label=\MakeLowercase{\alph*}), ref=\alph*}

\usepackage{amsmath}
\usepackage{amsthm}
\usepackage{amssymb}
\usepackage{stix2}

\usepackage[colorlinks=true, linkcolor=rosa, citecolor=azulC, urlcolor=dorado]{hyperref}
\usepackage[spanish, capitalize]{cleveref}
\crefname{section}{Sección}{Secciones}
\Crefname{section}{Sección}{Secciones}
\usepackage[backend=biber, style=numeric, sortcites, url=true]{biblatex}
\usepackage{csquotes,url}
\addbibresource{chapters/referencias.bib}

\usepackage{imakeidx}
\usepackage{etoolbox}
\makeindex[columns=1, intoc, title=PRUEBA]
\makeindex[name=trad, columns=1, intoc, title=Caracterizaciones]
\makeindex[name=sym, columns=3, intoc, title=Índice Simbólico]
\makeindex[name=alph, columns=2, intoc, title=Índice Alfabético]
\indexsetup{firstpagestyle=beginstyle}

\usepackage{thmtools}
\usepackage[framemethod=TikZ]{mdframed}

\definecolor{azul}{RGB}{0, 60, 113}
\definecolor{dorado}{RGB}{196, 151, 57}
\definecolor{morado}{RGB}{101, 43, 145}
\definecolor{rosa}{RGB}{146, 33, 209}
\definecolor{azulC}{RGB}{52, 90, 229}

\mdfsetup{linewidth=0}

\mdfdefinestyle{caja1}{
	roundcorner=5pt,
	backgroundcolor=azul!8,
	linewidth=0,
}

\mdfdefinestyle{caja2}{
	roundcorner=5pt,
	backgroundcolor=dorado!8,
	linewidth=0,
}

\mdfdefinestyle{caja3}{
	roundcorner=5pt,
	backgroundcolor=morado!8,
	linewidth=0,
}
\mdfdefinestyle{caja4}{
	roundcorner=5pt,
	backgroundcolor=gray!10,
	linewidth=0,
}

\declaretheoremstyle[
    headfont=\bfseries,
    notefont=\bfseries,
    notebraces={(}{)},
    bodyfont=\itshape,
    postheadspace=0.5em,
]{myTeoStyle}


\declaretheorem[style=myTeoStyle,mdframed={style=caja1},name=Definición,numberwithin=section]{definicion}
\declaretheorem[style=myTeoStyle,mdframed={style=caja2},name=Proposición,sharenumber=definicion]{proposicion}
\declaretheorem[style=myTeoStyle,mdframed={style=caja2},name=Lema,sharenumber=definicion]{lema}
\declaretheorem[style=myTeoStyle,mdframed={style=caja3},name=Corolario,sharenumber=definicion]{corolario}
\declaretheorem[style=myTeoStyle,mdframed={style=caja4},name=Observación,sharenumber=definicion]{observacion}
\declaretheorem[style=myTeoStyle,mdframed={style=caja4},name=Ejemplo,sharenumber=definicion]{ejemplo}
\declaretheorem[style=myTeoStyle,mdframed={style=caja4},name=Consideración,sharenumber=definicion]{consideracion}
\declaretheorem[style=myTeoStyle,mdframed={style=caja2},name=Teorema,sharenumber=definicion]{teorema}


\makeatletter
\renewenvironment{proof}{%
    \par\pushQED{\qed}%
    \renewcommand{\qedsymbol}{$\blacksquare$}%
    \normalfont \topsep6\p@\@plus6\p@\relax
    \trivlist
    \item[\hskip\labelsep\bfseries\itshape Demostración.]\ignorespaces
}{%
    \popQED\endtrivlist\@endpefalse
    \vspace{-0.3em}
}
\makeatother

\makeatletter
\let\c@figure\c@definicion
\let\thefigure\thedefinicion
\makeatother



\newcommand{\tq}{\text{ $|$ }}
\newcommand{\midcup}{\mbox{$\bigcup$}\,}
\newcommand{\midcap}{\mbox{$\bigcap$}\,}
\newcommand{\ms}[1]{\mathscr{#1}}
\newcommand{\mycoloneq}{\mathrel{\text{\ooalign{\raisebox{0.5ex}{\scalebox{0.52}{$\bullet$}}\cr \raisebox{0.05ex}{\scalebox{0.52}{$\bullet$}}}}\mkern-5mu =}}
\renewcommand{\emptyset}{\varnothing}
\renewcommand{\tau}{\ms{T}}

\newcommand{\mysetminusD}{\hbox{\tikz{\draw[line width=0.6pt,line cap=round] (3pt,0) -- (0,6pt);}}}
\newcommand{\mysetminusT}{\mysetminusD}
\newcommand{\mysetminusS}{\hbox{\tikz{\draw[line width=0.45pt,line cap=round] (2pt,0) -- (0,4pt);}}}
\newcommand{\mysetminusSS}{\hbox{\tikz{\draw[line width=0.4pt,line cap=round] (1.5pt,0) -- (0,3pt);}}}
\newcommand{\mysetminus}{\mathbin{\mathchoice{\mysetminusD}{\mysetminusT}{\mysetminusS}{\mysetminusSS}}}
\renewcommand{\setminus}{\mysetminus}


\DeclareMathOperator{\inte}{int}
\DeclareMathOperator{\ext}{ext}
\DeclareMathOperator{\cla}{cl}
\DeclareMathOperator{\der}{der}
\DeclareMathOperator{\fron}{fr}
\DeclareMathOperator{\scl}{sqcl}
\DeclareMathOperator{\cf}{cf}
\DeclareMathOperator{\Id}{Id}
\DeclareMathOperator{\ima}{ima}
\DeclareMathOperator{\dom}{dom}
\DeclareMathOperator{\St}{St}
\DeclareMathOperator{\Osq}{so}
\DeclareMathOperator{\T}{T}
\DeclareMathOperator{\US}{US}
\DeclareMathOperator{\AN}{AN}
\DeclareMathOperator{\OR}{OR}
\DeclareMathOperator{\CAR}{CAR}
\DeclareMathOperator{\zfc}{ZFC}
\DeclareMathOperator{\zf}{ZF}
\DeclareMathOperator{\HC}{CH}
\DeclareMathOperator{\Ma}{MA}
\DeclareMathOperator{\Ac}{AC}
\DeclareMathOperator{\Pm}{MC}
\DeclareMathOperator{\Pdm}{WMC}
\DeclareMathOperator{\Ad}{AD}
\DeclareMathOperator{\Mad}{MAD}
\DeclareMathOperator{\Gen}{Gen}
\DeclareMathOperator{\MP}{MP}

\newcommand{\CTT}{\textcolor{magenta}{(CITE)}}