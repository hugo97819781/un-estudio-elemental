\section{Topología}

\index[alph]{espacio!topológico}\index[alph]{espacio}
Un \textit{espacio topológico} (o simplemente, \textit{espacio}) es un par ordenado $(X,\tau)$, donde $X$ es un conjunto y $\tau \subseteq \ms{P}(X)$ es una colección cerrada bajo uniones arbitrarias, intersecciones finitas y tal que $\emptyset, X \in \tau$, en tal caso $\tau$ es una \textit{topología} para $X$. En la gran mayoría de ocasiones, se hará referencia al espacio $(X,\tau)$ únicamente con el nombre de su conjunto subyacente, $X$. En todo lo que resta del capítulo, $X$ y $Y$ serán espacios topológicos.

\index[alph]{base!local}\index[alph]{base!de vecindades}\index[alph]{vecindad}\index[alph]{conjunto!vecindad}
Una \textit{vecindad} para un punto $x \in X$ es un conjunto $N \subseteq X$ tal que existe un abierto $U$ de $X$ con la propiedad $x \in U \subseteq N$. Una \textit{base de vecindades} de un punto $x$ en $X$ es una colección $\mathcal{B}_x$ de vecindades de $x$ tal que para cada abierto $U$, con $x \in U$, existe $B \in \mathcal{B}_x$ de modo que $x \in B \subseteq U$. Una \textit{base local} de $x$ es una base de vecindades de $x$ compuesta de abiertos del espacio.

\index[alph]{función!abierta}\index[alph]{función!encaje}\index[alph]{encaje}\index[sym]{$X \cong Y$}\index[alph]{espacio!encajado en}\index[alph]{copia homeomorfa}
Si $X$ y $Y$ son homeomorfos, escribiremos $X \cong Y$. Una función $f:X \to Y$ es \textit{encaje} si es un homeomorfismo entre $X$ y $f[X] \subseteq Y$, en tal caso, se dice que $X$ \textit{está encajado} en $Y$ o que $Y$ contiene una \textit{copia homeomorfa} de $X$. Una función $g:X \to Y$ es \textit{abierta} si para cada $U$ abierto en $X$, $g[U]$ es abierto en $Y$.

\index[alph]{producto topológico}\index[alph]{suma topológica}\index[sym]{$\coprod_{\alpha \in I} X_\alpha$}\index[sym]{$\prod_{\alpha \in I} X_\alpha$}\index[alph]{función!proyección}\index[alph]{proyección!cartesiana}\index[sym]{$\pi_\alpha$}
La \textit{suma topológica} y el \textit{producto topológico}, de una colección no vacía de espacios $\{X_\alpha \tq \alpha \in I \}$, se denotarán $\coprod_{\alpha \in I} X_\alpha$ y $\prod_{\alpha \in I} X_\alpha$, respectivamente. Si $\beta \in I$, $\pi_\beta:\prod_{\alpha \in I} X_\alpha \to X_\beta$ será la $\beta$-ésima \textit{proyección cartesiana} ($f \mapsto f(\beta)$).

\index[alph]{operador!derivado}\index[alph]{operador!clausura}\index[alph]{operador!interior}\index[alph]{operador!exterior}\index[alph]{operador!frontera}\index[alph]{derivado}\index[alph]{clausura}\index[alph]{interior}\index[alph]{exterior}\index[alph]{frontera}\index[sym]{$\der(A)$}\index[sym]{$\cla(A)$}\index[sym]{$\inte(A)$}\index[sym]{$\ext(A)$}\index[sym]{$\fron(A)$}
Para cualquier $A \subseteq X$ utilizaremos las notaciones $\der(A)$, $\inte(A)$, $\cla(A)$, $\ext(A)$ y $\fron(A)$ para: el conjunto de puntos de acumulación, el \textit{interior}, la \textit{clausura}, el \textit{exterior} y la \textit{frontera} de $A$, respectivamente.

\index[alph]{propiedad!topológica}\index[alph]{invariante!topológico}\index[alph]{propiedad!topológica!hereditaria}\index[alph]{propiedad!topológica!débilmente hereditaria}\index[alph]{propiedad!topológica!productiva}\index[alph]{propiedad!topológica!finitamente productiva}\index[alph]{propiedad!topológica!factorizable}
Una \textit{propiedad}, o \textit{invariante topológico}, es una cualidad que se preserva bajo homeomorfismos. Una propiedad topológica es: \textit{hereditaria} (\textit{débilmente hereditaria}, respectivamente) si se preserva bajo subespacios (subespacios cerrados, respectivamente), \textit{productiva} (\textit{finitamente productiva}, respectivamente) si se preserva bajo productos arbitrarios (productos finitos, respectivamente); y es \textit{factorizable} si cada vez que un producto topológico cuenta con ella, cada factor también. Toda propiedad hereditaria es factorizable.

\subsection{Axiomas de numerabilidad y separación}

\index[alph]{peso}\index[alph]{densidad}\index[alph]{carácter}\index[alph]{espacio!de peso $\kappa$}\index[alph]{espacio!de densidad $\kappa$}\index[alph]{punto de caracter $\kappa$}\index[sym]{$w(X)$}\index[sym]{$\chi(x,X)$}\index[sym]{$d(X)$}\index[alph]{espacio!primero numerable}\index[alph]{espacio!segundo numerable}\index[sym]{$1\AN$}\index[sym]{$2\AN$}

Los números $w(X)$ y $d(X)$ denotan los mínimos cardinales $\kappa$ para los cuales $X$ tiene una base de tamaño $\kappa$ y un denso de cardinalidad $\kappa$, respectivamente. Para cada $x \in X$, $\chi(x,X)$ es el mínimo tamaño para cualquier base local de $x$. Estos números son el \textit{peso} de $X$, la \textit{densidad} de $X$, y el \textit{carácter} de $x \in X$, respectivamente. Se dice que el espacio $X$ es \textit{segundo numerable} ($2\AN$) si $w(X)\leq \omega$; \textit{primero numerable} ($1\AN$) cuando para cada $x \in X$, $\chi(x,X)\leq \omega$; y \textit{separable} si $d(X)\leq \omega$.

\begin{proposicion}\phantomsection\label{prop-baseKappa}
    Para todo espacio topológico $X$:
    \begin{enumerate}
        \item $d(X) \leq w(X)$.
        \item Cualquier base de $X$ contiene una base de tamaño $w(X)$.
    \end{enumerate}
\end{proposicion}
%\begin{proof}
%    (i) Fíjese ($\Ac$), para cada $B \in \mathcal{B}$, un elemento $x_B \in B$. De esta forma, si $U$ es cualquier abierto no vacío de $X$, existen un punto $x \in U$ y cierto $B \in \mathcal{B}$ de modo que $x \in B \subseteq U$. De aquí que $x_B \in U$, mostrando que $D:=\{ x_B \tq B \in \mathcal{B} \}$ es un denso de $X$ de cardinalidad menor o igual a $\kappa$.

%    (ii) Sea $R:=\{ (U,W) \in \mathcal{B}^2 \tq \exists C \in \mathcal{C} \, (U \subseteq C \subseteq W) \}$. Utilizando $\Ac$, para cada $(U,W) \in R$ fíjese $C_{U,W} \in W$ tal que $U \subseteq C_{U,W} \subseteq W$. De esta manera, $\mathcal{C}':=\{ C_{U,W} \tq U,W \in \mathcal{B}^2 \} \subseteq \mathcal{C}$ tiene tamaño no mayor a $\kappa$.
    
    
%    Sea $O$ un abierto de $X$ y supóngase que $x \in O$, entonces existe $W \in \mathcal{B}$ tal que $x \in W \subseteq O$; de donde, por ser $\mathcal{C}$ base, existe $C \in \mathcal{C}$ de modo que $x \in C \subseteq W \subseteq O$. De nuevo, por ser $\mathcal{B}$ base, existe cierto $U \in \mathcal{B}$ tal que $x \in U \subseteq C \subseteq W \subseteq O$. De esta forma, $(U,W) \in R$, y así $x \in C_{U,W} \subseteq O$, mostrando que $\mathcal{C}'$ es base de $X$.
%\end{proof}

\index[alph]{axioma!$\T_0$}\index[alph]{axioma!$\T_1$}\index[alph]{axioma!$\T_2$}\index[alph]{axioma!$\T_3$}\index[alph]{axioma!$\T_{3 {\scriptscriptstyle 1/2}}$}\index[alph]{axioma!$\T_4$}\index[alph]{espacio!de Hausdorff}\index[alph]{espacio!de Tychonoff}\index[alph]{espacio!completamente regular}\index[alph]{espacio!regular}\index[alph]{espacio!normal}

El espacio $X$ es \textit{regular} (\textit{completamente regular}, respectivamente) si cualquier cerrado $F$ y cualquier $x \in X \setminus F$ se separan por abiertos ajenos (se separan funcionalmente, respectivamente). Se dice que $X$ es \textit{normal} si cualquier par de cerrados ajenos se separan por abiertos ajenos. Los axiomas $\T_3,\T_{3 {\scriptscriptstyle 1/2}}$ y $\T_4$ son los axiomas de regularidad, regularidad completa y normalidad, respectivamente, adicionados con el axioma $\T_1$. Cuando $X$ es $\T_2$ se dice que es \textit{de Hausdorff}, y cuando es es $\T_{3 {\scriptscriptstyle 1/2}}$, se dice que es \textit{de Tychonoff}. Los axiomas $\T_0,\T_1,\T_2,\T_3,\T_{3 {\scriptscriptstyle 1/2}}$ y $\T_4$ son invariantes productivos, factorizables. Todos menos $\T_4$, son hereditarios, $\T_4$ sólo es es débilmente hereditario.

A continuación se enuncian dos de los resultados más notables en relación a los axiomas de separación. Las pruebas de los mismos pueden consultarse en \CTT, \CTT y \CTT, respectivamente.

\begin{teorema}[de inmersión de Tychonoff]\phantomsection\label{Teo-Inm-Tych}\index[alph]{Teorema!de inmersión de Tychonoff}\index[alph]{Tychonoff!Teorema de inmersión de}
    Un espacio $X$ es de Tychonoff si y sólo si se encaja en $[0,1]^{w(X)}$.
\end{teorema}

\begin{teorema}[extensión de Tietze]\phantomsection\label{Teo-Tietze}\index[alph]{Teorema!de extensión de Tietze}\index[alph]{Tietze!Teorema de extensión de}
    Un espacio $X$ es normal si y sólo si para cualquier cerrado $F \subseteq X$ y cualquier función continua $f:F \to \mathbb{R}$, existe $g:X \to \mathbb{R}$ continua y tal que $g \upharpoonright A = f$.
\end{teorema}

El siguiente Lema será utilizado en el CHAP 4, anexaremos su demostración, pues es una alicación sencilla del Teorema de extensión de Tietze.

\begin{corolario}[Lema de Jones]\phantomsection\label{lem-JonesS}\index[alph]{Lema!de Jones}\index[alph]{Jones!Lema de}
    Sea $X$ un espacio $\T_4$ y $A \subseteq X$ infinito, discreto y cerrado en $X$. Si $D$ es denso infinito en $X$, entonces $2^{|A|} \leq 2^{|D|}$. Particularmente, si $X$ es separable, $2^{|A|} \leq \mathfrak{c}$.
\end{corolario}
\begin{proof}
    Sea $F:=\mathbb{R}^S=\{ f \tq f:S \to \mathbb{R} \}$. Como $A$ es discreto, cada $f \in F$ es función continua. Ahora, como $A$ es cerrado, utilícese el Teorema de Tietze para fijar ($\Ac$), para cada $f \in F$, una extesión continua $g_f:X \to \mathbb{R}$ de $f$. Obsérvese que $f \mapsto g_f$ es inyectiva.

    Supóngase ahora que $h,k:X \to \mathbb{R}$ son continuas y tales que $h \upharpoonright D=k \upharpoonright D$. Sea $x \in X$ cualquiera, si $h(x)\neq k(y)$, existen dos abiertos ajenos de $\mathbb{R}$, a saber $U$ y $V$, tales que $h(x) \in U$ y $k(x) \in V$. Como $x \in h^{-1}[U] \cap k^{-1}[V]$ son abiertos no vacíos de $X$, existe $d \in D \cap h^{-1}[U] \cap k^{-1}[V]$, lo cual es imposible pues implica que $h(d)=k(d) \in U \cap V$. Por lo tanto, $h=k$; esto es, la asignación $h \mapsto \hat{h}$ es inyectiva. Por tanto, hay una inyección de $\mathbb{R}^S$ en $\mathbb{R}^D$, y así:
    \[ 2^{|S|} = 2^{\aleph_0 |S|} = (2^{\aleph_0})^{|S|} = |\mathbb{R}^S| \leq |\mathbb{R}^D| = (2^{\aleph_0})^{|D|} = 2^{\aleph_0 |D|} = 2^{|D|} \, \]
    finalizando la prueba.
\end{proof}

\index[alph]{espacio!cero-dimensional}
Un espacio $X$ es \textit{cero-dimensional} si y sólo si es $\T_1$ y contiene una base compuesta de conjuntos abiertos y cerrados a la vez, este invariante topológico es productivo y hereditario (por tanto, factorizable).

\begin{proposicion}\phantomsection\label{cero-dim-caracte}\phantomsection\label{cerodim-Tych}
    Un espacio $X$ es cero-dimensional y sólo si se encaja en $2^{w(X)}$. Por tanto, todo espacio cero-dimensional es de Tychonoff (\ref{Teo-Inm-Tych}).
\end{proposicion}
\begin{proof}
    (i) $\to$ (ii) Supóngase que $X$ es cero-dimensional y sea $\mathcal{B}$ una base compuesta de abiertos y cerrados para $X$, enumerada como $\{ B_\alpha \tq \alpha \in w(X) \}$ (usando \ref{prop-baseKappa}). Defínase la función $f:X \to 2^{w(X)}$, para cada $x \in X$, como $f(x)(\alpha)=1$ si y sólo si $x \in B_\alpha$.

    $f$ es continua, pues para cada $\alpha \in w(X)$, $\pi_\alpha f$ es la función característica del abierto y cerrado $B_\alpha \subseteq X$. Ahora, si $x \neq y$, entonces por ser $X$ espacio $\T_1$, existe $\beta \in w(X)$ de manera que $x \in B_\beta$, pero $y \notin B_\beta$; así que $f(x)(\beta) \neq f(y)(\beta)$ y en consecuencia $f(x) \neq f(y)$.
    
    Finalmente, verificaremos que para cualquier $\alpha \in w(X)$, $f[B_\alpha]$ es abierto en $f[X]$, lo suficiente para mostrar que $f$ es abierta, y con ello homeomorfismo, sobre su imagen. Si $f(x) \in f[B_\alpha]$ es arbitrario, $x \in B_\alpha$ (por ser $f$ inyectiva); y así $f(x) \in \pi_\alpha^{-1}[\{1\}] \cap f[X] \subseteq f[B_\alpha] \cap f[X]$. En efecto, si $f(y) \in \pi_\alpha^{-1}[\{1\}] \cap f[X]$, entonces $f(y)(\alpha)=1$, mostrando que $y \in B_\alpha$, y así, $f(y) \in f[B_\alpha]$. Por lo tanto, $f[B_\alpha]$ es abierto en $f[X]$, luego, $f$ es encaje de $X$ en $2^{w(X)}$.

    (ii) $\to$ (i) El espacio $2=\{0,1\}$ es cero-dimensional, así que por ser tal propiedad productiva y hereditaria, cualquier subespacio de $2^{w(X)}$ es cero-dimensional.
\end{proof}

\subsection{Convergencia de sucesiones}

\index[alph]{subsucesión}\index[sym]{$x_n \to a$}\index[sym]{espacio!de convergencia única}\index[sym]{$\lim(x_n)$}
Durante esta subsección, $(x_n)_{n\in \omega}$ será una sucesión en $X$. Un \textit{subsucesión} de $(x_n)_{n\in \omega}$ es una sucesión de la forma $(x_{f(n)})_{n \in \omega}$, donde $f:\omega \to \omega$ es estrictamente creciente. Se utilizará la notación $x_n \to a$ para indicar que $(x_n)_{n\in \omega}$ converge al punto $a$ del espacio. El espacio $X$ es \textit{de convergencia única} si para cualquier sucesión $(y_n)_{n \in \omega} \subseteq X$, $y_n \to a$ y $y_n \to b$ implican $a = b$ (en tal caso, escribimos $a = \lim_{n \to \infty} x_n$). Todo espacio $\T_2$ es de convergencia única. %\cite[Prop. 1.6.11]{engelTopo}.

\index[alph]{sucesión!convergente (conjunto)}\index[alph]{conjunto!convergente}\index[sym]{$A \to a$}\index[sym]{$\lim(A)$}
Un subconjunto numerable $A \in [X]^\omega$ es una \textit{sucesión convergente} si existe $a \in X$ de manera que, para cada abierto $U$ de $X$ con $a \in U$, $A \subseteq ^* U$; lo cual se denotará como $A \to a$, o bien $a=\lim(A)$ si $X$ es de convergencia única.

\begin{proposicion}\phantomsection\label{prop-sucesionesConvergentes}
    Sean $X$ un espacio y $a \in X$. Entonces:
    \begin{enumerate}
        \item Si $(x_n)_{n \in \omega} \subseteq X$ es infinita y $x_n \to a$, entonces $x[\omega] \to a$.
        \item Si $A \in [X]^\omega$, $A \to a$ y $x:\omega \to A$ es inyección, $x_n \to a$.
    \end{enumerate}
\end{proposicion}
%\begin{proof}
%    (i) $\to$ (ii) Supóngase que $x_n \to a$ y sea $U$ un abierto tal que $a \in U$. Entonces existe $N \in \omega$ tal que $\{x_n \tq n\geq N\} \subseteq U$, consecuentemente, ocurre que $B \setminus U \subseteq \{x_n \tq n<N\} =^* \emptyset$, y así, $x[\omega] \to a$.

%    (ii) $\to$ (i) Supóngase que $A \to a$ y sea $x:\omega \to A$ cualquier inyección. Si $U$ es cualquier abierto en $X$ con $a \in U$, se cumple que $A \setminus U \subseteq x[\omega]$ es finito. Como $x$ es biyección, existe $N \in \omega$ tal que $A \setminus U \subseteq \{x_n \tq n < N\}$. Debido a lo anterior, $\{x_n \tq n \geq N \} \subseteq U$, y así, $x_n \to a$.
%\end{proof}

\index[sym]{espacio!secuencialmente compacto}
Un espacio $X$ es \textit{secuencialmente compacto} si y sólo si cada sucesión en $X$ tiene una subsucesión convergente; equivalentemente (en virtud de \ref{prop-sucesionesConvergentes}), para cada $A \in [X]^\omega$ existen $a \in X$ y $B \in [A]^\omega$ con $B \to a$.

\index[alph]{clausura!secuencial}\index[sym]{$\scl(A)$}\index[alph]{operador!clausura secuencial}
Para cada $A \subseteq X$ denotarmos por $\scl(A)$ la \textit{clausura secuencial} de $A$, esto es, el conjunto de puntos $x \in X$ tales que existe $(y_n)_{n \in \omega} \subseteq A$ con $y_n \to x$. Se dice que $A \subseteq X$ es \textit{secuencialmente cerrado} si $\scl(A) \subseteq X$. Como consecuencia de la \cref{prop-sucesionesConvergentes} se tiene que:
\begin{corolario}
    Si $X$ es espacio $\T_1$ y $A \subseteq X$, entonces:
    \[ \scl(A) = A \cup \{ x \in X \tq \exists B \in [A]^\omega \, (B \to a) \} \, . \]
\end{corolario}

\index[alph]{clausura!secuencial!transfinita}\index[sym]{$\scl^\alpha(A)$}\index[alph]{operador!clausura secuencial!transfinita}
Para cada ordinal $\alpha \leq \omega_1$ defínase la \textit{clausura secuencial transfinita} de $A$, por recursión en $\omega_1$, como:
\begin{enumerate}
    \item $\scl^0(A)=A$,
    \item Para todo ordinal $\alpha<\omega_1$, $\scl^{\alpha+1}(A)=\scl(\scl^\alpha(A))$, y
    \item Para todo ordinal límite $\gamma \leq \omega_1$, $\scl^{\gamma}(A)=\midcup\{ \scl^\alpha (A) \tq \alpha < \gamma \}$.
\end{enumerate} 

\index[alph]{orden!secuencial}\index[alph]{secuencial!orden}\index[sym]{$\Osq(X)$}
En caso de existir, se define el \textit{orden secuencial} de $X$, $\Osq(X)$, como el mínimo ordinal $\alpha$ tal que para cada $A \subseteq X$, $\scl^\alpha(A)=\cla(A)$.

El espacio $X$ es \textit{secuencial} cuando todos sus subconjuntos secuencialmente cerrados son cerrados en $X$, equivalentemente, cuando $\Osq(X)$ está bien definido. El espacio $X$ es \textit{de Fréchet} si para cada $A \subseteq X$, $\cla(A)=\scl(A)$, equivalentemente, si $\Osq(X)=1$. Todo espacio $1\AN$ es de Fréchet y todo espacio de Fréchet es secuencial. La propiedad de Féchet es hereditaria, por ello, factorizable. %\cite[Ej.~2.1.H]{engelTopo}.

\subsection{Compacidad y extensiones unipuntuales}

\index[alph]{cubierta}\index[alph]{subcubierta}\index[alph]{cubierta!abierta}\index[alph]{espacio!compacto}\index[alph]{espacio!$\sigma$-compacto}\index[alph]{espacio!numerablemente compacto}\index[alph]{espacio!de Lindelöf}\index[alph]{función!acotada}\index[alph]{espacio!localmente compacto}
Una cubierta, de un conjunto $A$, es una colección $\mathcal{U} \subseteq \ms{P}(X)$ tal que $A \subseteq \midcup \mathcal{U}$. Una \textit{subcubierta} de $\mathcal{U}$ es una cubierta de $A$ contenida en $\mathcal{U}$. Una cubierta de $X$ es \textit{abierta} cuando sus elementos son abiertos de $X$. El espacio $X$ es \textit{compacto} si cualquiera de sus cubiertas abiertas tiene una subcubierta finita; es \textit{numerablemente compacto} si toda cubierta contable de $X$ tiene una subcubierta finita; y es \textit{de Lindelöf} si toda cubierta abierta de $X$ tiene una cubierta abierta contable. $X$ es \textit{$\sigma$-compacto} cuando es unión numerable de subespacios compactos. $X$ es localmente compacto cuando cada punto $x \in X$ tiene una base de vecindades compactas. Todos estos invariantes son factorizables y débilmente hereditarias.

\index[alph]{refinamiento}\index[alph]{refinamiento!abierto}\index[alph]{cubierta!refinamiento}\index[sym]{$\mathcal{V} \preccurlyeq \mathcal{U}$}
Dada una cubierta $\mathcal{U}$ para un espacio $X$, decimos que una cubierta $\mathcal{V} \subseteq \ms{P}(X)$ es \textit{refinamiento} de $\mathcal{U}$ cuando para cada $V \in \mathcal{V}$ existe $U \in \mathcal{U}$ de manera que $V \subseteq U$, esta situación será denotada por $\mathcal{V} \preccurlyeq \mathcal{U}$. Un refinamiento es \textit{abierto} cuando todos sus elementos son abiertos. Si $\mathcal{V} \preccurlyeq \mathcal{U}$, fijando ($\Ac$) para cada $V \in \mathcal{V}$ un elemento $U_V \in \mathcal{U}$ con $V \subseteq U_V$ se obtiene una subcubierta de $\mathcal{U}$ de tamaño a lo más $|\mathcal{V}|$, a saber, $\{U_V \tq V \in \mathcal{V}\}$.

\begin{lema}
    Sea $X$ un espacio $2\AN$. Si $\mathcal{U}$ es una cubierta de $X$ y para cada $x \in X$ tiene una vecindad $N \in \mathcal{U}$, $\mathcal{U}$ contiene una subcubierta contable.
\end{lema}
\begin{proof}
    Por los comentarios previos al enunciado de esta proposición, basta encontrar un refinamiento a lo más numerable de $\mathcal{U}$. Defínase la colección $\mathcal{V}:=\{  B \in \mathcal{B} \tq \exists U \in \mathcal{U} \, (B \subseteq U) \}$, para que $\mathcal{V} \preccurlyeq \mathcal{U}$, resta ver que $\mathcal{V}$ cubre a $X$.

    Efectivamente, si $x \in X$, existe una vecindad de $N \in \mathcal{U}$ de $x$ en $X$. Así, existen un abierto $U$ de $X$, y un elemento $B \in \mathcal{B}$, tales que $x \in B \subseteq U \subseteq N$. Esto prueba que $B \in \mathcal{V}$, por lo que $X \subseteq \midcup \mathcal{V}$.
\end{proof}

\newpage
\subsection{Extensiones unipuntuales}

\subsection{Espacios Metrizables}