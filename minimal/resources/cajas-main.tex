%Colores
\definecolor{azul}{RGB}{0, 60, 113}
\definecolor{dorado}{RGB}{196, 151, 57}
\definecolor{morado}{RGB}{101, 43, 145}
\definecolor{rosa}{RGB}{146, 33, 209}
\definecolor{azulC}{RGB}{52, 90, 229}

%Setup
\mdfsetup{innertopmargin=-2pt, innerbottommargin=3pt}
%\mdfsetup{
%    splittable=true,
%    usename=true,
%    skipabove=2pt,
%    skipbelow=2pt,
%    needspace=no
%}

%Estilos de Cajas
\mdfdefinestyle{caja1}{
	leftline=true,
	rightline=false,
	topline=false,
	bottomline=false,
	linecolor=morado!90,
	linewidth=2.5pt,
	backgroundcolor=morado!5,
}

\mdfdefinestyle{caja2}{
	roundcorner=5pt,
	backgroundcolor=dorado!8,
	linewidth=0,
	%shadow=true,
	%shadowcolor=dorado!70,
	%shadowsize=4pt,
}

\mdfdefinestyle{caja3}{
	topline=false,
	bottomline=true,
	leftline=true,
	rightline=false,
	linewidth=1.5pt,
	linecolor=azul!90,
}

\mdfdefinestyle{caja4}{
	topline=true,
	bottomline=true,
	leftline=false,
	rightline=false,
	linewidth=1.5pt,
	linecolor=gray!90,
}

%Definicion de entornos
\newmdtheoremenv[style=caja1]{definicion}{Definición}[section]
\newmdtheoremenv[style=caja2]{proposicion}[definicion]{Proposición}
\newmdtheoremenv[style=caja2]{lema}[definicion]{Lema}
\newmdtheoremenv[style=caja2]{corolario}[definicion]{Corolario}
\newmdtheoremenv[style=caja3]{observacion}[definicion]{Observación}
\newmdtheoremenv[style=caja3]{ejemplo}[definicion]{Ejemplo}
\newmdtheoremenv[style=caja4]{consideracion}[definicion]{Consideración}

%Entorno (complicado) de teorema
\newcounter{theot}
\newenvironment{teorema}[1][]{%
	%\setcounter{theot}{\value{TemTheot}}
	\stepcounter{definicion}
	%\setcounter{theot}{\value{definicion}}
	\renewcommand{\thetheot}{\thedefinicion}
	\refstepcounter{theot}
	\ifstrempty{#1}%
	{\mdfsetup{%
			frametitle={%
					\tikz[baseline=(current bounding box.east),outer sep=0pt]
					\node[anchor=east,rectangle,fill=dorado!35]
					{\strut Teorema~\thetheot};}}
	}%
	{\mdfsetup{%
			frametitle={%
					\tikz[baseline=(current bounding box.east),outer sep=0pt]
					\node[anchor=east,rectangle,fill=dorado!35]
					{\strut \textbf{Teorema~\thetheot~(#1)}};}}%
	}%
	\mdfsetup{innertopmargin=0pt,innerleftmargin=3pt,innerrightmargin=3pt,linecolor=dorado!35,%
		linewidth=2pt,topline=true,
		frametitleaboveskip=\dimexpr-\ht\strutbox\relax,}
	\begin{mdframed}[]\relax%
		}{\end{mdframed}}
\providecommand*{\theotautorefname}{Teorema}

\makeatletter
\renewenvironment{proof}{%
	\par\pushQED{\qed}%
	\renewcommand{\qedsymbol}{$\blacksquare$}%
	\normalfont \topsep6\p@\@plus6\p@\relax
	\trivlist
	\item[\hskip\labelsep\bfseries\itshape Demostración.]\ignorespaces
}{%
	\popQED\endtrivlist\@endpefalse
}
\makeatother